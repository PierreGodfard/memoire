\documentclass{amsart}
% package
\usepackage[utf8]{inputenc}
\usepackage[T1]{fontenc}
\usepackage[french]{babel}
%\usepackage[backend=bibtex,style=verbose-trad2]{biblatex}
\usepackage{cite}
\usepackage{amsmath,amssymb}
\usepackage{amsfonts}
\usepackage{amsthm}
\usepackage{calc}
\usepackage{mathtools}
\usepackage{relsize}

\usepackage{tikz}
\usetikzlibrary{positioning}
\usepackage{tikz-cd}


\tikzcdset{arrow style=tikz}

% Pour les liens hypertexte
\usepackage[hidelinks]{hyperref}
\newcommand{\sref}[2]{\hyperref[#2]{#1 \ref*{#2}}}

% Numérotation des théorèmes, propositions, définitions, lemmes, remarques consécutives par subsection
\theoremstyle{plain}
\newtheorem{theo}{Théorème}[section]
\newtheorem{theodefi}[theo]{Théorème\,-\,définition}
\newtheorem{prop}[theo]{Proposition}
\newtheorem{propdefi}[theo]{Proposition\,-\,définition}
\newtheorem{coro}[theo]{Corollaire}
\newtheorem{lem}[theo]{Lemme}
\newtheorem{lemdefi}[theo]{Lemme\,-\,définition}
\newtheorem{conj}[theo]{Conjecture}

\theoremstyle{definition}
\newtheorem{defi}[theo]{Définition}
\newtheorem{ex}[theo]{Exemple}
\newtheorem{cstr}[theo]{Construction}

\theoremstyle{remark}
\newtheorem{rem}[theo]{Remarque}

\renewcommand{\proofname}{Démonstration}

\renewcommand{\descriptionlabel}[1]{%
  \hspace\labelsep \upshape\bfseries #1%
}

% Commande ensemble de nombres
\newcommand{\ensemblenombre }[1]{\mathbb{#1}}
\newcommand{\N}{\ensemblenombre{N}}
\newcommand{\Np}{\ensemblenombre{N}\setminus\{0\}}
\newcommand{\Z}{\ensemblenombre{Z}}
\newcommand{\Q}{\ensemblenombre{Q}}

\newcommand{\M}{\mathcal{M}}
\newcommand{\MP}{\mathcal{P}}
\newcommand{\A}{\mathcal{A}}
\newcommand{\E}{\mathcal{E}}
\newcommand{\Ring}[1]{\mathcal{O}_{#1}}

\newcommand{\Ob}[1]{\mathrm{Ob}\:#1}
\newcommand{\Hom}[3]{\mathrm{Hom}_{#1}(#2,#3)}
\newcommand{\Iso}[3]{\mathrm{Iso}_{#1}(#2,#3)}
\newcommand{\Isos}[1]{\mathrm{Isos}(#1)}
\newcommand{\Aut}[1]{\mathrm{Aut}({#1})}
\newcommand{\id}{\mathrm{id}}
\newcommand{\op}{^\mathrm{op}}
\newcommand{\ab}{^\mathrm{ab}}
\newcommand{\Fon}[2]{\mathrm{Fon}(#1,#2)}
\newcommand{\Ens}{\mathrm{Ens}}
\newcommand{\DEns}{\widehat{\mathbf{\Delta}}}
\newcommand{\DCat}{\mathbf{\Delta}}
\newcommand{\Chp}{\mathrm{\mathrm{Ch}_+}}
\newcommand{\DAb}{\mathbf{\Delta}\mathrm{Ab}}
\newcommand{\Cat}{\mathrm{Cat}}
\newcommand{\CatEx}{\mathrm{CatEx}}
\newcommand{\Ab}{\mathrm{Ab}}
\newcommand{\mylim}{\mathrm{lim}}
\newcommand{\colim}{\mathrm{colim}}
\newcommand{\hocolim}[2]{\mathrm{hocolim}_{#1}\:{#2}}
\newcommand{\Top}{\mathrm{Top}}
\newcommand{\Grp}{\mathrm{Grp}}
\newcommand{\Supp}[1]{\mathrm{Supp}({#1})}
\newcommand{\brak}[2]{{<{#1},{#2}>}}
\newcommand{\Tot}[1]{{\mathrm{Tot}({#1})}}
\newcommand{\sk}[2]{\mathrm{sk}_{#1}\;{#2}}

\newcommand{\Ho}[1]{\mathrm{Ho}({#1})}
\newcommand{\Sing}{\mathrm{Sing}}
\newcommand{\xlongrightrightarrows}[2]{\mathop{\rightrightarrows}_{#1}^{#2}}


\newcommand{\coend}[2]{\bigsqcup_{\phi:[n]\ra [m]}{#1}_m\times {#2}^n\xlongrightrightarrows{\sqcup\phi^*\times\id}{\sqcup\id\times\phi_*}\bigsqcup_{[n]}{#1}_n\times {#2}^n}

\newcommand{\GL}[2]{\mathrm{GL}_{#1}(#2)}
\newcommand{\End}[2]{\mathrm{End}_{#1}(#2)}
\newcommand{\EGL}[2]{\mathrm{E}_{#1}(#2)}

\newcommand{\Proj}[1]{\mathrm{P}({#1})}
\newcommand{\Pgr}[1]{\mathrm{Pgr}({#1})}
\newcommand{\Modf}[1]{\mathrm{Modf}({#1})}
\newcommand{\Tor}{\mathrm{Tor}}
\newcommand{\fTor}[4]{\mathrm{Tor}_{#1}^{#2}({#3},{#4})}

\newcommand{\xrightarrowtail}[1]{\overset{#1}{\rightarrowtail}}
\newcommand{\xtwoheadrightarrow}[1]{\overset{#1}{\twoheadrightarrow}}
\newcommand{\xtwoheadleftarrow}[1]{\overset{#1}{\twoheadleftarrow}}
\newcommand{\ra}{\rightarrow}
\newcommand{\lra}{\longrightarrow}
\newcommand{\mono}{\rightarrowtail}
\newcommand{\epi}{\twoheadrightarrow}
\newcommand{\exa}[3]{0\ra {#1}\ra {#2}\ra {#3}\ra 0}
\newcommand{\exac}[3]{{#1}\rightarrowtail {#2}\twoheadrightarrow {#3}}
\newcommand{\exaname}[5]{0\ra {#1}\xrightarrow{#2} {#3}\xrightarrow{#4} {#5}\ra 0}
\newcommand{\exacname}[5]{{#1}\xrightarrowtail{#2} {#3}\xtwoheadrightarrow{#4} {#5}}

\newcommand{\myker}[1]{\mathrm{ker}({#1})}
\newcommand{\coker}[1]{\mathrm{coker}({#1})}
\newcommand{\im}[1]{\mathrm({#1})}

\newcommand{\Spec}[1]{\mathrm{Spec}({#1})}
\newcommand{\Open}[1]{\mathrm{D}({#1})}

% Commande "ensemble tel que"
\newcommand{\enstq}[2]{\left\{#1\,\middle|\,#2\right\}}


\title{K-théorie algébrique}
\author{Pierre Godfard}
\date{}


\begin{document}
\renewcommand{\abstractname}{Introduction}
\begin{abstract}
  [TBD]
\end{abstract}

\maketitle

\tableofcontents

\section*{Remarque préliminaire}

On suppose ici l'existence d'univers dans la théorie des ensembles (cardinaux fortement inaccessibles).
On note $\Omega$ le plus petit. Toutes les structures algébriques seront par défaut dans $\Omega$.
Pour les ensembles et la catégorie, nous diront qu'ils ou elles sont petit(e)s
quand ils ou elles sont dans $\Omega$.

\section{\texorpdfstring{Les groupes $K_0$ et $K_1$}{Les groupes KO et K1}}\label{sectionK0}

Dans cette section, nous allons définir et étudier les groupes $K_0$, $K_1$ et $K_2$ associés à un anneau. Nous définirons également le $K_0$ d'une catégorie exacte.

\subsection{\texorpdfstring{Les groupes $K_0$  et $K'_0$ d'un anneau}{Les groupes K0  et K'0 d'un anneau}}

Pour $A$ un anneau, nous noterons $\Proj{A}$ la catégorie des modules projectifs de type fini sur $A$ et $\Modf{A}$ la catégorie des modules de type fini sur $A$.
Si $A$ n'est pas commutatif, les modules sur $A$ seront toujours par défaut des $A$-modules à gauche.

\begin{defi}[$K_0$  et $K'_0$]
  Le groupe abélien $K_0(A)$ est le groupe de Grothendieck de $\Proj{A}$~: le quotient du groupe abélien libre $\bigoplus_P \Z\cdot [P]$ engendré par les objets de $\Proj{A}$,
  par les relations $[R] = [P] + [Q]$ pour chaque suite exacte $0\ra P\ra R\ra Q\ra 0$.

  On définit de même le groupe $K'_0(A)$ comme le groupe de Grothendieck de $\Modf{A}$.
\end{defi}

On appelle $K_0(A)$ le groupe de $K$-théorie de $A$ et $K'_0(A)$ le groupe de $K'$-théorie de $A$. En général, pour utiliser $K'_0(A)$,
on préfère se restreindre au cas où $A$ est noethérien, de telle sorte que $\Modf{A}$ soit une catégorie abélienne.

\begin{rem}
  Les catégories $\Proj{A}$ et $\Modf{A}$ ont un ensemble dans $\Omega$ de classes d'isomorphisme. Or si $f:M\ra M'$ est un isomorphisme, les suites exactes $\exa{0}{0}{0}$ et $\exa{M}{M'}{0}$,
  induisent $0=[0]$ et $[M]=[M']$ dans les groupes de Grothendieck. Donc les éléments des groupes $K_0(A)$  et $K'_0(A)$ forment des ensembles de $\Omega$.
\end{rem}

Le foncteur $\Proj{A}\times \Proj{A}\lra \Proj{A},\;(P,Q)\mapsto P\oplus Q$ induit un morphisme $\oplus:K_0(A)\times K_0(A)\lra K_0(A)$.
Or pour tout $A$-modules $M$ et $N$, on a une suite exacte $\exa{M}{M\oplus N}{N}$. Donc $\oplus$ est l'addition. De même avec $\Modf{A}$ et $K'_0$.

Si $A$ est commutatif, le foncteur $\Proj{A}\times \Proj{A}\lra \Proj{A},\;(P,Q)\mapsto P\otimes Q$ passe également au quotient en un morphisme
$\otimes:K_0(A)\times K_0(A)\lra K_0(A)$ (les modules projectifs sont plats).

De même, $\Proj{A}\times \Modf{A}\lra \Modf{A},\;(P,M)\mapsto P\otimes M$ induit $\otimes:K_0(A)\times K'_0(A)\lra K'_0(A)$.
On vérifie aisément que~:
\begin{prop}
  Si $A$ est commutatif, l'application $\otimes:K_0(A)\times K_0(A)\lra K_0(A)$ est bilinéaire et induit sur $K_0(A)$ une structure d'anneau de neutre $[0]$ et d'unité $[A]$.

  Dans ce cas, $\otimes:K_0(A)\times K'_0(A)\lra K'_0(A)$ fait de $K_0'(A)$ un $K_0(A)$-module.
\end{prop}

Ces definitions sont fonctorielles en $A$. En effet, si $f:A\ra B$ est un morphisme d'anneaux, le foncteur $-\otimes_A B:\Proj{A}\ra \Proj{B}$
induit un morphisme $f^*:K_0(A)\ra K_0(B)$, et on a, si $g:B\ra C$, $(gf)^*=g^*f^*$. Le morphisme $f^*$ est un morphisme d'anneau si $A$ et $B$ sont commutatifs.
De même, si $f:A\ra B$ est un morphisme plat, $-\otimes_A B:\Modf{A}\ra \Modf{B}$ induit un morphisme $f^*:K'_0(A)\ra K'_0(B)$ (de $K_0(A)$-modules si $A$ et $B$ sont commutatifs).
La notation "contravariante" $f^*$ s'explique par la definition de la $K$-théorie d'un schema.

Si $f:A\ra B$ est un morphisme projectif et fini (c'est-à-dire que $B$ est un $A$ module projectif de type fini),
alors le foncteur d'oubli $(-)_A:\Proj{B}\ra \Proj{A}, M\mapsto M$ induit un morphisme \textbf{de groupe} $f_*:K_0(B)\ra K_0(A)$.
De même, si $f:A\lra B$ est un morphisme fini, le foncteur d'oubli $(-)_A:\Modf{B}\ra \Modf{A}$ induit un morphisme \textbf{de groupe} $f_*:K'_0(B)\ra K'_0(A)$.

Si $f_*$ et $f^*$ sont définis pour $K_0$, et si $A$ et $B$ sont commutatifs, l'isomorphisme entre les foncteurs
$(P,N)\mapsto (P\otimes_AN)\otimes B$ et $(P,N)\mapsto P\otimes_B(N\otimes B)$
induit la formule de projection suivante. Pour tout $a$ dans $K_0(A)$ et $b$ dans $K_0(B)$~:
$$f_*(b\cdot f^*(a))=f_*(b)\cdot a\text{ dans }K_0(A)$$
De même, si $f_*$ est défini pour $K_0$ et $K'_0$, et $f^*$ est défini pour $K'_0$, alors pour tout $a$ dans $K'_0(A)$ et $b$ dans $K_0(B)$~:
$$f_*(b\cdot f^*(a))=f_*(b)\cdot a\text{ dans }K'_0(A)$$


On peut affaiblir les hypothèses sur $f$ sous lesquelles $f_*$ en $K$-théorie et $f^*$ en $K'$-théorie sont définis.
Pour les definitions de la dimension projective et de la $\Tor$-dimension, voir le \cite[Chp.4]{Weib2}.

\begin{defi}
  Un morphisme d'anneau $f:A\ra B$ est dit de $\Tor$-dimension finie si $B$ est un $A$-module de $\Tor$-dimension finie.
\end{defi}

Pour étendre $f^*$, nous utilisons le résultat suivant.

\begin{prop}\label{KprimeTordimfinie}
  Soit $f:A\ra B$ de $\Tor$-dimension finie avec $A$ noethérien.
  Soit $\mathcal{M}$ la sous-catégorie pleine de $\Modf{A}$ des modules $M$ tels que $\fTor{i}{A}{B}{M}=0$ pour tout $i\geq 1$. On note $K'_0(\mathcal{M})$
  son groupe de Grothendieck (cf. \sref{définition}{defiK0catexacte} si ce n'est pas clair). Alors l'inclusion $\mathcal{M}\ra \Modf{A}$ induit un isomorphisme $K'_0(\mathcal{M})\ra K'_0(A)$.
\end{prop}

\begin{proof}
  Nous allons décrire un morphisme inverse $K'_0(A)\ra K'_0(\mathcal{M})$. On note $n<\infty$ la $\Tor$-dimension de $B$ sur $A$. Soit $M$ un $A$-module de type fini.
  On se donne $P_*\ra M$ une résolution projective de $M$ par des $A$-modules de type fini (possible car $A$ est noethérien).
  On pose $\tilde{M} := \coker{d:P_{n+1}\ra P_n}$. Par la suite exacte longue des foncteurs $\Tor$,
  $\tilde{M}$ est un objet de $\mathcal{M}$.
  On obtient donc une résolution finie $0\ra \tilde{M}\ra P_{n-1}\ra \dotsb\ra P_0\ra M\ra 0$ de $M$ par des objets de $\mathcal{M}$, que l'on note $Q_*\ra M$.
  On pose alors~:
  $$u(M) := \sum_{i=0}^n (-1)^i[Q_i]\in K'_0(\mathcal{M})$$
  Montrons que $u(M)$ est indépendant de la résolution $P_*\ra M$. Soit $P'_*\ra M$ une autre résolution. On montre facilement qu'il existe une troisième
  résolution $P''_*\ra M$ et deux morphismes $a:P''_*\ra P_*$ et $a':P''_*\ra P'_*$ surjectifs en chaque degrés, par exemple en prenant une résolution
  projective du complexe $P\times_{M[0]}P'$ par des modules de type fini.
  Alors $\myker{a}_*\ra 0$ est une résolution de $0$ par des objets de $\mathcal{M}$.
  On note $\tilde{M}:= \coker{d:P_{n+1}\ra P_n}$, $\tilde{M''}:= \coker{d:P''_{n+1}\ra P''_n}$ et $\tilde{K}:= \coker{d:\myker{a}_{n+1}\ra \myker{a}_n}$.
  Alors $\exa{\tilde{K}}{\tilde{M''}}{\tilde{M}}$ est exacte. En effet la suite est exacte à droite par le lemme du serpent et exacte à gauche car $\tilde{K}$ et
  $\tilde{M''}$ s'injectent dans $P''_{n-1}$. Si on note $Q''_*\ra M$ la résolution finie associée à $P''_*$ et $K_*\ra 0$ celle associée à $\myker{a}_*\ra 0$,
  on a~:
  $$\sum_{i=0}^n (-1)^i[Q_i]=\sum_{i=0}^n (-1)^i[Q''_i]-\sum_{i=0}^n (-1)^i[K_i]=\sum_{i=0}^n (-1)^i[Q''_i]\text{ dans }K'_0(\mathcal{M})$$
  Montrons maintenant que si $\exa{M'}{M}{M''}$ est une suite exacte dans $\Modf{M}$, alors $u(M) = u(M'')+u(M')$.
  Par le lemme du fer à cheval, on dispose de résolutions respectant la suite exacte~:
  \begin{center}
    \begin{tikzcd}
      0\rar & P'_* \rar\dar & P_* \rar\dar & P''_* \rar\dar & 0\\
      0\rar & M'   \rar     & M   \rar     & M''   \rar     & 0
    \end{tikzcd}
  \end{center}
  Par le même argument que ci-dessus $\exa{\tilde{M'}}{\tilde{M}}{\tilde{M''}}$ est exacte, et donc $\exa{Q'_*}{Q_*}{Q''_*}$ est exacte.
  Il est alors clair que $u(M) = u(M'')+u(M')$.

  On vérifie immédiatement que $u$ est inverse à droite et à gauche de $K'_0(\mathcal{M})\ra K'_0(A)$.
\end{proof}

Ainsi, avec les hypothèses et notations de la proposition, le foncteur $-\otimes_A B:\mathcal{M}\ra \Modf{B}$ induit un morphisme $K'_0(\mathcal{M})\ra K'_0(B)$
et donc un morphisme $f^*:K'_0(A)\ra K'_0(B)$. Si $B$ est plat sur $A$, on retrouve la définition plus haut.

Pour étendre $f_*$, on donne une formule explicite.

\begin{propdefi}
  Soit $f:A\ra B$ fini et de $\Tor$-dimension finie avec $A$ noethérien. Alors tout $B$ module projectif $P$ admet
  une $A$-résolution projective finie par des modules de type fini $Q_*\ra P$. On pose~:
  $$f_*([P]) := \sum_{i} (-1)^i[Q_i]\in K_0(A)$$
  Alors $f_*:K_0(B)\ra K_0(A)$ est un morphisme de groupe.
\end{propdefi}

On remarque que la définition ci-dessus étend la définition plus haut dans le cas où $f$ est projectif de type fini.

\begin{proof}
  Soit $n$ la $\Tor$-dimension de $B$ sur $A$. Comme $A$ est noethérien, pour tout module de type fini $M$ sur $A$, sa dimension projective
  est égale à sa $\Tor$-dimension. De plus, pour tout $B$-module projectif de type fini $P$, $\Tor\dim(P)=n$. Donc tout tel module $P$ admet une
  $A$-résolution projective finie par des modules de type fini $Q_*\ra P$. 

  Montrons que $\sum_{i} (-1)^i[Q_i]$ est indépendant de la résolution $Q_*\ra P$. Comme dans la \sref{proposition}{KprimeTordimfinie}, on peut
  supposer qu'il existe $u:Q'_*\ra Q_*$ surjection entre les deux résolutions considérées. Alors,
  $K:=\myker{u}_*$ est un complexe fini acyclique de modules projectifs de type fini. Donc~:
  $$\sum_{i} (-1)^i[Q'_i]=\sum_{i} (-1)^i[Q_i]+\sum_{i} (-1)^i[K_i]=\sum_{i} (-1)^i[Q_i]$$

  Montrons maintenant que si $\exa{P'}{P}{P''}$ est une suite exacte dans $\Proj{B}$, alors $f_*[P] = f_*[P'']+f_*[P']$.
  Par le lemme du fer à cheval, on dispose de résolutions finies respectant la suite exacte~:
  \begin{center}
    \begin{tikzcd}
      0\rar & Q'_* \rar\dar & Q_* \rar\dar & Q''_* \rar\dar & 0\\
      0\rar & P'   \rar     & P   \rar     & P''   \rar     & 0
    \end{tikzcd}
  \end{center}
  On a alors clairement $f_*[P] = f_*[P'']+f_*[P']$.  
\end{proof}

On peut maintenant étendre la formule de projection.

\begin{prop}[formule de projection]
  Soit $f:A\ra B$ fini et de $\Tor$-dimension finie, avec $A$ et $B$ commutatifs et $A$ noethérien.
  Alors pour tout $a$ dans $K_0'(A)$ et $b$ dans $K_0(B)$~:
  $$f_*(b\cdot f^*(a))=f_*(b)\cdot a\text{ dans }K'_0(A)$$
  De même, pour tout $a$ dans $K_0(A)$ et $b$ dans $K_0(B)$~:
  $$f_*(b\cdot f^*(a))=f_*(b)\cdot a\text{ dans }K_0(A)$$
\end{prop}

\begin{proof}
  On reprend les notations de la \sref{proposition}{KprimeTordimfinie}. On démontre la première formule, la seconde se démontre de façon similaire.
  On se ramène au cas où $a=[M]$ avec $M$ dans $\mathcal{M}$ et $b=[P]$ avec $P$ dans $\Proj{B}$. On se donne $Q_*\ra P$ une résolution
  finie de $P$ par des $A$-modules projectifs de type fini. Or $M\otimes_A Q_*\ra M\otimes_A P$ est exacte
  car pour tout $i>0$, $\fTor{i}{A}{M}{P} = 0$. Donc~:
  \begin{align*}
    f_*(b\cdot f^*(a)) &= [P\otimes_B (B\otimes_A M)] \\
                       &= [P\otimes_A M] \\
                       &= \sum_i(-1)^i[Q_i\otimes_A M] \\
                       &= (\sum_i(-1)^i[Q_i])\cdot [M] \\
                       &= f_*(b)\cdot a
  \end{align*}
\end{proof}


Nous allons maintenant nous intéresser au cas particulier des anneaux réguliers. Nous verrons que si $A$ est régulier, alors $K_0(A)$ et $K'_0(A)$ coïncident.

\begin{defi}
  Un anneau noethérien (à gauche) $A$ est dit proj-régulier si tout $A$-module de type fini est de dimension projective finie.
\end{defi}

Dans le cas commutatif, cette definition coïncide avec la definition usuelle d'anneau régulier (cf. le \sref{corollaire}{regulierprojregulier} ci-dessous).

\begin{theo}\label{thmanneauxreguliers}
  Soit $(A,\mathfrak{m})$ un anneau local commutatif noethérien. S'équivalent~:
  \begin{description}
    \item[(i)] $A$ est de dimension projective finie~;
    \item[(ii)] $A$ est local régulier ($\mathrm{Krull}\dim A = \dim_{A/\mathfrak{m}} \mathfrak{m}/\mathfrak{m}^2$)~;
    \item[(iii)] $k:=A/\mathfrak{m}$ est de dimension projective finie sur $A$.
  \end{description} 
  Et dans ce cas, la dimension projective de $A$ est la dimension projective de $k:=A/\mathfrak{m}$ comme $A$-module.
\end{theo}

Nous renvoyons au \cite[Chp.4]{Weib2} pour la démonstration et les différentes propriétés élémentaires de la dimension projective qui seront utilisées par la suite.

\begin{coro}\label{regulierprojregulier}
  Un anneau commutatif noethérien $A$ est proj-régulier si et seulement si il est régulier (ie. pour tout $\mathfrak{m}$ idéal maximal,
  $\mathrm{ht}\:\mathfrak{m} = \dim_{A/\mathfrak{m}} \mathfrak{m}/\mathfrak{m}^2$).
\end{coro}

\begin{proof}
  $\mathbf{(\Leftarrow)}$ Soit $P_*\ra M$ une résolution projective par des modules de type fini d'un $A$-module de type fini $M$. On note $Z_n:=\coker{d:P_n\ra P_{n-1}}$.
  Comme pour tout $\mathfrak{m}$ idéal maximal, $M_\mathfrak{m}$ est de dimension projective finie, par le \sref{théorème}{thmanneauxreguliers},
  il existe $n\geq 0$ tel que $(Z_n)_\mathfrak{m}$ soit projectif.
  Mais, comme $A$ est noethérien, $Z_n$ est de présentation finie. Donc $(Z_n)_\mathfrak{m}$ est un $A_\mathfrak{m}$-module libre de type fini.
  Donc, il existe $f_\mathfrak{m}\in A\backslash\mathfrak{m}$ tel que $(Z_n)_{f_\mathfrak{m}}$ soit libre, et donc projectif. Or si $(Z_n)_{f_\mathfrak{m}}$
  est projectif, $(Z_m)_{f_\mathfrak{m}}$ est aussi projectif pour tout $m\geq n$. Donc, comme $\Spec{A}$ est quasi-compact, il existe $n\geq 0$ et $f_1,\dotsc,f_l$
  tels que $\Spec{A}=\bigcup_i\Open{f_i}$ et $(Z_n)_{f_i}$ soit projectif de type fini pour tout $i$. Alors $Z_n$ est projectif,
  et $M$ est de dimension projective finie sur $A$.

  $\mathbf{(\Rightarrow)}$ Soit $\mathfrak{m}$ un idéal maximal. Alors $A/\mathfrak{m}$ est de dimension projective finie sur $A$ et donc aussi sur $A_\mathfrak{m}$.
  Donc, d'après le \sref{théorème}{thmanneauxreguliers}, $A_\mathfrak{m}$ vérifie $\mathrm{Krull}\dim A_\mathfrak{m} = \dim_{A/\mathfrak{m}} \mathfrak{m}/\mathfrak{m}^2$.
  On a donc bien $\mathrm{ht}\:\mathfrak{m} = \dim_{A/\mathfrak{m}} \mathfrak{m}/\mathfrak{m}^2$.
\end{proof}


\begin{theo}\label{KegalKprimeAnneauxReguliers}
  Soit $A$ un anneau commutatif noethérien régulier. Alors le foncteur naturel $\Proj{A}\ra\Modf{A}$ induit un isomorphisme $K_0(A)\ra K'_0(A)$.
\end{theo}

\begin{proof}
  Nous allons décrire un inverse $u:K'_0(A)\ra K_0(A)$. Soit $M$ un $A$-module de type fini, et soit $P_*\ra M$ une résolution finie par des $A$-modules projectifs.
  On pose $u(M):=\sum_i(-1)^i[P_i]$.

  Montrons que $u(M)$ ne dépend pas de la résolution. Comme dans la preuve de la \sref{proposition}{KprimeTordimfinie}, il suffit
  de se restreindre au cas où il existe $a:P'_*\ra P_*$ surjection entre deux résolutions de $M$. Alors $\myker{a}_*\ra 0$ est une résolution projective.
  On a alors $\sum_i(-1)^i[P_i]=\sum_i(-1)^i[P'_i]-\sum_i(-1)^i[\myker{a}_i]=\sum_i(-1)^i[P'_i]$.

  Montrons maintenant que si $\exa{M'}{M}{M''}$ est une suite exacte dans $\Modf{M}$, alors $u(M) = u(M'')+u(M')$.
  Par le lemme du fer à cheval, on dispose de résolutions finies respectant la suite exacte~:
  \begin{center}
    \begin{tikzcd}
      0\rar & P'_* \rar\dar & P_* \rar\dar & P''_* \rar\dar & 0\\
      0\rar & M'   \rar     & M   \rar     & M''   \rar     & 0
    \end{tikzcd}
  \end{center}
  On a alors clairement $u(M) = u(M'')+u(M')$.
\end{proof}

On remarque que ce théorème ressemble beaucoup à la \sref{proposition}{KprimeTordimfinie}. Ces deux résultats sont encore vraie en $K$-théorie supérieure,
et sont des applications d'un même résultat (voir la \sref{sous-section}{soussectionresolution}).


\subsection{\texorpdfstring{Les groupes $K_0$ et $K'_0$ d'un schéma}{Les groupes K0 et K'0 d'un schéma}}

On peut étendre la definition des groupes de $K$ et $K'$-théorie des anneaux commutatifs aux schémas.
Pour $X$ un schéma, nous noterons $\Proj{X}$ la catégorie des modules projectifs localement de type fini sur $X$
et $\Modf{X}$ la catégorie des modules quasi-cohérents localement de type fini sur $X$.

\begin{defi}[$K_0$  et $K'_0$]
  Soit $X$ un schéma.
  Le groupe abélien $K_0(X)$ est le groupe de Grothendieck de $\Proj{X}$~: le quotient du groupe abélien libre $\bigoplus_\mathcal{P} \Z\cdot [\mathcal{P}]$
  engendré par les objets de $\Proj{X}$,
  par les relation $[\mathcal{R}] = [\mathcal{P}] + [\mathcal{Q}]$ pour chaque suite exacte $0\ra \mathcal{P}\ra \mathcal{R}\ra \mathcal{Q}\ra 0$.

  On définit de même le groupe $K'_0(X)$ comme le groupe de Grothendieck de $\Modf{X}$.
\end{defi}

De même que pour les anneaux, nous avons les propriétés suivantes~:
\begin{description}
  \item[(i)] $(K_0(X),\oplus,\otimes)$ est un anneau et le produit tensoriel fait de $K'_0(X)$ un $K_0(X)$-module~;
  \item[(ii)] Si $f:X\ra Y$ est un morphisme de schéma, le foncteur $f^*:\Proj{Y}\ra\Proj{X}$ induit un morphisme d'anneau~:
    $$f^*:K_0(Y)\lra K_0(X)$$
  \item[(ii')] Si $f:X\ra Y$ est un morphisme plat, le foncteur $f^*:\Modf{Y}\ra\Modf{X}$ induit un morphisme d'anneau~:
    $$f^*:K'_0(Y)\lra K'_0(X)$$
  \item[(iii)] Si $f:X\ra Y$ est un morphisme fini localement libre, le foncteur $f_*:\Proj{X}\ra\Proj{Y}$ induit un morphisme \textbf{de groupe}~:
    $$f_*:K_0(X)\lra K_0(Y)$$
    Et on a la formule de projection, pour tout $x$ dans $K_0(X)$ et $y$ dans $K_0(Y)$~:
    $$f_*(x\cdot f^*(y))=f_*(x)\cdot y\text{ dans }K_0(Y)$$
  \item[(iii')] Si $f:X\ra Y$ est un morphisme fini, le foncteur $f_*:\Modf{X}\ra\Modf{Y}$ induit un morphisme~:
    $$f_*:K'_0(X)\lra K'_0(Y)$$
    Et si $f$ est localement libre, on a la formule de projection, pour tout $x$ dans $K_0(Y)$ et $y$ dans $K'_0(Y)$~:
    $$f_*(x\cdot f^*(y))=f_*(x)\cdot y\text{ dans }K'_0(Y)$$
\end{description}


[à terminer, ajouter les morphismes propres et les morphismes de Tor dimension finie]

Nous allons maintenant étendre le cadre où $f^*$ et $f_*$ est définie en $K'$-théorie. Le cas de la $K$-théorie n'est pas traité ici.

Nous ne prouvons pas tout de suite ces résultats (la \sref{proposition}{propositionfonctorialiteKprimetransfert} et 
la \sref{proposition}{propositionfonctorialiteKprimetireenarriere}), car les démonstrations seraient pénible. En effet, on ne peut pas reprendre tel quel
les démonstrations faites dans le cadre des anneaux car les fibrés vectoriels ne sont pas en général des objets projectifs.
Nous préférons plutôt renvoyer à la \sref{sous-section}{soussectionfonctorialite} qui traite des mêmes problèmes dans le cadre plus général
de la $K'$-théorie supérieure. Dans ce cadre étendu, la construction $Q$ de Quillen efface en partie ces pénibilités.

\begin{defi}
  Soit $f:X\ra Y$ un morphisme de schéma. On dit que $f$ est de $\Tor$-dimension finie s'il existe un entier $n> 0$ tel que pour tout
  $x$ point de $X$ d'image $y$ dans $Y$, tout $\Ring{Y,y}$-module $M_y$, et tout entier $m\geq n$, on ait~:
  $$\fTor{m}{\Ring{Y,y}}{\Ring{X,x}}{M_y}=0$$
\end{defi}

Pour la définition et les propriétés élémentaires des fibrés en droite amples et très amples, voir \cite[chp. 13]{Gort}.

\begin{prop}\label{propositionfonctorialiteKprimetireenarriere}
  Soit $f:X\ra Y$ un morphisme de $\Tor$-dimension finie entre schémas noethériens et séparés.
  On suppose de plus que tout $\Ring{Y}$-module quasi-cohérent est quotient d'un fibré vectoriel,
  ce qui est par exemple le cas si $Y$ admet un faisceau en droite ample.

  Soit $\mathcal{M}$ la sous-catégorie pleine de $\Modf{Y}$ des modules $\mathcal{F}$ tels que~:
  $$\fTor{i}{\Ring{Y,f(x)}}{\Ring{X,x}}{\mathcal{F}}=0$$
  pour tout $i\geq 1$ et $x$ dans $X$. On note $K'_0(\mathcal{M})$
  son groupe de Grothendieck (cf. \sref{définition}{defiK0catexacte} si ce n'est pas clair).
  Alors l'inclusion $\mathcal{M}\ra \Modf{Y}$ induit un isomorphisme $K'_0(\mathcal{M})\ra K'_0(Y)$.

  Ainsi le foncteur~:
  \[
    \begin{array}{lcl}
      \M & \ra & \Modf{X} \\
      \mathcal{F} &\mapsto& f^*\mathcal{F}
    \end{array}
  \]
  est exact et induit un morphisme d'anneaux~:
  $$f^*:K'_0(Y)\lra K'_0(X)$$
\end{prop}

Pour la démonstration voir la \sref{proposition}{proptireenarriere} plus bas.

\begin{rem}
  On se place dans le cadre de la proposition. Soit $\mathcal{F}$ un objet de $\Modf{Y}$. Alors,
  il admet une résolution finie $\mathcal{G}_*\ra \mathcal{F}$ par des modules de $\M$. Et donc~:
  $$f^*[\mathcal{F}]=\sum_i\geq 0 [f^*\mathcal{G}_i]$$
\end{rem}

Nous admettons le résultat suivant de géométrie algébrique.

\begin{theo}[Serre]\label{theoremeSerre}
  Soit $f:X\ra Y$ un morphisme propre entre schémas noethériens, et soit $\mathcal{L}$ un fibré en droite ample sur $X$.
  Pour tout $\Ring{X}$-module $\mathcal{F}$ et entier $n\in\Z$, on note $\mathcal{F}(n):=\mathcal{F}\otimes_{\Ring{X}}\mathcal{L}^{\otimes n}$.

  Alors pour tout $\Ring{X}$-module $\mathcal{F}$~:
  \begin{description}
    \item[(i)] les $\Ring{Y}$-modules $R^qf_*(\mathcal{F})$ sont cohérents~;
    \item[(ii)] il existe un entier $n_0$ tel que pour tout $n\geq n_0$, $R^qf_*(\mathcal{F}(n))=0$ pour tout $q>0$~;
    \item[(iii)] il existe un entier $n_0$ tel que pour tout $n\geq n_0$, $f^*(f_*(\mathcal{F}(n)))\ra \mathcal{F}$ soit surjectif.
  \end{description}
\end{theo}

Une démonstration relativement courte d'un cas particulier est donnée dans \cite[III.7]{Hart}.
Pour se ramené au cas particulier, il suffit de voir que dans le cadre du théorème ci-dessous, $f$ est projectif (voir \cite[III.2.2.1]{EGA}
pour une discussion plus détaillée).

\begin{prop}\label{propositionfonctorialiteKprimetransfert}
  Soit $f:X\ra Y$ un morphisme entre schémas noethériens et séparés tel que \textbf{l'une} des conditions suivantes soit vérifiée~:
  \begin{description}
    \item[(a)] le morphisme $f$ est fini, ou~:
    \item[(b)] le morphisme $f$ est propre et le schéma $X$ admet un fibré en droite ample. 
  \end{description}
  
  Soit $\mathcal{M}$ la sous-catégorie pleine de $\Modf{X}$ des modules $\mathcal{F}$ tels que~:
  $$R^if_*(\mathcal{F})=0$$
  pour tout $i\geq 1$. On note $K'_0(\mathcal{M})$
  son groupe de Grothendieck (cf. \sref{définition}{defiK0catexacte} si ce n'est pas clair).
  Alors l'inclusion $\mathcal{M}\ra \Modf{X}$ induit un isomorphisme $K'_0(\mathcal{M})\ra K'_0(X)$.

  Ainsi le foncteur~:
  \[
    \begin{array}{lcl}
      \M & \ra & \Modf{Y} \\
      \mathcal{F} &\mapsto& f_*\mathcal{F}
    \end{array}
  \]
  est exact et induit un morphisme de groupe~:
  $$f_*:K'_0(X)\lra K'_0(Y)$$
\end{prop}

Pour la démonstration voir la \sref{proposition}{propositiontransfertschemas} plus bas.

\begin{rem}
  On se place dans le cadre de la proposition. Soit $\mathcal{F}$ un objet de $\Modf{X}$. Alors,
  il admet une résolution finie $\mathcal{F}\ra \mathcal{G}^*$ par des modules de $\M$. Et donc~:
  $$f_*[\mathcal{F}]=\sum_i\geq 0 [f_*\mathcal{G}^i]$$
\end{rem}

Nous allons maintenant aborder le cas des schémas réguliers. Si les résultats sont assez similaires au cas des anneaux,
comme ci-dessus, les schémas posent le problème de l'existence de résolutions par des fibrés vectoriels.
Mais, dans le cas régulier, le \sref{lemme}{RegulierSepareQuotient} résout ce problème et nous évite de passer par la construction $Q$.

\begin{defi}
  Un schéma $X$ est dit régulier si l'anneau local $\Ring{X,x}$ est régulier pour tout point $x$ dans $X$.
\end{defi}

\begin{rem}
  Un schéma, et même un anneau, dont les anneaux locaux sont réguliers n'est pas nécessairement noethérien. Par exemple, c'est le cas du localisé
  de $k[X_1,X_2,X_3,\dotsc]$ par le complémentaire de l'union $\bigcup_n(X_{n^2},X_{n^2+1},\dotsc,X_{(n+1)^2-1})$ (contre-exemple dû à Nagata).
\end{rem}

\begin{lem}\label{RegulierSepareQuotient}
  Soit $X$ un schéma noethérien, régulier et séparé. Alors tout faisceau cohérent sur $X$ est quotient d'un fibré vectoriel.
\end{lem}

La démonstration ci-dessous utilise des propriétés des diviseurs de Cartier et de Weil. Pour plus d'information, voir \cite[chp. 11]{Gort}.

\begin{proof}
  \begin{description}
    \item[(a)] Si $(A,\mathfrak{m})$ est local noethérien et normal, alors~:
    $$\dim(A)\leq 1 \Leftrightarrow \Spec{A}\backslash \{\mathfrak{m}\}\text{ est affine}$$
    En effet, si $\dim(A)=0$, c'est clair. Si $\dim(A)=1$, il existe $f\in \mathfrak{m}$ qui évite les idéaux premiers minimaux de $A$.
    On a alors $\Spec{A}\backslash \{\mathfrak{m}\}=\Spec{A_f}$. Si $\dim(A)\geq 2$, par le lemme de Hartogs (\cite[6.45]{Gort}),
    $$\Ring{\Spec{A}}(\Spec{A})\ra \Ring{\Spec{A}}(\Spec{A}\backslash \{\mathfrak{m}\})$$
    est un isomorphisme, donc $\Spec{A}\backslash \{\mathfrak{m}\}$ ne peut être affine.
    
    \item[(b)] Si $X$ est localement noethérien et normal, et si $U\hookrightarrow X$ est une immersion ouverte affine, alors toutes les composantes
    irréductibles de $X\backslash U$ sont de codimension $\leq 1$ dans $X$.
    
    En effet, si $\xi\in X\backslash U$ est le point générique d'une composante irréductible, on pose 
    $V=\Spec{\Ring{X,\xi}}\backslash\{\mathfrak{m}_{X,\xi}\}$. Par l'hypothèse, c'est un ouvert affine et $\Ring{X,\xi}$ est noethérien normal.
    Donc par (a), $\dim(\Ring{X,\xi})\leq 1$.
    
    \item[(c)] Soit $U\subseteq X$ affine avec $X$ noethérien, régulier et séparé. Alors il existe un fibré en droite $\mathcal{L}$ et $f\in \mathcal{L}(X)$
    tels que $U=\Open{f}$.
    
    Pour démontrer ce point nous pouvons supposer $X$ intègre, $X\neq U$ et $U$ non vide. Comme $X$ est séparé, $U\hookrightarrow X$ est affine.
    Ainsi, d'après (b),$X\backslash U$ est de codimension pure $1$. Donc, par définition, il existe un diviseur de Weil effectif $D$ tel que
    $X\backslash U=\Supp{D}$. Comme $X$ est régulier, il existe un diviseur de Cartier effectif $\tilde{D}$ dont le diviseur
    de Weil associé est $D$. Alors avec $f:=1_{\tilde{D}}\in \Ring{}(\tilde{D})(X)$ la section associée à $1\in\mathrm{K}(X)$ (corps des fractions de $X$) et 
    $\mathcal{L}:=\Ring{}(\tilde{D})(X)$, on a $U=\Open{f}$.

    \item[(d)] Conclusion~:

    On pose $X=\bigcup_{i\in I} U_i$ avec $I$ fini et chaque $U_i$ affine. Pour chaque $i$, à l'aide de (c), on se donne $\mathcal{L}_i$
    et $f_i\in\mathcal{L}_i(X)$ tels que $U_i=\Open{f_i}$.

    Soit $\mathcal{F}$ un faisceau cohérent sur $X$. Pour chaque $i\in I$, on note $a_{i1},\dotsc,a_{in_i}$ une famille de générateurs de
    $\mathcal{F}(U_i)$. Il existe alors, pour chaque $i$, des entiers $m_i\geq 0$ tel que pour tout $j$, $a_{ij}\otimes f_i^{m_i}=b_{ij\mid\Open{f_i}}$
    avec $b_{ij}\in \mathcal{F}\otimes \mathcal{L}^{\otimes m_i}(X)$.

    Les $b_{ij}$ induisent des surjections $\Ring{X}^{n_i}\lra \mathcal{F}\otimes\mathcal{L}^{\otimes m_i}$ sur $U_i$ et donc une surjection~:
    $$\phi:\bigoplus_{i\in I}\Ring{X}^{n_i}\otimes\mathcal{L}^{\otimes -m_i}\lra \mathcal{F}$$
  \end{description}
\end{proof}

\begin{lem}\label{resolutionsVB}
  Soit $X$ un schéma noethérien, régulier et séparé. Alors tout complexe fini de modules cohérents sur $X$ 
  admet une résolution finie par des fibrés vectoriels.  
\end{lem}

\begin{proof}
  Soit $\mathcal{F}_*$ un complexe de modules cohérents sur $X$, concentré en degrés $\{k,\dotsc,l-1\}$. Alors, d'après le 
  \sref{lemme}{RegulierSepareQuotient}, ce complexe admet une résolution $\mathcal{E}_*\ra \mathcal{F}_*$ par des fibrés vectoriels.
  Soit $X=\bigcup_{i\in I} U_i$ un recouvrement fini par des ouverts affines.
  Alors, sur chaque ouvert $U_i$, on a une résolution $\mathcal{E}(U_i)_*\ra \mathcal{F}(U_i)_*$ du complexe de
  $\Ring{X}(U_i)$-modules de type fini $\mathcal{F}(U_i)_*$. Or comme $\Ring{X}(U_i)$ est régulier, $\im{\mathcal{E}(U_i)_{l+1}\ra \mathcal{E}_l(U_i))}$
  est de dimension projective finie. Donc, il existe $n_i\geq 0$ tel que pour tout 
  $m\geq n_i$, $\coker{\mathcal{E}_{m+1}(U_i)\ra \mathcal{E}_m(U_i)}$ soit projectif.
  Si on choisi $n\geq n_i$ pour tout $i$, alors 
  $\coker{\mathcal{E}_{n+1}(U_i)\ra \mathcal{E}_n(U_i)}\ra \mathcal{E}_{n-1}(U_i)\ra\dotsb\ra \mathcal{E}_0(U_i)\ra \mathcal{F}_*$ est une 
  résolution finie de $\mathcal{F}$ par des fibrés vectoriels.
\end{proof}


\begin{lem}\label{ferAChevalVB}
  Soit $\exa{\mathcal{F}'}{\mathcal{F}}{\mathcal{F}''}$ une suite exacte de modules cohérents sur un schéma $X$ noethérien régulier séparé.
  Alors il existe des résolutions finies par des fibrés vectoriels respectant la suite exacte~:
  \begin{center}
    \begin{tikzcd}
      0\rar & \mathcal{P}'_* \rar\arrow[d, "u'"] & \mathcal{P}_* \rar\arrow[d, "u"] & \mathcal{P}''_* \rar\arrow[d, "u''"] & 0\\
      0\rar & \mathcal{F}'   \rar                & \mathcal{F}   \rar               & \mathcal{F}''   \rar                 & 0
    \end{tikzcd}
  \end{center}
\end{lem}

\begin{proof}
  Montrons d'abord le résultat suivante~:
  
  (*)~: Soit $\exa{\mathcal{F}'}{\mathcal{F}}{\mathcal{F}''}$ une suite exacte de modules cohérents. Alors, on peut la compléter en un diagramme~:
  \begin{center}
    \begin{tikzcd}
      0\rar & \mathcal{P}' \rar\arrow[d, "u'"] & \mathcal{P} \rar\arrow[d, "u"] & \mathcal{P}'' \rar\arrow[d, "u''"] & 0\\
      0\rar & \mathcal{F}' \rar                & \mathcal{F} \rar               & \mathcal{F}'' \rar                 & 0
    \end{tikzcd}
  \end{center}
  où les flèches verticales sont surjectives et les modules $\mathcal{P}'$,$\mathcal{P}$ et $\mathcal{P}''$ sont des fibrés vectoriels.
  
  Montrons l'existence d'un tel diagramme. D'après le \sref{lemme}{RegulierSepareQuotient}, il existe des surjections $v:\mathcal{P}''\ra\mathcal{F}$
  et $u':\mathcal{P}'\ra\mathcal{F}'$ avec $\mathcal{P}''$ et $\mathcal{P}'$ fibré vectoriels.
  On pose alors $\mathcal{P}:=\mathcal{P}'\oplus\mathcal{P}''$ et on complète le diagramme de manière évidente avec
  $\exa{\mathcal{P}'}{\mathcal{P}'\oplus\mathcal{P}''}{\mathcal{P}''}$ induit par le scindage, $u:=u'\oplus v$ et 
  $u'':=(\mathcal{F}\ra\mathcal{F}'')\circ v$.

  Maintenant, en appliquant (*) successivement aux la suite exacte $\exa{\myker{u'}}{\myker{u}}{\myker{u''}}$, on obtient des résolutions 
  respectant la suite exacte $\exa{\mathcal{F}'}{\mathcal{F}}{\mathcal{F}''}$. D'après la preuve du \sref{lemme}{resolutionsVB}, nous
  pouvons tronquer ces résolutions à un rang $n\geq 0$. Il reste à montrer que la suite~:
  $$\exa{\coker{\mathcal{P}'_{n+1}\ra\mathcal{P}'_{n}}}{\coker{\mathcal{P}_{n+1}\ra\mathcal{P}_{n}}}{\coker{\mathcal{P}''_{n+1}\ra\mathcal{P}''_{n}}}$$
  est exacte. Elle est exacte à droite par le lemme du serpent, et à gauche car les deux premiers termes s'injectent dans $\mathcal{P}_{n-1}$.
\end{proof}

\begin{theo}
  Soit $X$ un schéma noethérien, régulier et séparé. Alors le foncteur $\Proj{X}\ra \Modf{X}$ induit un isomorphisme de groupes
  $K_0(X)\simeq K'_0(X)$.
\end{theo}

\begin{proof}
  la preuve est essentiellement la même que pour le \sref{théorème}{KegalKprimeAnneauxReguliers}. Il faut, pour appliquer la preuve de ce dernier,
  montrer deux lemmes dans le cadre des modules sur le schéma $X$~: l'existence de résolutions finies pour les complexes finis de modules cohérents,
  c'est le \sref{lemme}{resolutionsVB} ci-dessus~; l'existence de résolutions finies respectant les suites exactes, 
  c'est le \sref{lemme}{ferAChevalVB} ci-dessus.
\end{proof}


\subsection{\texorpdfstring{Categories exactes et leurs $K_0$}{Categories exactes et leurs K0}}

Dans les deux sous-sections précédentes, nous avons défini les groupes de Grothendieck de différentes catégories~:
$\Proj{A}$, $\Modf{A}$, $\Proj{X}$, $\Modf{X}$, ou encore $\mathcal{M}$ de la \sref{proposition}{KprimeTordimfinie}.
Ces catégories ont en commun d'être des sous-catégories stable par extension d'une catégorie abélienne.

\begin{defi}
  Une catégorie exacte plongée est une sous-catégorie pleine $\M$ stable par extension d'une catégorie abélienne $\A$. C'est-à-dire, pour 
  toute suite exacte $\exa{M'}{M}{M''}$ dans $\A$ telle que $M'$ et $M''$ soient dans $\M$, il existe un objet $\tilde{M}$ de $\M$ isomorphe à $M$.
\end{defi}

Soit $\M\hookrightarrow\A$ une catégorie exacte plongée. On note $\E$ l'ensemble des suites $\exa{M'}{M}{M''}$ d'objets de $\M$ qui sont exactes dans $\A$.
On appelle monomorphisme admissible un morphisme $M'\lra M$ dans $\M$ qui apparaît comme le premier morphisme d'un élément $\exa{M'}{M}{M''}$ de $\E$.
On le notera alors $M'\mono M$.
On appelle épimorphisme admissible un morphisme $M\lra M''$ dans $\M$ qui apparaît comme le second morphisme d'un élément $\exa{M'}{M}{M''}$ de $\E$.
On le notera alors $M\epi M''$.
Alors $\M$ est additive et $\M$ et $\E$ vérifient les propriétés suivantes.
\begin{description}
  \item[(a1)] Si $\exa{M'}{M}{M''}$ est une suite dans $\M$ isomorphe à une suite exacte de $\E$, alors $\exa{M'}{M}{M''}$ est dans $\E$.
  \item[(a2)] Pour tous $M'$ et $M''$ dans $\M$, la suite naturelle $\exa{M'}{M'\oplus M''}{M''}$ est dans $\E$.
  \item[(a3)] Pour tout $\exaname{M'}{i}{M}{j}{M''}$ dans $\E$, $i$ est le noyau de $j$ dans $\M$ et $j$ est le conoyau de $i$ dans $\M$.
  \item[(b1)] Les monomorphismes admissibles sont stables par poussé en avant par un morphisme quelconque de $\mathcal{M}$. Les épimorphismes admissibles sont
              stables par tiré en arrière par un morphisme quelconque de $\mathcal{M}$.
  \item[(b2)] Les monomorphismes admissibles sont stables par composition. Les épimorphismes admissibles sont stables par composition. 
\end{description}

\begin{proof}
  Les points (a1)-(a3) sont immédiats. Soit $j:M\ra M''$ un épimorphisme admissible et $i:M'\ra M$ son noyau. Soit $f:N\ra M''$ un morphisme.
  Alors on a le diagramme suivant avec des lignes exactes dans $\A$ et le carré de droite est cartésien.
  \begin{center}
    \begin{tikzcd}
      0\rar & M' \rar\dar & M\times_{M''}N \rar\dar & N   \rar\dar & 0\\
      0\rar & M' \rar     & M              \rar     & M'' \rar     & 0
      \arrow["\mathlarger{\mathlarger{\mathlarger{\mathlarger{\lrcorner}}}}"{anchor=center, pos=0.125}, draw=none, from=1-3, to=2-4]
    \end{tikzcd}
  \end{center}
  Or, $N$ et $M'$ sont des objets de $\M$. Donc il existe $P$ dans $\M$ isomorphe à $M\times_{M''}N$. Ceci démontre (b1).

  Soit $j:M\ra M''$ et $p:M''\ra M'''$ deux épimorphismes admissibles. On note $i:N\ra M''$ le noyau de $p$.
  On a alors à nouveau le diagramme ci-dessus. On vérifie immédiatement que $M\times_{M''}N\ra M$ est le noyau de $p\circ j$.
\end{proof}

\begin{defi}[catégorie exacte]
  Une catégorie exacte est une catégorie additive $\M$ muni d'un ensemble $\E$ de suites $\exa{M'}{M}{M''}$ dans $\M$ vérifiant les axiomes
  (a1),(a2),(a3),(b1) et (b2).
  Un foncteur $F:\M\ra \M'$ entre catégories exactes est un foncteur additif préservant les suites exactes (ie. $F$ envoie $\E$ dans $\E'$).
\end{defi}

\begin{ex}
  Les catégories suivantes sont exactes~:
  \begin{itemize}
    \item une catégorie abélienne $\A$ munie de ses suites exactes~;
    \item une catégorie additive $\mathcal{N}$ munie des suites exactes scindés~;
    \item si $\M$ est une catégorie exacte, la catégorie $\M\op$ est exacte~;
    \item $\Proj{A}$ ou $\Modf{A}$ pour $A$ un anneau~;
    \item $\Proj{X}$ ou $\Modf{X}$ pour $X$ un schéma.
  \end{itemize}
\end{ex}

\begin{prop}
  Soit $\M$ une catégorie exacte et $\E$ son ensemble de suites exactes. Alors~:
  \begin{description}
    \item[(c1)] Si $f:M\ra M''$ a un noyau dans $\M$ et si $N\xrightarrow{u} M\xrightarrow{f} M''$ est un épimorphisme admissible,
                alors $f$ est un épimorphisme admissible.
    \item[(c2)] Si $f:M'\ra M$ a un conoyau dans $\M$ et si $M'\xrightarrow{f} M\xrightarrow{u} N$ est un monomorphisme admissible,
    alors $f$ est un monomorphisme admissible.
  \end{description}
\end{prop}

Nous ne démontrons pas ici cette proposition. La démonstration est élémentaire et relativement courte. Elle est rédigée dans \cite[A.1]{Kell}.

\begin{theo}[plongement]\label{plongementExacte}
  Pour toute catégorie exacte $\M$, il existe une catégorie abélienne $\A$ et un foncteur $i:\M\ra \A$ additif, exact et pleinement fidèle.
\end{theo}

Ce théorème sera admis ici, car sa démonstration a peu de rapport avec le reste du mémoire. Il permet d'utiliser les propriétés des catégories
abéliennes quand on travaille avec des catégories exactes (lemme du serpent par exemple).
Pour une démonstration, voir \cite[A.2]{Kell}.

Un autre exemple important de catégories exactes est le suivant.

\begin{propdefi}\label{propdeficategoriedessuitesexactes}
  Soit $\M$ une catégorie exacte et $\E$ l'ensemble de ses suites exactes. Alors $\E$ a une structure naturelle de catégorie additive où les
  morphismes de $\exac{M'}{M}{M''}$ vers $\exac{N'}{N}{N''}$ sont les diagrammes commutatifs~:
  \begin{center}
    \begin{tikzcd}
      M' \arrow[r,tail]\dar & M \arrow[r,two heads]\dar & M'' \dar \\
      N' \arrow[r,tail]     & N \arrow[r,two heads]     & N''
    \end{tikzcd}
  \end{center}
  On a alors $3$ foncteur $s,t,q:\E\ra \M$ donnés par~:
  \[
    \begin{array}{llcl}
      s:&\exac{M'}{M}{M''}&\mapsto& M' \\
      t:&\exac{M'}{M}{M''}&\mapsto& M  \\
      q:&\exac{M'}{M}{M''}&\mapsto& M''
    \end{array}
  \]
  On note $\mathcal{F}$ l'ensemble des suites $S=\exa{E'}{E}{E''}$ dans $\E$ tels que $s(S)$, $t(S)$
  et $q(S)$ soient exactes.
  
  Alors $\mathcal{F}$ fait de $\E$ une catégorie exacte et les foncteurs $s,t,q:\E\ra \M$ sont exactes.
\end{propdefi}

\begin{proof}
  Les propriétés (a1), (a2) et (a3) sont faciles à vérifier.
  Par le lemme "3x3" des catégories abéliennes et le \sref{théorème de plongement}{plongementExacte},
  $i:E'\ra E$ est un épimorphisme admissible dans $\E$ si et seulement si $s(i)$, $t(i)$ et $q(i)$ sont des épimorphismes 
  admissibles dans $\M$. De même pour les monomorphismes admissibles. Or la composition, le tiré en arrière et le poussé en avant se calculent
  termes à termes. Ceci montre (b1) et (b2). 
\end{proof}

Nous pouvons maintenant définir le $K_0$ d'une catégorie exacte.

\begin{defi}[$K_0$ d'une catégorie exacte]\label{defiK0catexacte}
  Soit $\M$ une catégorie exacte de suites exactes $\E$.
  On définit le groupe de $K$-théorie de $\M$, $K_0(\M)$, comme le quotient du groupe libre $\bigoplus_M\Z\cdot [M]$ sur les objets de $\M$, par les relations
  $[M]=[M']+[M'']$ pour chaque suite $\exac{M'}{M}{M''}$ dans $\E$.
\end{defi}

\begin{rem}\label{remarquefonctorialiteK0categorieexacte}
  La définition ci-dessus fait de $K_0$ un foncteur de la catégorie des catégories exactes et foncteurs exacts dans les groupes abéliens.
\end{rem}

\begin{rem}
  La exacte $\exac{0}{0}{0}$ implique $0=[0]$ dans $K_0(\M)$. Pour chaque isomorphisme $\eta:M\ra\tilde{M}$,
  la suite $\exacname{0}{}{M}{\eta}{\tilde{M}}$ est isomorphe à $\exacname{0}{}{0\oplus M}{0\oplus\id}{M}$
  qui est dans $\E$. Ceci montre que $[M]=[\tilde{M}]$ dans $K_0(\M)$. Ainsi, si les classes d'isomorphisme de $\M$ forment un ensemble
  dans un certain univers,
  $K_0(\M)$ est un ensemble dans ce même univers.
\end{rem}

\begin{prop}\label{CatExColimites}
  Soit $\M_{(-)}:I\ra \CatEx$, $i\mapsto \M_i$ un foncteur d'une catégorie filtrante $I$ dans la catégorie des petites catégories exactes
  et des foncteurs exactes. Alors la colimite $\M:=\colim_i\M_i$ existe. La catégorie sous-jacente est la colimite dans la catégorie des catégories
  et l'ensemble $\E$ des suites exactes est donnée par la colimite $\colim_i\E_i$, où $\E_i$ est l'ensembles des suites exactes dans $\M_i$.
  En d'autres termes $\exacname{M'}{k}{M}{p}{M''}$ est dans $\E$ si et seulement si seulement si pour un certain
  rang $i$ dans $\Ob{I}$, il existe $M'_i$,$M_i$,$M''_i$,$k_i$ et $p_i$  induisant respectivement $M'$,$M$,$M''$,$k$ et $p$ dans $\M$,
  tels que $\exacname{M'_i}{k_i}{M_i}{p_i}{M''_i}$ soit exacte dans $\M_i$.
\end{prop}
\begin{proof}
  On pose $\M$ la colimite de $\M_{(-)}$ dans la catégorie des petites catégories. On pose $\E$ la colimite $\colim_i\E_i$.
  Montrons que $\M$ est exacte. Il sera alors immédiat que $\M$ est la colimite dans $\CatEx$.
  Comme tout diagramme fini est réalisé à un rang $i$ dans $\Ob{i}$, les axiomes (a1),(a2) et (a3) passent à la colimite.
  L'axiome (b2) passe de même clairement à la colimite. L'axiome (b1) est vérifié par le \sref{lemme}{tireEnArriereExact} suivant. 
\end{proof}

\begin{lem}\label{tireEnArriereExact}
  Soit $F:\M\ra\M'$ un foncteur exact entre catégories exactes. Soit $j:M\twoheadrightarrow M''$ un épimorphisme admissible dans $\M$ et $f:N\ra M''$
  un morphisme. Alors le carré~:
  \begin{center}
    \begin{tikzcd}
      F(M\times_{M''}N) \arrow[two heads, r] \arrow[d] & F(N) \arrow[d, "F(f)"] \\
      F(M)              \arrow[two heads, r, "F(j)"]           & F(M'') 
    \end{tikzcd}
  \end{center}
  est cartésien. En d'autres termes, les foncteurs exactes préservent les tirés en arrière des épimorphismes admissibles.
\end{lem}

\begin{proof}
  On note $i:M'\ra M$ le noyau de $j$.
  On note $\alpha: F(M\times_{M''}N)\ra F(M)\times_{F(M'')}F(N)$ le morphisme canonique. Montrons que $\alpha$ a noyau nul.
  Soit $u:P\ra F(M\times_{M''}N)$ tel que $\alpha\circ u=0$. Alors comme $\exac{F(M')}{F(M\times_{M''}N)}{F(N)}$ est exacte,
  $u$ se factorise en $v: P\ra F(M')$ par $F(M')\rightarrowtail F(M\times_{M''}N)$. Or, $F(i)\circ v=0$, donc $v=0$ car
  $F(i)$ est un noyau.
  Or la composition de $\alpha$ avec $F(M)\times_{F(M'')}F(N)\ra F(N)$ est un épimorphisme admissible. Donc par (c1), 
  $\alpha$ est un épimorphisme admissible de noyau nul. Par (a3) c'est donc un isomorphisme.
\end{proof}

\begin{prop}\label{CatExColimitesK0}
  Soit $\M_{(-)}:I\ra \CatEx$, $i\mapsto \M_i$ un foncteur d'une catégorie filtrante $I$ dans la catégorie des petites catégories exactes.
  On note $\M$ sa colimite. Alors l'application induite $\colim_i K_0(\M_i)\ra K_0(\M)$ est un isomorphisme.
\end{prop}

\begin{proof}
  On a $\bigoplus_{M\in\Ob{\M}} \Z = \colim_i \bigoplus_{M\in\Ob{\M_i}}\Z$ et le sous-groupe $<[M']+[M'']-[M]>_{\exac{M'}{M}{M''}}$
  de $\bigoplus_{M\in\Ob{\M}}\Z$ est colimite des sous-groupes corresponds des $\bigoplus_{M\in\Ob{\M_i}}\Z$.
\end{proof}

Nous allons maintenant appliquer ce résultat de colimite pour effectuer un calcul.

\begin{defi}
  Soit $A=A_0\oplus A_1\oplus\dotsb$ un anneau gradué en degrés positifs. On note $\Pgr{A}$ la catégorie des $A$-modules $\Z$-gradués de type fini
  et projectifs
  (en tant que modules gradués\footnote{ce qui est équivalent à projectif en tant que $A$-module, voir \cite[pp.636-637]{Bass}. Ce ne sera pas utile ici}).
  C'est une catégorie exacte.
  On notera $t:\Pgr{A}\ra \Pgr{A}$, $(P_n)_n\mapsto (P_{n-1})_n$ le foncteur exact de décalage. On notera également par $t$ l'automorphisme induit
  sur $K_0(\Pgr{A})$.
\end{defi}

\begin{prop}\label{K0gradues}
  Soit $A=A_0\oplus A_1\oplus\dotsb$ un anneau gradué en degrés positifs.
  L'automorphisme $t$ fait de $K_0(\Pgr{A})$ un $\Z[t,t^{-1}]$-module. On a un isomorphisme de $\Z[t,t^{-1}]$-modules~:
  \[
  \begin{array}{llll}
    \phi:&\Z[t,t^{-1}]\otimes_{\Z}K_0(A_0) &\ra    & K_0(\Pgr{A}) \\
         &1\otimes x                       &\mapsto& (A\otimes_{A_0}-)_*(x)
  \end{array}
  \]
\end{prop}

\begin{proof}
  Pour $k$ dans $\Z$, on note $F_k$ le foncteur $\Pgr{A}\ra\Pgr{A}$, $P\mapsto <P_n>_{n\leq q}$, où $<P_n>_{n\leq q}$ est le sous-module de $P$
  engendré par les éléments homogènes de degrés $n\leq q$.
  Pour $q\geq 0$, on note $\Pgr{A}_q$ la sous-catégorie pleine de $\Pgr{A}$ des $P$ tels que $F_{-q-1}P=0$ et $F_qP=P$.
  On définit le foncteur exact $T:\Pgr{A}\ra \Pgr{A_0}$, $P\mapsto A_0\otimes_A P$. Pour tout $P$ et tout $n$, nous allons vérifier que le morphisme
  canonique suivant est un isomorphisme~:
  \[
  \begin{array}{lll}
    A[-n]\otimes_{A_0}T(P)_n &\ra& F_nP/F_{n-1}P \\
    a_m\otimes \overline{b_n}&\mapsto &\overline{a_m\cdot b_n}
  \end{array}
  \]
  pour $a_m$ et $b_n$ homogènes de degrés $m$ et $n$. On dispose d'un épimorphisme naturel de $A_0$-modules gradués $f:P\ra T(P)$.
  Or $T(P)$ est un $A_0$-module gradué projectif.
  Soit $g:T(P)\ra P$ une section. On note $h:T(P)\otimes_{A_0}A\ra P$ le morphisme de $A$-modules gradués de type fini induit.
  Le morphisme $T(h)$ s'identifie à $\id_{T(P)}$. Donc $T(\coker{h})$ est nul. Or $\coker{h}$ est de type fini et donc inférieurement borné.
  Donc $\coker{h}=0$. Donc $h$ est surjectif. Comme $P$ est projectif, on a $T(P)\otimes_{A_0}A\simeq \myker{h}\oplus P$. Mais par cet isomorphisme,
  $T(T(P)\otimes_{A_0}A)$ est isomorphe à $T(P)$. Donc $T(\myker{h})=0$. Or, comme quotient de $P$, $\myker{h}$ est de type fini et donc borné
  inférieurement. Donc $\myker{h}=0$. Ainsi on a un isomorphisme non-canonique $T(P)\otimes_{A_0}A\simeq P$. Via cet isomorphisme,
  $F_nP\simeq \oplus_{q\leq n} A[-q]\otimes_{A_0} T(P)_q$. On a donc bien $A[-n]\otimes_{A_0}T(P)_n\simeq F_nP/F_{n-1}P$, et ce morphisme correspond
  à celui défini plus haut car $h$ est induit par une section de $P\ra T(P)$.

  Nous pouvons maintenant calculer $K_0(\Pgr{A}_q)$. Soit $P$ un objet de $\Pgr{A}_q$. On a une filtration de $P$~:
  $$0=F_{-q-1}P\subseteq F_{-q}\subseteq \dotsb \subseteq F_qP=P$$
  Et donc, au vu de l'isomorphisme ci-dessus~:
  $$[P]=\sum_{n=-q}^q [F_nP/F_{n-1}P]=\sum_{n=-q}^q [A[-n]\otimes_{A_0}T(P)_n]=\sum_{n=-q}^q [\phi(t^n\otimes T(P)_n)]$$
  Ainsi, si on pose $\chi_q: K_0(\Pgr{A}_q)\ra \bigoplus_{n=-q}^q K_0(A_0)$, $P\mapsto \oplus_n T(P)_n$, 
  on a que $\phi_q$ et $\chi_q$ sont réciproques, où $\phi_q$ est la restriction de $\phi$ à $\bigoplus_{n=-q}^q \Z\cdot t^n\otimes K_0(A_0)$.
  Donc $\phi_q$ est un isomorphisme. Par la \sref{proposition}{CatExColimitesK0}, $\phi$ est un isomorphisme.
\end{proof}

\subsection{\texorpdfstring{Le groupe $K_1$ d'un anneau}{Le groupe K1 d'un anneau}}\label{soussectionK1}

Dans cette section, nous fixons un anneau $A$.

On note $\GL{n}{A}$ le groupe des matrices $n\times n$ inversibles à coefficients dans $A$.
On fait de $n\mapsto \GL{n}{A}$ un foncteur $(\N,\leq)\ra \Grp$ via les inclusions~:
\[
  \begin{array}{lcl}
    \GL{n}{A} &\ra     & \GL{n+1}{A} \\
    M         &\mapsto & \begin{pmatrix} M & 0 \\ 0 & 1 \end{pmatrix}
  \end{array}
\]
et on note $\GL{}{A}:=\colim_{n\in\N} \GL{n}{A}$.  

\begin{defi}[$K_1$ d'un anneau]
  Pour $A$ un anneau, on définit le $1^{\mathrm{er}}$ groupe de $K$-théorie de $A$, $K_1(A)$, par~:
  $$K_1(A):=\GL{}{A}\ab$$
\end{defi}

Nous allons maintenant donner une description du groupe des commutateurs $[\GL{}{A},\GL{}{A}]$.

\begin{defi}
  Pour $1\leq i,j \leq n$, $i\neq j$ et $a\in A$, on pose~:
  $$e_{ij}(a):= \id+aE_{ij}\in \GL{n}{A}$$
  On note $\EGL{n}{A}$ le sous-groupe de $\GL{n}{A}$ engendré par les $e_{ij}(a)$ pour $1\leq i,j \leq n$, $i\neq j$ et $a\in A$.
  On note $\EGL{}{A}:=\bigcup_n\EGL{n}{A}\subset\GL{}{A}$.
\end{defi}

\begin{ex}
  \begin{enumerate}
    \item Toutes les matrices de permutation paires sont dans $\EGL{}{A}$~;
    \item Si $M\in \GL{n}{A}$, alors~:
         $$\begin{pmatrix} M & 0 \\ 0 & M^{-1} \end{pmatrix}\in \EGL{2n}{A}$$
         En effet, on a la formule~:
         $$\begin{pmatrix} M & 0  \\ 0       & M^{-1} \end{pmatrix}=%
           \begin{pmatrix} 1 & M  \\ 0       & 1      \end{pmatrix}%
           \begin{pmatrix} 1 & 0  \\ -M^{-1} & 1     \end{pmatrix}%
           \begin{pmatrix} 1 & M  \\ 0       & 1      \end{pmatrix}%
           \begin{pmatrix} 0 & -1 \\ 1       & 0      \end{pmatrix}$$
  \end{enumerate}
\end{ex}

\begin{prop}[lemme de Whitehead]
  Pour $A$ un anneau, $\EGL{}{A}$ et parfait, et on a l'égalité~:
  $$\EGL{}{A}=[\GL{}{A},\GL{}{A}]$$
  Et donc~:
  $$K_1(A)=\GL{}{A}/\EGL{}{A}$$
\end{prop}

\begin{proof}
  \begin{description}
    \item[$\subseteq$:] on a $e_{ij}(a)=[e_{ik}(a),e_{kj}(a)]$ pour $i$,$j$ et $k$ distincts. Ceci montre également que $\EGL{}{A}$ et parfait.
    \item[$\supseteq$:] Pour $M,N\in \GL{n}{A}$, on a, dans $\GL{2n}{A}$~:
                      $$[M,N]=\begin{pmatrix} M         & 0  \\ 0 & M^{-1} \end{pmatrix}%
                              \begin{pmatrix} N         & 0  \\ 0 & N^{-1} \end{pmatrix}%
                              \begin{pmatrix} (NM)^{-1} & 0  \\ 0 & NM     \end{pmatrix}\in\EGL{2n}{A}$$ 
  \end{description}
\end{proof}

[ajouter lien avec K-théorie topologique]

\section{\texorpdfstring{Définition de la $K$-théorie supérieure}{Définition de la K-théorie supérieure}}

Cette section est dédiée à la construction $Q$ de la $K$-théorie supérieure, dûe à Daniel Quillen. La référence principale est l'article
original de Quillen \cite{Quil}.

\subsection{\texorpdfstring{La construction $Q$ de Quillen}{La construction Q de Quillen}}

Dans cette sous-section, $\M$ désigne une petite catégorie exacte.

\begin{propdefi}\label{propBicartesien}
  Un diagramme dans $\M$ de la forme~:
  \begin{center}
    \begin{tikzcd}[column sep = large, row sep = large]
      A \arrow[d, two heads, "p'"] \arrow[r, tail, "i"] & B \arrow[d, two heads, "p"] \\
      C \arrow[r, tail, "i'"]                           & D
    \end{tikzcd}
  \end{center}
  est cartésien si et seulement si il est cocartésien. Si c'est le cas, $p$ et $p'$ ont même noyau, et $i$ et $i'$ ont même conoyau.
  
  On appellera un tel diagramme bicartésien.

  De plus, tout diagramme cartésien de la forme $(1)$ ci-dessous ou cocartésien de la forme $(2)$ ci-dessous est bicartésien.
  \begin{center}
    \begin{tikzcd}[column sep = small, row sep = small]
      \cdot \arrow[dd] \arrow[rr] &     & \cdot \arrow[dd, two heads] & \cdot \arrow[dd, two heads] \arrow[rr, tail] &     & \cdot \arrow[dd] \\
                                  & (1) &                             &                                              & (2) &                  \\
      \cdot \arrow[rr, tail]      &     & \cdot                       & \cdot \arrow[rr]                             &     & \cdot
      \arrow["\mathlarger{\mathlarger{\mathlarger{\mathlarger{\lrcorner}}}}"{anchor=center, pos=0.125}, draw=none, from=1-1, to=3-3]
      \arrow["\mathlarger{\mathlarger{\mathlarger{\mathlarger{\lrcorner}}}}"{anchor=center, pos=0.125, rotate=180}, draw=none, from=3-6, to=1-4]
    \end{tikzcd}
  \end{center}
\end{propdefi}

\begin{proof}
  Supposons seulement que le diagramme soit cartésien, que $i'$ soit un monomorphisme admissible et $p$ un épimorphisme admissible (cas $(1)$).
  On note $k:N\rightarrowtail B$ le noyau de $p$. Un morphisme $u:P\ra A$ qui vérifie
  $(P\xrightarrow{u} A\ra C)=0$ s'identifie à un morphisme $P\ra B$ qui vérifie $(P\ra B\ra D)=0$, c'est à dire un morphisme $P\ra N$.
  Donc $p$ et $p'$ ont même noyau. De plus, par l'axiome (b1), $p'$ est un épimorphisme admissible.

  Maintenant, un couple de morphismes $u:C\ra P$, $v:B\ra P$ s'identifie à un morphisme $v:B\ra P$ tel que $A\ra B\ra P$ se factorise
  par $C$. C'est-à-dire, comme $p$ et $p'$ ont même noyau, un morphisme de $w:D\ra P$. Ainsi, le diagramme est cocartésien.
  
  Montrons que $i$ est un monomorphisme admissible. Soit $j':D\twoheadrightarrow Q$ le conoyau de $i'$. Un morphisme $u:P\ra B$
  vérifie $j'\circ p\circ u=0$ si et seulement si $p\circ u$ se factorise par $i':C\ra D$ si et seulement si $u$ se factorise par $i$.
  Donc $i$ est le noyau de $j'\circ p$, qui est un épimorphisme admissible. Donc $i$ est un monomorphisme admissible.
  
  Les autres énoncés sont duaux de ceux démontrés.
\end{proof}

%Or les deux premières colonnes sont exactes, donc par le \sref{théorème de plongement}{plongementExacte} et le lemme "$3\times 3$" des catégories
%abéliennes, on a que $v$ est un isomorphisme. Ainsi, $i$ et $i'$ on même conoyau.

%\begin{center}
%  \begin{tikzcd}
%    N \arrow[d, tail, "k'"]      \arrow[r, equal]     & N \arrow[d, tail, "k"]      \arrow[r]                 & 0 \arrow[d]      \\
%    A \arrow[d, two heads, "p'"] \arrow[r, tail, "i"] & B \arrow[d, two heads, "p"] \arrow[r, two heads,"j"]  & M \arrow[d, "v"] \\
%    B \arrow[r, tail, "i'"]                           & C                           \arrow[r, two heads,"j'"] & M'
%  \end{tikzcd}
%\end{center}

\begin{defi}[catégorie $Q\M$]
  On définit la catégorie $Q\M$ comme la catégorie dont~:
  \begin{enumerate}
    \item les objets sont les objets de $\M$~;
    \item l'ensemble $\Hom{Q\M}{M}{M'}$ est l'ensemble des diagrammes~:
          $$M\twoheadleftarrow N \rightarrowtail M'$$
          à isomorphisme près, où un isomorphisme entre $M\twoheadleftarrow N \rightarrowtail M'$ et $M\twoheadleftarrow N' \rightarrowtail M'$
          est la donnée d'un isomorphisme $\eta:N\ra N'$ dans $\M$ faisant commuter le diagramme~:
          \begin{center}
            \begin{tikzcd}[row sep = small]
                & N \arrow[dd, "\eta"] \arrow[dl, two heads] \arrow[dr, tail] &    \\
              M &                                                             & M' \\
                & N'                   \arrow[ul, two heads] \arrow[ur, tail] &    \\
            \end{tikzcd}
          \end{center}
    \item la composition de $M\twoheadleftarrow N \rightarrowtail M'$ et $M'\twoheadleftarrow N' \rightarrowtail M''$ est donnée par le diagramme~:
    \begin{center}
      \begin{tikzcd}
          &                                          & N\times_{M'}N'\arrow[dl, two heads] \arrow[dr, tail] &                                        &     \\
        M & N \arrow[l, two heads] \arrow[r, tail]   & M'                                                   & N'\arrow[l, two heads] \arrow[r, tail] & M'' \\
      \end{tikzcd}
    \end{center}
  \end{enumerate}
\end{defi}

\begin{rem}
  La composition dans $Q\M$ est bien définie par fonctorialité des limites. L'associativité est facile à vérifier.
\end{rem}

\begin{defi}
  Pour $i:M\rightarrowtail M'$, on note $i_!:M\ra M'$ le morphisme associé dans $Q\M$. Un tel morphisme sera appelé une injection.

  Pour $j:M\twoheadrightarrow M''$, on note $j^!:M''\ra M$ le morphisme associé dans $Q\M$. Un tel morphisme sera appelé une surjection.
\end{defi}

\begin{prop}
  Tout morphisme $u$ de $Q\M$ se factorise en $u=i_!j^!$ uniquement à unique isomorphisme près. De même,
  $u$ se factorise uniquement sous la forme $u=j^!i_!$ à unique isomorphisme près.

  Un morphisme $u$ est un isomorphisme si et seulement si c'est une injection et une surjection.

  De plus, on a $\Iso{Q\M}{M}{M'}\simeq \Iso{\M}{M}{M'}$.
\end{prop}

\begin{proof}
  Le premier point est la définition. Le second point découle de la bijection entre factorisations $u=j^!i_!$ et $u=i_!j^!$ induite par les propriétés
  des carrés bicartésiens (\sref{proposition}{propBicartesien}).

  La caractérisation des isomorphismes est immédiate dès que l'on remarque~:
  si $u'=\tilde{i}_!\tilde{j}^!$ est inverse de $u=i_!j^!$, on pose $a_!b^!=\tilde{j}^!i_!$. Alors $\id=u'u= (\tilde{i}a)_!(bj)^!$~;
  donc $\tilde{i}a$ et $bj$ sont des isomorphismes dans $\M$, donc $\tilde{i}$ et $j$ également.
\end{proof}

Une conséquence immédiate de la proposition ci-dessus et de la \sref{proposition}{propBicartesien} est la propriété universelle
vérifiée par $Q\M$ dans $\Cat$.

\begin{prop}[Propriété universelle de $Q\M$]\label{propUnivQM}
  La donnée d'un foncteur $F:Q\M\ra D$ avec $D$ une petite catégorie est équivalente à la donnée~:
  \begin{enumerate}
    \item d'une application $F:\Ob\M\ra \Ob D$~;
    \item de morphismes $Fi_!:FM'\ra FM$ pour chaque $i:M\rightarrowtail M'$~;
    \item de morphismes $Fj^!:FM''\ra FM$ pour chaque $j:M\twoheadrightarrow M''$~;
    \item tels que $F(ii')_!=Fi_!\circ Fi'_!$ et $F(j'j)^!=Fj'^!\circ Fj^!$~;
    \item et tels que pour chaque diagramme bicartésien~:
    \begin{center}
      \begin{tikzcd}[column sep = large, row sep = large]
        \cdot \arrow[d, two heads, "j'"] \arrow[r, tail, "i'"] & \cdot \arrow[d, two heads, "j"] \\
        \cdot \arrow[r, tail, "i"]                             & \cdot
      \end{tikzcd}
    \end{center}
    On ai $Fj^!\circ Fi_!=Fi'_!\circ Fj'^!$.
  \end{enumerate}
\end{prop}

Nous allons maintenant donner une autre interprétation des morphismes.

\begin{propdefi}
  On appelle sous-objet de $M\in \Ob \M$ une classe d'isomorphisme de monomorphismes admissibles $M'\rightarrowtail M$, et quotient de $M$
  une classe d'isomorphisme d'épimorphismes admissibles $M\twoheadrightarrow M''$. Les sous-objets sont en bijection avec les quotients via les
  suites exactes.

  On définit une relation d'ordre sur les sous-objets de $M$ par $M_0\leq M_1$ si et seulement si il existe $M_0\rightarrowtail M_1$ monomorphisme
  admissible au-dessus  de $M$.
  
  L'ensemble partiellement ordonné des niveaux de $M$ est définie par~:
  \begin{itemize}
    \item[$\bullet$] Objets~: couples $(M_0,M_1)$ de sous objets avec $M_0\leq M_1$~;
    \item[$\bullet$] Morphismes~: $(M_0,M_1)\leq (M'_0,M'_1)$ si et seulement si $M'_0\leq M_0\leq M_1\leq M'_1$. 
  \end{itemize}

  On a une équivalence entre $Q\M\downarrow M$ et la catégorie des niveaux donnée par~:
  $$(M_0,M_1)\mapsto (M_1/M_0\twoheadleftarrow M_1\rightarrowtail M)$$
\end{propdefi}

La démonstration est immédiate. Pour la notation $Q\M\downarrow M$,
voir le début de la \sref{sous-section}{thmAetB}.

\begin{prop}
  On a un isomorphisme naturel de catégories~:
  \[
    \begin{array}{ccc}
      Q\M\op                                        &\simeq & Q\M \\
      M                                             &\mapsto& M   \\
      M\xtwoheadleftarrow{p} N\xrightarrowtail{i} M' &\mapsto& M \xrightarrowtail{p\op} N\xtwoheadleftarrow{i\op}M'
    \end{array}
  \]
\end{prop}

Nous pouvons maintenant donner une définition de la $K$-théorie supérieure. Pour cela, nous allons considérer la catégorie
$Q\M$ comme un ensemble simplicial. Pour des rappels sur ce point et sur les groupes d'homotopie des ensembles simpliciaux, voir le début de
la \sref{sous-section suivante}{thmAetB}.

\begin{defi}[$K$-théorie supérieure]
  Soit $\M$ une catégorie exacte et $O$ un objet nul de $\M$. On définit, pour $n\geq 0$, le $n$-ème groupe de $K$-théorie de $\M$ comme~:
  $$K_n(\M):=\pi_n(Q\M,O)$$
  où la catégorie $Q\M$ est vu comme un ensemble simplicial via le foncteur nerf, voir la \sref{sous-section}{thmAetB}.
\end{defi}

\begin{rem}
  Comme $Q\M$ est connexe, la classe d'isomorphisme de $K_n(\M)$ est indépendante de $O$. On a même mieux~: si $O'$ est un autre élément nul
  dans $\M$, il existe un unique isomorphisme $u:O\ra O'$ dans $\M$~; ce dernier induit un isomorphisme $u:O\ra O'$ dans $Q\M$,
  et donc un isomorphisme canonique $\pi_n(Q\M,O)\simeq\pi_n(Q\M,O')$.
\end{rem}

Il reste à vérifier que la nouvelle définition de $K_0$ coïncide avec celle donnée dans la \sref{section}{sectionK0}. C'est l'objet
du théorème ci-dessous.

Nous allons maintenant étudier le groupoïde associé à $Q\M$. Soit $O\in \Ob\M$ un objet nul.

Nous notons, pour $M$ objet de $Q\M$, $i_M:O\rightarrowtail M$ et $j_M:M\twoheadrightarrow O$ les uniques morphismes.
La démonstration du théorème utilise la \sref{proposition}{pi1categorie}.
\begin{theo}\label{theoremeK0constructionQ}
  L'application~:
  $$
  \begin{array}{lcl}
    K_0(\M) &\ra     & \pi_1(Q\M,O) \\
    {[{M}]}     &\mapsto & [i_{M!}]^{-1}[j_M^!]
  \end{array}
  $$
  induit un isomorphisme de groupes.
\end{theo}

\begin{proof}
  L'idée de la preuve est de montrer qu'on dispose d'une équivalence de catégorie~:
  $$[K_0(\M),\Isos{\Ens}]\simeq [Q\M,\Isos{\Ens}]$$
  Où $\Isos{\Ens}$ est la catégorie des bijections entre ensembles.
  Pour cela, on remarque dans un premier temps que $[Q\M,\Isos{\Ens}]$ est équivalente à sa sous-catégorie pleine $\mathcal{F}$,
  formée des foncteurs $F$ tels que~:
  $$\forall M,\; F(M)=F(O)\text{ et }F(i_{M!})=\id_{F(O)}$$
  En effet, un foncteur réciproque à l'inclusion est donné par~·
  $$\gamma:F\mapsto (u:M\ra M'\mapsto F(i_{M'!})^{-1}\circ F(u)\circ F(i_{M!}))$$
  Étudions les éléments de $\mathcal{F}$. Si $i:M\rightarrowtail M'$, $i\circ i_{M'}=i_M$, et donc $F(i_!)=id_{F(O)}$. 
  Soit $\exacname{M'}{i}{M}{j}{M''}$ une suite exacte. On a un carré bicartésien~:
  \begin{center}
    \begin{tikzcd}[column sep = large, row sep = large]
      M' \arrow[d, two heads, "j_M'"] \arrow[r, tail, "i"] & M \arrow[d, two heads, "j"] \\
      0  \arrow[r, tail, "i_{M''}"]                       & M''
    \end{tikzcd}
  \end{center}
  On a donc $j^!i_{M''!}=i_!j_{M'}^!$, puis $F(j^!)=F(j_{M'}^!)$. De plus $j_M^!=j^!j_{M''}^!$. Donc $F(j_M^!)=F(j_{M'}^!)F(j_{M''}^!)$.
  Nous disposons donc d'un morphisme naturel en $F$~:
  $$
    \begin{array}{lcl}
      K_0(\M) &\ra     & \Aut{F(O)} \\
      {[{M}]} &\mapsto & F(j_M^!)
    \end{array}
  $$
  Ainsi, nous avons un foncteur:
  $$ \alpha : \mathcal{F}\ra [K_0(\M),\Ens]$$
  On remarque également que les éléments $F$ de $\mathcal{F}$ sont entièrement déterminés par $F(O)$ et leurs valeurs sur les $j_M^!$.
  Nous allons maintenant décrire un morphisme réciproque. 
  
  Soit $S$ un $K_0(\M)$-ensemble. On pose~:
  \[
    \begin{array}{cccc}
      F_S: & Q\M                                       &\ra    & \Isos{\Ens} \\
           & M                                         &\mapsto& S   \\
           & i_!                                       &\mapsto& id_S \\
           & j^!                                       &\mapsto& [\myker{j}]\cdot(-)
    \end{array}
  \]
  Vérifions que $F_S$ est un foncteur. Si $\cdot\xtwoheadrightarrow{j}\cdot\xtwoheadrightarrow{j'}\cdot$, la suite
  $\exacname{\myker{j}}{}{\myker{j'j}}{j}{\myker{j'}}$
  est exacte, donc $F((jj')^!)=F(j^!)F(j'^!)$. Si on a un carré bicartésien~:
  \begin{center}
    \begin{tikzcd}[column sep = large, row sep = large]
      \cdot \arrow[d, two heads, "j'"] \arrow[r, tail, "i'"] & \cdot \arrow[d, two heads, "j"] \\
      \cdot \arrow[r, tail, "i"]                             & \cdot
    \end{tikzcd}
  \end{center}
  On a $\myker{j}\simeq\myker{j'}$ et donc $F(i'_!j'^!)=F(j^!i_!)$. Ainsi, par la propriété universelle de $Q\M$ énoncée dans la
  \sref{proposition}{propUnivQM}, $F_S$ est un foncteur.
  On dispose donc d'un foncteur~:
  $$\beta: [K_0(\M),\Ens]\ra \mathcal{F}$$
  Or on voit que $\alpha$ et $\beta$ sont réciproques. En effet, on a un isomorphisme naturel $\alpha(\beta(S))\simeq S$ 
  induit par $\myker{j_M}\simeq M$.
  De même, on a un isomorphisme naturel $\beta(\alpha(F))\simeq F$ induit par les $F(j^!)=F(j^!_{M'})$ et $\myker{j}\simeq M'$
  pour $\exacname{M'}{i}{M}{j}{M''}$ exacte.

  On note $\delta:[Q\M,\Isos{\Ens}] \ra [\pi_1(Q\M,O),\Ens]$ l'équivalence naturelle. L'équivalence~:
  $$\delta\circ(\mathcal{F}\ra [Q\M,\Isos{\Ens}])\circ\beta:[K_0(\M),\Ens]\ra [\pi_1(Q\M,O),\Ens]$$
  Est induite par un isomorphe de groupe $\phi: \pi_1(Q\M,O)\ra K_0(\M)$ (résultat classique sur les actions de groupe).
  On vérifie, en appliquant l'équivalence au $K_0(\M)$-ensemble $K_0(\M)$, que $\phi$ envoie $[i_{M!}]^{-1}[j_M^!]$ sur ${[{M}]}$.
  Ce qui conclut.
\end{proof}

\subsection{Les théorèmes A et B de Quillen}\label{thmAetB}

Comme la définition de la $K$-théorie supérieure repose sur de la topologie, cette section donne quelques rappels sans démonstrations
et prouve $2$ théorèmes (dits A et B) sur lesquels reposeront la majorité des résultats sur la construction $Q$.

On note, pour $C$ une catégorie et $x$ un objet de $C$, $C\downarrow x$ la catégorie au-dessus de $x$ des couples $(y,u:y\ra x)$.
De même, on note $C\uparrow x$ la catégorie au-dessous de $x$ des couples $(y,u:x\ra y)$.

Pour $f:C\ra D$ un foncteur et $d$ un objet de $D$, on note $f\downarrow d$ la catégorie des couples $(c,u:fc\ra d)$. La catégorie
$f\downarrow d$ est le tiré en arrière de $D\downarrow d$ par $f$. De même pour $f\uparrow d$. On note parfois
$C\downarrow d$ pour $f\downarrow d$, quand le foncteur $f$ est évident, par exemple quand $C$ est une sous-catégorie pleine de $D$.

Pour $f:C\ra D$ un foncteur et $d$ un objet de $D$, on note $f^{-1}d$ la sous-catégorie de $C$ formée des objets $c$ tels que $fc=d$ et des
morphismes $u:c\ra c'$ tels que $fu=\id_d$.

\begin{propdefi}\label{definitionnerf}
  Le foncteur nerf $N:\Cat\ra \DEns$ est défini par
  $$NC_\bullet:=\Hom{\Cat}{[\bullet]}{C}$$
  où $[\bullet]$ désigne la catégorie cosimpliciale
  $\DCat\ra \Cat,[n]\mapsto (0\ra 1\ra\dotsb\ra n)$.
  Ce foncteur est pleinement fidèle et admet un adjoint à gauche $\tau_{\leq 1}:\DEns\ra\Cat$.
\end{propdefi}

Pour la démonstration, voir le \cite[chp. 1]{Goer}.

\begin{rem}
  Un fait classique est qu'un ensemble simplicial $X$ est dans l'image essentielle du foncteur nerf
  si et seulement si il a la propriété d'unique relèvement le long des inclusions $\Lambda_k^n\hookrightarrow\Delta^n$,
  ie. pour tout $u:\Lambda_k^n\ra X$, il existe un unique $v:\Delta^n\ra X$ qui l'étend.
\end{rem}

Dans la suite, comme $N$ est pleinement fidèle, nous l'omettrons et notérons $C$ pour l'ensemble simplicial associé à une catégorie $C$.

Nous admettrons les structures de modèles de Kan-Quillen sur $\DEns$ et $\Top$.

Nous noterons $|\bullet|:\DEns\ra \Top$ le foncteur réalisation et $\Sing:\Top\ra \DEns$ le foncteur complexe singulier.
Pour simplifier des démonstrations, nous utiliserons l'équivalence de Quillen~:
\begin{center}
  \begin{tikzcd}
    {|\bullet|:\Ho{\DEns}} \ar[r,bend left,""{name=A, below}] & {\Ho{\Top}:\Sing} \ar[l,bend left,""{name=B,above}] \ar[from=A, to=B, phantom,"\perp"]
  \end{tikzcd}
\end{center}

Pour $X$ ensemble simplicial et $x\in X_0$, nous noterons $\pi_n(X,x)$ pour $\pi_n(|X|,x)$. Nous utiliserons également les groupes d'homotopie
simpliciaux sur les complexes de Kan.

Voir \cite[chp. 1]{Goer} pour plus d'informations sur ces sujets.

Nous allons maintenant nous intéresser aux propriétés homotopiques des catégories.

\begin{prop}
  \begin{description}
    \item[(1)] Une transformation naturelle $\theta:f\Rightarrow g$ entre $f$ et $g:C\ra D$ induit une homotopie~:
              $$C\times \Delta^1\ra D$$ 
    \item[(2)] Si $f:C\ra D$ a un adjoint à droite ou à gauche, alors $f$ est une équivalence d'homotopie~;
    \item[(3)] Une catégorie $C$ avec un objet initial ou final est contractile. 
  \end{description}
\end{prop}

\begin{proof}
  \begin{description}
    \item[(1)] $N:\Cat\ra \DEns$ préserve les produits et la donnée de $\theta$ est équivalente à celle d'un
              foncteur $C\times (0\ra 1) \ra D$~;
    \item[(2)] On note $f':D\ra C$ l'adjoint. On a alors par (1) des homotopies $ff'\simeq \id_D$ et $f'f\simeq \id_C$~;
    \item[(3)] Le foncteur $C\ra (*)$ a alors un adjoint à droite ou à gauche. 
  \end{description}
\end{proof}

\begin{prop}\label{propositioncolimitecategories}
  \begin{description}
    \item[(1)] Si $C_{(-)}:I\ra \Cat$ est un foncteur avec $I$ petite catégorie filtrante et $C$ sa colimite, l'application naturelle~·
              $$\colim_I NC_{(-)}\lra NC\text{ est un isomorphisme.}$$ 
    \item[(2)] Dans ce même contexte, si on se donne des objets compatibles $(x_i\in C_i)_{i\in I}$ et $x\in C$ associé, les applications naturelles~:
              $$\colim_{i\in I} \pi_n(C_{i},x_i)\lra \pi_n(C,x)\text{ sont des isomorphismes.}$$
    \item[(3)] Une catégorie $I$ filtrante est contractile. 
  \end{description}
\end{prop}

\begin{proof}
  \begin{description}
    \item[(1)] Il suffit de montrer que $\colim_I NC_{(-)}$ est une catégorie, c'est-à-dire que cet ensemble simplicial
               a la propriété d'unique extension le long
               des $\Lambda_k^n\ra\Delta^n$. Or, si $f:\Lambda_k^n\ra \colim_I NC_{(-)}$ est une application, comme $\Lambda_k^n$ est un objet
               compacte de $\DEns$, $f$ proviens de $f_i:\Lambda_k^n\ra C_i$ pour un certain $i\in \Ob I$. On peut alors l'étendre en
               $g_i:\Delta^n\ra C_i$, lequel induit $g:\Delta^n\ra \colim_I NC_{(-)}$. Si $g$ et $g'$ conviennent, alors ils proviennent
               respectivement de $g_j:\Delta^n\ra C_j$ et $g'_j:\Delta^n\ra C_j$, pour un même $j$ car $I$ est filtrante.
               Quitte à changer $j$, on peut supposer que $g_{j|\Lambda_k^n}=g'_{j'|\Lambda_k^n}$. Or $C_j$ est une catégorie, 
               donc $g_j=g_j'$ et $g=g'$.
    \item[(2)] Le foncteur remplacement fibrant $R:\DEns\ra \DEns$ induit par l'application de l'argument du petit objet aux inclusions
               $\Lambda_k^n\ra\Delta^n$ commute aux colimites filtrantes. En effet, comme on l'a vu ci-dessus~:
               $$\Hom{\DEns}{\Lambda_k^n}{C}\simeq \colim_i\Hom{\DEns}{\Lambda_k^n}{C_i}$$
               Donc, avec $\mathcal{J}=\{\Lambda_k^n\ra\Delta^n\}$, et $G^*$ la construction du petit objet,
               on a $G^1(\mathcal{J},C\ra *)=\colim_i G^1(\mathcal{J},C_i\ra *)$, et donc par induction et passage à la colimite~:
               $RC = \colim_i RC_i$.

               Nous nous sommes donc ramenés à montrer que les groupes d'homotopie simpliciaux commutent aux colimites filtrantes de complexes de Kan.
               Ce qui se montre de manière équivalente au (1).
    \item[(3)] On pose $F$ le foncteur $I\ra \Cat,\;i\mapsto I\downarrow i$. On a $I=\colim_I F$. Or chaque $I\downarrow i$ a un objet final et 
               est donc contractile. Donc par (2), $I$ est contractile.
  \end{description}
\end{proof}

Il existe une autre manière de décrire le $\pi_1$ d'une catégorie. On note $\pi:\Cat \ra \mathrm{Groupoides}$ l'adjoint à gauche de l'oubli.

\begin{prop}\label{pi1categorie}
  Soit $C$ une catégorie et $\pi C$ le groupoïde engendré. Pour $x$ un objet de $C$, on a un isomorphisme~:
  $$\mathrm{Aut}_{\pi C}(x)\simeq \pi_1(C,x)$$
\end{prop}

\begin{proof}
  On note $X$ l'ensemble simplicial obtenu en ajoutant un $2$-simplexe $\sigma_a$ à $C$ le long de chaque application $(a,\id,-):\Lambda_2^2\ra C$,
  pour $a$ morphisme de $C$. On a alors $\tau_{\leq 1}X=\pi C$. On utilise alors le résultat suivant~:
  
  $(*)$ Si $X$ est un ensemble simplicial tel que $\tau_{\leq 1} X$ est un groupoïde, si et $f:\Lambda_k^2\ra X$ une application,
  on note $g:X\ra Y$ le poussé en avant de $\Lambda_k^2\ra \Delta^2$ le long de $f$. Alors $\tau_{\leq 1}g$ est un isomorphisme.
  
  Par $(*)$, on dispose d'un remplacement fibrant $K$ de $X$, et donc de $C$, tel que $\tau_{\leq 1} K=\pi C$. Or c'est un résultat classique sur
  les complexes de Kan que $\tau_{\leq 1} K= \pi_{\leq 1} K$. Donc, on a~:
  $$\pi_1(C,x)=\pi_1(K,x)=\mathrm{Aut}_{\pi_{\leq 1} K}(x)=\mathrm{Aut}_{\pi C}(x)$$

  Il reste à montrer $(*)$. Pour cela, par exemple si $k=2$, on note $f=(a,b,-)$ et $(a,b,c):\Delta^2\ra Y$ l'extension.
  On a $\tau_{\leq 1}Y=(\tau_{\leq 1}X)[c]/(ac=b)$. Or, dans $\tau_{\leq 1}Y$, $[c] = [a]^{-1}[b]\in \tau_{\leq 1}X$.
  Donc $\tau_{\leq 1}Y=\tau_{\leq 1}X$.
\end{proof}

Pour la suites, nous aurons besoin de deux résultats sur les ensembles bisimpliciaux.

\begin{defi}
  Un ensemble bisimplicial est un foncteur~:
  $$\DCat\op\times\DCat\op\lra \Ens$$
  Pour $X_{\bullet\bullet}$ un ensemble bisimplicial, on appelle sa réalisation $|X|$ le coégalisateur suivant dans $\DEns$~:
  $$\coend{X}{\Delta}$$
  Où $X_n$ désigne l'ensemble simplicial $X_{n,\bullet}$.
  
  Sa diagonale est l'ensemble simplicial $d(X):[n]\mapsto X_{n,n}$
\end{defi}

\begin{prop}
  Soit $X$ un ensemble bisimplicial. Alors les applications~:
  $$\alpha_n:X_n\times \Delta^n\ra d(X),\;(x_m,\phi:[m]\ra [n])\mapsto (\phi\times \id_{[m]})^*(x_m)$$
  induisent un isomorphisme $|X|\ra d(X)$.
\end{prop}

\begin{proof}
  Soit $\phi:[n]\ra [m]$ et $(x,\chi:[r]\ra [n])\in X_{m,r}\times \Delta^n_r$. Alors~:
  $$\alpha_n(\phi^*x,\chi)=((\phi\chi)\times \id)^*x=\alpha_m(x,\phi\chi)$$
  Donc les $\alpha_n$ induisent bien une application $|X|\ra d(X)$. Cette dernière est clairement fonctorielle en $X$.
  Or $X\mapsto |X|$ et $X\mapsto d(X)$ commutent aux colimites, et dans la catégorie $[\DCat\op\times\DCat\op,\Ens]$
  des ensembles bisimpliciaux, tout ensemble $X$ est colimite de $\DCat\times\DCat\downarrow X$, où l'inclusion
  $\DCat\times\DCat\ra [\DCat\op\times\DCat\op,\Ens]$ est donnée par Yoneda.
  Donc, nous sommes ramenées à $X_{\bullet_1\bullet_2}=\Delta^r_{\bullet_1}\times\Delta^s_{\bullet_2}$. Or~:
  $$d(\Delta^r_{\bullet_1}\times\Delta^s_{\bullet_2})=\Delta^r\times\Delta^s$$
  Et~:
  \begin{align*}
    |\Delta^r_{\bullet_1}\times\Delta^s_{\bullet_2}|_l&=\mathrm{coeq}\Big(\bigsqcup_{\phi:[n]\ra [m]}\Delta^r_m\times\Delta^s_l\times\Delta^n_l
    \xlongrightrightarrows{\sqcup\phi^*\times\id}{\sqcup\id\times\phi_*}
    \bigsqcup_{[n]}\Delta^r_n\times\Delta^s_l\times\Delta^n_l\Big)\\
    &=\mathrm{coeq}\Big(\bigsqcup_{\phi:[n]\ra [m]}\Delta^r_m\times\Delta^n_l
     \xlongrightrightarrows{\sqcup\phi^*\times\id}{\sqcup\id\times\phi_*}
     \bigsqcup_{[n]}\Delta^r_n\times\Delta^n_l\Big)\times \Delta^s_l\\
     &=\mathrm{coeq}\Big(\bigsqcup_{\phi:[n]\ra [m]}\Delta_m^r\times\Delta^n
     \xlongrightrightarrows{\sqcup\phi^*\times\id}{\sqcup\id\times\phi_*}
     \bigsqcup_{[n]}\Delta_n^r\times\Delta^n\Big)_l\times \Delta^s_l\\
     &=\Delta^r_l\times \Delta^s_l
  \end{align*}
  Où la dernière égalité est une égalité classique sur les extensions de Yoneda. Plus précisément, on utilise que si $C$ est une petite
  catégorie et $X$ un objet de $[C\op,\Ens]$, $Y:C\ra [C\op,\Ens]$ le plongement de Yoneda, et $\Pi:C\downarrow X\ra C$
  l'oubli, alors~:
  $$X=\colim_C Y\circ\Pi=\mathrm{coeq}\Big(\bigsqcup_{\phi:c\ra d} X(d)\times Y(c)
  \xlongrightrightarrows{\sqcup X(\phi)\times\id}{\sqcup\id\times Y(\phi)}
  \bigsqcup_{c} X(c)\times Y(c) \Big)$$
  On vérifie facilement que $|\Delta^r_{\bullet_1}\times\Delta^s_{\bullet_2}|\ra d(\Delta^r_{\bullet_1}\times\Delta^s_{\bullet_2})$
  est l'identité via ces deux identifications.
\end{proof}

\begin{lem}\label{diagonalequiv}
  Soit $\phi:X_{\bullet\bullet}\ra Y_{\bullet\bullet}$ un morphisme entre ensembles bisimpliciaux. Si pour tout $n\geq 0$,
  $\phi_n:X_{n\bullet}\ra Y_{n\bullet}$ est une équivalence d'homotopie faible, alors $d(\phi):d(X)\ra d(Y)$ est une équivalence d'homotopie faible.
\end{lem}

\begin{proof}\cite[IV.1.7]{Goer}
  Introduisons d'abord quelques notations.
  Soit $Z$ un ensemble bisimplicial. Pour $p\geq 0$ on note $d(Z)^{(p)}$ l'image de $\bigsqcup_ {n\leq p} Z_n\times\Delta^n$ dans $d(Z)$.
  Pour $r\geq 0$ et $p\geq 0$, on note $s_{[r]}Z_p=\bigcup_{i\leq r}s_i(Z_p)\subset Z_{p+1}$. On a alors les $3$ diagrammes cocartésiens suivants~:
  \begin{center}
    \begin{tikzcd}[column sep=0.2ex,row sep=scriptsize]
      s_{[r]}Z_{p-1} \arrow[dd, hook] \arrow[rr, "s_{r+1}"] &   &s_{[r]}Z_p \arrow[dd, hook] \\
                                                            &(*)& \\
      Z_p \arrow[rr, "s_{r+1}"]                             &   &s_{[r+1]}Z_p
      \arrow["\mathlarger{\mathlarger{\mathlarger{\mathlarger{\lrcorner}}}}"{anchor=center, pos=0.125, rotate=180}, draw=none, from=3-3, to=1-1]\\

      (s_{[p]}Z_p\times\Delta^{p+1})\cup (Z_{p+1}\times\partial\Delta^{p+1}) \arrow[dd, hook] \arrow[rr] & &d(Z)^{(p)} \arrow[dd, hook] \\
      &(**)&\\
      Z_{p+1}\times \Delta^{p+1} \arrow[rr]                                                           & &d(Z)^{(p+1)}
      \arrow["\mathlarger{\mathlarger{\mathlarger{\mathlarger{\lrcorner}}}}"{anchor=center, pos=0.125, rotate=180}, draw=none, from=6-3, to=4-1]\\

      s_{[p]}Z_p\times\partial \Delta^{p+1} \arrow[dd, hook] \arrow[rr] & &Z_{p+1}\times\partial\Delta^{p+1} \arrow[dd, hook] \\
      &(***)&\\
      s_{[p]}Z_p\times\Delta^{p+1} \arrow[rr]                                                           & &(s_{[p]}Z_p\times\Delta^{p+1})\cup (Z_{p+1}\times\partial\Delta^{p+1})
      \arrow["\mathlarger{\mathlarger{\mathlarger{\mathlarger{\lrcorner}}}}"{anchor=center, pos=0.125, rotate=180}, draw=none, from=9-3, to=7-1]\\
    \end{tikzcd}
  \end{center}
  Montrons par récurrence sur $p$ que $d(X)^{(p)}\ra d(Y)^{(p)}$ est une équivalence d'homotopie faible. Cela conclura, car les $d(X)^{(p)}$
  filtrent $d(X)$.

  Pour initialiser, on remarque que $d(X)^{(p)}=d(\sk{p}{X})$, et en particulier, $d(X)^{(0)}=X_0$.

  Soit $p\geq 0$, et supposons que $d(X^{(i)})\ra d(Y^{(i)})$ soit une équivalence d'homotopie faible pour tout $i\leq p$.
  \begin{description}
    \item[(1)] En appliquant successivement $(*)$ et le \sref{lemme de collage}{lemmedecollage}, on a que $s_{[r]}X_p\ra s_{[r]}Y_p$
              est une équivalence d'homotopie faible pour tout $r\leq p$~;
    \item[(2)] En utilisant $(***)$ et le \sref{lemme de collage}{lemmedecollage}, on a que~:
              $$(s_{[p]}X_p\times\Delta^{p+1})\cup (X_{p+1}\times\partial\Delta^{p+1})\ra (s_{[p]}Y_p\times\Delta^{p+1})\cup (Y_{p+1}\times\partial\Delta^{p+1})$$
              est une équivalence faible~;
    \item[(3)] Par $(**)$ et le \sref{lemme de collage}{lemmedecollage},
    $$d(X^{(p+1)})\ra d(Y^{(p+1)})$$
    est une équivalence faible.
  \end{description}
  Ceci conclut la récurrence.
\end{proof}

Nous faisons ici quelques rappels sur les foncteurs (pré)(co)fibrés.

\begin{defi}[foncteur (pré)fibré]
  Un foncteur $f:C\ra D$ est préfibré si pour tout $y$ objet de $D$, le foncteur pleinement fidèle~:
  $$f^{-1}(y)\ra f\uparrow y,\; x\mapsto (x,\id_y)$$
  admet un adjoint à droite, alors noté $(x,v:y\ra fx)\mapsto v^*x$.

  Alors, si $v:y\ra y'$, on peut restreindre cet adjoint à $f^{-1}(y')$, et on obtient un foncteur~:
  $$v^*:f^{-1}(y')\lra f^{-1}(y)$$
  que l'on appelle changement de base de $y'$ à $y$.

  Un foncteur $f:C\ra D$ est fibré s'il est préfibré, et si pour tout $u$ et $v$ composables dans $D$,
  l'application naturelle $u^*v^*\ra (vu)^*$ est un isomorphisme.
\end{defi}

On ajoute ci-dessous la définition complètement duale de foncteur (pré)cofibré.

\begin{defi}[foncteur (pré)cofibré]
  Un foncteur $f:C\ra D$ est précofibré si pour tout $y$ objet de $D$, le foncteur pleinement fidèle~:
  $$f^{-1}(y)\ra f\downarrow y,\; x\mapsto (x,\id_y)$$
  admet un adjoint à gauche, alors noté $(x,v:fx\ra y)\mapsto v_*x$.

  Alors, si $v:y'\ra y$, on peut restreindre cet adjoint à $f^{-1}(y')$, et on obtient un foncteur~:
  $$v_*:f^{-1}(y')\lra f^{-1}(y)$$
  que l'on appelle changement de cobase de $y'$ à $y$.

  Un foncteur $f:C\ra D$ est cofibré s'il est précofibré, et si pour tout $u$ et $v$ composables dans $D$,
  l'application naturelle $(vu)_*\ra v_*u_*$ est un isomorphisme.
\end{defi}

\begin{rem}
  Pour $f:C\ra D$ (pré)(co)cofibré, on dit souvent que $C$ est une catégorie (pré)(co)cofibrée au-dessus de $D$.
\end{rem}

\begin{rem}
  Les définitions ci-dessus sont celles de \cite[§1]{Quil}. Ces définitions ont l'avantage de faire directement apparaître le résultat suivant
  (et son dual évident)~:

  $(*)$ Si $f:C\ra D$ est préfibré, alors pour tout $y$ objet de $D$, $f^{-1}y\ra f\uparrow y$ est une équivalence d'homotopie.

  Pour comprendre l'existence des applications naturelles
  $u^*v^*\ra (vu)^*$, il peut être commode de voir l'adjonction à l'aide des morphismes (pré)cartésiens. Soit $f:C\ra D$.
  
  Un morphisme $\xi:\tilde{z}\ra z$ dans $C$ est $f$-précartésien si pour tout $w:x\ra z$ avec $f(w)=f(\xi)$, il existe
  un unique $u:x\ra \tilde{z}$ tel que $w=\xi u$ et $f(u)=\id_{f\tilde{z}}$.

  \begin{center}
    \begin{tikzcd}[row sep=large,column sep=large]
      x & {\tilde{z}} & z \\
      {f\tilde{z}} & {f\tilde{z}} & fz
      \arrow[maps to, from=1-1, to=2-1]
      \arrow[maps to, from=1-2, to=2-2]
      \arrow[maps to, from=1-3, to=2-3]
      \arrow["w", bend left=20, from=1-1, to=1-3]
      \arrow["\xi"', from=1-2, to=1-3]
      \arrow["\id"', from=2-1, to=2-2]
      \arrow["f\xi"', from=2-2, to=2-3]
      \arrow["{\exists ! u}"', dashed, from=1-1, to=1-2]
    \end{tikzcd}
  \end{center}
  
  Un morphisme $\xi:\tilde{z}\ra z$ dans $C$ est $f$-cartésien si pour tout $w:x\ra z$ et $a:fx\ra f\tilde{z}$ avec $f(w)=f(\xi)a$, il existe
  un unique $u:x\ra \tilde{z}$ tel que $w=\xi u$ et $f(u)=a$.

  \begin{center}
    \begin{tikzcd}[row sep=large,column sep=large]
      x & {\tilde{z}} & z \\
      {fx} & {f\tilde{z}} & fz
      \arrow[maps to, from=1-1, to=2-1]
      \arrow[maps to, from=1-2, to=2-2]
      \arrow[maps to, from=1-3, to=2-3]
      \arrow["w", bend left=20, from=1-1, to=1-3]
      \arrow["\xi"', from=1-2, to=1-3]
      \arrow["a"', from=2-1, to=2-2]
      \arrow["f\xi"', from=2-2, to=2-3]
      \arrow["{\exists ! u}"', dashed, from=1-1, to=1-2]
    \end{tikzcd}
  \end{center}

  Soit maintenant $y$ un objet de $D$.
  Alors $f^{-1}(y)\ra f\uparrow y$ admet un adjoint à droite si et seulement si pour tout $z$ objet de $C$ et $v:y\ra fz$,
  il existe $\xi:v^* z\ra z$ $f$-précartésien avec $f(\xi)=v$.

  \begin{center}
    \begin{tikzcd}[row sep=large,column sep=large]
      {\exists v^*z} & z \\
      y & fz
      \arrow["v", from=2-1, to=2-2]
      \arrow[maps to, from=1-2, to=2-2]
      \arrow[maps to, from=1-1, to=2-1]
      \arrow["\xi", from=1-1, to=1-2]
    \end{tikzcd}
  \end{center}

  Il est maintenant aisé de vérifier que $f$ est fibré si et seulement si pour tout $y$ objet de $D$, $z$ objet de $C$ et $v:y\ra fz$,
  il existe $\xi:v^* z\ra z$ $f$-cartésien avec $f(\xi)=v$.
\end{rem}

\begin{propdefi}[catégorie (co)fibrée associée à un préfaisceau]
  Soit $p:C\op\ra\Ens$ un foncteur avec $C$ une petite catégorie (souvent appelé préfaisceau sur $C$). On définit la catégorie fibrée
  $$f:F(p)\ra C$$
  dont les objets sont les couples $(c,x)$ avec $x\in p(C)$, et les morphismes de $(c,x)$ dans $(d,y)$ sont les morphismes
  $u:c\ra d$ dans $C$ tels que $p(u)(y)=x$.

  On appelle $f:F(p)\ra C$ la catégorie fibrée associée à $p$.

  De même, à partir d'un foncteur $q:C\ra \Ens$, on construit la catégorie cofibrée
  $$g:G(q)\ra C$$
  dont les objets sont les couples $(c,x)$ avec $x\in q(C)$, et les morphismes de $(c,x)$ dans $(d,y)$ sont les morphismes
  $u:c\ra d$ dans $C$ tels que $p(u)(x)=y$.

  On appelle $g:G(q)\ra C$ la catégorie cofibrée associée à $q$.
\end{propdefi}

Nous pouvons maintenant énoncer un premier résultat sur le type d'homotopie de catégories.

\begin{theo}[Théorème A de Quillen]\label{theoremeA}
  Soit $f:C\ra D$ un foncteur entre petites catégories. Si pour tout objet $y$ de $D$, $f\uparrow y$ est une catégorie contractile,
  alors $f$ est une équivalence d'homotopie.
\end{theo}

\begin{proof}
  On note $S(f)$ la catégorie cofibrée au dessus de $D\op\times C$ définie par le foncteur~:
  \begin{align*}
    D\op\times C &\ra     \Ens \\
    (y,x)        &\mapsto \Hom{D}{y}{fx}
  \end{align*}
  On dispose de foncteurs~:
  $$D\op\overset{p_2}{\longleftarrow} S(f) \overset{p_1}{\lra} C$$
  L'ensemble simplicial $S(f)$ est la diagonale de l'ensemble bisimplicial $T(f)$ défini par~:
  $$T(f)_{pq}:=\{(y_p\ra\dotsb\ra y_0\ra fx_0,\;x_0\ra\dotsb\ra x_q)\}$$
  
  On dispose d'un morphisme d'ensembles bisimpliciaux $m:T(f)_{\bullet_1\bullet_2}\ra C_{\bullet_2}$, où $(C_{\bullet_2})_{pq}=NC_q$.
  Alors, $d(m)$ s'identifie à $p_1:S(f)\ra C$. Or, on a, pour $q\geq 0$~:
  $$m_{\bullet q}:\bigsqcup_{x_0\ra\dotsb\ra x_q} (D\downarrow f(x_0))\op \ra \bigsqcup_{x_0\ra\dotsb\ra x_q} *$$
  C'est une équivalence d'homotopie, car chaque $(D\downarrow f(x_0))\op$ a un objet initial et est donc contractile.
  Ainsi, par le \sref{lemme}{diagonalequiv}, $d(m)=p_1:S(f)\ra C$ est une équivalence d'homotopie.

  De même, on dispose d'un morphisme d'ensembles bisimpliciaux $n:T(f)_{\bullet_1\bullet_2}\ra D\op_{\bullet_1}$. Et pour tout $p\geq 0$~:
  $$n_{p\bullet}:\bigsqcup_{y_p\ra\dotsb\ra y_0} f\uparrow y_0 \ra \bigsqcup_{y_p\ra\dotsb\ra y_0} *$$
  Or, par hypothèse, chaque $f\uparrow y_0$ est contractile. Donc $n_{p\bullet}$ est une équivalence faible.
  Donc, par le \sref{lemme}{diagonalequiv}, $d(n)=p_2:S(f)\ra D\op$ est une équivalence d'homotopie.

  Or, la factorisation~:
  $$D\op\times C \overset{\id\times f}{\lra} D\op\times D\overset{\Hom{D}{-}{-}}{\lra} \Ens$$
  induit le diagramme commutatif suivant dans $\Cat$~:
  \begin{center}
    \begin{tikzcd}[row sep=large,column sep=large]
      D\op & {S(f)} & C \\
      D\op & {S(\id_D)} & D
      \arrow[equal, from=1-1, to=2-1]
      \arrow["f", from=1-3, to=2-3]
      \arrow["{p_1}","\sim"', from=1-2, to=1-3]
      \arrow["{p_2}"',"\sim", from=1-2, to=1-1]
      \arrow["\sim"',from=2-2, to=2-1]
      \arrow["\sim",from=2-2, to=2-3]
      \arrow["{f_*}", from=1-2, to=2-2]
    \end{tikzcd}
  \end{center}
  Par les résultats ci-dessus, les flèches avec des $\sim$ sont des équivalences d'homotopie faibles.
  Donc $f$ également.
\end{proof}

\begin{rem}
  Le théorème A admet la forme duale suivante~:

  Si $f:C\ra D$ un foncteur entre petites catégories et si pour tout objet $y$ de $D$, $f\downarrow y$ est une catégorie contractile,
  alors $f$ est une équivalence d'homotopie.
\end{rem}

\begin{coro}
  Si $f:C\ra D$ un foncteur pré(co)fibré entre petites catégories et si pour tout objet $y$ de $D$, $f^{-1}(y)$ est une catégorie contractile,
  alors $f$ est une équivalence d'homotopie.
\end{coro}
\begin{proof}
  Dans ce cas $f^{-1}(y)\ra f\uparrow y$ ou $f^{-1}(y)\ra f\downarrow y$ est une équivalence faible d'homotopie. On peut donc utiliser
  le \sref{théorème}{theoremeA}.
\end{proof}

Pour énoncer le théorème B de Quillen, nous aurons besoin du formalisme des carrés homotopiquement cartésiens dans $\DEns$.
Nous avons fait le choix de le mettre dans \sref{l'annexe}{carrehomotopiquementcart}.

\begin{defi}
  Soit $Z:I\ra \DEns$ un foncteur avec $I$ petite catégorie. On définit l'ensemble bisimplicial~:
  $$\mathrm{BE}_IZ_{m,n}:=\bigsqcup_{i_0\ra\dotsb\ra i_m}Z(i_0)_n$$
  Et on note sa diagonale~:
  $$\hocolim{I}{Z}:=d(\mathrm{BE}_IZ)$$
\end{defi}

\begin{lem}\label{lemmetheoremeB}
  Soit $X:I\ra \DEns$ un foncteur, avec $I$ une petite catégorie, tel que pour tout $\alpha:i\ra j$ dans $I$,
  $X(\alpha): X(i)\ra X(j)$ soit une équivalence d'homotopie faible. Alors pour tout $j$ objet de $I$, le diagramme cartésien~:
  \begin{center}
    \begin{tikzcd}[column sep=0.2ex,row sep=small]
      {X(j)} && {\hocolim{I}{X}} \\
      & {(D)} \\
      {*} && I
      \arrow["j", from=3-1, to=3-3]
      \arrow[from=1-1, to=3-1]
      \arrow["\pi", from=1-3, to=3-3]
      \arrow[from=1-1, to=1-3]
      \arrow["\mathlarger{\mathlarger{\mathlarger{\mathlarger{\lrcorner}}}}"{anchor=center, pos=0.05}, draw=none, from=1-1, to=3-3]
    \end{tikzcd}
  \end{center}
  est homotopiquement cartésien.
\end{lem}

\begin{rem}
  Le diagramme $(D)$ est la diagonale du diagramme cartésien suivant d'ensembles bisimpliciaux~:
  \begin{center}
    \begin{tikzcd}[column sep=large,row sep=large]
      {X(j)_n} & {\Big(\displaystyle\bigsqcup_{i_0\ra\dotsb\ra i_n}X(i_0)\Big)_n} \\
      {*} & {\Big(\displaystyle\bigsqcup_{i_0\ra\dotsb\ra i_n}*\Big)_n}
      \arrow["j\ra\dotsb\ra j", from=2-1, to=2-2]
      \arrow[from=1-1, to=2-1]
      \arrow["\pi", from=1-2, to=2-2]
      \arrow["j\ra\dotsb\ra j",from=1-1, to=1-2]
      \arrow["\mathlarger{\mathlarger{\mathlarger{\mathlarger{\lrcorner}}}}"{anchor=center, pos=0.05}, draw=none, from=1-1, to=2-2]
    \end{tikzcd}
  \end{center}
\end{rem}

\begin{proof}
  \cite[IV.5.7]{Goer}. On remarque d'abord le fait suivant~:

  $(*)$ le tiré en arrière par $\pi$ commute aux colimites dans $\DEns\downarrow I$.
  
  En effet, il suffit, pour montrer $(*)$, de vérifier que $-\times_x y$ commute aux sommes disjointes et coégalisateurs dans $\Ens\downarrow x$,
  ce qui est facile.

  On factorise $j$ à l'aide de l'argument du petit objet appliqué à $\mathcal{J}=\{\Lambda_k^n\ra\Delta^n\}$ en
  \begin{tikzcd}
    {*} & U & I
    \arrow["i", "\sim"', hook, from=1-1, to=1-2]
    \arrow["p", two heads, from=1-2, to=1-3]
  \end{tikzcd}.
  Notre objectif est de montrer que l'application induite $X(j)\ra U\times_I\hocolim{I}{X}$ est acyclique.
  Par la \sref{proposition}{criterehomocartesienpropreadroite}, ceci conclura.

  Or $i$ est construit comme un élément de $\mathcal{J}-\mathrm{Cell}$ colimite de poussés en avant de la forme~:
  \begin{center}
    \begin{tikzcd}[column sep=large,row sep=large]
      {\bigsqcup\Lambda_k^n} & {G^m(\mathcal{J},j)} \\
      {\bigsqcup\Delta^n} & {G^{m+1}(\mathcal{J},j)}
      \arrow["\sim" sloped, hook, from=1-1, to=2-1]
      \arrow["\sim" sloped, hook, from=1-2, to=2-2]
      \arrow[from=1-1, to=1-2]
      \arrow[from=2-1, to=2-2]
      \arrow["\mathlarger{\mathlarger{\mathlarger{\mathlarger{\lrcorner}}}}"{anchor=center, pos=0.05, rotate=180}, draw=none, from=2-2, to=1-1]
    \end{tikzcd}  
  \end{center}
  Et donc, par $(*)$, en tirant en arrière par $\pi$~:
  \begin{center}
    \begin{tikzcd}[column sep=large,row sep=large]
      {\bigsqcup\Lambda_k^n\times_I\hocolim{I}{X}} & {G^m(\mathcal{J},j)\times_I\hocolim{I}{X}} \\
      {\bigsqcup\Delta^n\times_I\hocolim{I}{X}} & {G^{m+1}(\mathcal{J},j)\times_I\hocolim{I}{X}}
      \arrow["u", hook, from=1-1, to=2-1]
      \arrow["v", hook, from=1-2, to=2-2]
      \arrow[from=1-1, to=1-2]
      \arrow[from=2-1, to=2-2]
      \arrow["\mathlarger{\mathlarger{\mathlarger{\mathlarger{\lrcorner}}}}"{anchor=center, pos=0.05, rotate=180}, draw=none, from=2-2, to=1-1]
    \end{tikzcd}  
  \end{center}
  Or, le produit fibré dans $\Ens$, et donc également dans $\DEns$, préserve les injections. Donc $u$ est une cofibration.
  Il nous reste à montrer que $u$ est acyclique. Car alors $v$ sera également une cofibration acyclique, et par passage à la colimite,
  $i_*:X(j)\ra U\times_I\hocolim{I}{X}$ également.

  Nous sommes donc ramenés au problème suivant~:

  $(**)$ si 
  \begin{tikzcd}
    {\Lambda^k_n} & {\Delta^n} & I
    \arrow["w", "\sim" ',hook, from=1-1, to=1-2]
    \arrow["\sigma", from=1-2, to=1-3]
  \end{tikzcd}
  alors $w_*:\Lambda^k_n\times_I\hocolim{I}{X}\ra \Delta^n\times_I\hocolim{I}{X}$ est une cofibration acyclique.

  Or $\sigma: \Delta^n\ra I$ correspond à un foncteur $\sigma: [n]\ra I$. De plus, le carré cartésien suivant d'ensembles bisimpliciaux
  \begin{center}
    \begin{tikzcd}[column sep=large,row sep=large]
      {\Big(\displaystyle\bigsqcup_{u_0\ra\dotsb\ra u_n}X(\sigma(u_0))\Big)_n} & {\Big(\displaystyle\bigsqcup_{i_0\ra\dotsb\ra i_n}X(i_0)\Big)_n} \\
      {\Big(\displaystyle\bigsqcup_{u_0\ra\dotsb\ra u_n}*\Big)_n} & {\Big(\displaystyle\bigsqcup_{i_0\ra\dotsb\ra i_n}*\Big)_n}
      \arrow["\sigma", from=2-1, to=2-2]
      \arrow[from=1-1, to=2-1]
      \arrow["\pi", from=1-2, to=2-2]
      \arrow["\sigma",from=1-1, to=1-2]
      \arrow["\mathlarger{\mathlarger{\mathlarger{\mathlarger{\lrcorner}}}}"{anchor=center, pos=0.05}, draw=none, from=1-1, to=2-2]
    \end{tikzcd}
  \end{center}
  induit un isomorphisme $\Delta^n\times_I\hocolim{I}{X}\simeq \hocolim{[n]}{X\circ \sigma}$. On a alors le diagramme suivant d'ensembles bisimpliciaux
  \begin{center}
    \begin{tikzcd}[column sep=large,row sep=large]
      {\displaystyle\bigsqcup_{u_0\ra\dotsb\ra u_r\in\Lambda_k^n}X(\sigma(0))} & {\displaystyle\bigsqcup_{u_0\ra\dotsb\ra u_r\in\Delta^n}X(\sigma(0))} \\
      {\displaystyle\bigsqcup_{u_0\ra\dotsb\ra u_r\in\Lambda_k^n}X(\sigma(u_0))} & {\displaystyle\bigsqcup_{u_0\ra\dotsb\ra u_r\in\Delta^n}X(\sigma(u_0))} \\
      {\displaystyle\bigsqcup_{u_0\ra\dotsb\ra u_r\in\Lambda_k^n}*} & {\displaystyle\bigsqcup_{u_0\ra\dotsb\ra u_r\in\Delta^n}*}
      \arrow[from=2-1, to=3-1]
      \arrow[from=2-2, to=3-2]
      \arrow["w", from=3-1, to=3-2]
      \arrow["{w_*}", from=2-1, to=2-2]
      \arrow["{\theta_*}", from=1-1, to=2-1]
      \arrow["{\theta_*}", from=1-2, to=2-2]
      \arrow["j", from=1-1, to=1-2]
      \arrow["\mathlarger{\mathlarger{\mathlarger{\mathlarger{\lrcorner}}}}"{anchor=center, pos=0.05}, draw=none, from=2-1, to=3-2]
    \end{tikzcd}
  \end{center}
  où $\theta:X(\sigma(0))\ra X\circ\sigma$ est la transformation naturelle évidente. Alors $\theta_*$ est une équivalence faible terme à terme par hypothèse.
  D'après le \sref{lemme}{diagonalequiv}, $d(\theta_*)$ est une équivalence faible. Or $d(j)$ s'identifie à
  $\Lambda_k^n\times X(\sigma(0))\ra \Delta^n\times X(\sigma(0))$, qui est une équivalence faible.
  Donc $d(w_*)$ est une équivalence faible, ce qui conclut.
\end{proof}

\begin{theo}[Théorème B de Quillen]\label{theoremeB}
  Soit $f:C\ra D$ un foncteur entre petites catégories. Si pour tout $u:y\ra y'$ dans $D$, $u^*:f\uparrow y'\ra f\uparrow y$
  est une équivalence faible d'homotopie, alors pour tout objet $y$ de $D$, le diagramme cartésien~:
  \begin{center}
    \begin{tikzcd}[column sep=large,row sep=large]
      {f\uparrow y} & {C} \\
      {D\uparrow y} & D
      \arrow["j'", from=2-1, to=2-2]
      \arrow["f_*", from=1-1, to=2-1]
      \arrow["f", from=1-2, to=2-2]
      \arrow["j", from=1-1, to=1-2]
      \arrow["\mathlarger{\mathlarger{\mathlarger{\mathlarger{\lrcorner}}}}"{anchor=center, pos=0.05}, draw=none, from=1-1, to=2-2]
    \end{tikzcd}
  \end{center}
  est homotopiquement cartésien.

  En particulier, on dispose, pour tout $x$ objet de $f^{-1}(y)$, d'une suite exacte~:
  $$\dotsb\ra \pi_{i+1}(D,y)\ra \pi_{i}(f\uparrow y,\bar{x})\overset{j_*}{\ra} \pi_{i}(C,x)\overset{f_*}{\ra} \pi_{i}(D,y)\ra \dotsb$$
  où $\bar{x}=(x,\id_y)$.
\end{theo}

\begin{rem}
  Le théorème A découle du théorème B via la suite exacte. La suite exacte est une conséquence directe du diagramme homotopiquement
  cartésien car $D\uparrow y$ est contractile.
\end{rem}

\begin{proof}
  On reprend le cadre de la démonstration du \sref{théorème}{theoremeA}~: $S(f)$, $T(f)$ et le diagramme suivant.
  \begin{center}
    \begin{tikzcd}[row sep=large,column sep=large]
      D\op & {S(f)} & C \\
      D\op & {S(\id_D)} & D
      \arrow[equal, from=1-1, to=2-1]
      \arrow["f", from=1-3, to=2-3]
      \arrow["{p_1}","\sim"', from=1-2, to=1-3]
      \arrow["{p_2}"', from=1-2, to=1-1]
      \arrow["\sim"',from=2-2, to=2-1]
      \arrow["\sim",from=2-2, to=2-3]
      \arrow["{f_*}", from=1-2, to=2-2]
    \end{tikzcd}
  \end{center}
  On rappelle que $p_2=d(n:T(f)\ra D\op)$ où $n$ est donné par~:
  $$n_{p\bullet}:\bigsqcup_{y_p\ra\dotsb\ra y_0} f\uparrow y_0 \ra \bigsqcup_{y_p\ra\dotsb\ra y_0} *$$
  Donc $p_2=\hocolim{D\op}{N(f\uparrow -)}\ra D\op$. Donc, par le \sref{lemme}{lemmetheoremeB}, le carré
  \begin{center}
    \begin{tikzcd}[column sep=large,row sep=large]
      {f\uparrow y} & {S(f)} \\
      {*} & D\op
      \arrow["y", from=2-1, to=2-2]
      \arrow[from=1-1, to=2-1]
      \arrow["p_2", from=1-2, to=2-2]
      \arrow[from=1-1, to=1-2]
      \arrow["\mathlarger{\mathlarger{\mathlarger{\mathlarger{\lrcorner}}}}"{anchor=center, pos=0.05}, draw=none, from=1-1, to=2-2]
    \end{tikzcd}
  \end{center}
  est homotopiquement cartésien.
  Or, on a le diagramme commutatif suivant~:
  \begin{center}
    \begin{tikzcd}[column sep=tiny,row sep=tiny]
      {f\uparrow y} && {S(f)} && C \\
      &(1)&& (2)&\\
      {D\uparrow y} && {S(\id_D)} && D \\
      &(3)&&& \\
      {*} && D\op
      \arrow["\sim", from=1-3, to=1-5]
      \arrow["\sim", from=3-3, to=3-5]
      \arrow["f", from=1-5, to=3-5]
      \arrow["{f_*}", from=1-3, to=3-3]
      \arrow["\mathlarger{\mathlarger{\mathlarger{\mathlarger{\lrcorner}}}}"{anchor=center, pos=0.05}, draw=none, from=1-3, to=3-5]
      \arrow[from=1-1, to=1-3]
      \arrow[from=3-1, to=3-3]
      \arrow[from=1-1, to=3-1]
      \arrow["\sim" sloped, from=3-1, to=5-1]
      \arrow["\sim" sloped, from=3-3, to=5-3]
      \arrow["y", from=5-1, to=5-3]
      \arrow["\mathlarger{\mathlarger{\mathlarger{\mathlarger{\lrcorner}}}}"{anchor=center, pos=0.05}, draw=none, from=3-1, to=5-3]
      \arrow["\mathlarger{\mathlarger{\mathlarger{\mathlarger{\lrcorner}}}}"{anchor=center, pos=0.05}, draw=none, from=1-1, to=3-3]
    \end{tikzcd}
  \end{center}
  où les flèches notées $\sim$ sont des équivalences faibles. Le carré formé par $(1)$ et $(3)$ est homotopiquement cartésien,
  donc celui formé par $(1)$ également, et donc celui formé par $(1)$ et $(2)$ aussi. C'est le résultat recherché.
\end{proof}

\begin{rem}
  Le théorème B admet la forme duale suivante~:

  Soit $f:C\ra D$ un foncteur entre petites catégories. Si pour tout $u:y\ra y'$ dans $D$, $u_*:f\downarrow y\ra f\downarrow y'$
  est une équivalence faible d'homotopie, alors pour tout objet $y$ de $D$, le diagramme cartésien~:
  \begin{center}
    \begin{tikzcd}[column sep=large,row sep=large]
      {f\downarrow y} & {C} \\
      {D\downarrow y} & D
      \arrow["j'", from=2-1, to=2-2]
      \arrow["f_*", from=1-1, to=2-1]
      \arrow["f", from=1-2, to=2-2]
      \arrow["j", from=1-1, to=1-2]
      \arrow["\mathlarger{\mathlarger{\mathlarger{\mathlarger{\lrcorner}}}}"{anchor=center, pos=0.05}, draw=none, from=1-1, to=2-2]
    \end{tikzcd}
  \end{center}
  est homotopiquement cartésien.
\end{rem}

\begin{coro}
  Si $f:C\ra D$ un foncteur préfibré (respectivement précofibré) entre petites catégories et si pour tout $u:y\ra y'$ dans $D$,
  $u^*:f^{-1} y'\ra f^{-1} y$ (respectivement $u_*:f^{-1} y\ra f^{-1} y'$)
  est une équivalence faible d'homotopie, alors pour tout objet $y$ de $D$, $f^{-1}(y)$ est la fibre homotopique de $f$ au dessus de $y$.
  On a alors, pour tout $x$ objet de $f^{-1}(y)$, une suite exacte~:
  $$\dotsb\ra \pi_{i+1}(D,y)\ra \pi_{i}(f^{-1} y,x)\overset{j_*}{\ra} \pi_{i}(C,x)\overset{f_*}{\ra} \pi_{i}(D,y)\ra \dotsb$$
\end{coro}
\begin{proof}
  Dans ce cas $f^{-1}(y)\ra f\uparrow y$ ou $f^{-1}(y)\ra f\downarrow y$ est une équivalence faible d'homotopie. Plaçons nous dans 
  le premier cas, le second est similaire. On a un carré~:
  \begin{center}
    \begin{tikzcd}[column sep=large,row sep=large]
      {f^{-1}y'} & {f\uparrow y'} \\
      {f^{-1}y}  & {f\uparrow y}
      \arrow["\sim", from=2-1, to=2-2]
      \arrow["u^*","\sim"' sloped, from=1-1, to=2-1]
      \arrow["u^*", from=1-2, to=2-2]
      \arrow["\sim", from=1-1, to=1-2]
    \end{tikzcd}
  \end{center}
  Les deux foncteurs induits de $f^{-1}y'$ dans $f\uparrow y$ sont~:
  \[
    \begin{array}{lcl}
      F:x'&\mapsto& (x',u: y\ra y'=fx)\\
      G:x'&\mapsto& (u^*x',\id_y)             
    \end{array}
  \]
  Or on dispose d'une transformation naturelle $G\Rightarrow F,\; u^*x'\ra x'$ induite par l'adjonction.
  Donc le carré commute à homotopie près. Donc $u^*:f\uparrow y'\ra f\uparrow y$ est une équivalence d'homotopie. On peut maintenant appliquer le théorème B.
\end{proof}

\subsection{Premières propriétés et exemples}\label{premieresprops}

Dans cette sous-section, nous étudions la fonctorialité de la construction $Q$, certaines propriétés élémentaires,
sa commutation aux colimites filtrantes et le comportement relativement aux filtrations et suites exactes de foncteurs exacts.
Nous terminons par la définition de la $K$-théorie supérieure des anneaux et une version supérieure de la \sref{proposition}{K0gradues}
sur la $K$-théorie des modules projectifs gradués de type fini sur un anneau gradué.

\begin{propdefi}[fonctorialité]
  Un foncteur exact $f:\M\ra\M'$ entre deux catégories exactes induit un morphisme en $K$-théorie pour tout $i\geq 0$~:
  $$f_*:K_i\M\ra K_i\M'$$
  Ce morphisme ne dépend que de la classe d'isomorphisme du foncteur $f$, et coïncide avec le morphisme définit dans la 
  \sref{remarque}{remarquefonctorialiteK0categorieexacte} pour $i=0$.
\end{propdefi}

\begin{proof}
  En effet, $f$ induit un foncteur $f:Q\M\ra Q\M'$, et donc $f_*:K_i\M\ra K_i\M'$. Si $f\simeq g$ est un isomorphisme,
  il induit également un isomorphisme entre les foncteurs $f,g:Q\M\ra Q\M'$, et donc $f_*=g_*$.
  La coïncidence dans le cas $i=0$ découle immédiatement de la formule de l'isomorphisme donné dans le \sref{théorème}{theoremeK0constructionQ}.
\end{proof}

\begin{prop}
  Por tout $i\geq 0$, on dispose d'un isomorphisme naturel en $\M$ catégorie exacte~:
  $$K_i\M\simeq K_i\M\op$$
\end{prop}

\begin{proof}
  On a un isomorphisme naturel $Q\M\simeq Q\M\op$.
\end{proof}

\begin{prop}
  L'isomorphisme naturel $Q(\M\times\M')\simeq Q\M\times Q\M'$ induit, pour tout $i\geq 0$, un isomorphisme naturel~:
  \[
    \begin{array}{lcl}
      K_i(\M\times\M') &\simeq  & K_i(\M)\oplus K_i(\M') \\
      x                &\mapsto & \mathrm{pr}_{1*}x + \mathrm{pr}_{2*}x
    \end{array}
  \]
  Le foncteur~:
  $$\oplus:\M\times\M\lra M$$
  est exact et induit un morphisme fonctoriel~:
  $$\oplus:K_i(\M)\oplus K_i(\M)\simeq K_i(\M\times\M)\lra K_i\M$$
  Ce morphisme coïncide avec l'addition.
\end{prop}

\begin{proof}
  Pour montrer la coïncidence avec l'addition, on remarque que pour tout $a$,$b$,$c$ et $d$ dans $K_i\M$~:
  $$(a+b)\oplus (c+d)=(a\oplus c)+(d\oplus d)$$
  On a donc, pour $a$ et $b$ dans $K_i\M$~:
  $$a\oplus b = (a+0)\oplus (0+b)=a\oplus 0 + 0\oplus b$$
  Or, les foncteurs $\id_\M$, $M\mapsto 0\oplus M$ et $M\mapsto M\oplus 0$ sont isomorphes.
  Donc $a\oplus b=a+b$.
\end{proof}

Nous allons maintenant étudier la commutation aux colimites filtrantes. Il s'agit essentiellement d'étendre la \sref{proposition}{CatExColimitesK0}.

\begin{prop}\label{CatExColimitesKi}
  Soit $\M_{(-)}:I\ra \CatEx$, $i\mapsto \M_i$ un foncteur d'une catégorie filtrante $I$ dans la catégorie des petites catégories exactes.
  On note $\M$ sa colimite. Alors, pour tout $n\geq 0$, l'application induite $\colim_i K_n(\M_i)\ra K_n(\M)$ est un isomorphisme.
\end{prop}

\begin{proof}
  On utilise la démonstration de la \sref{porposition}{CatExColimitesK0}. On a que $\E = \colim_i \E_i$, et donc que
  $Q\M = \colim_i Q\M_i$, ce qui conclut par la \sref{proposition}{propositioncolimitecategories}.
\end{proof}

Nous utilisons ci-dessous la construction de la catégorie des suites exactes $\E$ d'une catégorie exacte $\M$
de la \sref{proposition-définition}{propdeficategoriedessuitesexactes}.

\begin{theo}
  Soit $\M$ une catégorie exacte et $\E$ sa catégorie exacte des suites exactes. On note $s,q:\E\ra \M$ comme
  dans la \sref{proposition-définition}{propdeficategoriedessuitesexactes}.
  Alors le foncteur induit~:
  $$(s,q):Q\E\lra Q\M\times Q\M$$
  est une équivalence d'homotopie.
\end{theo}

\begin{proof}
  Par le \sref{théorème A}{theoremeA}, il suffit de montrer que pour tout $M$, $N$ dans $\M$,
  $(s,q)\uparrow (M,N)$ est contractile. On note $\mathcal{C}:=(s,q)\uparrow (M,N)$. Les objets de $\mathcal{C}$ sont
  les $(E,u,v)$ avec $E$ objet de $\E$, $u:sE\ra M$ et $v:qE\ra N$.

  On note $\mathcal{C}'$ la sous-catégorie pleine de $\mathcal{C}$ formée des $(E,u,v)$ avec $u$ une surjection.

  On note $\mathcal{C}''$ la sous-catégorie pleine de $\mathcal{C}$ formée des $(E,u,v)$ avec $u$ une surjection et $v$ une injection.
  \begin{description}
    \item[(A)] L'inclusion $\mathcal{C}'\ra \mathcal{C}$ a un adjoint à gauche.
    Soit $X=(E,u,v)$. Alors $u$ de la forme~:
    $$u:sE\overset{i}{\rightarrowtail} M'\overset{j}{\twoheadleftarrow} M$$
    On a alors le diagramme suivant~:
    \begin{center}
      % https://q.uiver.app/?q=WzAsMTIsWzAsMCwiRToiXSxbMCwxLCJpXypFOiJdLFsxLDAsIjAiXSxbMSwxLCIwIl0sWzIsMCwic0UiXSxbMywwLCJ0RSJdLFs0LDAsInFFIl0sWzUsMCwiMCJdLFs1LDEsIjAiXSxbNCwxLCJxRSJdLFszLDEsIlQiXSxbMiwxLCJNJyJdLFs0LDExLCJpIiwwLHsic3R5bGUiOnsidGFpbCI6eyJuYW1lIjoibW9ubyJ9fX1dLFs0LDVdLFs1LDZdLFs2LDddLFsyLDRdLFszLDExXSxbNSwxMCwiIiwyLHsic3R5bGUiOnsidGFpbCI6eyJuYW1lIjoibW9ubyJ9fX1dLFs2LDksImVxdWFsIiwyXSxbMTAsOV0sWzksOF0sWzExLDEwXSxbMTAsNCwiIiwxLHsic3R5bGUiOnsibmFtZSI6ImNvcm5lciJ9fV1d
      \begin{tikzcd}
        {E:} & 0 & sE & tE & qE & 0 \\
        {i_*E:} & 0 & {M'} & T & qE & 0
        \arrow["i", tail, from=1-3, to=2-3]
        \arrow[from=1-3, to=1-4]
        \arrow[from=1-4, to=1-5]
        \arrow[from=1-5, to=1-6]
        \arrow[from=1-2, to=1-3]
        \arrow[from=2-2, to=2-3]
        \arrow[tail, from=1-4, to=2-4]
        \arrow[equal, from=1-5, to=2-5]
        \arrow[from=2-4, to=2-5]
        \arrow[from=2-5, to=2-6]
        \arrow[from=2-3, to=2-4]
        \arrow["\mathlarger{\mathlarger{\mathlarger{\mathlarger{\lrcorner}}}}"{anchor=center, pos=0.05, rotate=180}, draw=none, from=2-4, to=1-3]
      \end{tikzcd}
    \end{center}
    Et on pose $\overline{X}:=(i_*E,j^!,v)$. On dispose de $X\ra \overline{X}$.

    Soit maintenant $X'=(E',j'^!,v')$ dans $\mathcal{C}'$, et $X\ra X'$ un morphisme représenté par~:
    $$E\rightarrowtail E_0\twoheadleftarrow E'$$
    On a alors~:
    $$u:sE\rightarrowtail sE_0\twoheadleftarrow sE'\overset{j'}{\twoheadleftarrow} M$$
    On donc, à isomorphisme $sE_0\simeq M'$ près, $i=sE\rightarrowtail sE_0$ et $j=M\twoheadrightarrow sE_0$.
    Alors, $E\rightarrowtail E_0$ se factorise en $E\rightarrowtail i_*E\rightarrowtail E_0$. Nous avons donc une factorisation~:
    $$X\lra\overline{X}\lra X'$$
    Il reste à montrer l'unicité de $\overline{X}\ra X'$. Pour cela, on remarque que $\mathcal{C}\downarrow X'=Q\E\downarrow E'$
    est équivalente à la catégorie des couches admissibles de $E'$. Ainsi, une factorisation $X\ra X''\ra X'$
    avec $X''$ dans $\mathcal{C}'$ correspond à une couche $(E''_0,E''_1)\geq (E_0,E_1)$ telle que
    $sE''_1=sE'$, où $(E_0,E_1)$ est la couche de $X\ra X'$.
    Parmi ces couches $(E''_0,E''_1)$, il en existe une minimale telle que $tE''_0=tE_0$ et $tE''_1=sE'+tE_1$.
    Cette couche correspond à $\overline{X}\ra X'$. Ce qui conclut.

    \item[(B)] L'inclusion $\mathcal{C}''\ra\mathcal{C}'$ admet un adjoint à gauche.
    On procède comme pour (A), mais de façon duale (ie. dans $\M\op$).
    
    \item[(C)] L'objet $(0,j_M^!,i_{N!})$ est initial dans $\mathcal{C}''$.
    En effet, soit $(E, j^!,i_!)$ un objet de $\mathcal{C}''$. Alors l'unique morphisme de $(0,j_M^!,i_{N!})$ dans 
    $(E, j^!,i_!)$ est donné par le diagramme~:
    \begin{center}
      % https://q.uiver.app/?q=WzAsMTQsWzAsMiwiMCJdLFs0LDFdLFs0LDIsIjAiXSxbMywyLCJxRSJdLFsyLDIsInRFIl0sWzEsMiwic0UiXSxbMCwwLCIwIl0sWzQsMCwiMCJdLFsxLDAsIjAiXSxbMiwwLCIwIl0sWzMsMCwiMCJdLFsxLDMsIk0iXSxbMywzLCJOIl0sWzIsMSwicUUiXSxbMCw1XSxbNCwzXSxbMywyXSxbNSw0XSxbNSw4LCIiLDEseyJzdHlsZSI6eyJoZWFkIjp7Im5hbWUiOiJlcGkifX19XSxbMTAsMywiIiwxLHsic3R5bGUiOnsidGFpbCI6eyJuYW1lIjoibW9ubyJ9fX1dLFsxMSw1LCIiLDEseyJzdHlsZSI6eyJoZWFkIjp7Im5hbWUiOiJlcGkifX19XSxbMywxMiwiIiwxLHsic3R5bGUiOnsidGFpbCI6eyJuYW1lIjoibW9ubyJ9fX1dLFs5LDEzLCIiLDEseyJzdHlsZSI6eyJ0YWlsIjp7Im5hbWUiOiJtb25vIn19fV0sWzQsMTMsIiIsMSx7InN0eWxlIjp7ImhlYWQiOnsibmFtZSI6ImVwaSJ9fX1dLFs2LDhdLFs4LDldLFs5LDEwXSxbMTAsN11d
      \begin{tikzcd}
        0 & 0 & 0 & 0 & 0 \\
        && qE && {} \\
        0 & sE & tE & qE & 0 \\
        & M && N
        \arrow[from=3-1, to=3-2]
        \arrow[from=3-3, to=3-4]
        \arrow[from=3-4, to=3-5]
        \arrow[from=3-2, to=3-3]
        \arrow[two heads, from=3-2, to=1-2]
        \arrow[tail, from=1-4, to=3-4]
        \arrow[two heads, from=4-2, to=3-2]
        \arrow[tail, from=3-4, to=4-4]
        \arrow[tail, from=1-3, to=2-3]
        \arrow[two heads, from=3-3, to=2-3]
        \arrow[from=1-1, to=1-2]
        \arrow[from=1-2, to=1-3]
        \arrow[from=1-3, to=1-4]
        \arrow[from=1-4, to=1-5]
      \end{tikzcd}
    \end{center}
  \end{description}
\end{proof}

\begin{coro}\label{corollaireadditivite}
  Soit $\M$ et $\M'$ deux catégories exactes, et~:
  \begin{center}
    % https://q.uiver.app/?q=WzAsNSxbMCwwLCIwIl0sWzEsMCwiRiciXSxbMiwwLCJGIl0sWzMsMCwiRicnIl0sWzQsMCwiMCJdLFswLDFdLFsxLDJdLFsyLDNdLFszLDRdXQ==
    \begin{tikzcd}
      0 & {F'} & F & {F''} & 0
      \arrow[from=1-1, to=1-2]
      \arrow[from=1-2, to=1-3]
      \arrow[from=1-3, to=1-4]
      \arrow[from=1-4, to=1-5]
    \end{tikzcd}
  \end{center}
  une suite exacte de foncteurs (exacts) de $\M'$ dans $\M$. Alors, pour tout $i\geq 0$~:
  $$F_*=F'_*+F''_*:K_i\M'\ra K_i\M$$
\end{coro}

\begin{proof}
  La suite exacte de foncteurs induit un foncteur $\tilde{F}:\M'\ra\E$. Il suffit donc de traiter le cas où $\M'=\E$, et de la suite de foncteurs~:
  \begin{center}
    % https://q.uiver.app/?q=WzAsNSxbMCwwLCIwIl0sWzEsMCwiRiciXSxbMiwwLCJGIl0sWzMsMCwiRicnIl0sWzQsMCwiMCJdLFswLDFdLFsxLDJdLFsyLDNdLFszLDRdXQ==
    \begin{tikzcd}
      0 & {s} & t & {q} & 0
      \arrow[from=1-1, to=1-2]
      \arrow[from=1-2, to=1-3]
      \arrow[from=1-3, to=1-4]
      \arrow[from=1-4, to=1-5]
    \end{tikzcd}
  \end{center}
  On pose~:
  \[
  \begin{array}{clcl}
    f: & \M\times\M & \ra     & \E                           \\
       & (M',M'')   & \mapsto & \exac{M'}{M'\oplus M''}{M''}
  \end{array}
  \]
  Alors, $tf$ est isomorphe à $\oplus (s,q)f$, donc~:
  $$t_*f_*=(s_*+q_*)f_*:K_i\M\times K_i\M\ra K_i\M$$
  mais, $(s,q)f$ est isomorphe à l'identité, et $(s,q)$ est une équivalence d'homotopie. Donc $f$ est une équivalence d'homotopie.
  Donc~:
  $$t_*=s_*+q_*$$
\end{proof}

\begin{defi}
  Soit $F:C\ra\M$ un foncteur avec $\M$ une catégorie exacte. Une filtrations~·
  $$0=F_0\subseteq F_1\subseteq \dotsb\subseteq F_n=F$$
  de $F$ est la donnée de transformations naturelles~:
  \begin{center}
    % https://q.uiver.app/?q=WzAsNCxbMCwwLCIwPUZfMCJdLFsxLDAsIkZfMSJdLFszLDAsIkZfe24tMX0iXSxbNCwwLCJGX249RiJdLFswLDEsInVfMSJdLFsyLDMsInVfbiJdLFsxLDIsIiIsMCx7InN0eWxlIjp7ImJvZHkiOnsibmFtZSI6ImRhc2hlZCJ9fX1dXQ==
    \begin{tikzcd}
      {0=F_0} & {F_1} && {F_{n-1}} & {F_n=F}
      \arrow["{u_1}", from=1-1, to=1-2]
      \arrow["{u_n}", from=1-4, to=1-5]
      \arrow[dashed, from=1-2, to=1-4]
    \end{tikzcd}
  \end{center}
  où les $u_i$ sont des monomorphismes admissibles en chaque point.
\end{defi}

\begin{rem}
  Dans ce cas, on peut définir les quotients $F_p/F_q:C\ra\M$ pour $q\leq p$, et si $C$ est une catégorie exacte et les $F_i$
  sont exacts, alors les quotients $F_p/F_q:C\ra\M$ sont également exacts, par le lemme du serpent.
\end{rem}

On peut ainsi étendre le corollaire ci-dessus par récurrence. On obtient les deux corollaires suivants.

\begin{coro}[additivité pour les filtrations caractéristiques]\label{corollaireadditivitefiltration}
  Soit $F:\M'\ra\M$ foncteur exact et $0=F_0\subseteq F_1\subseteq \dotsb\subseteq F_n=F$ filtration admissible de foncteurs exacts.
  Alors~:
  $$F_*=\sum_{p=1}^n(F_p/F_{p-1})_*:K_i\M'\ra K_i\M$$
\end{coro}

\begin{coro}[additivité pour les suites exactes caractéristiques]\label{corollaireadditivitesuitesexacte}
  Si $0\ra F_0\ra\dotsb F_n\ra 0$ est une suite exacte de foncteurs exacts $\M'\ra \M$, alors~:
  $$\sum_{p=0}^n(F_p)_*=0:K_i\M'\ra K_i\M$$
\end{coro}

Nous allons maintenant définir la $K$-théorie supérieure des anneaux et des schémas.

\begin{defi}
  Pour $A$ un anneau et $i\geq 0$. On définit le $i$-ème groupe de $K$-théorie de $A$ comme~:
  $$K_i(A):=K_i(\Proj{A})$$
  Pour $f:A\ra A'$ un morphisme d'anneaux, le foncteur exact $A'\otimes_A -: \Proj{A}\ra\Proj{A'}$ induit un morphisme~:
  $$f^*:K_i(A)\ra K_i(A')$$
\end{defi}

\begin{defi}
  Pour $X$ un schéma et $i\geq 0$. On définit le $i$-ème groupe de $K$-théorie de $X$ comme~:
  $$K_i(A):=K_i(\Proj{X})$$
  Pour $f:X'\ra X$ un morphisme de schémas, le foncteur exact $f^* -: \Proj{X}\ra\Proj{X'}$ induit un morphisme~:
  $$f^*:K_i(X)\ra K_i(X')$$
\end{defi}

On reprend maintenant le cadre de la \sref{proposition}{K0gradues}.

\begin{prop}\label{Kngradues}
  Soit $A=A_0\oplus A_1\oplus\dotsb$ un anneau gradué en degrés positifs et $i\geq 0$ un entier.
  L'automorphisme de translation $t$ fait de $K_i(\Pgr{A})$ un $\Z[t,t^{-1}]$-module. On a un isomorphisme de $\Z[t,t^{-1}]$-modules~:
  \[
  \begin{array}{llll}
    \phi:&\Z[t,t^{-1}]\otimes_{\Z}K_i(A_0) &\ra    & K_i(\Pgr{A}) \\
         &1\otimes x                       &\mapsto& (A\otimes_{A_0}-)_*(x)
  \end{array}
  \]
\end{prop}

\begin{proof}
  On reprend les notations de la preuve de la \sref{proposition}{K0gradues}.

  On dispose d'une filtration de l'identité sur $\Pgr{A}_q$, donnée pour $P$ dans $\Pgr{A}_q$ par~:
  $$0=F_{-q-1}P\subseteq F_{-q}\subseteq \dotsb \subseteq F_qP=P$$
  Et donc, par le \sref{corollaire}{corollaireadditivitefiltration}~:
  $$\id_{K_i\Pgr{A}_q}=\sum_{n=-q}^q (F_n/F_{n-1})_*=\sum_{n=-q}^q (A[-n]\otimes_{A_0}T(-)_n)_*=\sum_{n=-q}^q (\phi(t^n\otimes T(-)_n))_*$$
  Ainsi, si on pose $\chi_q: K_i(\Pgr{A}_q)\ra \bigoplus_{n=-q}^q K_i(A_0)$ induit par $P\mapsto \oplus_n T(P)_n$, 
  on a que $\phi_q$ et $\chi_q$ sont réciproques, où $\phi_q$ est la restriction de $\phi$ à $\bigoplus_{n=-q}^q \Z\cdot t^n\otimes K_i(A_0)$.
  Donc $\phi_q$ est un isomorphisme. Par la \sref{proposition}{CatExColimitesKi}, $\phi$ est un isomorphisme.
\end{proof}

\section{\texorpdfstring{Construction $+$ de Quillen}{Construction + de Quillen}}\label{sectionplus}

\subsection{\texorpdfstring{La construction $+$ en topologie}{La construction + en topologie}}

Dans cette section, nous définissons une construction $+$ en topologie. Nous utilisons un certain nombre de résultats
de topologie, notamment l'homologie à coefficients et les tours de Postnikov.
Voir \sref{l'annexe}{annexetopologie} pour plus d'information.

\begin{defi}\label{definitionplustopologie}
  Soit $X$ connexe dans $\DEns$, $x\in X_0$, et $P\triangleleft \pi_1(X,x)$ un sous-groupe normal parfait.
  Un morphisme $f:X\ra X^+$ est une construction $+$ pour $P$ si~:
  \begin{description}
    \item[$(1)$] $0\ra P\ra \pi_1(X,x) \ra \pi_1(X^+,fx)\ra 0$ est exacte~;
    \item[$(2)$] Pour tout $L$ système de coefficients sur $X^+$, l'application $H_*(X,f^*L)\ra H_*(X^+,L)$ est un isomorphisme.
  \end{description}
\end{defi}

\begin{rem}
  Comme $X$ est supposé connexe, le point $(1)$ ne dépend pas de $x\in X_0$.
\end{rem}

\begin{prop}\label{constructionpluscasfacile}
  Soit $X$ connexe dans $\DEns$ et $x\in X_0$ tels que $\pi_1(X,x)$ soit un groupe parfait. Alors une construction $+$ existe pour $\pi_1(X,x)$.
\end{prop}

\begin{proof}
  Quitte à changer $X$, on peut supposer que $X$ est un complexe de Kan.
  On se donne $I$ un ensemble de générateurs de $\pi_1(X,x)$. On définit $Y$ comme la somme amalgamée~:
  \begin{center}
    \begin{tikzcd}[column sep = large, row sep = large]
      {\bigvee_{\gamma\in I}\partial\Delta^2} & X \\
      {\bigvee_{\gamma\in I}\Delta^2} & Y
      \arrow[hook, from=1-1, to=2-1]
      \arrow[from=1-2, to=2-2]
      \arrow["{\vee(\gamma,*,*)}", from=1-1, to=1-2]
      \arrow[from=2-1, to=2-2]
      \arrow["\mathlarger{\mathlarger{\mathlarger{\mathlarger{\lrcorner}}}}"{anchor=center, rotate=180, pos=0.05}, draw=none, from=2-2, to=1-1]
    \end{tikzcd}
  \end{center}
  Alors, par le \sref{lemme d'extension}{lemmedextension} $\pi_1(Y)=0$.
  Maintenant, comme $\bigvee_{\gamma\in I}\Delta^2$ est contractile,
  ${\bigvee_{\gamma\in I}\partial\Delta^2}\ra X\ra Y$ est une suite cofibre. On a donc les suites exactes suivantes en homologie sur $\Z$~:
  $$0\ra H_i(X)\ra H_i(Y)\ra 0\text{ pour }i\geq 3$$
  $$H_2({\bigvee_{\gamma\in I}\partial\Delta^2})\ra H_2(X)\ra H_2(Y)\ra H_1({\bigvee_{\gamma\in I}\partial\Delta^2})\ra H_1(X)$$
  Or, par le \sref{théorème de Hurewicz}{theoremeHurewiczpi1}, $H_1(X)=0$. De plus $H_2({\bigvee_{\gamma\in I}\partial\Delta^2})=0$
  et $H_1({\bigvee_{\gamma\in I}\partial\Delta^2})$ est $\Z$-libre.
  Donc, la suite se scinde et on peut choisir un isomorphisme~:
  $$H_2(Y)\simeq H_2(X)\oplus \bigoplus_{\gamma\in I}\Z\cdot [\gamma]$$
  Or, $\pi_0(Y)=*$ et $\pi_1(Y)=0$, donc par le \sref{théorème de Hurewicz}{theoremeHurewitzpin}, on a l'isomorphisme de Hurewicz $\mathcal{H}_2:\pi_2(Y,fx)\simeq H_2(Y)$.
  On se donne $\tilde{Y}$ complexe de Kan équivalent à $Y$
  et, pour chaque $\gamma\in I$, $[s_\gamma]\in \pi_2(\tilde{Y},fx)$ tel que $\mathcal{H}_2([s_\gamma])=[\gamma]$.
  On définit alors $X^+$ comme la somme amalgamée~:
  \begin{center}
    \begin{tikzcd}[column sep = large, row sep = large]
      {\bigvee_{\gamma\in I}\partial\Delta^3} & \tilde{Y} \\
      {\bigvee_{\gamma\in I}\Delta^3} & X^+
      \arrow[hook, from=1-1, to=2-1]
      \arrow[from=1-2, to=2-2]
      \arrow["{\vee(s_\gamma,*,*,*)}", from=1-1, to=1-2]
      \arrow[from=2-1, to=2-2]
      \arrow["\mathlarger{\mathlarger{\mathlarger{\mathlarger{\lrcorner}}}}"{anchor=center, rotate=180, pos=0.05}, draw=none, from=2-2, to=1-1]
    \end{tikzcd}
  \end{center}
  Par le \sref{lemme d'extension}{lemmedextension}, $\pi_1(X^+)=0$. Or la suite ${\bigvee_{\gamma\in I}\partial\Delta^3}\ra \tilde{Y}\ra X^+$ est une suite cofibre.
  Comme $H_i(\bigvee_{\gamma\in I}\partial\Delta^3)=0$ pour $i\geq 3$, et $H_2(\bigvee_{\gamma\in I}\partial\Delta^3)\hookrightarrow H_2(\tilde{Y})$
  est une injection, on a $H_i(\tilde{Y})\simeq H_i(X^+)$ et donc $H_i(X)\simeq H_i(X^+)$ pour $i\geq 3$. C'est également clair pour $H_1$ et $H_0$.
  Pour $H_2$, on a le diagramme suivant~:
  \begin{center}
    \begin{tikzcd}[column sep = large, row sep = large]
      & {H_1(\bigvee_I\partial\Delta^2)} \\
      {H_2(\bigvee_{I}\partial\Delta^3)} & {H_2(\tilde{Y})} & {H_2(X^+)} & {} \\
      & {H_2(X)}
      \arrow[hook, from=2-1, to=2-2]
      \arrow[two heads, from=2-2, to=2-3]
      \arrow[hook, from=3-2, to=2-2]
      \arrow[from=3-2, to=2-3]
      \arrow[two heads, from=2-2, to=1-2]
      \arrow["\sim" sloped, from=2-1, to=1-2]
    \end{tikzcd}
  \end{center}
  La flèche $H_2(\bigvee_{I}\partial\Delta^3)\ra H_1(\bigvee_I\partial\Delta^2)$ est un isomorphisme par construction de
  $\bigvee_I\partial\Delta^3\ra \tilde{Y}$. Comme la ligne et la colonne sont exactes, ceci implique que la flèche
  $H_2(X)\ra H_2(X^+)$ est aussi un isomorphisme.

  On a donc que $H_*(-;\Z)$ envoie $X\ra X^+$ sur un isomorphisme. Ainsi, par le \sref{théorème des coefficients universels}{coeffsuniv},
  $H_*(-;M)$ envoie $X\ra X^+$ sur un isomorphisme pour tout $M$ $\Z$-module. Or, comme $\pi_1(X^+)=0$, un système de coefficients
  sur $X^+$ n'est autre qu'un $\Z$-module. Donc $X\ra X^+$ est une construction $+$ pour $\pi_1(X,x)$.
\end{proof}

\begin{prop}
  Soit $X$ connexe dans $\DEns$, $x\in X_0$ et $P\triangleleft \pi_1(X,x)$ sous-groupe normal parfait. Alors une construction $+$
  pour $P$ existe.
\end{prop}

\begin{proof}
  On note $p:\tilde{X}\ra X$ le revêtement connexe correspondant à $\pi_1(X,x)/P$. On se donne $\tilde{x}\in \tilde{X}$ au-dessus de $x$.
  On a alors $\pi_1(\tilde{X},\tilde{x})=P$.

  On pose $\tilde{f}:\tilde{X}\hookrightarrow \tilde{X}^+$ une construction $+$ pour $\pi_1(\tilde{X},\tilde{x})$ qui soit une cofibration.
  On définit $X^+$ comme la somme amalgamée suivante~:
  \begin{center}
    \begin{tikzcd}[column sep = large, row sep = large]
      {\tilde{X}} & \tilde{X}^+ \\
      {X} & X^+
      \arrow["p", from=1-1, to=2-1]
      \arrow["p'", from=1-2, to=2-2]
      \arrow[hook, "\tilde{f}", from=1-1, to=1-2]
      \arrow[hook, "f", from=2-1, to=2-2]
      \arrow["\mathlarger{\mathlarger{\mathlarger{\mathlarger{\lrcorner}}}}"{anchor=center, rotate=180, pos=0.05}, draw=none, from=2-2, to=1-1]
    \end{tikzcd}
  \end{center}
  Comme tous les objets de $\DEns$ sont cofibrants, par le dual de la \sref{proposition}{homocartesienfibrant}, 
  $\DEns$ est propre à gauche.
  Ainsi, par le dual de la \sref{proposition}{criterehomocartesienpropreadroite}, le carré ci-dessus est homotopiquement cocartésien.
  Par le \sref{théorème de Van-Kampen}{theoremedeVanKampen}, la suite~:
  $$\exa{P}{\pi_1(X,x)}{\pi_1(X^+,fx)}\text{ est exacte}$$
  Comme le carré est homotopiquement cocartésien, et comme $\tilde{X}\hookrightarrow\tilde{X}\times\Delta^1\sqcup_{X}X\overset{\sim}{\twoheadrightarrow}X$,
  on a le diagramme commutatif suivant à carrés cocartésiens~:
  \begin{center}
    \begin{tikzcd}[column sep = large, row sep = large]
      {\tilde{X}} & \tilde{X}^+ \\
      {\tilde{X}\times\Delta^1\sqcup_{X}X} & {X'^+} \\
      {X} & {X^+}                            
      \arrow[hook, from=1-1, to=2-1]
      \arrow["p''",hook, from=1-2, to=2-2]
      \arrow[hook, from=1-1, to=1-2]
      \arrow[hook, from=2-1, to=2-2]
      \arrow[two heads, "\sim" sloped,from=2-1, to=3-1]
      \arrow[hook, from=3-1, to=3-2]
      \arrow["\sim" sloped, from=2-2, to=3-2]
      \arrow["\mathlarger{\mathlarger{\mathlarger{\mathlarger{\lrcorner}}}}"{anchor=center, rotate=180, pos=0.05}, draw=none, from=2-2, to=1-1]
      \arrow["\mathlarger{\mathlarger{\mathlarger{\mathlarger{\lrcorner}}}}"{anchor=center, rotate=180, pos=0.05}, draw=none, from=3-2, to=2-1]
    \end{tikzcd}
  \end{center}
  Où les applications marqué de $\sim$ sont des équivalences.
  Maintenant, soit $L$ un système de coefficients locaux sur $X^+$. Comme on peut identifier $\tilde{X}^+\setminus \tilde{X}$
  et $X'^+\setminus \tilde{X}\times\Delta^1\sqcup_{X}X$, on a~:
  $$C_*(\tilde{X}^+,\tilde{X};p''^*L)\simeq C_*(X'^+, \tilde{X}\times\Delta^1\sqcup_{X}X;L)$$
  Et donc~: $$H_*(\tilde{X}^+,\tilde{X};p'^*L)\lra H_*(X^+,X;L)$$ est un isomorphisme.
  Or, comme $H_*(\tilde{X};\tilde{f}^*p'^*L)\ra H_*(\tilde{X}^+;p'^*L)$ est un isomorphisme, $H_*(\tilde{X}^+,\tilde{X};p'^*L)=0$.
  Donc $H_*(X;f^*L)\ra H_*(X^+;L)$ est un isomorphisme.
\end{proof}

Nous énonçons ici une propriété de stabilité par produit qui sera utiles par la suite. La démonstration utilise la suite
spectrale d'une filtration, voir \cite[5.3]{Weib2}.

\begin{prop}\label{propositionconstructionplusproduit}
  Soit $X$ connexe dans $\DEns$, $x\in X_0$ et $P\triangleleft \pi_1(X,x)$ sous-groupe normal parfait.
  Soit $X\hookrightarrow X^+$ la construction $+$ pour $P$ comme donné par la construction précédente.

  Soit $Z$ un ensemble simplicial connexe. Alors $X\times Z\hookrightarrow X^+\times Z$ est une construction $+$ pour
  $P\times\{1\}$.
\end{prop}

\begin{proof}
  La propriété souhaitée sur les groupes fondamentaux est immédiate. Il s'agit donc de vérifier l'isomorphisme
  en homologie.
  
  En reprenant la construction de la proposition précédente, on voit qu'il suffit de montrer que~:
  $$\tilde{X}\times Z\hookrightarrow\tilde{X}^+\times Z$$
  est un isomorphisme en homologie pour tout système de coefficients locaux.
  Or $\tilde{X}^+$ est simplement connexe. Donc tout système de coefficients locaux sur $\tilde{X}^+\times Z$
  est induit par un système dur $Z$. Soit $L$ un tel système sur $Z$. Par le morphisme d'Eilenberg-Zilber (voir \cite[IV.2.5]{Goer}),
  il existe un isomorphisme en homologie fonctoriel en $W$ ensemble simplicial~:
  $$C_*(W;\Z)\otimes C_*(Z;L)\lra C_*(X\times Z;L)$$
  De plus, si on filtre $C_*(W;\Z)$ par degrés, on obtient une filtration fonctorielle en $W$ du complexe $C_*(W;\Z)\otimes C_*(Z;L)$,
  dont la suite spectrale associée $E^r_{pq}(W)$ vérifie~:
  \[
    \begin{array}{lcl}
      E^0_{pq}&=&C_p(W;\Z)\otimes C_q(Z;L)\\
      E^1_{pq}&=&C_p(W;H_q(Z;L)) \\
      E^2_{pq}&=&H_p(W;H_q(Z;L))
    \end{array}
  \]
  Ainsi, comme le morphisme~:
  $$H_p(\tilde{X};H_q(Z;L))\lra H_p(\tilde{X}^+;H_q(Z;L))$$
  est un isomorphisme pour tout $p$ et tout $q$, le morphisme de suites spectrales~:
  $$E^r_{pq}(\tilde{X})\ra E^r_{pq}(\tilde{X}^+)$$
  induit un isomorphisme entre les limites. Donc $H_*(\tilde{X}\times Z;L)\ra H_*(\tilde{X}^+\times Z;L)$ est un isomorphisme.
\end{proof}

Nous allons maintenant énoncer un résultat d'unicité.

\begin{prop}\label{topologiepluspropuniv}
  Soit $X$ connexe dans $\DEns$, $x\in X_0$ et $P\triangleleft \pi_1(X,x)$ un sous-groupe normal parfait. Soit $f:X\hookrightarrow X^+$ une construction $+$
  pour $P$. Alors pour tout $g:X\ra Y$ avec $P\subseteq \myker{\pi_1(g,x)}$ et $Y$ complexe de Kan, il existe $h:X^+\ra Y$ tel que $h\circ f=g$.
  \begin{center}
    \begin{tikzcd}[column sep = small, row sep = large]
      X && Y \\
      & {X^+}
      \arrow["f",hook, from=1-1, to=2-2]
      \arrow["{\exists h}", dashed, from=2-2, to=1-3]
      \arrow["g", from=1-1, to=1-3]
    \end{tikzcd}
  \end{center}
  De plus, tout autre $h'$ a le même type d'homotopie que $h$ dans $\Ho{\DEns\uparrow X}$. 
\end{prop}

\begin{rem}
  La dernière affirmation est équivalente à~: $h'$ est homotope à $h$ relativement à $X$. 
\end{rem}

\begin{proof}
  Sans perte de généralité, on peut supposer que $Y$ est connexe.
  On se donne $\tilde{X}\ra X$ revêtement associé à $\pi_1(X,x)/P$ et $\tilde{Y}\ra Y$ revêtement universels. Alors~:
  \begin{center}
    \begin{tikzcd}[column sep = large, row sep = large]
      {\tilde{X}} & {\tilde{Y}} \\
      X & Y
      \arrow["{\exists \tilde{g}}", dashed, from=1-1, to=1-2]
      \arrow[two heads, from=1-1, to=2-1]
      \arrow[two heads, from=1-2, to=2-2]
      \arrow["g", from=2-1, to=2-2]
    \end{tikzcd}
  \end{center}
  Maintenant, $\tilde{Y}$ est simplement connexe et par le \sref{théorème}{Postnikovfibrationsprincipales}, il dispose donc d'une tour de Postnikov de fibrations
  principales.
  \begin{center}
    \begin{tikzcd}
      && {\tilde{Y}(n)} \\
      {\tilde{Y}} && {\tilde{Y}(n-1)} & {K(n+1,\pi_n(\tilde{Y}))} \\
      && {\tilde{Y}(2)} & {K(4,\pi_3(\tilde{Y}))} \\
      && {\tilde{Y}(1)\simeq *} & {K(3,\pi_2(\tilde{Y}))}
      \arrow[two heads, from=1-3, to=2-3]
      \arrow["{u_{n-1}}", from=2-3, to=2-4]
      \arrow[dotted, two heads, from=2-3, to=3-3]
      \arrow["{u_2}", from=3-3, to=3-4]
      \arrow[two heads, from=3-3, to=4-3]
      \arrow["{u_1}", from=4-3, to=4-4]
      \arrow[from=2-1, to=1-3]
      \arrow[from=2-1, to=2-3]
      \arrow[from=2-1, to=3-3]
      \arrow[from=2-1, to=4-3]
    \end{tikzcd}
  \end{center}
  On choisit des espaces d'Eilenberg-MacLane qui soient fibrants.
  Supposons que l'on ait défini des applications compatibles $\tilde{h}(i):\tilde{X}^+\ra \tilde{Y}(i)$, pour $1\leq i\leq n-1$, tels que pour chaque $i$,
  le diagramme suivant soit commutatif~:
  \begin{center}
    \begin{tikzcd}
      {\tilde{X}} & {\tilde{Y}} & {\tilde{Y}(i)} \\
      & {\tilde{X}^+}
      \arrow[hook, from=1-1, to=2-2]
      \arrow["{\tilde{h}(i)}", from=2-2, to=1-3]
      \arrow["{\tilde{g}}", from=1-1, to=1-2]
      \arrow[from=1-2, to=1-3]
    \end{tikzcd}
  \end{center}
  On a alors le diagramme commutatif suivant~:
  \begin{center}
    \begin{tikzcd}
      {\tilde{X}}   & {\tilde{Y}(n)}   & \mathrm{F}(u_{n-1})\\
      {\tilde{X}^+} & {\tilde{Y}(n-1)} & \mathrm{Path}(u_{n-1}) & {K(n+1,\pi_n(\tilde{Y}))}
      \arrow[from=1-2, to=2-2]
      \arrow[hook, from=1-3, to=2-3]
      \arrow["\sim", from=1-2, to=1-3]
      \arrow["\sim", from=2-2, to=2-3]
      \arrow[two heads, "p", from=2-3, to=2-4]
      \arrow["{\tilde{h}(n-1)}", from=2-1, to=2-2]
      \arrow[from=1-1, to=1-2]
      \arrow[from=1-1, to=2-1]
      \arrow["\sim", from=1-2, to=1-3]
    \end{tikzcd}
  \end{center}
  Où $\mathrm{Path}(u_{n-1})={\tilde{Y}(n-1)\times_{K(n+1,\pi_n(\tilde{Y}))} K(n+1,\pi_n(\tilde{Y}))^{\Delta^1}}$ et $\mathrm{F}(u_{n-1})$
  est la fibre de $p$.
  Comme le L à droite est une suite fibre, ce diagramme nous donne un morphisme $\tilde{X}^+\cup C\tilde{X}\ra K(n+1,\pi_n(\tilde{Y}))$
  où $C\tilde{X}:=\tilde{X}\times\Delta^1 / \tilde{X}\times\{1\}$ est le cône de $\tilde{X}$.
  La classe d'homotopie de ce morphisme correspond, par le \sref{théorème}{EMfoncteur}, à un élément de
  $H^{n+1}(\tilde{X}^+\cup C\tilde{X};\pi_2(\tilde{Y}))=H^{n+1}(\tilde{X}^+,\tilde{X};\pi_2(\tilde{Y}))=0$
  (par le \sref{théorème}{coeffsunivcohom}). Donc $\tilde{X}^+\cup C\tilde{X}\ra K(n+1,\pi_n(\tilde{Y}))$
  est homotope à une application constante et on peut l'étendre en $C(\tilde{X}^+)\ra K(n+1,\pi_n(\tilde{Y}))$,
  c'est-à-dire prolonger $\tilde{X}\ra \mathrm{F}(u_{n-1})$ à $\tilde{X}^+$. On peut donc trouver $\tilde{h}(n):\tilde{X}^+\ra \tilde{Y}(n)$
  qui complète le diagramme~:
  \begin{center}
    \begin{tikzcd}
      {\tilde{X}} && {\tilde{Y}(n)} \\
      {\tilde{X}^+} && {\tilde{Y}(n-1)}
      \arrow[two heads, from=1-3, to=2-3]
      \arrow["{\tilde{h}(n-1)}"', from=2-1, to=2-3]
      \arrow[from=1-1, to=1-3]
      \arrow[from=1-1, to=2-1]
      \arrow["{\tilde{h}(n)}", from=2-1, to=1-3]
    \end{tikzcd}
  \end{center}
  Ainsi, nous avons construit des applications compatibles $\tilde{h}(i):\tilde{X}^+\ra \tilde{Y}(i)$, pour $i\geq 1$.
  Or par le \sref{corollaire}{limitetourdePostnikov} $\lim_{i\in \N}\tilde{Y}(i)=\tilde{Y}$.
  Donc, on a construit $\tilde{h}:\tilde{X}^+\ra \tilde{Y}$.
  Donc, par le diagramme suivant, on a construit $h$~:
  \begin{center}
    \begin{tikzcd}[column sep = large, row sep = large]
      {\tilde{X}} & {\tilde{X}^+} & {\tilde{Y}} \\
      X & {X^+} & Y
      \arrow["{\tilde{h}}", from=1-2, to=1-3]
      \arrow[hook, from=1-1, to=1-2]
      \arrow[two heads, from=1-1, to=2-1]
      \arrow[two heads, from=1-3, to=2-3]
      \arrow[hook, from=2-1, to=2-2]
      \arrow["{\exists h}", dashed, from=2-2, to=2-3]
      \arrow["g"', bend right=20, from=2-1, to=2-3]
      \arrow[two heads, from=1-2, to=2-2]
      \arrow["\mathlarger{\mathlarger{\mathlarger{\mathlarger{\lrcorner}}}}"{anchor=center, rotate=180, pos=0.05}, draw=none, from=2-2, to=1-1]
    \end{tikzcd}
  \end{center}
  Montrons maintenant l'unicité à homotopie près. Soit $h'$ une autre telle application. Par propriété de relèvement le long des revêtements, on
  peut trouver $\tilde{h}$ et $\tilde{h'}$ qui complètent le diagramme~:
  \begin{center}
    \begin{tikzcd}[column sep = large, row sep = large]
      {\tilde{X}} & {\tilde{X}^+} & {\tilde{Y}} \\
      X & {X^+} & Y
      \arrow["{\tilde{h}}", shift left=1, from=1-2, to=1-3]
      \arrow[hook, from=1-1, to=1-2]
      \arrow[two heads, from=1-1, to=2-1]
      \arrow[two heads, from=1-3, to=2-3]
      \arrow[hook, from=2-1, to=2-2]
      \arrow["h", shift left=1, from=2-2, to=2-3]
      \arrow[two heads, from=1-2, to=2-2]
      \arrow["\lrcorner"{anchor=center, pos=0.125, rotate=180}, draw=none, from=2-2, to=1-1]
      \arrow["{h'}"', shift right=1, from=2-2, to=2-3]
      \arrow["{\tilde{h'}}"', shift right=1, from=1-2, to=1-3]
    \end{tikzcd}
  \end{center}
  Il suffit donc de montrer que $\tilde{h}$ et $\tilde{h'}$ sont homotopes relativement à $\tilde{X}$.
  Pour cela, on se donne une homotopie~:
  $$H(1):\tilde{h}(1)\Rightarrow\tilde{h'}(1):\tilde{X}^+\times\Delta^1\ra \tilde{Y}(1)$$
  C'est toujours possible car $\tilde{Y}(1)$ est contractile.
  Or, on remarque que 
  $$\tilde{X}\times\Delta^1\cup\tilde{X}^+\times\partial\Delta^1\hookrightarrow\tilde{X}^+\times\Delta^1$$
  est une équivalence en homologie, et donc aussi en cohomologie.
  En effet, la suite exacte de complexes~:
  \begin{center}
    \begin{tikzcd}[row sep = small, column sep = small]
      0 \ra C_*(\tilde{X}\times\Delta^1) \arrow[r]
      & C_*(\tilde{X}\times \Delta^1\cup\tilde{X}^+\times \{0\})\oplus C_*(\tilde{X}\times \Delta^1\cup\tilde{X}^+\times \{1\})
      \arrow[d, phantom, ""{coordinate, name=Z}]
      \arrow[d,
      rounded corners,
      to path={ -- ([xshift=2ex]\tikztostart.east)
      |- (Z) [near end]\tikztonodes
      -| ([xshift=-2ex]\tikztotarget.west)
      -- (\tikztotarget)}] \\
      &C_*(\tilde{X}\times\Delta^1\cup\tilde{X}^+\times\partial\Delta^1) \ra 0
      \end{tikzcd}
  \end{center}
  induit, car $\tilde{X}\times \Delta^1\cup\tilde{X}^+\times \{-\}\simeq \tilde{X}^+$, des suites exactes~:
  $$H_n(\tilde{X}^+)\overset{(1,-1)}{\hookrightarrow}H_n(\tilde{X}^+)\oplus H_n(\tilde{X}^+)\twoheadrightarrow%
    H_n(\tilde{X}\times\Delta^1\cup\tilde{X}^+\times\partial\Delta^1)$$
  Donc $\tilde{X}^+\ra \tilde{X}\times\Delta^1\cup\tilde{X}^+\times\partial\Delta^1$ est un isomorphisme en homologie.
  Or $\tilde{X}^+ \simeq \tilde{X}^+ \times \Delta^1$.
  Donc $\tilde{X}\times\Delta^1\cup\tilde{X}^+\times\partial\Delta^1\hookrightarrow\tilde{X}^+\times\Delta^1$ également.
  Ainsi, on peut, comme ci-dessus, relever successivement l'homotopie à
  chaque $\tilde{Y}(n)$, et en passant à la limite, à $\tilde{Y}$.
\end{proof}

On a directement le corollaire suivant.

\begin{coro}\label{uniciteplustopologie}
  Soit $X$ connexe dans $\DEns$, $x\in X_0$ et $P\triangleleft \pi_1(X,x)$ un sous-groupe normal parfait.
  Alors une construction $+$ pour $P$ est unique à équivalence d'homotopie sous $X$ près.
\end{coro}

Plus loin, nous auront besoin d'une version améliorée de ces résultats. On rappelle qu'un espace $X$ est simple si en chaque point $x\in X_0$ et chaque
$n\geq 1$, $\pi_1(X,x)$ agit trivialement sur $\pi_n(X,x)$. En particulier, $\pi_1(X,x)$ est abélien pour tout $x\in X_0$.

\begin{lem}\label{lemmeplussurZ}
  Soit $X$ connexe dans $\DEns$ et $x\in X_0$ tels que le sous-groupe $C:=[\pi_1(X,x),\pi_1(X,x)]$ soit parfait.
  Soit $f:X\ra X^+$ une construction $+$ pour $C$ telle que $X^+$ soit simple.
  Si $g:X\ra Y$ vérifie~:
  \begin{description}
    \item[$(1)$] $Y$ est simple et connexe~;
    \item[$(2)$] $g_*:H_*(X,\Z)\ra H_*(Y,\Z)$ est un isomorphisme~;
  \end{description}
  alors $g$ est aussi une construction $+$ pour $C$. En particulier $Y$ et $X$ sont équivalents relativement à $X$.
\end{lem}

\begin{rem}
  La preuve de ce lemme repose sur le \sref{théorème}{Postnikovfibrationsprincipales},
  lequel admet la version générale du théorème de Hurewicz pour les paires.
  Voir la référence donnée dans la preuve de cette proposition pour plus d'information.
\end{rem}

\begin{proof}
  Soit $h:X\ra Z$ avec $Z$ simple. Alors par le \sref{théorème}{Postnikovfibrationsprincipales},
  $Z$ admet une tour de Postnikov de fibrations principales~:
  \begin{center}
    \begin{tikzcd}
      && {Z(n)} \\
      {Z} && {Z(n-1)} & {K(n+1,\pi_n(Z))} \\
      && {Z(1)} & {K(3,\pi_2(Z))} \\
      && {Z(0)\simeq *} & {K(2,\pi_1(Z))}
      \arrow[two heads, from=1-3, to=2-3]
      \arrow["{u_{n-1}}", from=2-3, to=2-4]
      \arrow[dotted, two heads, from=2-3, to=3-3]
      \arrow["{u_1}", from=3-3, to=3-4]
      \arrow[two heads, from=3-3, to=4-3]
      \arrow["{u_0}", from=4-3, to=4-4]
      \arrow[from=2-1, to=1-3]
      \arrow[from=2-1, to=2-3]
      \arrow[from=2-1, to=3-3]
      \arrow[from=2-1, to=4-3]
    \end{tikzcd}
  \end{center}
  Comme $g_*:H_*(X,\Z)\ra H_*(Y,\Z)$ est un isomorphisme, c'est également le cas de $g^*:H^*(Y,M)\ra H^*(X,M)$ pour tout $M$ $\Z$-module
  par le \sref{theorème}{coeffsunivcohom}. Donc, comme dans la preuve de la \sref{proposition}{topologiepluspropuniv},
  on dispose de $h_Y:Y\ra Z$ qui complète le diagramme commutatif~:
  \begin{center}
    \begin{tikzcd}[column sep = small, row sep = large]
      X && Z \\
      & Y
      \arrow["h", from=1-1, to=1-3]
      \arrow["g"', from=1-1, to=2-2]
      \arrow["{h_Y}"', from=2-2, to=1-3]
    \end{tikzcd}
  \end{center}
  Comme $X\ra X^+$ vérifie également $(1)$ et $(2)$, on a également un morphisme $h_{X^+}:X^+\ra Z$ qui complète le diagramme~:
  \begin{center}
    \begin{tikzcd}[column sep = small, row sep = large]
      X && Z \\
      & {X^+}
      \arrow["h", from=1-1, to=1-3]
      \arrow["f"', from=1-1, to=2-2]
      \arrow["{h_{X^+}}"', from=2-2, to=1-3]
    \end{tikzcd}
  \end{center}
  Maintenant, on pose $h=f$, $Z=X^+$ pour obtenir un morphisme $f_Y:Y\ra X^+$, et $h=g$, $Z=Y$ pour obtenir un morphisme $g_{X^+}:X^+\ra Y$.
  Alors, $g_{X^+}\circ f_Y$ et $\id_{X^+}$ sont tout deux des morphismes $X^+\ra X^+$ dans $\DEns\uparrow X$. Or, $X^+$ étant simple,
  il admet une tour de Postnikov de fibrations principales. Donc, comme dans la preuve de l'unicité de la \sref{proposition}{topologiepluspropuniv},
  $g_{X^+}\circ f_Y$ et $\id_{X^+}$ sont homotopes sous $X$. De même pour $f_Y\circ g_{X^+}$ et $\id_{Y}$.
  Ainsi, $X^+$ et $Y$ sont homotopes sous $X$, et $g:X\ra Y$ est une construction $+$ pour $C$.
\end{proof}

\subsection{\texorpdfstring{La construction $+$ de la $K$-théorie}{La construction + de la K-théorie}}

Dans cette section, nous verrons certains groupes comme des ensembles simpliciaux via \sref{le foncteur nerf}{definitionnerf}.

\begin{defi}
  Soit $A$ un anneau. On note $f_A:\GL{}{A}\ra\GL{}{A}^+$ une construction $+$ pour $\EGL{}{A}$ (\sref{définition}{definitionplustopologie}).
  Pour $n\geq 1$, on pose~:
  $$K_n(A):=\pi_{n}(\GL{}{A},f_A*)$$
  où $*\in \GL{}{A}_0$ est l'unique point.

  Si $u:A\ra B$ est un morphisme d'anneau, par la \sref{proposition}{topologiepluspropuniv}, il existe $u^*$ complétant le carré~:
  \begin{center}
    % https://q.uiver.app/?q=WzAsNCxbMCwwLCJcXEdMe317QX0iXSxbMSwwLCJcXEdMe317Qn0iXSxbMCwxLCJcXEdMe317QX1eKyJdLFsxLDEsIlxcR0x7fXtCfV4rIl0sWzAsMSwiXFxHTHt9e3V9Il0sWzIsMywidV8qIl0sWzAsMiwiZl9BIiwyXSxbMSwzLCJmX0IiXV0=
    \begin{tikzcd}[column sep = large, row sep = large]
    	{\GL{}{A}} & {\GL{}{B}} \\
    	{\GL{}{A}^+} & {\GL{}{B}^+}
    	\arrow["{\GL{}{u}}", from=1-1, to=1-2]
    	\arrow["{u^*}", from=2-1, to=2-2]
    	\arrow["{f_A}"', from=1-1, to=2-1]
    	\arrow["{f_B}", from=1-2, to=2-2]
    \end{tikzcd}
  \end{center}
  qui induit des morphismes de groupe $u^*:K_n(A)\ra K_n(B)$ pour $n\geq 1$.
\end{defi}

\begin{rem}
  Cette définition définit les groupes $K_n(A)$, et pas seulement à conjugaison près. En effet, l'unicité de la construction $+$ donnée par le
  \sref{corollaire}{uniciteplustopologie} est à homotopie sous $\GL{}{A}$ près. De même pour les morphismes $u^*$.
\end{rem}

Nous auront besoin, pour démontrer le \sref{théorème $+=Q$ }{theoremeplusegalQ}, de montrer que $\GL{}{A}^+$ est un espace simple.
C'est l'objet de la fin de cette sous-section.

\begin{lem}\label{lemmeplussimple}
  Soit $G$ un groupe tel que $[G,G]$ soit parfait. On note $f:G\hookrightarrow G^+$ une construction $+$ pour $G$ telle que
  $G^+$ soit un complexe de Kan. On choisit $\gamma\in [G,G]$, et on note $c_\gamma$ la conjugaison par $\gamma$ dans $G$.
  Alors il existe une homotopie \textbf{pointée} $H:G\times\Delta^1\ra G^+$ entre $f$ et $f\circ c_\gamma$.
\end{lem}

\begin{proof}
  On remarque d'abord que $\gamma$ induit une transformation naturelle entre $\id_G$ et $c_\gamma$~:
  $$h:G\times(0\ra 1)\ra G,\;*\overset{\gamma}{\ra} *$$
  Mais $h$ n'est pas pointée (cela fera toute la différence dans la \sref{proposition}{GLAplussimple}).
  Comme $f_*[\gamma]=0$, on dispose de $\tau:\Delta^2\ra G^+$ tel que~:
  \begin{center}
    \begin{tikzcd}[row sep = 0.50em, column sep = 1em]
      & 1 \\
      & {\tau} \\[-0.30em]
      0 && 2
      \arrow["{s_0(*)}", from=3-1, to=1-2]
      \arrow["{s_0(*)}", from=1-2, to=3-3]
      \arrow["\gamma"', from=3-1, to=3-3]
    \end{tikzcd}
  \end{center}
  Maintenant, on peut compléter le diagramme~:

  \begin{minipage}{0.70\textwidth}
    \begin{center}
      % https://q.uiver.app/?q=WzAsMyxbMCwwLCJHXFx0aW1lc1xcTGFtYmRhXzJeMlxcY3VwXFx7KlxcfVxcdGltZXNcXERlbHRhXjIiXSxbMSwwLCJHXisiXSxbMCwxLCJHXFx0aW1lc1xcRGVsdGFeMiJdLFswLDIsIiIsMCx7InN0eWxlIjp7InRhaWwiOnsibmFtZSI6Imhvb2siLCJzaWRlIjoidG9wIn19fV0sWzAsMSwiKHNfMChmKSxmXFxjaXJjIGgsLSlcXGN1cCBcXHRhdSJdLFsyLDEsIlxcZXhpc3RzIFxcdGhldGEiLDIseyJzdHlsZSI6eyJib2R5Ijp7Im5hbWUiOiJkYXNoZWQifX19XV0=
      \begin{tikzcd}[column sep = 6em]
        {G\times\Lambda_0^2\cup\{*\}\times\Delta^2} & {G^+} \\
        {G\times\Delta^2}
        \arrow[hook, from=1-1, to=2-1]
        \arrow["{(-,f\circ h,s_0(f))\cup \tau}", from=1-1, to=1-2]
        \arrow["{\exists \theta}"', dashed, from=2-1, to=1-2]
      \end{tikzcd}
    \end{center}
  \end{minipage}\hfill
  \begin{minipage}{0.26\textwidth}
    \begin{center}
      % https://q.uiver.app/?q=WzAsNCxbMSwxLCJcXGV4aXN0c1xcdGhldGEiXSxbMSwwLCIxIl0sWzAsMiwiMCJdLFsyLDIsIjIiXSxbMiwzLCJmXFxjaXJjIGgiLDJdLFsxLDMsInNfMChmKSJdLFsyLDEsIiIsMix7InN0eWxlIjp7ImJvZHkiOnsibmFtZSI6ImRhc2hlZCJ9fX1dXQ==
      \begin{tikzcd}[row sep = 0.50em, column sep = 1em]
        & 1 \\
        & \exists\theta \\[-0.30em]
        0 && 2
        \arrow["{f\circ h}"', from=3-1, to=3-3]
        \arrow["{s_0(f)}", from=3-1, to=1-2]
        \arrow["H", dashed, from=1-2, to=3-3]
      \end{tikzcd}
    \end{center}
  \end{minipage}

  Alors $H:=\theta\circ (\id_G\times d^0)$ convient.
\end{proof}

\begin{prop}\label{GLAplussimple}
  Soit $A$ un anneau. Alors $\GL{}{A}^+$ est simple.
\end{prop}

\begin{proof}
  \begin{description}
    \item[(A)] On commence par construire une application~:
    $$\mu^+: \GL{}{A}^+\times\GL{}{A}^+\ra\GL{}{A}^+$$
    Pour cela,
    on pose $\mu:\GL{}{A}\times\GL{}{A}\ra \GL{}{A}$ l'application $(M,N)\mapsto P$, avec $P_{2n+1,2m+1}=M_{n,m}$
    $P_{2n,2m}=N_{n,m}$ et $P_{i,j}=0$ sinon. Or, en utilisant la \sref{proposition}{propositionconstructionplusproduit},
    on remarque que~:
    $$\GL{}{A}\times \GL{}{A}\ra \GL{}{A}^+\times\GL{}{A}^+$$
    est une construction $+$. On peut donc étendre $\mu$ en $\mu^+:\GL{}{A}^+\times\GL{}{A}^+\ra\GL{}{A}^+$, comme souhaité.
    \item[(B)] De la chaîne de cofibrations~:
    $$\GL{1}{A}\hookrightarrow \GL{2}{A}\hookrightarrow\dotsb\hookrightarrow\GL{n}{A}\hookrightarrow\dotsb$$
    On peut déduire, par la \sref{proposition}{topologiepluspropuniv}, une chaîne de cofibrations entre complexes de Kan~:
    $$\GL{1}{A}^+\hookrightarrow \GL{2}{A}^+\hookrightarrow\dotsb\hookrightarrow\GL{n}{A}^+\hookrightarrow\dotsb$$
    On note $G$ sa colimite. Alors par fonctorialité des colimites, on a une application $u:\GL{}{A}\ra G$.
    De plus~:
    $$\pi_1(\GL{}{A})\ra \pi_1(G)=\GL{}{A}\ra\colim_{n}\pi_1(\GL{n}{A}^+)$$
    Et~:
    $$H_*(\GL{}{A},u^*L)\ra H_*(G,L)=\colim_{n}(H_*(\GL{n}{A},u^*L)\ra H_*(\GL{n}{A}^+,L))$$
    Donc $u:\GL{}{A}\ra G$ est une construction $+$ pour $[\GL{}{A},\GL{}{A}]$. Ainsi, on peut supposer que~:
    $$\GL{}{A}^+ =\bigcup_n\GL{n}{A}^+$$
    \item[(C)] Fixons $n\geq 1$. On note $\mu_n:\GL{n}{A}\times\GL{n}{A}\ra\GL{}{A}^+$ la restriction de $\mu$,
    et $i_n:\GL{n}{A}\ra\GL{}{A}^+$ l'application naturelle. Alors, $\mu_n(*,-)$ et $i_n$ sont conjugués par une permutation $\gamma\in \EGL{2n}{A}$.
    Ainsi, on dispose, par le \sref{lemme}{lemmeplussimple}, d'une homotopie pointée $h:\GL{}{A}\times\Delta^1\ra \GL{}{A}^+$,
    telle que sa restriction $H:\GL{n}{A}\times\Delta^1\ra \GL{}{A}^+$ soit une homotopie pointée entre $\mu_n(*,-)$ et $i_n$. Or, l'application~:
    $$\GL{n}{A}\times\Delta^1\cup\GL{n}{A}^+\times\partial\Delta^1\ra\GL{n}{A}^+\times\Delta^1$$
    est un isomorphisme sur tous les groupes d'homologie (voir la fin de la démonstration de la \sref{proposition}{topologiepluspropuniv}).
    Donc on peut étendre $H$ en~:
    $$H_n^+:\GL{n}{A}^+\times\Delta^1\ra \GL{}{A}^+$$
    homotopie entre $\mu_n^+(*,-)$ et $i_n^+$ (extensions données par la \sref{proposition}{topologiepluspropuniv}). On pourrait faire de même
    avec $\mu_n^+(-,*)$ et $i_n^+$.
    \item[(D)] Maintenant, nous fixons $m\geq 1$, $\eta\in \pi_1(\GL{}{A}^+,*)$ et $\delta\in\pi_m(\GL{}{A}^+,*)$. Par (B), il existe
    $n\geq 1$ tel que $\delta$ et $\eta$ soient supportés dans $\GL{n}{A}^+$. Alors~:
    $$\mu^+(\eta,\delta):\Delta^1\times\Delta^m\ra \GL{}{A}^+,\;\mu^+(*,-)\simeq i_n^+\;\mathrm{rel}\: *\text{ et }\mu^+(-,*)\simeq i_n^+\;\mathrm{rel}\: *\text{ (C)}$$
    induisent une homotopie entre $\eta\cdot\delta$ (action) et $\mu(\eta(1),\delta)=\mu(*,\delta)$. Or, $\mu(*,\delta)\simeq \delta\;\mathrm{rel}\: *$.
    Donc $[\eta\cdot\delta]=[\delta]$.
  \end{description}
\end{proof}

\section{\texorpdfstring{Le théorème "$+=Q$"}{Le théorème "+=Q"}}\label{sectionplusegalQ}

Cette section reprend la démonstration du théorème $+=Q$ donnée dans \cite{Gray}.

\subsection{Catégories monoïdales, actions et localisations}

Dans cette sous-section, nous introduisons la notion de catégorie monoïdale, et d'action sur une catégorie.

\begin{defi}
  Un catégorie monoïdale symétrique est une catégorie $S$ munie d'un foncteur~:
  $$+:S\times S\lra S$$
  ainsi que d'un objet $0$, et d'isomorphismes naturels~:
  \begin{description}
    \item[(associateur)] $\alpha_{A,B,C}:A+ (B+ C)\simeq (A+ B)+ C$~;
    \item[(identités)]  $\eta_A:A+ 0\simeq A$ et $\epsilon_A:0+A\simeq A$~;
    \item[(tressage)] $\beta_{A,B}:A+B\simeq B+A$ tel que $\beta_{A,B}\beta_{B,A}=\id_{B+A}$~;
  \end{description}
  tels que les diagrammes suivants commutent~:
  \begin{center}
    % https://q.uiver.app/?q=WzAsNSxbMCwwLCJBKyhCKyhDK0QpKSJdLFsxLDAsIkErKChCK0MpK0QpIl0sWzIsMCwiKEErKEIrQykpK0QiXSxbMiwxLCIoKEErQikrQykrRCJdLFswLDEsIihBK0IpKyhDK0QpIl0sWzAsNCwiXFxzaW0iXSxbMCwxLCJcXHNpbSIsMl0sWzEsMiwiXFxzaW0iLDJdLFs0LDMsIlxcc2ltIl0sWzIsMywiXFxzaW0iLDJdXQ==
    \begin{tikzcd}
    	{A+(B+(C+D))} & {A+((B+C)+D)} & {(A+(B+C))+D} \\
    	{(A+B)+(C+D)} && {((A+B)+C)+D}
    	\arrow["\sim" sloped, from=1-1, to=2-1]
    	\arrow["\sim"', from=1-1, to=1-2]
    	\arrow["\sim"', from=1-2, to=1-3]
    	\arrow["\sim", from=2-1, to=2-3]
    	\arrow["\sim"' sloped, from=1-3, to=2-3]
    \end{tikzcd}
  \end{center}
  \begin{center}
    % https://q.uiver.app/?q=WzAsMyxbMCwwLCJ4KygxK3kpIl0sWzIsMCwiKHgrMSkreSJdLFsxLDEsIngreSJdLFswLDEsIlxcc2ltIl0sWzAsMiwiXFxzaW0iLDJdLFsxLDIsIlxcc2ltIl1d
    \begin{tikzcd}
    	{A+(0+B)} && {(A+0)+B} \\
    	& {A+B}
    	\arrow["\sim", from=1-1, to=1-3]
    	\arrow["\sim"' sloped, from=1-1, to=2-2]
    	\arrow["\sim" sloped, from=1-3, to=2-2]
    \end{tikzcd}
  \end{center}
  \begin{center}
    % https://q.uiver.app/?q=WzAsNixbMCwwLCIoQStCKStDIl0sWzEsMCwiQysoQStCKSJdLFsyLDAsIihDK0EpK0IiXSxbMiwxLCJCKyhDK0EpIl0sWzAsMSwiKEIrQSkrQyJdLFsxLDEsIkIrKEErQykiXSxbMCwxLCJcXHNpbSJdLFsxLDIsIlxcc2ltIl0sWzIsMywiXFxzaW0iXSxbNSwzLCJcXHNpbSIsMl0sWzQsNSwiXFxzaW0iLDJdLFswLDQsIlxcc2ltIiwyXV0=
    \begin{tikzcd}
    	{(A+B)+C} & {C+(A+B)} & {(C+A)+B} \\
    	{(B+A)+C} & {B+(A+C)} & {B+(C+A)}
    	\arrow["\sim", from=1-1, to=1-2]
    	\arrow["\sim", from=1-2, to=1-3]
    	\arrow["\sim" sloped, from=1-3, to=2-3]
    	\arrow["\sim"', from=2-2, to=2-3]
    	\arrow["\sim"', from=2-1, to=2-2]
    	\arrow["\sim"' sloped, from=1-1, to=2-1]
    \end{tikzcd}
  \end{center}
\end{defi}

\begin{rem}
  Les diagrammes commutatifs de la définition permettent de montrer une forme d'unicité à unique isomorphisme près sur les formules formelles
  composées de $+$,$()$ et $0$~: le théorème de cohérence pour les catégories monoïdales. Pour un énoncé précis et une démonstration,
  voir \cite[Chp. VII]{Macl}.
\end{rem}

\begin{defi}
  Un foncteur monoïdal $f:S\lra T$ entre deux catégories monoïdales symétriques est la donnée d'un foncteur~:
  $$f:S\lra T$$
  ainsi que d'un isomorphisme naturel $f(A+B)\simeq fA+fB$ et d'un isomorphisme $f0\sim 0$ tels que les diagrammes suivants commutent~:
  \begin{center}
    % https://q.uiver.app/?q=WzAsNixbMCwwLCJmKChBK0IpK0MpIl0sWzEsMCwiZihBK0IpK2ZDIl0sWzIsMCwiKGZBK2ZCKStmQyJdLFsyLDEsImZBKyhmQitmQykiXSxbMSwxLCJmQStmKEIrQykiXSxbMCwxLCJmKEErKEIrQykpIl0sWzAsNSwiXFxzaW0iXSxbMCwxLCJcXHNpbSIsMl0sWzEsMiwiXFxzaW0iLDJdLFsyLDMsIlxcc2ltIiwyXSxbNSw0LCJcXHNpbSJdLFs0LDMsIlxcc2ltIl1d
    \begin{tikzcd}
    	{f((A+B)+C)} & {f(A+B)+fC} & {(fA+fB)+fC} \\
    	{f(A+(B+C))} & {fA+f(B+C)} & {fA+(fB+fC)}
    	\arrow["\sim" sloped, from=1-1, to=2-1]
    	\arrow["\sim"', from=1-1, to=1-2]
    	\arrow["\sim"', from=1-2, to=1-3]
    	\arrow["\sim"' sloped, from=1-3, to=2-3]
    	\arrow["\sim", from=2-1, to=2-2]
    	\arrow["\sim", from=2-2, to=2-3]
    \end{tikzcd}
  \end{center}
  \begin{center}
    % https://q.uiver.app/?q=WzAsNCxbMCwwLCJmKDArQSkiXSxbMSwwLCJmMCtmQSJdLFsxLDEsIjArZkEiXSxbMCwxLCJmQSJdLFswLDEsIlxcc2ltIl0sWzAsMywiXFxzaW0iLDJdLFszLDIsIlxcc2ltIiwyXSxbMSwyLCJcXHNpbSJdXQ==
    \begin{tikzcd}
      {f(0+A)} & {f0+fA} \\
      fA & {0+fA}
      \arrow["\sim", from=1-1, to=1-2]
      \arrow["\sim"' sloped, from=1-1, to=2-1]
      \arrow["\sim"', from=2-1, to=2-2]
      \arrow["\sim" sloped, from=1-2, to=2-2]
    \end{tikzcd}
  \end{center}
  \begin{center}
    % https://q.uiver.app/?q=WzAsNCxbMCwwLCJmKEErQikiXSxbMSwwLCJmKEIrQSkiXSxbMSwxLCJmQitmQSJdLFswLDEsImZBK2ZCIl0sWzAsMSwiXFxzaW0iXSxbMSwyLCJcXHNpbSJdLFszLDIsIlxcc2ltIiwyXSxbMCwzLCJcXHNpbSIsMl1d
    \begin{tikzcd}
      {f(A+B)} & {f(B+A)} \\
      {fA+fB} & {fB+fA}
      \arrow["\sim", from=1-1, to=1-2]
      \arrow["\sim" sloped, from=1-2, to=2-2]
      \arrow["\sim"', from=2-1, to=2-2]
      \arrow["\sim"' sloped, from=1-1, to=2-1]
    \end{tikzcd}
  \end{center}
\end{defi}

\begin{defi}
  Une action à gauche d'une catégorie monoïdale symétrique $S$ sur une catégorie $X$ est la donnée d'un foncteur~:
  $$+:S\times X\ra X$$
  et de deux isomorphismes naturels $A+(B+F)\simeq (A+B)+F$ et $0+F\simeq F$ (pour $A$, $B$ dans $S$ et $F$ dans $X$),
  tels que les diagrammes suivants commutent (avec $F$ dans $X$, les autres dans $S$)~:
  \begin{center}
    \begin{tikzcd}
    	{A+(B+(C+F))} & {A+((B+C)+F)} & {(A+(B+C))+F} \\
    	{(A+B)+(C+F)} && {((A+B)+C)+F}
    	\arrow["\sim" sloped, from=1-1, to=2-1]
    	\arrow["\sim"', from=1-1, to=1-2]
    	\arrow["\sim"', from=1-2, to=1-3]
    	\arrow["\sim", from=2-1, to=2-3]
    	\arrow["\sim"' sloped, from=1-3, to=2-3]
    \end{tikzcd}
  \end{center}
  \begin{center}
    \begin{tikzcd}
    	{A+(0+F)} && {(A+0)+F} \\
    	& {A+F}
    	\arrow["\sim", from=1-1, to=1-3]
    	\arrow["\sim"' sloped, from=1-1, to=2-2]
    	\arrow["\sim" sloped, from=1-3, to=2-2]
    \end{tikzcd}
  \end{center}
\end{defi}

\begin{ex}
  Par exemple, $+:S\times S\ra S$ induit une action naturelle de $S$ sur $S$.
\end{ex}

\begin{defi}
  Un foncteur $g:X\lra Y$ entre deux catégories munies d'une $S$-action préserve l'action s'il existe un isomorphisme naturel~:
  $A+gF\simeq g(A+F)$ tel que les diagrammes suivants commutent~:
  \begin{center}
    % https://q.uiver.app/?q=WzAsNSxbMCwwLCIoQStCKStnRiJdLFsxLDAsImcoKEErQikrRikiXSxbMiwwLCJnKEErKEIrRikpIl0sWzIsMSwiQStnKEIrRikiXSxbMCwxLCJBKyhCK2dGKSJdLFs0LDMsIlxcc2ltIl0sWzIsMywiXFxzaW0iLDJdLFsxLDIsIlxcc2ltIiwyXSxbMCwxLCJcXHNpbSIsMl0sWzAsNCwiXFxzaW0iXV0=
    \begin{tikzcd}
      {(A+B)+gF} & {g((A+B)+F)} & {g(A+(B+F))} \\
      {A+(B+gF)} && {A+g(B+F)}
      \arrow["\sim", from=2-1, to=2-3]
      \arrow["\sim"' sloped, from=1-3, to=2-3]
      \arrow["\sim"', from=1-2, to=1-3]
      \arrow["\sim"', from=1-1, to=1-2]
      \arrow["\sim" sloped, from=1-1, to=2-1]
    \end{tikzcd}
  \end{center}
  \begin{center}
    % https://q.uiver.app/?q=WzAsMyxbMCwwLCIwK2dGIl0sWzEsMSwiZ0YiXSxbMiwwLCJnKDArRikiXSxbMCwxLCJcXHNpbSJdLFsyLDEsIlxcc2ltIiwyXSxbMCwyLCJcXHNpbSIsMl1d
    \begin{tikzcd}
      {0+gF} && {g(0+F)} \\
      & gF
      \arrow["\sim" sloped, from=1-1, to=2-2]
      \arrow["\sim"' sloped, from=1-3, to=2-2]
      \arrow["\sim"', from=1-1, to=1-3]
    \end{tikzcd}
  \end{center}
\end{defi}

Nous allons maintenant introduire la notion d'action inversible et de localisation par une action.
Mais avant de la donner dans la cadre des catégories, voyons ce que sont ces définitions dans le cadre des actions
de monoïdes symétriques sur des ensemble.

Ce cas est inclus dans celui des catégories. En effet, on peut voir un ensemble comme une catégorie
discrète (ie. les seuls morphismes sont les identités). Une structure de catégorie monoïdale sur un ensemble est alors 
équivalente à une structure de monoïde.

\begin{propdefi}
  Soit $S$ un monoïde symétrique agissant sur un ensemble $X$. On note $S^{-1}X$ le quotient de $S\times X$
  par l'action de $S$ sur les deux coordonnées~:
  $$S^{-1}X=S\times X/(t,x)\sim (s+t,s+x)$$
  L'ensemble $S^{-1}X$ admet une $S$-action donnée par $s+(t,x)=(t,s+x)$, et on a un morphisme de $S$-ensembles~:
  \[
  \begin{array}{lcl}
    X & \lra    & S^{-1}X \\
    x & \mapsto & (0,x)
  \end{array}
  \]
  L'action d'un monoïde commutatif $T$ sur un ensemble $Y$ est dite inversible si $s+(-):Y\ra Y$ est un isomorphisme pour tout $s$ dans $T$.
  
  Alors l'action de $S$ sur $S^{-1}X$ est inversible et l'action de $S$ sur $X$ est inversible si et seulement si $X\ra S^{-1}X$
  est un isomorphisme.
\end{propdefi}

\begin{proof}
  L'action de $S$ sur la deuxième coordonnée de $S\times X$ passe au quotient car $u+(s+t,s+x) = (s+t,s+(u+x))$.
  Un inverse à $s+(-)$ dans $S^{-1}X$ est donné par $(t,x)\mapsto (s+t,x)$. La dernière affirmation est immédiate.
\end{proof}

\begin{defi}
  Soit $S$ une catégorie monoïdale symétrique agissant sur un ensemble $X$. On dit que l'action est inversible si pour tout $A$ dans $S$,
  le foncteur~:
  \[
  \begin{array}{lcl}
    X & \lra    & X \\
    F & \mapsto & A+F
  \end{array}
  \]
  est une équivalence d'homotopie (où la catégorie $X$ est vue comme un ensemble simplicial via le \sref{foncteur nerf}{definitionnerf}).
\end{defi}

\begin{defi}
  Soit $S$ catégorie symétrique monoïdale agissant sur une catégorie $X$. On définit $\brak{S}{X}$ comme la catégorie dont les~:
  \begin{description}
    \item[$\bullet$] objets sont les objets de X~;
    \item[$\bullet$] morphismes de $F$ dans $G$ sont les quadruplets $(F,G,A,A+F\ra G)$ avec $A$ dans $S$, à isomorphismes près.
                    Un isomorphisme  entre $(F,G,A,A+F\ra G)$ et $(F,G,B,B+F\ra G)$ est la donnée d'un isomorphisme $\phi:A\simeq B$ tel que le diagramme~:
    \begin{center}
      % https://q.uiver.app/?q=WzAsMyxbMCwwLCJBK0YiXSxbMSwxLCJHIl0sWzIsMCwiQitGIl0sWzAsMV0sWzIsMV0sWzAsMiwiXFxzaW0iLDJdXQ==
      \begin{tikzcd}
        {A+F} && {B+F} \\
        & G
        \arrow[from=1-1, to=2-2]
        \arrow[from=1-3, to=2-2]
        \arrow["\sim"', "\phi+\id_F", from=1-1, to=1-3]
      \end{tikzcd}
    \end{center}
    soit commutatif.
    \item[$\bullet$] les compositions sont comme suit. La composée de $(F,G,A,A+F\ra G)$ et $(G,H,B,B+G\ra H)$ est $(F,H,B+A,B+A+F\ra B+G\ra H)$. 
  \end{description}
\end{defi}

\begin{rem}
  Dans la suite on notera parfois $(A,A+F\ra G)$ pour $(F,G,A,A+F\ra G)$.
\end{rem}

\begin{propdefi}
  Soit $S$ une catégorie symétrique monoïdale agissant sur une catégorie $X$. On pose $S^{-1}X:=\brak{S}{S\times X}$,
  où $S$ agit sur les deux facteurs de $S\times X$.

  Alors $S$ agit sur $S^{-1}X$ par $A+(B,F):=(B,A+F)$, l'action est inversible et on dispose d'un foncteur préservant l'action~:
  \[
  \begin{array}{lcl}
    X & \lra    & S^{-1}X \\
    F & \mapsto & (0,F)
  \end{array}
  \]
\end{propdefi}

\begin{proof}
  L'action de $S$ sur les morphismes est donnée comme suit. Soit $u:\alpha\ra\beta$ morphisme dans $S$ et $v=((A,F),(B,G),C,(C+A,C+F)\ra (B,G))$
  morphisme dans $S^{-1}X$. Alors~:
  \begin{align*}
  u+v:=(&(\alpha+A,\alpha+F),(\beta+B,\beta+G),C,\\
          &(\alpha+C+A,\alpha+C+F)\ra (\beta+C+A,\beta+C+F)\ra (\beta+B,\beta +G))
  \end{align*}
  On vérifie facilement que les bons diagrammes commutent.

  Un inverse à homotopie près à $s_A:A+(-)$ dans $S^{-1}X$ est donné par $i_A:(B,F)\ra (A+B,F)$. En effet la collection de morphismes~:
  $$((B,F),(A+B,A+F),A,(A+B,A+F)\overset{=}{\ra}(A+B,A+F))$$
  induit une transformation naturelle entre $\id_{S^{-1}X}$ et $i_A\circ s_A=s_A\circ i_A$.
\end{proof}

On voit que le quotient par $S$ dans le cas des ensembles est ici remplacé par la construction $\brak{S}{(-)}$.
Montrons par un petit calcul que ces deux construction coïncident à homotopie près dans le cas des ensembles. Dans ce calcul,
on notera $S^{-1}X$ pour le quotient.

Dans ce cadre, les morphismes dans $\brak{S}{S\times X}$ sont comme suit. Un morphisme $(t,x)\ra (s,y)$ est la donnée d'un $u\in S$
tel que $(u+t,u+x)=(s,y)$. On peut donc identifier naturellement $\pi_0(\brak{S}{S\times X})$ à $S^{-1}X$.
On a donc un foncteur naturel $p_X:\brak{S}{S\times X}\ra S^{-1}X$. On vérifie aisément qu'il préserve l'action de $S$.

Montrons maintenant que $p_X$ est une équivalence d'homotopie. Pour cela, soit $(t,x)$ dans $S\times X$, on note $[(t,x)]$ sa classe dans $S^{-1}X$.
On a alors $\brak{S}{S\times X}\downarrow [(t,x)]=p_X^{-1}[(t,x)]$. D'après le \sref{théorème A}{theoremeA}, il suffit de montrer que
$p_X^{-1}[(t,x)]$ est contractile. Pour cela nous allons montrer que c'est une catégorie filtrante. En effet, si $u,v:(s,y) \ra (r,z)$
on a $s+u=s+v:(s,y)\ra (s+r,s+z)$~; si $(s,y)$ et $(r,z)$ sont dans $p_X^{-1}[(t,x)]$ on a $(s,y)\sim\dotsb\sim (r,z)$, où $\sim$ est induit
par l'action de $S$~; alors si on note $\sigma$ la somme de toutes les premières coordonnées qui apparaissent dans la chaîne,
on a $\sigma +s=\sigma + r$ et $\sigma+y=\sigma+z$.

Revenons maintenant au cadre des catégories. Dans la majorité de la suite, nous nous restreindrons au cas où $S$ est un groupoïde.

\begin{prop}\label{groupoideSScontractile}
  Soit $S$ une catégorie monoïdale symétrique. Si $S$ est un groupoïde, alors $0$ est initial dans $\brak{S}{S}$, et
  en particulier, $\brak{S}{S}$ est contractile.
\end{prop}

\begin{proof}
  Soit $A$ dans $S$. Un morphisme $0\ra A$ dans $\brak{S}{S}$ est un $B$ et un isomorphisme $B+0\simeq A$, à isomorphisme près.
  Or, si $C$, $C+0\simeq A$ induit aussi un morphisme, il existe un unique isomorphisme $C\simeq B$.
  Donc il existe au plus un unique morphisme $0\ra A$. Or $A+0\simeq A$ en induit un.
\end{proof}

\begin{defi}
  Soit $S$ une catégorie monoïdale symétrique agissant sur $X$ catégorie. On définit la première projection~:
  \[
  \begin{array}{llcl}
    \rho:&S^{-1}X                  &\ra     & \brak{S}{S} \\
         &(B,F)                    &\mapsto & B \\
         &(A,(A+B,A+F)\ra (B',F')) &\mapsto & (A,A+B\ra B') 
  \end{array}
  \]
\end{defi}

\begin{lem}\label{lemmeactioncategorie}
  Soit $S$ une catégorie monoïdale symétrique agissant sur $X$ catégorie. Alors, sous les deux conditions suivantes $\rho$ est cofibrée~:
  \begin{description}
    \item[$(1)$] tous les morphismes de $S$ sont des monomorphismes~;
    \item[$(2)$] les foncteurs de translation $A+(-):S\ra S$ sont fidèles.
  \end{description}
\end{lem}

\begin{proof}
  On se donne un morphisme $(C,C+B\ra B'',C+F\ra F'')$ dans $S^{-1}X$ et une factorisation de la première coordonnée en $(A,A+B\ra B')$
  et $(A',A'+B'\ra B'')$.
  On a alors le diagramme suivant~:
  \begin{center}
    % https://q.uiver.app/?q=WzAsNixbMCwwLCIoQixGKSJdLFswLDEsIkIiXSxbMSwwLCIoQicsRitBKSJdLFsyLDAsIihCJycsRicnKSJdLFsyLDEsIkInJyJdLFsxLDEsIkInIl0sWzAsMiwiKEEsQStCXFxyYSBCJyxcXGlkX3tGK0F9KSIsMl0sWzIsMywiIiwwLHsic3R5bGUiOnsiYm9keSI6eyJuYW1lIjoiZGFzaGVkIn19fV0sWzMsNCwiIiwwLHsic3R5bGUiOnsidGFpbCI6eyJuYW1lIjoibWFwcyB0byJ9fX1dLFsyLDUsIiIsMCx7InN0eWxlIjp7InRhaWwiOnsibmFtZSI6Im1hcHMgdG8ifX19XSxbMCwxLCIiLDIseyJzdHlsZSI6eyJ0YWlsIjp7Im5hbWUiOiJtYXBzIHRvIn19fV0sWzEsNSwiKEEsQStCXFxyYSBCJykiLDJdLFs1LDQsIihBJyxBJytCJ1xccmEgQicnKSIsMl0sWzAsMywiKEMsQitDXFxyYSBCJycsRitDXFxyYSBGJycpIiwwLHsiY3VydmUiOi0yfV1d
    \begin{tikzcd}[column sep = 15ex]
      {(B,F)} & {(B',A+F)} & {(B'',F'')} \\
      B & {B'} & {B''}
      \arrow["{(A,A+B\ra B',\id_{A+F})}"', from=1-1, to=1-2]
      \arrow[dashed, from=1-2, to=1-3]
      \arrow[maps to, from=1-3, to=2-3]
      \arrow[maps to, from=1-2, to=2-2]
      \arrow[maps to, from=1-1, to=2-1]
      \arrow["{(A,A+B\ra B')}"', from=2-1, to=2-2]
      \arrow["{(A',A'+B'\ra B'')}"', from=2-2, to=2-3]
      \arrow["{(C,C+B\ra B'',C+F\ra F'')}", bend left =12pt, from=1-1, to=1-3]
    \end{tikzcd}
  \end{center}
  On cherche à trouver une flèche en pointillés qui fasse commuter le diagramme et à montrer son unicité.
  On dispose d'au moins un triangle commutatif dans $S$~:
  \begin{center}
    % https://q.uiver.app/?q=WzAsMyxbMCwwLCJBJytBK0IiXSxbMiwwLCJDK0IiXSxbMSwxLCJCJyciXSxbMCwxLCJcXHNpbSJdLFsxLDJdLFswLDJdXQ==
    \begin{tikzcd}
      {A'+A+B} && {C+B} \\
      & {B''}
      \arrow["\sim", from=1-1, to=1-3]
      \arrow[from=1-3, to=2-2]
      \arrow[from=1-1, to=2-2]
    \end{tikzcd}
  \end{center}
  Donc, comme tous les morphismes sont des monomorphismes et les translations sont fidèles,
  il existe un unique isomorphisme $C\simeq A'+A$ qui induise celui du diagramme.
  Ainsi, le morphisme $A'+A+F\simeq C+F\ra F''$ est la flèche en pointillés recherchée.
  L'unicité suit immédiatement.
\end{proof}

\begin{theo}\label{theoremeactioninversible}
  Soit $S$ un groupoïde symétrique monoïdal agissant sur une catégorie $X$. 
  On suppose que les translations $A+(-):S\ra S$ sont fidèles. Alors s'équivalent~:
  \begin{description}
    \item[$(1)$] $X\lra S^{-1}X$ est une équivalence d'homotopie~;
    \item[$(2)$] L'action de $S$ sur $X$ est inversible. 
  \end{description}
\end{theo}

\begin{proof}
  On a, pour chaque $B$, un isomorphisme naturel~:
  \[
  \begin{array}{lcl}
    X &\simeq  & \rho^{-1}B \\
    F &\mapsto & (B,F)
  \end{array}
  \]
  Et alors le changement de cobase $\rho^{-1}B\lra \rho^{-1}B'$ induit par $(A,A+B\ra B')$ s'identifie à la translation par $A+(-)$
  (voir la démonstration du \sref{lemme}{lemmeactioncategorie} ci-dessus).
  Ainsi, si l'action est inversible, le \sref{théorème B}{theoremeB} implique que le diagramme~:
  \begin{center}
    % https://q.uiver.app/?q=WzAsNCxbMCwwLCJYPVxccmhvXnstMX0wIl0sWzEsMCwiU157LTF9WCJdLFsxLDEsIlxcYnJha3tTfXtTfSJdLFswLDEsIioiXSxbMywyLCIwIl0sWzAsM10sWzAsMV0sWzEsMl1d
    \begin{tikzcd}
      {X=\rho^{-1}0} & {S^{-1}X} \\
      {*} & {\brak{S}{S}}
      \arrow["0", from=2-1, to=2-2]
      \arrow[from=1-1, to=2-1]
      \arrow[from=1-1, to=1-2]
      \arrow[from=1-2, to=2-2]
    \end{tikzcd}
  \end{center}
  est homotopiquement cartésien. Or, comme $S$ est un groupoïde, par la \sref{proposition}{groupoideSScontractile}, $\brak{S}{S}$ est contractile.
  Donc $X\lra S^{-1}X$ est une équivalence d'homotopie.

  Réciproquement, si $X\lra S^{-1}X$ est une équivalence d'homotopie, alors les translations sur $X$ s'identifient à homotopie près à celles sur
  $S^{-1}X$. Or, l'action de $S$ sur $S^{-1}X$ est inversible. Cela conclut.
\end{proof}

\subsection{Calculs d'homologie de localisés}
\label{soussectioncalculhomologielocalises}

Le but de cette sous-section est de calculer $H_*(S^{-1}X)$ en fonction de $H_*(X)$ sous certaines hypothèses sur l'action de $S$ sur $X$.

On se fixe, dans cette sous-section, une catégorie monoïdale symétrique $S$ agissant sur une catégorie $X$.

Si $u:s\ra s'$ est un morphisme dans $S$, il induit une transformation naturelle $s+(-)\Rightarrow s'+(-):X\times\Delta^1\ra X$,
et donc une équivalence d'homotopie entre $s+(-)$ et $s'+(-)$. Ainsi, on a une action naturelle de $\pi_0(S)$ sur $X$ vu comme objet de $\Ho{\DEns}$.
L'ensemble $\pi_0(S)$ est naturellement un monoïde symétrique via $+$.

En particulier, $\pi_0(S)$ agit sur $H_*(X)$. De même, $\pi_0(S)$ agit sur $H_*(S^{-1}X)$, et comme l'action de $S$ y est inversible,
$\pi_0(S)$ agit sur $H_*(S^{-1}X)$ par automorphisme.

Donc, le morphisme $X\ra S^{-1}X$ induit un morphisme~:
$$\pi_0(S)^{-1}H_*(X)\lra H_*(S^{-1}X)$$
La localisation désigne ici la localisation du $\Z[\pi_0(S)]$-module $H_*(X)$ par la partie multiplicative $\pi_0(S)$.

On peut maintenant énoncer le résultat principale de la sous-section.

\begin{theo}\label{theoremehomologielocalisation}
  Soit $S$ une groupoïde monoïdal symétrique agissant sur une catégorie $X$, tels que les translations $A+(-)$ soient fidèles pour tout $A$ dans $S$.
  Alors le morphisme construit ci-dessus~:
  $$\pi_0(S)^{-1}H_*(X)\lra H_*(S^{-1}X)$$
  est un isomorphisme.
\end{theo}

\begin{defi}\label{definitionmbar}
  Si $M$ est un $\Z[\pi_0(S)]$-module, on pose~:
  \[
  \begin{array}{llcl}
    \overline{M}:& \brak{S}{S}   & \ra     & \Ab \\
            & B             & \mapsto & M \\
            & (A,A+B\ra B') & \mapsto & [A]+(-)
  \end{array}
  \]
\end{defi}

\begin{proof}[Démonstration du théorème]
  D'après le \sref{lemme}{lemmeactioncategorie} et la démonstration du \sref{théorème}{theoremeactioninversible}~:
  \begin{description}
    \item[$(i)$] $\rho:S^{-1}X\lra \brak{S}{S}$ est cofibrée~;
    \item[$(ii)$] on a une identification naturelle $\rho^{-1}B\simeq X$~;
    \item[$(iii)$] via cette identification, les changements de cobase d'identifient aux translations.   
  \end{description}
  On considère l'ensemble bisimplicial $T_{\bullet,\bullet}$ définit par~:
  \begin{align*}
    T_{pq}:=\{&((A_0,F_0)\ra\dotsb\ra (A_q,F_q),A_q\ra B_0,B_0\ra\dotsb\ra B_p) \\
               &\mid B_i\in \brak{S}{S},\; (A_j,F_j)\in S^{-1}X\}
  \end{align*}
  et le bicomplexe de chaîne associé~:
  $$C_{pq}:=\Z T_{pq}$$
  On a une action naturelle de $S$ sur $T_{pq}$ par~:
  \begin{align*}
    U+&((A_0,F_0)\ra\dotsb\ra (A_q,F_q),A_q\ra B_0,B_0\ra\dotsb\ra B_p) \\
      &:=((A_0,U+F_0)\ra\dotsb\ra (A_q,U+F_q),A_q\ra B_0,B_0\ra\dotsb\ra B_p)
  \end{align*}
  Cette action passe naturellement à $C_{\bullet,\bullet}$.
  
  On filtre $\Tot{C}$ par la seconde coordonnée~:
  $$F_q\Tot{C}_n:=\bigoplus_{i+j=n,j\leq q}C_{ij}$$
  On note $E^r_{pq}$ la suite spectrale de cette filtration, voir \cite[5.4]{Weib2} (attention, les indices sont transposés ici).
  On a~:
  $$E^{0}_{pq}=C_{pq}=\bigoplus_{(A_0,F_0)\ra\dotsb\ra (A_q,F_q)}\Z N(\brak{S}{S}\uparrow A_q)_p$$
  Et $\partial^0:E^{0}_{pq}\ra E^{0}_{(p-1)q}$ s'identifie sur chaque coordonnée à la différentielle de $\Z N(\brak{S}{S}\uparrow B_q)$.
  Donc, on a immédiatement~:
  $$E^1_{pq}=\bigoplus_{(A_0,F_0)\ra\dotsb\ra (A_q,F_q)}  H_p(\brak{S}{S}\uparrow A_q)
  =\begin{cases}
    0&\text{ si }q>0 \\
    \Z N(S^{-1}X)_p&\text{ sinon}
  \end{cases}$$
  On remarque que l'action $S$ induit ici une structure de bicomplexe de $\Z[\pi_0(S)]$-modules sur $E^1_{\bullet\bullet}$. 
  Du calcul de $E^1_{pq}$ on déduit~:
  $$E^2_{pq}=\begin{cases}
    0&\text{ si }q>0 \\
    H_p(S^{-1}X)&\text{ sinon}
  \end{cases}$$
  On a donc un isomorphisme $H_p(\Tot{C})\simeq H_p(S^{-1}X)$. L'action induite de $\pi_0(S)$ sur $H_p(S^{-1}X)$
  est alors celle décrite au début de la sous-section.

  On filtre maintenant $\Tot{C}$ par la première coordonnée~:
  $$F'_p\Tot{C}_n:=\bigoplus_{i+j=n,i\leq p}C_{ij}$$
  On note $E'^r_{pq}$ la suite spectrale associée.
  On a~:
  $$E'^{0}_{pq}=C_{pq}=\bigoplus_{B_0\ra\dotsb\ra B_p}\Z N(\rho\downarrow B_0)_q$$
  Et $\partial'^0:E'^{0}_{pq}\ra E'^{0}_{p(q-1)}$ s'identifie sur chaque coordonnée à la différentielle de $\Z N(\rho\downarrow B_0)$.
  Donc, on a immédiatement~:
  $$E'^1_{pq}=\bigoplus_{B_0\ra\dotsb\ra B_p} H_q(\rho\downarrow B_0)\simeq \bigoplus_{B_0\ra\dotsb\ra B_p} H_q(X)$$
  En effet, comme $\rho$ est cofibrée et par le point $(iii)$ ci-dessus, $\rho\downarrow B\ra \rho^{-1}B\ra X$ est une chaîne fonctorielle d'équivalences
  faibles.
  On remarque ici aussi que l'action $S$ induit une structure de bicomplexe de $\Z[\pi_0(S)]$-modules sur $E'^1_{\bullet\bullet}$.
  On en déduit~:
  $$E'^2_{pq}=H_p(\brak{S}{S};\overline{H_q(X)})$$
  où $\overline{H_q(X)}$ est donné par la construction de la \sref{définition}{definitionmbar}.
  L'action de $\pi_0(S)$ sur $E'^2_{pq}$ est induite par l'action par translation sur $H_q(X)$.
  Comme l'action de $\pi_0(S)$ sur $E^1_{\bullet\bullet}$ et l'action sur $E'^1_{\bullet\bullet}$ proviennent d'une même action
  de $S$ sur $C$, on a convergence d'une suite spectrale de $\Z[\pi_0(S)]$-modules~:
  $$H_p(\brak{S}{S};\overline{H_q(X)})\Longrightarrow H_p(S^{-1}X)$$
  Or, l'action de $\pi_0(S)$ sur $H_p(S^{-1}X)$ est inversible, et la localisation est exacte, et de plus~:
  $$\pi_0(S)^{-1}E'^1_{pq}=\bigoplus_{B_0\ra\dotsb\ra B_p} \pi_0(S)^{-1}H_q(X)$$
  Donc, on a également convergence de la suite spectrale~:
  $$H_p(\brak{S}{S};\overline{\pi_0(S)^{-1}H_q(X)})\Longrightarrow H_p(S^{-1}X)$$
  Or, l'action de $<S,S>$ sur $\overline{\pi_0(S)^{-1}H_q(X)}$ est inversible, et $\brak{S}{S}$ est contractile.
  Donc $H_p(\brak{S}{S};\overline{\pi_0(S)^{-1}H_q(X)})$ est une homologie à coefficients d'un espace contractile.
  Donc~:
  $$H_p(\brak{S}{S};\overline{\pi_0(S)^{-1}H_q(X)})=\begin{cases}
    \pi_0(S)^{-1}H_q(X)&\text{ si }p=0 \\
    0                    &\text{sinon}
  \end{cases}$$
  On a donc un isomorphisme pour $p\geq 0$~:
  $$v:\pi_0(S)^{-1}H_p(X)\simeq H_p(S^{-1}X)$$

  Il reste à montrer que cet isomorphisme est induit par le morphisme~:
  $$\pi_0(S)^{-1}H_p(X)\ra H_p(S^{-1}X)$$
  construit plus haut.
  Pour cela, on remarque que la suite $X\ra S^{-1}X\ra \brak{S}{S}$ est fonctorielle en $S$.
  Or on a un morphisme de catégories monoïdales symétriques $*\ra S$. On a donc un diagramme commutatif~:
  \begin{center}
    % https://q.uiver.app/?q=WzAsNixbMCwwLCJYIl0sWzEsMCwiWCJdLFsyLDAsIioiXSxbMCwxLCJYIl0sWzEsMSwiU157LTF9WCJdLFsyLDEsIlxcYnJha3tTfVN7fSJdLFsyLDUsIjAiXSxbMSw0LCJpIl0sWzEsMiwiZyJdLFs0LDUsImYiXSxbMCwxXSxbMCwzXSxbMyw0XV0=
    \begin{tikzcd}
      X & X & {*} \\
      X & {S^{-1}X} & {\brak{S}S{}}
      \arrow["0", from=1-3, to=2-3]
      \arrow["i", from=1-2, to=2-2]
      \arrow["g", from=1-2, to=1-3]
      \arrow["f", from=2-2, to=2-3]
      \arrow[equal, from=1-1, to=1-2]
      \arrow[equal, from=1-1, to=2-1]
      \arrow[from=2-1, to=2-2]
    \end{tikzcd}
  \end{center}
  Ce diagramme induit un morphisme d'ensemble bisimpliciaux $T(g)\ra T(f)$, et donc un morphisme entre les suites spectrales.
  Nous disposons donc du diagramme commutatif suivant~:
  \begin{center}
    % https://q.uiver.app/?q=WzAsOSxbMCwwLCJIX3EoWCk9SF8wKCo7XFxvdmVybGluZXtIX3EoWCl9KSJdLFsxLDAsIkhfcShcXFRvdHtcXFogVChnKX0pIl0sWzIsMCwiSF9xKFgpIl0sWzIsMSwiSF9xKFNeey0xfVgpIl0sWzEsMSwiSF9xKFxcVG90e1xcWiBUKGYpfSkiXSxbMCwxLCJIXzAoXFxicmFre1N9e1N9O1xcb3ZlcmxpbmV7SF9xKFgpfSkiXSxbMiwyLCJIX3EoU157LTF9WCkiXSxbMSwyLCJIX3EoXFxUb3R7XFxaIFQoZil9KSJdLFswLDIsIihcXHBpXzAoUykpXnstMX1IX3EoWCkiXSxbMiwzLCJIX3EoaSkiXSxbMCw1XSxbMCwxLCJcXHNpbSJdLFsyLDEsIlxcc2ltIiwyXSxbMyw0LCJcXHNpbSIsMl0sWzUsNF0sWzMsNiwiZXF1YWwiXSxbNiw3LCJcXHNpbSJdLFs0LDcsImVxdWFsIiwyXSxbNSw4XSxbOCw3LCJcXHNpbSJdLFsxLDRdLFswLDIsImVxdWFsIiwyLHsiY3VydmUiOi00fV0sWzgsNiwidSIsMCx7ImN1cnZlIjo0fV1d
    \begin{tikzcd}
      {H_q(X)=H_0(*;\overline{H_q(X)})} & {H_q(\Tot{\Z T(g)})} & {H_q(X)} \\
      {H_0(\brak{S}{S};\overline{H_q(X)})} & {H_q(\Tot{\Z T(f)})} & {H_q(S^{-1}X)} \\
      {\pi_0(S)^{-1}H_q(X)} & {H_q(\Tot{\Z T(f)})} & {H_q(S^{-1}X)}
      \arrow["{H_q(i)}", from=1-3, to=2-3]
      \arrow[from=1-1, to=2-1]
      \arrow["\sim", from=1-1, to=1-2]
      \arrow["\sim"', from=1-3, to=1-2]
      \arrow["\sim"', from=2-3, to=2-2]
      \arrow[from=2-1, to=2-2]
      \arrow[equal, from=2-3, to=3-3]
      \arrow["\sim", from=3-3, to=3-2]
      \arrow[equal, from=2-2, to=3-2]
      \arrow[from=2-1, to=3-1]
      \arrow["\sim", from=3-1, to=3-2]
      \arrow[from=1-2, to=2-2]
      \arrow[equal, bend left=24pt, from=1-1, to=1-3]
      \arrow["v", "\sim"', bend right=24pt, from=3-1, to=3-3]
      \arrow["u"', bend right = 80pt,from=1-1, to=3-1]
    \end{tikzcd}
  \end{center}
  où $u$ est la localisation et $v$ est l'isomorphisme induit par la suite spectrale.
  On voit que $v$ est induit par $H_q(i)$, ce qui conclut.
\end{proof}

\subsection{Action sur les fibres}

\begin{rem}
  On se donne $S$ une catégorie monoïdale symétrique agissant trivialement sur une catégorie $Y$, ie. $S\times Y\ra Y=\mathrm{pr}_2$.
  Alors $S^{-1}Y=\brak{S}{Y\times S}=Y\times\brak{S}{S}$. On dispose donc d'un foncteur~:
  $$S^{-1}Y\ra Y$$
  qui est une équivalence d'homotopie si $S$ est un groupoïde.
\end{rem}

\begin{lem}\label{lemmeactionfibre}
  Soit $f:X\ra Y$ un foncteur. On se donne une action de $S$ groupoïde symétrique monoïdal sur $X$ tel que, naturellement en $A$ dans $S$
  et $F$ dans $X$, on ait $f(A+F)=fF$. Alors $S$ agit naturellement sur les fibres de $f$.
  On suppose de plus que~:
  \begin{description}
    \item[$\bullet$] $f$ est cofibré~;
    \item[$\bullet$] pour tout morphisme $v$ dans $Y$, le foncteur changement de cobase $v_*$ est compatible à l'action de $S$ sur les fibres.
  \end{description}
  Alors le foncteur $S^{-1}f:S^{-1}X\ra Y$ donné par $S^{-1}X\ra S^{-1}Y\ra Y$ est cofibré, et la fibre au-dessus de $G$ s'identifie naturellement
  à $S^{-1}(f^{-1}G)$.
\end{lem}

\begin{proof}
  Montrons que $S^{-1}f$ est cofibré. On se donne un morphisme $(C,C+B\ra B'',C+F\ra F'')$ entre $(B,F)$ et $(B'',F'')$,
  et une factorisation $v\circ u$ de $w:fF\ra fF''$ associé. On a alors le diagramme suivant~:
  \begin{center}
    % https://q.uiver.app/?q=WzAsNixbMCwwLCIoQixGKSJdLFsxLDAsIihCLHVfKkYpIl0sWzIsMCwiKEInJyxGJycpIl0sWzIsMSwiZkYnJyJdLFsxLDEsIkcnIl0sWzAsMSwiZkYiXSxbNSw0LCJ1Il0sWzQsMywidiJdLFs1LDMsInciLDIseyJjdXJ2ZSI6Mn1dLFsxLDQsIiIsMix7InN0eWxlIjp7InRhaWwiOnsibmFtZSI6Im1hcHMgdG8ifX19XSxbMiwzLCIiLDIseyJzdHlsZSI6eyJ0YWlsIjp7Im5hbWUiOiJtYXBzIHRvIn19fV0sWzAsNSwiIiwyLHsic3R5bGUiOnsidGFpbCI6eyJuYW1lIjoibWFwcyB0byJ9fX1dLFswLDIsIihDLEMrQlxccmEgQicnLEMrRlxccmEgRicnKSIsMCx7ImN1cnZlIjotMn1dLFsxLDIsIlxcZXhpc3RzICEiLDIseyJzdHlsZSI6eyJib2R5Ijp7Im5hbWUiOiJkYXNoZWQifX19XSxbMCwxLCIoMCwwK0JcXHJhIEIsMCtGXFxyYSB1XypGKSIsMl1d
    \begin{tikzcd}[column sep=12ex]
      {(B,F)} &[6ex] {(B,u_*F)} & {(B'',F'')} \\
      fF & {G'} & {fF''}
      \arrow["u", from=2-1, to=2-2]
      \arrow["v", from=2-2, to=2-3]
      \arrow["w"',bend right=12pt, from=2-1, to=2-3]
      \arrow[maps to, from=1-2, to=2-2]
      \arrow[maps to, from=1-3, to=2-3]
      \arrow[maps to, from=1-1, to=2-1]
      \arrow["{(C,C+B\ra B'',C+F\ra F'')}", bend left=12pt, from=1-1, to=1-3]
      \arrow["{\exists !}"', dashed, from=1-2, to=1-3]
      \arrow["{(0,0+B\ra B,0+F\ra u_*F)}"', from=1-1, to=1-2]
    \end{tikzcd}
  \end{center}
  Il reste à justifier l'existence et l'unicité de la flèche en pointillés. Pour l'existence il suffit de choisir~:
  $$(C,C+B\ra B'',C+u_*F\simeq u_*(C+F)\ra F'')$$
  Pour l'unicité, on suppose que $(C',C'+B\ra B'',C'+u_*F\ra F'')$ convient également. Alors il existe un isomorphisme $\eta:C'\simeq C$
  qui fasse commuter le diagramme~:
  \begin{center}
    % https://q.uiver.app/?q=WzAsMyxbMCwwLCJDK0YiXSxbMiwwLCJDJytGIl0sWzEsMSwiRicnIl0sWzAsMSwiXFxldGErRiJdLFswLDJdLFsxLDJdXQ==
    \begin{tikzcd}[column sep=tiny]
      {C+F} && {C'+F} \\
      & {F''}
      \arrow["{\eta+F}", from=1-1, to=1-3]
      \arrow[from=1-1, to=2-2]
      \arrow[from=1-3, to=2-2]
    \end{tikzcd}
  \end{center}
  Alors, on peut appliquer l'adjonction de $u_*$. Comme $u_*$ est compatible à l'action de $f$, on obtient le diagramme suivant~:
  \begin{center}
    % https://q.uiver.app/?q=WzAsNSxbMCwxLCJ1XyooQytGKSJdLFsyLDEsInVfKihDJytGKSJdLFsxLDIsIkYnJyJdLFswLDAsIkMrdV8qRiJdLFsyLDAsIkMnK3VfKkYiXSxbMCwxLCJcXHNpbSJdLFswLDJdLFsxLDJdLFszLDQsIlxcc2ltIl0sWzMsMCwiXFxzaW0iXSxbNCwxLCJcXHNpbSJdXQ==
    \begin{tikzcd}[column sep=tiny]
      {C+u_*F} && {C'+u_*F} \\
      {u_*(C+F)} && {u_*(C'+F)} \\
      & {F''}
      \arrow["\sim", from=2-1, to=2-3]
      \arrow[from=2-1, to=3-2]
      \arrow[from=2-3, to=3-2]
      \arrow["\sim", from=1-1, to=1-3]
      \arrow["\sim" sloped, from=1-1, to=2-1]
      \arrow["\sim" sloped, from=1-3, to=2-3]
    \end{tikzcd}
  \end{center}
  On en déduit que $(C',C'+B\ra B'',C'+u_*F\ra F'')$ est le même morphisme que $(C,C+B\ra B'',C+u_*F\ra F'')$.

  L'identification des fibres est immédiate.
\end{proof}

Dans la suite de cette sous-section, nous fixons $S$ un groupoïde symétrique monoïdal agissant sur une catégorie $X$, tel que~:
\begin{description}
  \item[$(1)$] toutes les flèches de $X$ sont des monomorphismes~;
  \item[$(2)$] pour tout $F$ dans $X$, le foncteur $S\ra X$, $B\mapsto B+F$ est fidèle~;
  \item[$(3)$] les translations $A+(-):S\ra S$ sont fidèles. 
\end{description}
Alors, on peut en déduire, par une preuve similaire à celle du \sref{lemme}{lemmeactioncategorie}, que la projection sur le second facteur~:
$$q:S^{-1}X\ra \brak{S}{X}$$
est cofibrée.
De plus, la fibre en $F$ s'identifie naturellement à $S$ par $B\mapsto (B,F)$, et via ces identifications, 
le changement de cobase par $(A,A+F\ra F')$ est la translation par $A$.

Nous faisons agir $S$ sur $S^{-1}X$ par la première coordonnée, et nous notons $S^{-1}S^{-1}X$ la localisation associée.
Cette action est, à homotopie près, l'inverse de l'action canonique (ie. sur la deuxième coordonnée).
Ainsi, par le \sref{théorème}{theoremeactioninversible}, le morphisme naturel~:
\[
\begin{array}{lcl}
    S^{-1}X &\ra     & S^{-1}S^{-1}X \\
    (B,F)   &\mapsto & (0,(B,F))
\end{array}
\]
est une équivalence d'homotopie.

Maintenant, le \sref{lemme}{lemmeactionfibre} s'applique à $q$. Ainsi, le foncteur~:
$$S^{-1}S^{-1}X\lra \brak{S}{X}$$
est fibré, les fibres s'identifient à $S^{-1}S$, et les changements de cobase aux translations. Or, ces translations sont des équivalences
d'homotopie. Donc, par le \sref{théorème B}{theoremeB}, pour tout $F$ dans $X$,
le diagramme commutatif suivant est homotopiquement cartésien~:
\begin{center}
  % https://q.uiver.app/?q=WzAsNCxbMCwwLCJTXnstMX1TIl0sWzIsMCwiU157LTF9U157LTF9WCJdLFsyLDIsIlxcYnJha3tTfXtYfSJdLFswLDIsIlxcYnJha3tTfXtTfSJdLFszLDIsIkFcXG1hcHN0byBGIl0sWzAsMywiXFxtYXRocm17cHJ9XzEiLDJdLFsxLDJdLFswLDEsIihBLEIpXFxtYXBzdG8gKEEsKEIsRikpIl1d
  \begin{tikzcd}[column sep = 6ex]
    {S^{-1}S} && {S^{-1}S^{-1}X} \\
    \\
    {\brak{S}{S}} && {\brak{S}{X}}
    \arrow["{A\mapsto F}", from=3-1, to=3-3]
    \arrow["{\mathrm{pr}_1}"', from=1-1, to=3-1]
    \arrow[from=1-3, to=3-3]
    \arrow["{(A,B)\mapsto (A,(B,F))}", from=1-1, to=1-3]
  \end{tikzcd}
\end{center}
Considérons maintenant le carré suivant~:
\begin{center}
  % https://q.uiver.app/?q=WzAsNCxbMCwwLCJTXnstMX1TIl0sWzIsMCwiU157LTF9WCJdLFsyLDIsIlNeey0xfVNeey0xfVgiXSxbMCwyLCJTXnstMX1TIl0sWzMsMl0sWzAsMywic19XIiwyXSxbMSwyXSxbMCwxLCIoQSxCKVxcbWFwc3RvIChBLEIrRikiXV0=
  \begin{tikzcd}[column sep = 6ex]
    {S^{-1}S} && {S^{-1}X} \\
    \\
    {S^{-1}S} && {S^{-1}S^{-1}X}
    \arrow[from=3-1, to=3-3]
    \arrow["{s_W}"', from=1-1, to=3-1]
    \arrow[from=1-3, to=3-3]
    \arrow["{(A,B)\mapsto (A,B+F)}", from=1-1, to=1-3]
  \end{tikzcd}
\end{center}
où~:
$$s_W: (A,B)   \mapsto  (B,A)$$
et les $\sim$ désignent des équivalences d'homotopie.
Ce carré commute à homotopie près, comme le montrent les $2$ transformations naturelles suivantes~:
$$(0,(A,B+F))\ra (B,(B+A,B+F))\leftarrow (B,(A,F))$$

Donc, si $\brak{S}{X}$ est contractile, $S^{-1}S\ra S^{-1}S^{-1}X$ est une équivalence d'homotopie, et donc
$S^{-1}S\ra S^{-1}X$ également. Nous avons démontré le théorème suivant.

\begin{theo}\label{theoremebrakSX}
  Soit $S$ un groupoïde symétrique monoïdal agissant sur une catégorie $X$ tels que~:
  \begin{description}
    \item[$(1)$] toutes les flèches de $X$ sont des monomorphismes~;
    \item[$(2)$] pour tout $F$ dans $X$, le foncteur $S\ra X$, $B\mapsto B+F$ est fidèle~;
    \item[$(3)$] les translations $A+(-):S\ra S$ sont fidèles~;
    \item[$(4)$] la catégorie $\brak{S}{X}$ est contractile. 
  \end{description}
  Alors pour tout $F$ objet de $X$, le foncteur~:
  \[
  \begin{array}{lcl}
    S^{-1}S & \ra     & S^{-1}X \\
    (A,B)   & \mapsto & (A,B+F)   
  \end{array}
  \]
  est une équivalence d'homotopie.
\end{theo}

\subsection{\texorpdfstring{Construction $S^{-1}S$ de la $K$-théorie}{Construction S-1S de la K-théorie}}

Nous allons ici démontrer un premier pas important vers la démonstration de $+=Q$.
Nous allons montrer que pour $A$ un anneau, $\GL{}{A}^+$ peut se réaliser comme un $S^{-1}S$ pour un certain groupoïde monoïdal symétrique $S$.

\begin{ex}\label{exempleSS}
  Soit $P$ une catégorie exacte où toutes les suites exactes soient scindées.
  On pose $S:=\Isos{P}$ la catégorie des isomorphismes de $P$. La somme directe $\oplus$ fait de $S$ un groupoïdes monoïdal symétrique.
  Nous avons alors une structure de $H$-espace sur $S^{-1}S$, donnée par le foncteur~:
  \[
  \begin{array}{lcl}
    S^{-1}S\times S^{-1}S &\ra     & S^{-1}S \\
    ((A,B),(C,D))         &\mapsto & (A\oplus C,B\oplus D)
  \end{array}  
  \]
\end{ex}

\begin{rem}
  Ici, un $H$-espace désigne un ensemble simplicial pointé $(X,e)$ muni d'une application~:
  $$\mu:X\times X\lra X$$
  telle que l'on ai les homotopies pointées suivantes~:
  $$\mu(-,e)\simeq \id_X\;\mathrm{rel}\;e\;\;\mu(e,-)\simeq \id_X\;\mathrm{rel}\;e$$
  Dans le cas de $S^{-1}S$ ci-dessus, il faut choisir $\oplus$ tel que $0\oplus 0=0$.
  C'est toujours possible car $0\oplus 0\simeq 0$.

  Nous utiliserons plus loin un résultat classique~: un $H$-espace est un espace simple.
  On peut résumer la démonstration de ce fait ainsi. Soit $(X,e)$ un $H$-espace,
  $\gamma : (\Delta^1,\partial\Delta^1)\ra (X,e)$, et $\sigma : (\Delta^n,\partial\Delta^n)\ra (X,e)$.
  Alors on remarque que $\gamma\cdot \sigma$ est homotope à $\mu(\gamma,\sigma)\circ (d^0\times\id_{\Delta^n})$.
  Or ce dernier simplexe est $\mu(e,\sigma)$ qui est homotope à $\sigma$.
\end{rem}

Dans la suite de cette sous-section, nous fixons un anneau $A$, et nous posons $S:=\Isos{\Proj{A}}$.
Pour rappel, $\Proj{A}$ est la catégorie exacte des $A$-modules projectifs de type fini.

\begin{lem}
  On a une identification compatible à l'action de $\pi_0(S)$~:
  \[
  \begin{array}{lcl}
    \pi_0(S^{-1}S) &\simeq  & K_0(A) \\
    (M,N)          &\mapsto & [N]-[M]
  \end{array}
  \]
\end{lem}

\begin{proof}
  L'ensemble $\pi_0(S^{-1}S)$ est un monoïde via $\oplus$.
  C'est même un groupe car $s_W:(A,B)\ra (B,A)$ est un inverse à homotopie près.
  Les seules relations sur $K_0(A)$ sont les $[U]+[V]=[U\oplus V]$.
  On peut donc définir un inverse par $[N]\mapsto (0,N)$.
\end{proof}

Nous noterons $(S^{-1}S)_0$ la composante connexe de $0$. On a un isomorphisme $S^{-1}S\simeq K_0(A)\times (S^{-1}S)_0$.

On note $S_n$ la composante connexe de $A^n$ dans $S_n$. C'est un groupoïde équivalent à $\GL{n}{A}$.

\begin{defi}
  On définit une catégorie $L$ dont les~:
  \begin{description}
    \item[$\bullet$] objets sont les couples $(n,N)$ avec $n\geq 0$ et $N$ dans $S_n$~;
    \item[$\bullet$] morphismes de $(n,N)$ dans $(n+m,M)$ sont les isomorphismes $A^m\oplus N\simeq M$. 
  \end{description}  
  On notera $L_n$ la sous-catégorie pleine de $L$ des $(M,m)$ pour $m\leq n$.
\end{defi}

Comme le montre la proposition suivante, $L$ est un "alias" de $\GL{}{A}$.

\begin{prop}
  Pour chaque $n\geq 0$, on se donne une équivalence $\eta_n:S_n\ra \GL{n}{A}$, telles
  que ces équivalences soient compatibles aux foncteurs~:
  \[
  \begin{array}{lcl}
    S_n & \ra     & S_{n+m}     \\
    M   & \mapsto & A^m\oplus M
  \end{array}
  \]
  On définit alors un foncteur $F$ par~:
  \[
  \begin{array}{lcl}
    L                          & \ra     & \GL{}{A}                                                      \\
    (n,N)                      & \mapsto & *                                                             \\
    \alpha:A^m\oplus N\simeq M & \mapsto & \eta_{n+m}(M)\circ \alpha\circ \eta_{n+m}(A^m\oplus N)^{-1}
  \end{array}
  \]
  Le foncteur $F$ est bien défini et induit une équivalence d'homotopie.
\end{prop}

\begin{proof}
  Le foncteur est bien défini par la compatibilité entre les $\eta_n$.
  Par le \sref{théorème A}{theoremeA}, il suffit de montrer que $F\downarrow *$ est contractile.
  Pour cela, il suffit de montrer que $F\downarrow *$ est filtrante.
  Or, on remarque que les $\Hom{F\downarrow *}{x}{y}$ sont soit $\{*\}$, soit $\varnothing$, et que tout objet $(N,u)$
  admet un morphisme dans un $(R^m,\id)$ pour $m$ assez grand. Donc la catégorie $F\downarrow *$ est filtrante.
\end{proof}

Nous allons maintenant faire le lien entre $L$ et $S^{-1}S$.

\begin{theo}
  Soit $\psi$ le foncteur~:
  \[
    \begin{array}{lcl}
      L                          & \ra     & (S^{-1}S)_0                                             \\
      (n,B)                      & \mapsto & (R^n,B)                                                 \\
      \alpha:R^m\oplus N\simeq N & \mapsto & (R^m, R^m\oplus R^n=R^{m+n},\alpha:R^m\oplus N\simeq M)
    \end{array}
  \]
  Alors $\psi$ est un isomorphisme en homologie. 
\end{theo}

\begin{rem}
  On remarque que les inclusions $S_n\ra L_n$ ont des adjoints à gauche, ce sont donc des équivalences d'homotopie.
  En particulier, si on note $i_n$ le foncteur~:
  \[
    \begin{array}{lcl}
      S_n & \ra     & S_{n+1}     \\
      M   & \mapsto & A\oplus M
    \end{array}
  \]
  On en déduit un diagramme $\N\ra \Cat, n\mapsto S_n$, lequel induit un isomorphisme~:
  $$\colim_\N H_*(S_n)\simeq H_*(L)$$
\end{rem}

\begin{proof}[Démonstration du théorème]
  Il s'agit, par la remarque ci-dessus, de montrer que le morphisme induit~:
  $$\colim_\N H_*(S_n)\lra H_*((S^{-1}S)_0)$$
  est un isomorphisme.
  On note $e$ la classe de $A$ dans $\pi_0(S)$. Comme $<e>$ est cofinal dans le monoïde $\pi_0(S)$,
  le \sref{théorème}{theoremehomologielocalisation} montre que $S\ra S^{-1}S$ induit un isomorphismes~:
  $$H_*(S)_e\simeq H_*(S^{-1}S)$$
  Le morphisme $H_*(S_n)\ra H_*((S^{-1}S)_0)$ considéré plus haut se décompose en~:
  \begin{center}
    % https://q.uiver.app/?q=WzAsMyxbMCwwLCJIXyooU19uKSJdLFsxLDAsIkhfKigoU157LTF9UylfbikiXSxbMiwwLCJIXyooKFNeey0xfVMpXzApIl0sWzEsMiwiMS9lXm4iXSxbMCwxXV0=
    \begin{tikzcd}
      {H_*(S_n)} & {H_*((S^{-1}S)_n)} & {H_*((S^{-1}S)_0)}
      \arrow["{1/e^n}", from=1-2, to=1-3]
      \arrow[from=1-1, to=1-2]
    \end{tikzcd}
  \end{center}
  où $(S^{-1}S)_n$ désigne la composante connexe de $A^n$ et la première flèche est induite par l'inclusion naturelle.
  
  Soit $(M,N)$ un objet de $(S^{-1}S)_0$. Quitte à appliquer un foncteur $(U,V)\mapsto (P\oplus U,P\oplus V)$,
  qui est homotope à l'identité, on peut remplacer $(M,N)$ par un objet de la forme $(A^n, N')$.
  Maintenant, il existe $k\geq 0$ tel que $A^k\oplus N'\simeq A^{n+k}$. Ainsi,
  dans $H_0(S^{-1}S)$, $(M,N)$ est de la forme $[x]/e^k$, pour $[x]$ dans $H_0(S_k)$, ie. $(M,N)$ est dans l'image d'un $S_k$.
  On peut appliquer le même raisonnement à des sommes de simplexes dans $H_p((S^{-1}S)_0)$, en redressant les morphismes sur la première coordonnée.
  
  Donc le morphisme~:
  $$\colim_\N H_*(S_n)\lra H_*((S^{-1}S)_0)$$
  est surjectif.

  Soit maintenant $[x]/e^k$ dans le noyau de $H_p(S_k)\ra H_p((S^{-1}S)_0)$. Alors, comme $H_*(S)_e\simeq H_*(S^{-1}S)$,
  il existe $n\geq 0$ tel que $e^n\cdot [x]=0$. Donc $[x]=0$ dans $H_p(S_{k+n})$.

  Donc le morphisme~:
  $$\colim_\N H_*(S_n)\lra H_*((S^{-1}S)_0)$$
  est injectif.
\end{proof}

\begin{theo}\label{theoremeplusegalSS}
  On a une équivalence d'homotopie $\GL{}{A}^+\simeq (S^{-1}S)_0$.
\end{theo}

\begin{proof}
  On peut remplacer $\GL{}{A}$ par $L$, qui lui est homotope. Or, on a construit $\psi:L\ra (S^{-1}S)_0$ tel que~:
  \begin{description}
    \item[$\bullet$] $\psi$ est un isomorphisme en homologie sur $\Z$~;
    \item[$\bullet$] $(S^{-1}S)_0$ est un espace simple.
  \end{description}
  Ainsi, par le \sref{lemme}{lemmeplussurZ} et la \sref{proposition}{GLAplussimple}, $\psi$ est une construction $+$ pour $\EGL{}{A}$.
\end{proof}

\subsection{\texorpdfstring{Lien entre la construction $Q$ et $S^{-1}S$}{Lien entre la construction Q et S-1S}}

Dans cette sous-section, on se place dans le cadre de \sref{l'exemple}{exempleSS}~: on fixe une catégorie exacte $P$
où toutes les suites exactes sont scindées, et on pose $S:=\Isos{P}$.

\begin{defi}
  Pour $C$ dans $P$, on note $E_C$ la catégorie dont les~:
  \begin{description}
    \item[$\bullet$] objets sont les suites exactes $\exa{A}{B}{C}$ de $P$ ($C$ est fixé)~;
    \item[$\bullet$] les morphismes de $\exa{A}{B}{C}$ dans $\exa{A'}{B'}{C}$ sont les diagrammes commutatifs~:
          \begin{center}
            % https://q.uiver.app/?q=WzAsMTAsWzAsMCwiMCJdLFsxLDAsIkEiXSxbMiwwLCJCIl0sWzMsMCwiQyJdLFs0LDAsIjAiXSxbNCwxLCIwIl0sWzAsMSwiMCJdLFsxLDEsIkEnIl0sWzIsMSwiQiciXSxbMywxLCJDIl0sWzMsOV0sWzIsOCwiXFxzaW0iXSxbMSw3LCJcXHNpbSJdLFswLDFdLFsxLDJdLFsyLDNdLFszLDRdLFs5LDVdLFs4LDldLFs3LDhdLFs2LDddXQ==
            \begin{tikzcd}
              0 & A & B & C & 0 \\
              0 & {A'} & {B'} & C & 0
              \arrow[equal, from=1-4, to=2-4]
              \arrow["\sim" sloped, from=1-3, to=2-3]
              \arrow["\sim" sloped, from=1-2, to=2-2]
              \arrow[from=1-1, to=1-2]
              \arrow[from=1-2, to=1-3]
              \arrow[from=1-3, to=1-4]
              \arrow[from=1-4, to=1-5]
              \arrow[from=2-4, to=2-5]
              \arrow[from=2-3, to=2-4]
              \arrow[from=2-2, to=2-3]
              \arrow[from=2-1, to=2-2]
            \end{tikzcd}
          \end{center}
          où les morphismes $\sim$ sont des isomorphismes.
  \end{description}
\end{defi}

On remarque que~:
\[
  \begin{array}{lcl}
    S & \ra     & E_0           \\
    A & \mapsto & \exa{A}{A}{0}
  \end{array}
\]
est une équivalence de catégorie. Dans la suite on identifiera parfois $E_0$ à $S$

\begin{defi}
  On définit la catégorie $E$ dont les~:
  \begin{description}
    \item[$\bullet$] objets sont les suites exactes $\exa{A}{B}{C}$ de $P$ ($C$ n'est pas fixé)~;
    \item[$\bullet$] les morphismes de $\exa{A}{B}{C}$ dans $\exa{A'}{B'}{C}$ sont les diagrammes commutatifs~:
          \begin{center}
            % https://q.uiver.app/?q=WzAsMTUsWzAsMCwiMCJdLFswLDEsIjAiXSxbMCwyLCIwIl0sWzQsMCwiMCJdLFs0LDEsIjAiXSxbNCwyLCIwIl0sWzEsMCwiQSJdLFsyLDAsIkIiXSxbMywwLCJDIl0sWzMsMiwiQyciXSxbMiwyLCJCJyJdLFsxLDIsIkEnIl0sWzEsMSwiQSciXSxbMiwxLCJCIl0sWzMsMSwiQ18xIl0sWzAsNl0sWzYsN10sWzcsOF0sWzgsM10sWzEyLDYsIiIsMCx7InN0eWxlIjp7InRhaWwiOnsibmFtZSI6Im1vbm8ifX19XSxbMTQsOCwiIiwwLHsic3R5bGUiOnsiaGVhZCI6eyJuYW1lIjoiZXBpIn19fV0sWzEzLDEwLCIiLDAseyJzdHlsZSI6eyJ0YWlsIjp7Im5hbWUiOiJtb25vIn19fV0sWzEzLDcsImVxdWFsIiwyXSxbMTQsOSwiIiwyLHsic3R5bGUiOnsidGFpbCI6eyJuYW1lIjoibW9ubyJ9fX1dLFsxMiwxMSwiZXF1YWwiLDJdLFsxLDEyXSxbMiwxMV0sWzE0LDRdLFs5LDVdLFsxMSwxMF0sWzEwLDldLFsxMiwxM10sWzEzLDE0XV0=
            \begin{tikzcd}
              0 & A & B & C & 0 \\
              0 & {A'} & B & {C_1} & 0 \\
              0 & {A'} & {B'} & {C'} & 0
              \arrow[from=1-1, to=1-2]
              \arrow[from=1-2, to=1-3]
              \arrow[from=1-3, to=1-4]
              \arrow[from=1-4, to=1-5]
              \arrow[tail, from=2-2, to=1-2]
              \arrow[two heads, from=2-4, to=1-4]
              \arrow[tail, from=2-3, to=3-3]
              \arrow[equal, from=2-3, to=1-3]
              \arrow[tail, from=2-4, to=3-4]
              \arrow[equal, from=2-2, to=3-2]
              \arrow[from=2-1, to=2-2]
              \arrow[from=3-1, to=3-2]
              \arrow[from=2-4, to=2-5]
              \arrow[from=3-4, to=3-5]
              \arrow[from=3-2, to=3-3]
              \arrow[from=3-3, to=3-4]
              \arrow[from=2-2, to=2-3]
              \arrow[from=2-3, to=2-4]
            \end{tikzcd}
          \end{center}
          à isomorphisme près.
  \end{description}
\end{defi}

\begin{rem}
On dispose donc d'un foncteur~:
\[
  \begin{array}{lcl}
  E             & \ra     & QP \\
  \exa{A}{B}{C} & \mapsto & C
\end{array}
\]
Alors $E_C$ est la fibre de $E\ra QP$ au-dessus de $C$.  
\end{rem}

La définition suivante éclaire un peut la catégorie $E$.

\begin{defi}
  Soit $X$ une catégorie. On définit $\mathrm{Sub}(X)$ comme la catégorie dont les objets sont les morphismes de $X$,
  et les morphismes de $f$ dans $g$ sont les paires de morphismes $h$, $k$ dans $X$ tels que $g=hfk$.
\end{defi}

La catégorie $E$ est alors équivalente à la catégorie $\mathrm{Sub}(X)$, où $X$ est la sous-catégorie des monomorphismes de $P$.
Le foncteur $E\ra QP$ devient alors~:
$$\mathrm{coker}:\mathrm{Sub}(X)\ra QP$$

Nous allons maintenant montrer que $E\ra QP$ est fibré et que tous les morphismes sont cartésiens.
Pour cela nous utilisons le lemme suivant.

\begin{lem}
  Soit $f:X\ra Y$ un foncteur tel que~:
  \begin{description}
    \item[$(1)$] pour tout $u:y\ra y'$ dans $Y$ et $x$ dans $X$ tel que $fx=y$, il existe $a:x\ra x'$ dans $X$ tel que $f(a)=u$~;
    \item[$(2)$] pour tout $a:x\ra x'$ et $b:x\ra x'$ morphismes dans $X$ tels que $f(a)=f(b)$, il existe un unique isomorphisme $\eta$ tel
    que $b=\eta a$ et $f(\eta)$ soit une identité.
  \end{description}
  Alors $f$ est fibré et tous les morphismes de $X$ sont cartésiens.
\end{lem}

\begin{proof}
  On se donne un morphismes $w$ dans $Y$ se factorisant en $w=vu$ et $a$ un morphisme de $X$ tel que $f(a)=w$. On a alors le diagramme commutatif suivant~:
  \begin{center}
    % https://q.uiver.app/?q=WzAsOCxbMCwwLCJ4XzEiXSxbMSwwLCJ4XzIiXSxbMiwwLCJ4XzMiXSxbMywwLCJ4XzMiXSxbMywxLCJ5XzMiXSxbMiwxLCJ5XzMiXSxbMSwxLCJ5XzIiXSxbMCwxLCJ5XzEiXSxbNyw0LCJ3IiwyLHsiY3VydmUiOjN9XSxbNSw0LCJlcXVhbCIsMl0sWzcsNiwidSIsMl0sWzYsNSwidiIsMl0sWzAsNywiIiwyLHsic3R5bGUiOnsidGFpbCI6eyJuYW1lIjoibWFwcyB0byJ9fX1dLFsxLDYsIiIsMix7InN0eWxlIjp7InRhaWwiOnsibmFtZSI6Im1hcHMgdG8ifX19XSxbMiw1LCIiLDIseyJzdHlsZSI6eyJ0YWlsIjp7Im5hbWUiOiJtYXBzIHRvIn19fV0sWzMsNCwiIiwwLHsic3R5bGUiOnsidGFpbCI6eyJuYW1lIjoibWFwcyB0byJ9fX1dLFswLDMsImEiLDAseyJjdXJ2ZSI6LTN9XSxbMCwxLCJcXGV4aXN0cyBiIiwyLHsic3R5bGUiOnsiYm9keSI6eyJuYW1lIjoiZGFzaGVkIn19fV0sWzEsMiwiXFxleGlzdHMgYyIsMix7InN0eWxlIjp7ImJvZHkiOnsibmFtZSI6ImRhc2hlZCJ9fX1dLFsyLDMsIlxcZXhpc3RzIFxcZXRhIiwyLHsic3R5bGUiOnsiYm9keSI6eyJuYW1lIjoiZGFzaGVkIn19fV1d
    \begin{tikzcd}[row sep=large, column sep=large]
      {x_1} & {x_2} & {x_3} & {x_3} \\
      {y_1} & {y_2} & {y_3} & {y_3}
      \arrow["w"', bend right=18pt, from=2-1, to=2-4]
      \arrow[equal, from=2-3, to=2-4]
      \arrow["u"', from=2-1, to=2-2]
      \arrow["v"', from=2-2, to=2-3]
      \arrow[maps to, from=1-1, to=2-1]
      \arrow[maps to, from=1-2, to=2-2]
      \arrow[maps to, from=1-3, to=2-3]
      \arrow[maps to, from=1-4, to=2-4]
      \arrow["a", bend left=18pt, from=1-1, to=1-4]
      \arrow["{\exists b}"', dashed, from=1-1, to=1-2]
      \arrow["{\exists c}"', dashed, from=1-2, to=1-3]
      \arrow["{\exists \eta}"', "\sim", dashed, from=1-3, to=1-4]
    \end{tikzcd}
  \end{center}
  L'existence de $b$ et de $c$ est garantie par $(1)$, l'existence de $\eta$ par $(2)$.
  L'unicité de $\eta c$ est garantie par l'unicité dans $(2)$.
  Tous les morphismes sont cartésiens car l'on peut choisir $b$ quelconque.
\end{proof}

Nous allons maintenant appliquer le lemme à $E\ra QP$. Le point $(1)$ est facile~: il suffit de traiter le cas d'une injection et le cas d'une
surjection, tous deux faciles.
Montrons $(2)$ dans le cas d'une surjection $C\twoheadleftarrow C'$. Dans ce cas, les morphismes 
de $\exa{A}{B}{C}$ dans $\exa{A'}{B'}{C'}$ sont juste les isomorphismes
$B\simeq B'$ compatibles à $C\twoheadleftarrow C'$, et $(2)$ est vérifié.

Maintenant, si on considère une injection $C\rightarrowtail C'$. Les morphismes possibles sont cette fois définis par le choix
d'un morphisme compatible $B\ra B'$. Or tous les tels morphismes $B'\ra B$ sont des monomorphismes et ont la même image,
donc $(2)$ est vérifié.

Dans le cas général, on considère $C\rightarrowtail C_1\twoheadleftarrow C'$. On déduit alors $(2)$ des deux cas particuliers
(ce n'est pas complètement immédiat, il faut écrire le diagramme).
\begin{defi}
  Nous munissons $E$ d'une action de $S$ donnée par~:
  $$A'+(\exa{A}{B}{C}):=\exa{A'\oplus A}{A'\oplus B}{C}$$
\end{defi}

Alors, d'après le dual du \sref{lemme}{lemmeactionfibre}, le foncteur~:
$$S^{-1}E\ra QP$$
est fibré et la fibre au-dessus de $C$ s'identifie à $S^{-1}E_C$.

\begin{theo}\label{theoremebrakSEC}
  Pour tout $C$ dans $P$, $\brak{S}{E_C}$ est contractile.
\end{theo}

\begin{proof}
  \begin{description}
    \item[$(i)$] $\brak{S}{E_C}$ est connexe.
    
    On pose, pour $F_i=\exa{A_i}{B_i}{C}$~:
    $$F_1\star F_2:=(\exa{A_1\oplus A_2}{B_1\times_C B_2}{C})$$
    Nous avons ainsi définit un foncteur~:
    $$\star:\brak{S}{E_C}\times \brak{S}{E_C}\lra \brak{S}{E_C}$$
    Maintenant, les projection $F_1\star F_2\twoheadrightarrow F_i$ sont scindées. On a donc des isomorphismes
    $A_1+F_2\simeq F_1\star F_2$ et $A_2+F_1\simeq F_1\star F_2$. Nous venons de connecter $F_1$ et $F_2$.

    \item[$(ii)$] $\brak{S}{E_C}$ est un $H$-espace et admet un inverse homotopique.
    
    Le produit est donnée par $\star$. L'élément $\exa{0}{C}{C}$ est neutre pour $\star$.
    Pour l'inverse, on montre que tout $H$-espace connexe $M$ admet un inverse homotopique. Considérons le diagramme~:
    \begin{center}
      % https://q.uiver.app/?q=WzAsNixbMCwwLCJNIl0sWzEsMCwiTVxcdGltZXMgTSJdLFsyLDAsIk0iXSxbMiwxLCJNIl0sWzAsMSwiTSJdLFsxLDEsIk1cXHRpbWVzIE0iXSxbMCwxLCJ4XFxtYXBzdG8oeCxlKSJdLFsxLDIsIlxcbWF0aHJte3ByfV8yIl0sWzUsMywiXFxtYXRocm17cHJ9XzIiXSxbNCw1LCJ4XFxtYXBzdG8oeCxlKSJdLFswLDQsImVxdWFsIiwxXSxbMiwzLCJlcXVhbCIsMV0sWzEsNSwiKHgseSlcXFxcXFxkb3duYXJyb3dcXFxcKHh5LHgpIl1d
      \begin{tikzcd}[column sep=large,row sep=large]
        M & {M\times M} & M \\
        M & {M\times M} & M
        \arrow["{x\mapsto(x,e)}", from=1-1, to=1-2]
        \arrow["{\mathrm{pr}_2}", from=1-2, to=1-3]
        \arrow["{\mathrm{pr}_2}", from=2-2, to=2-3]
        \arrow["{x\mapsto(x,e)}", from=2-1, to=2-2]
        \arrow[equal, from=1-1, to=2-1]
        \arrow[equal, from=1-3, to=2-3]
        \arrow["{\begin{matrix*}(x,y)\\ \downarrow\\(xy,x)\end{matrix*}}","f"', from=1-2, to=2-2]
      \end{tikzcd}
    \end{center}
    Alors les lignes sont des fibrations. Donc par la suite exacte longue, $f$ est une équivalence d'homotopie et admet un inverse homotopique
    $h$ (quitte à remplacer $M$ par un complexe de Kan). L'inverse est donné par $x\mapsto \mathrm{pr}_1(h(e,x))$.

    \item[$(iii)$] Le morphisme $x\mapsto x\star x$ est homotope à l'identité.
    
    En effet, on dispose d'une transformation naturelle $F\ra F\star F$ induite par $F\oplus F\simeq F\star F$.

    \item[$(iv)$] $\brak{S}{E_C}$ est contractile.
    
    On considère les endomorphismes de $\brak{S}{E_C}$ dans $\Ho{\DEns}$. L'ensemble $[\brak{S}{E_C},\brak{S}{E_C}]$
    est un groupe car $\brak{S}{E_C}$ est un $H$-espace admettant un inverse homotopique. Et tout élément de ce groupe est idempotent
    par $(iii)$. Donc $[\brak{S}{E_C},\brak{S}{E_C}]=*$. Donc $\brak{S}{E_C}$ est contractile.
  \end{description}
\end{proof}

\begin{theo}\label{theoremecarreSSQP}
  Le diagramme commutatif~:
  \begin{center}
    % https://q.uiver.app/?q=WzAsNCxbMCwwLCJTXnstMX1TIl0sWzEsMCwiU157LTF9RSJdLFsxLDEsIlFQIl0sWzAsMSwiKiJdLFswLDFdLFsxLDJdLFswLDNdLFszLDJdXQ==
    \begin{tikzcd}
      {S^{-1}S} & {S^{-1}E} \\
      {*} & QP
      \arrow[from=1-1, to=1-2]
      \arrow[from=1-2, to=2-2]
      \arrow[from=1-1, to=2-1]
      \arrow[from=2-1, to=2-2]
    \end{tikzcd}
  \end{center}
  est homotopiquement cartésien. La flèche $S^{-1}S\ra S^{-1}E$ est donnée par l'équivalence $S\simeq E_0$.
\end{theo}

\begin{proof}
  Pour appliquer le \sref{théorème B}{theoremeB}, nous allons montrer que les changements de base sont des équivalences faibles.
  Il suffit de le faire pour les changements de base par $j:0\twoheadleftarrow C$ et $i:0\rightarrowtail C$.

  On pose~:
  \[
  \begin{array}{clcl}
    f: & E_0 & \ra     & E_C                   \\
       & A   & \mapsto & \exa{A}{A\oplus C}{C}
  \end{array}
  \]
  Comme, d'après le \sref{théorème}{theoremebrakSEC}, $\brak{S}{E_C}$ est contractile, d'après le \sref{théorème}{theoremebrakSX},
  $S^{-1}f$ est une équivalence d'homotopie.

  Considérons le cas de $j$. Nous avons~:
  \[
    \begin{array}{clcl}
      j^*: & E_C           & \ra     & E_0 \\
           & \exa{A}{B}{C} & \mapsto & B
    \end{array}
  \]
  Alors on a~:
  \[
    \begin{array}{clcl}
      j^*\circ S^{-1}f: & S^{-1}E_0 & \ra     & S^{-1}E_0      \\
                        & (A',A)    & \mapsto & (A',A\oplus C)
    \end{array}
  \]
  Ce morphisme est une équivalence d'homotopie. Donc $j^*$ également.

  Considérons le cas de $i$. Nous avons~:
  \[
    \begin{array}{clcl}
      i^*: & E_C           & \ra     & E_0 \\
           & \exa{A}{B}{C} & \mapsto & A
    \end{array}
  \]
  Alors on a~:
  \[
    \begin{array}{clcl}
      i^*\circ S^{-1}f: & S^{-1}E_0 & \ra     & S^{-1}E_0      \\
                        & (A',A)    & \mapsto & (A',A)
    \end{array}
  \]
  Ce morphisme est une équivalence d'homotopie. Donc $i^*$ également.
\end{proof}

\begin{theo}\label{theoremeSEcontractile}
  $S^{-1}E$ est contractile.
\end{theo}

\begin{proof}
  On note $X$ la sous-catégorie des monomorphismes de $P$. On pose~:
  \[
    \begin{array}{lcl}
      \mathrm{Sub}(X) & \ra     & X \\
      f:x\ra y        & \mapsto & y
    \end{array}
  \]
  Comme $X$ admet un objet initial, ie. $0$, ce foncteur a un adjoint à droite donné par $x\mapsto 0\ra x$.
  Donc $\mathrm{Sub}(X)$ est homotope à $X$. Or, comme $X$ a un objet initial, il est contractile.
  Donc $\mathrm{Sub}(X)$ est contractile. Donc $E$, qui lui est équivalente, est également contractile.

  Ainsi, l'action de $S$ sur $E$ est inversible. Donc, d'après le \sref{théorème}{theoremeactioninversible}, $E\ra S^{-1}E$
  est une équivalence faible. Donc $S^{-1}E$ est contractile.
\end{proof}s

\begin{theo}[Théorème $+=Q$]\label{theoremeplusegalQ}
  $\Omega QP$ est homotope à $K_0(A)\times \GL{}{A}^+$.
  En particulier, $\pi_{n+1}(QP)\simeq \pi_n(\GL{}{A}^+)$, et les construction $+$ et $Q$ de la $K$-théorie coïncident.
  Ici, $\Omega$ désigne l'espace des lacets.
\end{theo}

\begin{proof}
  D'après le \sref{théorème}{theoremeplusegalSS}, il suffit de montrer que $\Omega QP$ est homotope à $S^{-1}S$.
  Or, d'après le \sref{théorème}{theoremecarreSSQP} et le \sref{théorème}{theoremeSEcontractile},
  $S^{-1}S$ est la limite homotopique de $*\rightrightarrows QP$. Or $\Omega QP$ est aussi une limite homotopique de $*\rightrightarrows QP$.
\end{proof}

\section{\texorpdfstring{Propriétés de la $K$-théorie supérieure}{Propriétés de la K-théorie supérieure}}

\subsection{\texorpdfstring{Rappels: définition de la $K$-théorie et de la $K'$-théorie des anneaux et schémas}
{Rappels: définition de la K-théorie et de la K'-théorie des anneaux et schémas}}

Dans la \sref{section}{sectionK0}, nous avons introduit la notion de catégorie exacte et montré que les définitions
du $K_0$ et du $K'_0$ pour les anneaux et les schémas s'interprètent comme le $K_0$ de différentes catégories exactes~:
$\Proj{A}$ et $\Modf{A}$ pour un anneau $A$, $\Proj{X}$ et $\Modf{X}$ pour un schéma $X$.

Maintenant, la construction $Q$ étend la construction du $K_0$, et est définie pour n'importe quelle catégorie exacte.
On peut donc naturellement étendre les définitions.

On rappelle que pour un anneau $A$, $\Proj{A}$ désigne la catégorie des $A$-modules à gauche projectifs de type fini,
et $\Modf{A}$ désigne la catégorie des $A$-modules de type fini.

\begin{defi}
  Soit $A$ un anneau, non nécessairement commutatif. Les groupes de $K$-théorie de $A$ sont les groupes~:
  $$K_nA:=K_n\Proj{A}\text{ pour }n\geq 0$$
  Les groupes de $K'$-théorie de $A$ sont les groupes~:
  $$K'_nA:=K_n\Modf{A}\text{ pour }n\geq 0$$
\end{defi}

\begin{rem}
  Dans le cas de la $K$-théorie des anneaux, nous avons vu qu'il existe une construction équivalente, la construction $+$
  (voir la \sref{section}{sectionplus} et la \sref{section}{sectionplusegalQ}).
  Cette dernière nous donne notamment la forme de $K_1A$ pour un anneau $A$~:
  $$K_1A=\GL{}{A}\ab=\GL{}{A}/\EGL{}{A}$$
  Pour les définitions de $\GL{}{A}$ et $\EGL{}{A}$, voir la \sref{sous-section}{soussectionK1}.
\end{rem}

On rappelle que pour un schéma $X$, $\Proj{X}$ désigne la catégorie des $\Ring{X}$-modules localement libres de type fini,
et $\Modf{X}$ désigne la catégorie des $\Ring{X}$-modules quasi-cohérents de type fini.

\begin{defi}
  Soit $X$ un schéma. Les groupes de $K$-théorie de $X$ sont les groupes~:
  $$K_nA:=K_n\Proj{X}\text{ pour }n\geq 0$$
  Les groupes de $K'$-théorie de $X$ sont les groupes~:
  $$K'_nA:=K_n\Modf{X}\text{ pour }n\geq 0$$
\end{defi}

\subsection{Réduction par résolution}\label{soussectionresolution}

Le but de cette sous-section est d'introduire les méthodes de résolution, qui permettent
notamment d'établir des résultats de fonctorialité de la $K$-théorie ou $K'$-théorie des anneaux ou schémas.
L'idée général est de pouvoir établir des arguments dans le style de la \sref{proposition}{KprimeTordimfinie} pour les groupes supérieurs.

La proposition suivante découle de la définition d'une catégorie exacte.

\begin{propdefi}
  Soit $\M$ une catégorie exacte et $\MP$ une sous-catégorie additive pleine de $\M$.
  On dit que $\MP$ est une sous-catégorie pleine stable par extension de $\M$ si~:
  \begin{description}
    \item[(i)] il existe un objet nul $O$ de $\M$ dans $\MP$~;
    \item[(ii)] Pour toute suite exacte $\exac{M'}{M}{M''}$ dans $\M$,
               si $M'$ et $M''$ sont isomorphes à des objets de $\M$, alors $M$ également.   
  \end{description}
  Dans ce cas, $\MP$ est une catégorie exacte avec $\E_{\MP}=\E_{\M}\cap\MP$, et
  $Q\MP$ est une sous-catégorie de $Q\M$.
\end{propdefi}

\begin{rem}
  En général, $Q\MP$ n'est pas une sous-catégorie pleine de $Q\M$. Par exemple, ce n'est pas le cas pour
  $\MP:=\Proj{A}$ et $\M:=\Modf{A}$, où $A$ est un anneau.
\end{rem}

\begin{theo}\label{theoremeresolutionbasique}
  Soit $\MP$ une sous-catégorie pleine stable par extension d'une catégorie exacte $\M$. On suppose~:
  \begin{description}
    \item[(i)] Si $\exac{M'}{M}{M''}$ est une suite exacte dans $\M$, et si $M$ est dans $\MP$, alors $M'$ également~;
    \item[(ii)] Pour tout $M''$ dans $\M$, il existe une suite exacte $\exac{M'}{M}{M''}$ avec $M$ dans $\MP$.
  \end{description}
  Alors le foncteur~:
  $$Q\MP\lra Q\M$$
  est une équivalence d'homotopie.
\end{theo}

\begin{rem}
  Un exemple classique de couple $(\M,\MP)$ vérifiant les hypothèses du \sref{théorème}{theoremeresolutionbasique} est le suivant~:
  \begin{description}
    \item[$\M$:] modules de dimension projective inférieure ou égale à $n$~;
    \item[$\MP$:] modules de dimension projective inférieure ou égale à $n-1$~; 
  \end{description}
  où $n\geq 1$ est un entier, et les modules considérés sont sur un anneau $A$ ou un schéma $X$. 
\end{rem}

\begin{proof}[Démonstration du théorème]
  On décompose le foncteur $Q\MP\lra Q\M$ en~:
  $$Q\MP\overset{g}{\lra}\mathcal{C}\overset{f}{\lra}  Q\M$$
  où $\mathcal{C}$ est la sous-catégorie pleine de $Q\M$ d'objets $\Ob{Q\MP}$.

  Montrons que $g$ est une équivalence d'homotopie. Soit $P$ un objet de $\mathcal{C}$. La catégorie $g\downarrow P$ est équivalente à
  la catégorie $J$ dont les~:
  \begin{description}
    \item[$\bullet$] objets sont les couches $\M$-admissibles $(M_0,M_1)$ de $P$ tels que $M_1/M_0$ soit dans $\MP$~;
    \item[$\bullet$] morphismes sont donnés par la relation d'ordre $\leq$ usuelle sur les couches. 
  \end{description}
  Par le point \textbf{(i)} de l'énoncé, pour tout $(M_0,M_1)$ dans $J$, $M_0$ et $M_1$ sont dans $\MP$. Nous avons donc~:
  $$(M_0,M_1)\leq (0,M_1)\geq (0,0)$$
  Or, ces morphismes sont naturels. Donc $g\downarrow P$ est contractile. Ainsi, par le \sref{théorème}{theoremeA},
  $g$ est une équivalence d'homotopie.

  Soit $M$ un objet de $\M$. On pose $\mathcal{F}:=f\uparrow M$. Les objets de $\mathcal{F}$ sont les couples $(P,u)$ avec $P$ dans $\MP$ et $u:M\ra P$
  une application dans $Q\M$. Soit $\mathcal{F}'$ la sous-catégorie pleine de $\mathcal{F}$ formée des $(P,u)$ où $u$ est une surjection.
  Pour $X=(P,u)$ dans $\mathcal{F}$ avec~:
  $$u:M\overset{j}{\twoheadleftarrow} \bar{P}\overset{i}{\rightarrowtail} P$$
  on note $\overline{X}:=(\overline{P},j^!)$ (par \textbf{(i)}, $\overline{P}$ est dans $\MP$). On vérifie aisément que $X\mapsto \overline{X}$
  est un adjoint à gauche à l'inclusion $\mathcal{F}'\subset \mathcal{F}$.

  Donc il suffit de montrer que $\mathcal{F}'$ est contractile. Or la contrainte que $u$ soit une surjection pour tout $(P,u)$
  dans $\mathcal{F}'$ est assez forte. En effet, ceci implique que tout morphisme $(P,u)\ra (P',u')$ dans $\mathcal{F}'$
  est représenté par une surjection $v:P\twoheadleftarrow P'$. Ainsi, ${\mathcal{F}'}\op$ est équivalente à la catégorie dont les~:
  \begin{description}
    \item[$\bullet$] objets sont les épimorphismes $P\twoheadrightarrow M$ avec $P$ dans $\MP$~;
    \item[$\bullet$] morphismes de $P\twoheadrightarrow M$ dans $P'\twoheadrightarrow M$ sont les
                    épimorphismes $P\twoheadrightarrow P'$ au-dessus de $M$. 
  \end{description}
  Or, d'après \textbf{(ii)}, il existe $P_0\twoheadrightarrow M$ dans ${\mathcal{F}'}\op$. De plus, pour tout $P\twoheadrightarrow M$
  dans ${\mathcal{F}'}\op$, $P_0\times_MP\twoheadrightarrow M$ est également dans ${\mathcal{F}'}\op$. En effet, on dispose d'une suite exacte~:
  $$\exac{\myker{P\twoheadrightarrow M}}{P_0\times_MP}{P_0}$$
  et $\myker{P\twoheadrightarrow M}$ est dans $\MP$ par \textbf{(i)}.
  Donc, nous avons les morphismes fonctoriels dans ${\mathcal{F}'}\op$~:
  $$(P\twoheadrightarrow M)\longleftarrow (P_0\times_MP\twoheadrightarrow M)\lra (P_0\twoheadrightarrow M)$$
  Donc ${\mathcal{F}'}\op$ est contractile.
\end{proof}

\begin{coro}\label{corollaireresolution}
  Soit $\MP$ une sous-catégorie pleine stable par extension d'une catégorie exacte $\M$. On suppose~:
  \begin{description}
    \item[(a)] pour tout suite exacte $\exac{M'}{M}{M''}$, si $M$ et $M''$ sont dans $\MP$, alors $M'$ aussi~;
    \item[(b)] si $j:M\twoheadrightarrow P$ est un épimorphisme avec $P$ dans $\MP$, alors il existe $P'$ dans $\MP$ et
           un épimorphisme $j':P'\twoheadrightarrow P$ admettant une factorisation~:
           $$P'\overset{f}{\ra} M \overset{j}{\twoheadrightarrow} P$$
  \end{description}
  Pour $n\geq 1$, on note $\MP_n$ la sous-catégorie pleine de $\M$ formée des $M$ ayant une $\MP$-résolution
  de longueur inférieure ou égale à $n$. On pose $\MP_\infty:=\bigcup_n\MP_n$. Alors, pour $i\geq 0$, les inclusions naturelles
  induisent des isomorphismes~:
  $$K_i\MP\simeq K_i\MP_1 \simeq\dotsb\simeq K_i\MP_\infty$$
\end{coro}

\begin{lem}\label{lemmeresolution}
  On reprend les hypothèses du corollaire. Alors, pour toute suite exacte $\exac{M'}{M}{M''}$ dans $\M$~:
  \begin{description}
    \item[$(1)$] Si $M\in \MP_n$ et $M''\in\MP_{n+1}$, alors $M'\in\MP_n$~;
    \item[$(2)$] Si $M',M''\in \MP_{n+1}$, alors $M\in\MP_{n+1}$~;
    \item[$(3)$] Si $M,M''\in\MP_{n+1}$, alors $M'\in\MP_{n+1}$. 
  \end{description}
\end{lem}

\begin{proof}[Démonstration du corollaire]
  Il s'agit d'appliquer le \sref{théorème}{theoremeresolutionbasique} aux inclusions $\MP_n\subset{\MP_{n+1}}$.
  D'après le point $(2)$ du \sref{lemme}{lemmeresolution}, ces deux catégories sont exacte et l'inclusion est une inclusion de
  sous-catégorie pleine exacte. Le point \textbf{(i)} découle directement du point $(1)$ du même du même lemme. Si $M$ est un objet de $\MP_{n+1}$,
  il admet une $\MP$-résolution, et donc en particulier, il existe $P$ dans $\MP$ et $u:P\twoheadrightarrow M$~;
  donc par le point $(1)$ du lemme, $u$ est $\MP_{n+1}$-admissible. Ceci démontre le point \textbf{(ii)}.
\end{proof}

\begin{proof}[Démonstration du lemme]
  \begin{description}
    \item[(A)] Si $\MP_0,\dotsc,\MP_n$ vérifient le point $(2)$ du lemme, alors $\MP_{n+1}=(\MP_n)_1$.
    
    Il s'agit de montrer que $(\MP_n)_1\subset \MP_{n+1}$. Soit $M''$ dans $\M$ et une suite exacte $\exac{M'}{M}{M''}$
    avec $M'$ et $M$ dans $\MP_n$. Considérons une présentation $\exac{K}{P}{M}$ de $M$ avec $P$ dans $\MP$ et $K$ dans $\MP_{n-1}$.
    Nous avons alors le diagramme suivant~:
    \begin{center}
      % https://q.uiver.app/?q=WzAsOCxbMCwyLCJNJyJdLFsxLDIsIk0iXSxbMiwyLCJNJyciXSxbMiwxLCJNJyciXSxbMSwxLCJQIl0sWzAsMSwiTSdcXHRpbWVzX01QIl0sWzEsMCwiSyJdLFswLDAsIksiXSxbNiw0LCIiLDAseyJzdHlsZSI6eyJ0YWlsIjp7Im5hbWUiOiJtb25vIn19fV0sWzQsMSwiIiwwLHsic3R5bGUiOnsiaGVhZCI6eyJuYW1lIjoiZXBpIn19fV0sWzQsMywiIiwwLHsic3R5bGUiOnsiaGVhZCI6eyJuYW1lIjoiZXBpIn19fV0sWzEsMiwiIiwwLHsic3R5bGUiOnsiaGVhZCI6eyJuYW1lIjoiZXBpIn19fV0sWzMsMiwiZXF1YWwiLDFdLFs1LDQsIiIsMSx7InN0eWxlIjp7InRhaWwiOnsibmFtZSI6Im1vbm8ifX19XSxbMCwxLCIiLDEseyJzdHlsZSI6eyJ0YWlsIjp7Im5hbWUiOiJtb25vIn19fV0sWzUsMCwiIiwxLHsic3R5bGUiOnsiaGVhZCI6eyJuYW1lIjoiZXBpIn19fV0sWzcsNiwiZXF1YWwiLDJdLFs3LDUsIlxcZXhpc3RzIiwyLHsic3R5bGUiOnsidGFpbCI6eyJuYW1lIjoibW9ubyJ9fX1dXQ==
      \begin{tikzcd}
        K & K \\
        {M'\times_MP} & P & {M''} \\
        {M'} & M & {M''}
        \arrow[tail, from=1-2, to=2-2]
        \arrow[two heads, from=2-2, to=3-2]
        \arrow[two heads, from=2-2, to=2-3]
        \arrow[two heads, from=3-2, to=3-3]
        \arrow[equal, from=2-3, to=3-3]
        \arrow[tail, from=2-1, to=2-2]
        \arrow[tail, from=3-1, to=3-2]
        \arrow[two heads, from=2-1, to=3-1]
        \arrow[equal, from=1-1, to=1-2]
        \arrow["\exists"', tail, from=1-1, to=2-1]
      \end{tikzcd}
    \end{center}
    où les lignes et colonnes sont exactes. Par le point $(2)$ du lemme appliqué à $\MP_n$ et la première colonne,
    $M'\times_MP$ est dans $\MP_n$. Donc comme la ligne du milieu est exacte, $M$ est dans $\MP_{n+1}$.

    \item[(B)] Si $\MP$ vérifie $(1)$, $(2)$, $(3)$ et (b) alors $\MP_1$ également.
    
    \item[(B.1)] Si $\MP$ vérifie $(2)$ et $(3)$, alors $\MP_1$ vérifie $(1)$.
    
    Soit $\exac{M'}{M}{M''}$ avec $M$ dans $\MP$ et $M''$ dans $\MP_1$. On alors un carré commutatif suivant à lignes et colonnes exactes~:
    \begin{center}
      % https://q.uiver.app/?q=WzAsOSxbMCwyLCJNJyJdLFsxLDIsIk0iXSxbMiwyLCJNJyciXSxbMiwxLCJQIl0sWzEsMSwiRiJdLFswLDEsIk0nIl0sWzEsMCwiUCciXSxbMCwwLCIwIl0sWzIsMCwiUCciXSxbNiw0LCIiLDAseyJzdHlsZSI6eyJ0YWlsIjp7Im5hbWUiOiJtb25vIn19fV0sWzQsMSwiIiwwLHsic3R5bGUiOnsiaGVhZCI6eyJuYW1lIjoiZXBpIn19fV0sWzQsMywiIiwwLHsic3R5bGUiOnsiaGVhZCI6eyJuYW1lIjoiZXBpIn19fV0sWzEsMiwiIiwwLHsic3R5bGUiOnsiaGVhZCI6eyJuYW1lIjoiZXBpIn19fV0sWzMsMiwiIiwxLHsic3R5bGUiOnsiaGVhZCI6eyJuYW1lIjoiZXBpIn19fV0sWzUsNCwiIiwxLHsic3R5bGUiOnsidGFpbCI6eyJuYW1lIjoibW9ubyJ9fX1dLFswLDEsIiIsMSx7InN0eWxlIjp7InRhaWwiOnsibmFtZSI6Im1vbm8ifX19XSxbNSwwLCJlcXVhbCIsMl0sWzcsNiwiZXF1YWwiLDIseyJzdHlsZSI6eyJ0YWlsIjp7Im5hbWUiOiJtb25vIn19fV0sWzcsNSwiIiwyLHsic3R5bGUiOnsidGFpbCI6eyJuYW1lIjoibW9ubyJ9fX1dLFs4LDMsIiIsMSx7InN0eWxlIjp7InRhaWwiOnsibmFtZSI6Im1vbm8ifX19XSxbNiw4LCJlcXVhbCJdXQ==
      \begin{tikzcd}
        0 & {P'} & {P'} \\
        {M'} & F & P \\
        {M'} & M & {M''}
        \arrow[tail, from=1-2, to=2-2]
        \arrow[two heads, from=2-2, to=3-2]
        \arrow[two heads, from=2-2, to=2-3]
        \arrow[two heads, from=3-2, to=3-3]
        \arrow[two heads, from=2-3, to=3-3]
        \arrow[tail, from=2-1, to=2-2]
        \arrow[tail, from=3-1, to=3-2]
        \arrow[equal, from=2-1, to=3-1]
        \arrow[equal, from=1-1, to=1-2]
        \arrow[tail, from=1-1, to=2-1]
        \arrow[tail, from=1-3, to=2-3]
        \arrow[equal, from=1-2, to=1-3]
      \end{tikzcd}
    \end{center}
    avec $P$ et $P'$ dans $\MP$. Comme $\MP$ vérifie $(2)$, $F$ est dans $\MP$. Maintenant, comme $\MP$ vérifie $(3)$, $M'$ est dans $\MP$.

    \item[(B.2)] Si $\MP$ vérifie $(2)$ et (b), alors $\MP_1$ vérifie $(2)$.
    
    Soit $\exac{M'}{M}{M''}$ avec $M'$ et $M''$ dans $\MP$. Il existe $P\twoheadrightarrow M$
    avec $P$ dans $\MP$. Alors, on a $M\times_{M''}P\twoheadrightarrow P$. Donc, comme $\MP$ vérifie (b),
    il existe un épimorphisme $P''\ra M\times_{M''}P\twoheadrightarrow P$ avec $P''$ dans $\MP$.
    On peut alors former le carré commutatif suivant à lignes et colonnes exactes~:
    \begin{center}
      % https://q.uiver.app/?q=WzAsOSxbMCwyLCJNJyJdLFsxLDIsIk0iXSxbMiwyLCJNJyciXSxbMiwxLCJQJyciXSxbMSwxLCJQJ1xcb3BsdXMgUCcnIl0sWzAsMSwiUCciXSxbMSwwLCJSIl0sWzAsMCwiUiciXSxbMiwwLCJSJyciXSxbNiw0LCIiLDAseyJzdHlsZSI6eyJ0YWlsIjp7Im5hbWUiOiJtb25vIn19fV0sWzQsMSwiIiwwLHsic3R5bGUiOnsiaGVhZCI6eyJuYW1lIjoiZXBpIn19fV0sWzQsMywiIiwwLHsic3R5bGUiOnsiaGVhZCI6eyJuYW1lIjoiZXBpIn19fV0sWzEsMiwiIiwwLHsic3R5bGUiOnsiaGVhZCI6eyJuYW1lIjoiZXBpIn19fV0sWzMsMiwiIiwxLHsic3R5bGUiOnsiaGVhZCI6eyJuYW1lIjoiZXBpIn19fV0sWzUsNCwiIiwxLHsic3R5bGUiOnsidGFpbCI6eyJuYW1lIjoibW9ubyJ9fX1dLFswLDEsIiIsMSx7InN0eWxlIjp7InRhaWwiOnsibmFtZSI6Im1vbm8ifX19XSxbNSwwLCIiLDIseyJzdHlsZSI6eyJoZWFkIjp7Im5hbWUiOiJlcGkifX19XSxbNyw2LCIiLDIseyJzdHlsZSI6eyJ0YWlsIjp7Im5hbWUiOiJtb25vIn19fV0sWzcsNSwiIiwyLHsic3R5bGUiOnsidGFpbCI6eyJuYW1lIjoibW9ubyJ9fX1dLFs4LDMsIiIsMSx7InN0eWxlIjp7InRhaWwiOnsibmFtZSI6Im1vbm8ifX19XSxbNiw4LCIiLDAseyJzdHlsZSI6eyJoZWFkIjp7Im5hbWUiOiJlcGkifX19XV0=
      \begin{tikzcd}
        {R'} & R & {R''} \\
        {P'} & {P'\oplus P''} & {P''} \\
        {M'} & M & {M''}
        \arrow[tail, from=1-2, to=2-2]
        \arrow[two heads, from=2-2, to=3-2]
        \arrow[two heads, from=2-2, to=2-3]
        \arrow[two heads, from=3-2, to=3-3]
        \arrow[two heads, from=2-3, to=3-3]
        \arrow[tail, from=2-1, to=2-2]
        \arrow[tail, from=3-1, to=3-2]
        \arrow[two heads, from=2-1, to=3-1]
        \arrow[tail, from=1-1, to=1-2]
        \arrow[tail, from=1-1, to=2-1]
        \arrow[tail, from=1-3, to=2-3]
        \arrow[two heads, from=1-2, to=1-3]
      \end{tikzcd}
    \end{center}
    avec $P'$, $R'$ et $R''$ dans $\MP$. Alors, comme $\MP$ vérifie $(2)$, $R$ est également dans $\MP$. Donc $M$ est dans $\MP_1$.
    
    \item[(B.3)] Si $\MP_1$ vérifie $(1)$, alors $\MP_1$ vérifie $(3)$.

    Soit $\exac{M'}{M}{M''}$ avec $M$ et $M''$ dans $\MP$. on peut alors former le diagramme suivant à lignes et colonnes exactes~:
    \begin{center}
      % https://q.uiver.app/?q=WzAsOSxbMCwyLCJNJyJdLFsxLDIsIk0iXSxbMiwyLCJNJyciXSxbMiwxLCJNJyciXSxbMSwxLCJQIl0sWzAsMSwiSyJdLFsxLDAsIlAnIl0sWzAsMCwiUCciXSxbMiwwLCIwIl0sWzYsNCwiIiwwLHsic3R5bGUiOnsidGFpbCI6eyJuYW1lIjoibW9ubyJ9fX1dLFs0LDEsIiIsMCx7InN0eWxlIjp7ImhlYWQiOnsibmFtZSI6ImVwaSJ9fX1dLFs0LDMsIiIsMCx7InN0eWxlIjp7ImhlYWQiOnsibmFtZSI6ImVwaSJ9fX1dLFsxLDIsIiIsMCx7InN0eWxlIjp7ImhlYWQiOnsibmFtZSI6ImVwaSJ9fX1dLFszLDIsImVxdWFsIiwyXSxbNSw0LCIiLDEseyJzdHlsZSI6eyJ0YWlsIjp7Im5hbWUiOiJtb25vIn19fV0sWzAsMSwiIiwxLHsic3R5bGUiOnsidGFpbCI6eyJuYW1lIjoibW9ubyJ9fX1dLFs1LDAsIiIsMix7InN0eWxlIjp7ImhlYWQiOnsibmFtZSI6ImVwaSJ9fX1dLFs3LDYsImVxdWFsIl0sWzcsNSwiIiwyLHsic3R5bGUiOnsidGFpbCI6eyJuYW1lIjoibW9ubyJ9fX1dLFs4LDMsIiIsMSx7InN0eWxlIjp7InRhaWwiOnsibmFtZSI6Im1vbm8ifX19XSxbNiw4LCIiLDAseyJzdHlsZSI6eyJoZWFkIjp7Im5hbWUiOiJlcGkifX19XV0=
      \begin{tikzcd}
        {P'} & {P'} & 0 \\
        K & P & {M''} \\
        {M'} & M & {M''}
        \arrow[tail, from=1-2, to=2-2]
        \arrow[two heads, from=2-2, to=3-2]
        \arrow[two heads, from=2-2, to=2-3]
        \arrow[two heads, from=3-2, to=3-3]
        \arrow[equal, from=2-3, to=3-3]
        \arrow[tail, from=2-1, to=2-2]
        \arrow[tail, from=3-1, to=3-2]
        \arrow[two heads, from=2-1, to=3-1]
        \arrow[equal, from=1-1, to=1-2]
        \arrow[tail, from=1-1, to=2-1]
        \arrow[tail, from=1-3, to=2-3]
        \arrow[two heads, from=1-2, to=1-3]
      \end{tikzcd}
    \end{center}
    Comme $\MP_1$ vérifie $(1)$, $K$ est dans $\MP$, donc $M'$ est dans $\MP_1$.

    \item[(B.b)] Si $\MP$ vérifie (b), alors $\MP_1$ vérifie (b).
    
    Soit $M\twoheadrightarrow M''$ avec $M''$ dans $\MP_1$. Alors, il existe $P\twoheadrightarrow M''$ avec $P$ dans $\MP$.
    En appliquant (b) à $M\times_{M''}P\twoheadrightarrow P$, on obtient le diagramme commutatif suivant~:
    \begin{center}
      % https://q.uiver.app/?q=WzAsNSxbMiwyLCJNJyciXSxbMSwyLCJNIl0sWzIsMSwiUCJdLFsxLDEsIk1cXHRpbWVzX3tNJyd9UCJdLFswLDAsIlxcZXhpc3RzIFAnIl0sWzEsMCwiIiwwLHsic3R5bGUiOnsiaGVhZCI6eyJuYW1lIjoiZXBpIn19fV0sWzIsMCwiIiwyLHsic3R5bGUiOnsiaGVhZCI6eyJuYW1lIjoiZXBpIn19fV0sWzMsMiwiIiwyLHsic3R5bGUiOnsiaGVhZCI6eyJuYW1lIjoiZXBpIn19fV0sWzMsMSwiIiwwLHsic3R5bGUiOnsiaGVhZCI6eyJuYW1lIjoiZXBpIn19fV0sWzQsMiwiIiwwLHsiY3VydmUiOi0yLCJzdHlsZSI6eyJoZWFkIjp7Im5hbWUiOiJlcGkifX19XSxbNCwxLCJmIiwyLHsiY3VydmUiOjJ9XSxbNCwzLCJcXGV4aXN0cyJdXQ==
      \begin{tikzcd}
        {\exists P'} \\
        & {M\times_{M''}P} & P \\
        & M & {M''}
        \arrow[two heads, from=3-2, to=3-3]
        \arrow[two heads, from=2-3, to=3-3]
        \arrow[two heads, from=2-2, to=2-3]
        \arrow[two heads, from=2-2, to=3-2]
        \arrow[bend left=12pt, two heads, from=1-1, to=2-3]
        \arrow["f"', bend right=12pt, from=1-1, to=3-2]
        \arrow["\exists", from=1-1, to=2-2]
      \end{tikzcd}
    \end{center}
    Alors $P'\ra M''$ et $f$ sont les morphismes recherchés.
  \end{description}
\end{proof}

Nous allons maintenant appliquer ce théorème aux cas particuliers des anneaux et des schémas.

\begin{theo}\label{theoremeresolutionanneaux}
  Soit $A$ un anneau. Pour $n\geq 0$, on note $\Proj{A}_n$ la sous-catégorie pleine de $\Modf{A}$ formée des $A$-modules admettant une $\Proj{A}$-résolution
  de longueur inférieure ou égale à $n$. On note $\Proj{A}_\infty:=\bigcup_n\Proj{A}_n$. On a alors pour $i\geq 0$~:
  $$K_iA\simeq\dotsb\simeq K_i\Proj{A}_n\simeq K_i\Proj{A}_\infty$$
  En particulier, si $A$ est noethérien régulier, on a, pour tout $i\geq 0$~:
  $$K_iA\simeq K'_iA$$
\end{theo}

\begin{proof}
  Il s'agit d'appliquer le \sref{corollaire}{corollaireresolution}. Le point (a) découle du fait que les modules projectifs
  scindent les suites exactes, et le point (b) du fait que tout $A$-module de type fini est quotient d'un $A$-module projectif.
  Pour le cas où $A$ est régulier, il suffit de voir qu'alors $\Proj{A}_\infty=\Modf{A}$.
\end{proof}

Voici un résultat similaire dans le cadre des schémas.

\begin{theo}
  Soit $X$ un schéma noethérien, régulier et séparé. Alors pour tout $i\geq 0$~:
  $$K_iX\simeq K'_iX$$
\end{theo}

\begin{proof}
  Appliquons le \sref{corollaire}{corollaireresolution}. Le point (a) découle du fait que les fibrés vectorielles scindent localement
  les suites exactes. Le point (b) découle du \sref{lemme}{RegulierSepareQuotient}. On conclut avec le \sref{lemme}{resolutionsVB}
  qui montre que $\Proj{X}_\infty=\Modf{X}$.
\end{proof}

\subsection{\texorpdfstring{Fonctorialité de la $K$-théorie et de la $K'$-théorie pour les anneaux et schémas}
{Fonctorialité de la K-théorie et de la K'-théorie pour les anneaux et schémas}}\label{soussectionfonctorialite}

Dans cette sous-section, nous allons utiliser les résultats de la sous-section précédente sur les résolutions pour étudier
la fonctorialité de la $K$-théorie supérieure.

L'obstruction à la construction, par exemple, de $f_*:K_*B\ra K_*A$ pour $f:A\ra B$ un morphisme quelconque d'anneau est essentiellement
la même que dans le cas du $K_0$. En effet, sur cette exemple, si on cherche naïvement à déduire $f^*$ d'un foncteur exact
de $\Proj{B}$ dans $\Proj{A}$, il devient nécessaire que $B$ soit un $A$-module projectif.

\begin{prop}[transferts en $K$-théorie des anneaux]
  Soit $f:A\ra B$ un morphisme fini entre anneaux noethériens (à gauche), tel que $B$ soit de $\Tor$-dimension finie sur $A$.
  Alors la restriction définit un foncteur exact~:
  $$\Proj{B}_\infty\lra \Proj{A}_\infty$$
  Qui induit un morphismes pour tout $i\geq 0$~:
  $$f_*:K_iB\ra K_iA$$
  Appelé application de transfert.
  Si $g:B\ra C$ est un autre tel morphismes, alors $(gf)_*=f_*g_*$.
\end{prop}

Cette proposition est un corollaire direct du \sref{théorème}{theoremeresolutionanneaux}.

Nous pouvons de même étendre le cadre où $f^*$ est défini en $K'$-théorie. Pour cela, nous passons par le résultat plus général suivant~:

\begin{theo}\label{theoremeresolutionfoncteurderive}
  Soit $\M$ une catégorie exacte, $\A$ une catégorie abélienne et soit $\enstq{T_n}{n\geq 1}$ une suite de foncteur exacts
  de $\M$ dans $\A$. On suppose donné pour chaque suite exacte $\exa{M'}{M}{M''}$ de $\M$ une suite exacte longue~:
  $$\dotsb\lra T_2M'\lra T_2M\lra T_2M''\lra T_1M'\lra T_1M\lra T_1M''$$
  tels que ces suites exactes longues soient fonctorielles en la suite courte $\exa{M'}{M}{M''}$.

  Soit $\MP$ la sous-catégorie pleine de $\M$ formée des objets $P$ tels que, pour tout $n\geq 0$, $T_n(P)=0$. On suppose de plus que~:
  \begin{description}
    \item[$\bullet$] pour tout $M$ dans $\M$, il existe $P\twoheadrightarrow M$ avec $P$ dans $\MP$~;
    \item[$\bullet$] pour tout $M$ dans $\M$, $T_n(M)=0$ pour $n$ assez grand. 
  \end{description}
  Alors l'inclusion $\MP\subset\M$ induit pour tout $i\geq 0$ un isomorphisme~:
  $$K_i\MP\simeq K_i\M$$
\end{theo}

\begin{proof}
  Ce théorème est encore un autre corollaire du \sref{théorème}{theoremeresolutionbasique}. En effet, pour $n\geq 0$, on pose~:
  $$\MP_n:=\enstq{M}{T_jM=0\;\forall j>n}$$
  Alors pour $n\geq 0$, $\MP_n\subset\MP_{n+1}$ vérifie les hypothèses \textbf{(i)} et \textbf{(ii)} du théorème. Pour \textbf{(i)}, il s'agit
  simplement de calculer sur la suite exacte. L'hypothèse \textbf{(ii)} découle du premier point.

  Maintenant, il suffit d'observer que $\MP=MP_0$ et $\M=\colim_n\MP_n$ par le second point.
\end{proof}

\begin{coro}[tiré en arrière en $K'$-théorie des anneaux]\label{corotireenarriere}
  Soit $f:A\ra B$ un morphisme entre anneaux noethériens (à gauche), tel que $B$ soit de $\Tor$-dimension finie sur $A$.
  Alors, avec $\mathcal{M}$ la sous-catégorie pleine de $\Modf{A}$ formée des $A$-modules $M$ tels que pour tout $n> 0$~:
  $$\fTor{n}{A}{M}{B}=0$$
  on a un isomorphisme pour tout $i\geq 0$~:
  $$K_i\M\simeq K'_iA$$
  En particulier, le foncteur $B\otimes_A-:\M\ra\Modf{B}$ induit alors pour tout $i\geq 0$ un morphisme~:
  $$f^*:K'_iA\lra K'_iB$$
\end{coro}

\begin{defi}[produit pour les anneaux en degré $0$]
  Soit $A$ un anneau. Pour tout $A$-module projectif de type fini $P$, on dispose de foncteurs exacts~:
  $$P\otimes_A(-):\Proj{A}\ra\Proj{A}\text{ et }P\otimes_A(-):\Modf{A}\ra\Modf{A}$$
  De plus, on a, pour tout suite exacte $\exa{P'}{P}{P''}$, une suite exacte de foncteurs~:
  $$\exa{P'\otimes_A(-)}{P\otimes_A(-)}{P''\otimes_A(-)}$$
  Ainsi, par le \sref{corollaire}{corollaireadditivite}, on dispose pour tout $i\geq 0$ de produits~:
  $$K_0A\otimes K_iA\ra K_iA\text{ et }K_0A\otimes K'_iA\ra K'_iA$$
\end{defi}

\begin{prop}[formule de projection]
  Soit $f:A\ra B$ fini et de $\Tor$-dimension finie, avec $A$ et $B$ commutatifs et $A$ noethérien. Soit $i\geq 0$ un entier.
  Alors pour tout $a$ dans $K_i'(A)$ et $b$ dans $K_0(B)$~:
  $$f_*(b\cdot f^*(a))=f_*(b)\cdot a\text{ dans }K'_i(A)$$
  De même, pour tout $a$ dans $K_i(A)$ et $b$ dans $K_0(B)$~:
  $$f_*(b\cdot f^*(a))=f_*(b)\cdot a\text{ dans }K_i(A)$$
\end{prop}

\begin{proof}
  Plaçons nous d'abord dans le cas de la $K'$-théorie. Il suffit de le montrer pour $a$ dans $K_i\M$ et $b=[P]$ avec $\M$ comme 
  dans le \sref{corollaire}{corotireenarriere}. Alors l'énoncé découle de l'isomorphisme de foncteurs~:
  $$(P\otimes_B(B\otimes_A(-)))_A\simeq (P)_A\otimes_A(-)$$
  où $(-)_A:\Modf{B}\ra\Modf{A}$ est le foncteur oubli.

  Le cas de la $K$-théorie découle du même isomorphisme.
\end{proof}

Passons maintenant au cadre des schémas.

\begin{prop}[tiré en arrière en $K'$-théorie des schémas]\label{proptireenarriere}
  Soit $f:X\ra Y$ un morphisme de $\Tor$-dimension finie entre schémas noethériens et séparés.
  On suppose de plus que tout module quasi-cohérent sur $Y$ est quotient d'un fibré vectoriel, ce qui est par exemple le cas
  si $Y$ admet un fibré en droite ample.
  Alors, avec $\mathcal{M}$ la sous-catégorie pleine de $\Modf{Y}$ formée des $\Ring{Y}$-modules $\mathcal{F}$ tels que pour tout $n>0$
  et tout point $x$ dans $X$~:
  $$\fTor{n}{\Ring{Y,f(x)}}{\mathcal{F}_x}{\Ring{X,x}}=0$$
  on a un isomorphisme pour tout $i\geq 0$~:
  $$K_i\M\simeq K'_iA$$
  En particulier, le foncteur $B\otimes_A-:\M\ra\Modf{B}$ induit alors pour tout $i\geq 0$ un morphisme~:
  $$f^*:K'_iA\lra K'_iB$$
\end{prop}

\begin{proof}
  Il suffit d'appliquer le \sref{théorème}{theoremeresolutionfoncteurderive} au foncteurs~:
  $$T_n(\mathcal{F}):=\prod_{x\in X}\fTor{n}{\Ring{Y,f(x)}}{\mathcal{F}_x}{\Ring{X,x}}$$
\end{proof}

\begin{lem}\label{lemmetransfertschemas}
  Soit $f:X\ra Y$ un morphisme propre entre schémas noethériens, et soit $\mathcal{L}$ un fibré en droite ample sur $X$.
  Alors tout module cohérent $\mathcal{F}$ sur $X$ s'injecte dans une $\Ring{X}$-module cohérent $f_*$-acyclique.
\end{lem}

\begin{proof}
  Quitte à changer $\mathcal{L}$ en une de ses puissances, on peut supposer qu'il est engendré par ses sections globales.
  
  Fixons un module cohérent $\mathcal{F}$ sur $X$. Par le \sref{théorème de Serre}{theoremeSerre}, il existe un entier $n_0$
  tel que pour tout $n\geq n_0$ et tout $q>0$, $R^qf_*(\mathcal{F}(n))=0$. on fixe un entier $n\geq n_0$.
  Soit maintenant une surjection $\Ring{X}^r\twoheadrightarrow \mathcal{L}^{\otimes n}$. Son noyau est un fibré vectoriel.
  En passant au dual la suite exacte obtenue, et en multipliant par $\mathcal{L}^{\otimes n}$, on obtient une suite exacte de fibrés vectoriels~:
  $$\exa{\Ring{X}}{(\mathcal{L}^{\otimes n})^r}{E}$$
  Comme $E$ est plat, on obtient, en multipliant par $\mathcal{F}$, une suite exacte~:
  $$\exa{\mathcal{F}}{\mathcal{F}(n)^r}{E\otimes \mathcal{F}}$$
  Et $\mathcal{F}(n)^r$ est $f_*$-acyclique.
\end{proof}

\begin{prop}[transferts en $K'$-théorie des schémas]\label{propositiontransfertschemas}
  Soit $f:X\ra Y$ un morphisme entre schémas noethériens et séparés tel que \textbf{l'une} des conditions suivantes soit vérifiée~:
  \begin{description}
    \item[(a)] le morphisme $f$ est fini, ou~:
    \item[(b)] le morphisme $f$ est propre et le schéma $X$ admet un fibré en droite ample. 
  \end{description}
  
  Soit $\mathcal{M}$ la sous-catégorie pleine de $\Modf{X}$ des modules $\mathcal{F}$ tels que~:
  $$R^if_*(\mathcal{F})=0$$
  Alors l'inclusion $\mathcal{M}\ra \Modf{X}$ induit, pour chaque $i\geq 0$, un isomorphisme~:
  $$K'_i(\mathcal{M})\ra K'_i(X)$$

  Ainsi le foncteur~:
  \[
    \begin{array}{lcl}
      \M & \ra & \Modf{Y} \\
      \mathcal{F} &\mapsto& f_*\mathcal{F}
    \end{array}
  \]
  est exact et induit, pour tout $i\geq 0$, un morphisme de groupe~:
  $$f_*:K'_i(X)\lra K'_i(Y)$$
\end{prop}

\begin{proof}
  Si $f$ est fini, le foncteur $f_*$ est exact et on a $\mathcal{M}=\Modf{X}$. Dans le cas (b), il suffit d'appliquer
  le dual du \sref{théorème}{theoremeresolutionfoncteurderive} au foncteurs $R^nf_*$. Le premier point du théorème est vérifié
  d'après le \sref{lemme}{lemmetransfertschemas}, et le second point découle du \sref{théorème de Serre}{theoremeSerre}.
\end{proof}

\subsection{Le théorème de localisation}

Dans cette sous-section, nous présentons un des résultats principales de l'article de Quillen sur la construction $Q$ \cite{Quil}~:
le théorème de localisation. Nous renvoyons à l'article pour la démonstration car la preuve est élémentaire et n'utilise que
les résultats de la \sref{sous-section}{thmAetB} et est relativement longue.

Pour comprendre ce résultat, il faut savoir ce qu'est une catégorie abélienne quotient. Nous mettons ici la définition.
Pour plus d'information, voir l'article de Gabriel \cite[chp. III]{Gabr}.

\begin{defi}
  Soit $\A$ une catégorie abélienne, et $\mathcal{C}$ une sous-catégorie pleine de $\mathcal{A}$.
  On dit que $\mathcal{C}$ est une sous-catégorie épaisse de $\A$ si pour toute suite exacte
  $\exa{M'}{M}{M''}$ dans $\A$, $M$ est dans $\mathcal{C}$ si et seulement si $M'$ et $M''$ sont dans $\mathcal{C}$.
\end{defi}

On a très facilement le fait suivant.

\begin{prop}
  Si $\mathcal{C}$ est une sous-catégorie épaisse d'une catégorie abélienne $\A$, alors $\mathcal{C}$ est abélienne.
\end{prop}

\begin{defi}
  Soit $\mathcal{C}$ une sous-catégorie épaisse d'une catégorie abélienne $\A$. On définit la catégorie abélienne quotient $\A/\mathcal{C}$ comme
  la catégorie dont les~:
  \begin{description}
    \item[$\bullet$] objets sont les objets de $\A$~;
    \item[$\bullet$] morphismes de $M$ dans $N$ sont données par~:
    $$\Hom{\A/\mathcal{C}}{M}{N}:=\colim_{\begin{cases}M'\subset M,\;N'\subset N \\ M/M',N'\in\mathcal{C}\end{cases}}\Hom{\A}{M'}{N/N'}$$ 
  \end{description}
\end{defi}

La composition est définie naturellement, pour plus de détails, voir \cite[chp. III]{Gabr}.

\begin{prop}
  Dans le cadre de la définition ci-dessus, $\A/\mathcal{C}$ est abélienne et il existe un foncteur exact $T:\A\ra\A/\mathcal{C}$.
  De plus, ce foncteur vérifie la propriété universelle suivante. Pour tout $G:\A\ra \mathcal{B}$ foncteur exact tel
  que $GM$ est nul pour tout $M$ dans $\mathcal{C}$, il existe une unique factorisation de $G$ par $\A/\mathcal{C}$~:
  \begin{center}
    % https://q.uiver.app/?q=WzAsMyxbMCwwLCJcXEEiXSxbMiwwLCJcXG1hdGhjYWx7Qn0iXSxbMSwxLCJcXEEvXFxtYXRoY2Fse0N9Il0sWzAsMSwiRyJdLFswLDIsIlQiLDJdLFsyLDEsIlxcZXhpc3RzIUgiLDIseyJzdHlsZSI6eyJib2R5Ijp7Im5hbWUiOiJkYXNoZWQifX19XV0=
    \begin{tikzcd}
      \A && {\mathcal{B}} \\
      & {\A/\mathcal{C}}
      \arrow["G", from=1-1, to=1-3]
      \arrow["T"', from=1-1, to=2-2]
      \arrow["{\exists!H}"', dashed, from=2-2, to=1-3]
    \end{tikzcd}
  \end{center}
  De plus, $TM$ est nul si et seulement si $M$ est dans $\mathcal{C}$.
\end{prop}

Voir \cite[chp. III]{Gabr} pour la démonstrations.

\begin{ex}
  \begin{description}
    \item[$\bullet$] Soit $A$ un anneau noethérien (à gauche) et $S$ une partie centrale multiplicative de $A$. On pose $\mathcal{A}:=\Modf{A}$
    et $\mathcal{C}$ la sous-catégorie pleine des modules de $S$-torsion. Alors $T:\A\ra\A/\mathcal{C}$ s'identifie au foncteur de localisation~:
    $$\Modf{A}\lra \Modf{S^{-1}A}$$
    \item[$\bullet$] Soit $X$ un schéma noethérien et $j:U\hookrightarrow X$ une immersion ouverte. On pose $\mathcal{A}:=\Modf{X}$
    et $\mathcal{C}$ la sous-catégorie pleine des modules supportés dans $X\setminus U$.
    Alors $T:\A\ra\A/\mathcal{C}$ s'identifie au foncteur de localisation~:
    $$j^*:\Modf{X}\lra \Modf{U}$$ 
  \end{description}
\end{ex}

Ces deux exemples découlent de la variante abélienne suivante d'un lemme classique sur les catégories localisés.

\begin{lem}
  On se donne l'adjonction suivante entre catégories abéliennes~:
  \begin{center}
    % https://q.uiver.app/?q=WzAsMixbMCwwLCJcXGxhbWJkYTpcXG1hdGhjYWx7QX0iXSxbMiwwLCJcXG1hdGhjYWx7Qn06RyJdLFswLDEsIiIsMCx7ImN1cnZlIjotMX1dLFsxLDAsIiIsMCx7ImN1cnZlIjotMX1dLFswLDEsIiIsMSx7InN0eWxlIjp7Im5hbWUiOiJhZGp1bmN0aW9uIn19XV0=
    \begin{tikzcd}
      {\lambda:\mathcal{A}} && {\mathcal{B}:G}
      \arrow[bend left=12pt, from=1-1, to=1-3]
      \arrow[bend left=12pt, from=1-3, to=1-1]
      \arrow["\dashv"{anchor=center}, draw=none, from=1-1, to=1-3]
    \end{tikzcd}
  \end{center}
  On suppose de plus que~:
  \begin{description}
    \item[$\bullet$] $G$ est pleinement fidèle~;
    \item[$\bullet$] $\lambda$ est exact.
  \end{description}

  On pose $\mathcal{W}:=\lambda^{-1}(\Isos{\mathcal{B}})$. On note $\mathcal{C}$ la sous-catégorie pleine de $\A$
  formée des objets $M$ tels que $\lambda(M)$ soit nul.
  On a alors~:
  \begin{description}
    \item[$(1)$] $\mathcal{B}$ s'identifie à la localisation $\A[\mathcal{W}^{-1}]$ de $\A$ par $\mathcal{W}$~;
    \item[$(2)$] $\mathcal{B}$ est équivalente à $\A/\mathcal{C}$.
  \end{description}
\end{lem}

\begin{proof}
  Le point $(1)$ est complètement indépendant du fait que les catégories soient abélienne. Il découle directement
  du fait que $G$ est pleinement fidèle et $\mathcal{W}=\lambda^{-1}(\Isos{\mathcal{B}})$. C'est un résultat classique de localisation,
  appelé "localisation réflexive", voir \cite{Gabr2}.

  Quitte à changer $\mathcal{B}$ par une catégorie équivalente, on peut supposer que $\lambda$ induit un isomorphisme sur les objets.
  Maintenant, il suffit de vérifier que $\lambda$ vérifie la propriété universelle de la catégorie quotient.
  
  Soit $J:\A\ra\mathcal{D}$ un foncteur exact tel que $J(M)$ est nul pour tout $M$ dans $\mathcal{C}$.
  Alors, soit $w:M\ra N$ un morphisme dans $\mathcal{W}$, alors $\lambda(w)$ est un isomorphisme,
  donc, comme $\lambda$ est exact, son noyau et son conoyau sont dans $\mathcal{C}$.
  Mais alors, comme $J$ est exact, $J(w)$ est un isomorphisme.
  Donc il existe une factorisation $J:\A\ra\mathcal{B}\ra\mathcal{D}$. On vérifie immédiatement qu'elle est unique.
\end{proof}

\begin{proof}[Démonstration des exemples]
  Pour l'exemple de l'anneau $A$, on pose~:
  \begin{center}
    % https://q.uiver.app/?q=WzAsMixbMCwwLCJcXGxhbWJkYTpcXG1hdGhjYWx7QX0iXSxbMiwwLCJcXG1hdGhjYWx7Qn06RyJdLFswLDEsIiIsMCx7ImN1cnZlIjotMX1dLFsxLDAsIiIsMCx7ImN1cnZlIjotMX1dLFswLDEsIiIsMSx7InN0eWxlIjp7Im5hbWUiOiJhZGp1bmN0aW9uIn19XV0=
    \begin{tikzcd}
      {\lambda:A-\mathrm{mod}} && {S^{-1}A-\mathrm{mod}:G}
      \arrow[bend left=12pt, from=1-1, to=1-3]
      \arrow[bend left=12pt, from=1-3, to=1-1]
      \arrow["\dashv"{anchor=center}, draw=none, from=1-1, to=1-3]
    \end{tikzcd}
  \end{center}
  L'exemple rentre dans le cadre du lemme et on a donc le résultat pour la catégorie des $A$-modules.
  Pour le résultat sur la catégorie $\Modf{A}$ des $A$-modules de type fini, il suffit d'observer la définition de la catégorie
  quotient, $\Hom{}{M}{N}$ pour $M$ et $N$ de type fini est le même que l'on soit dans $\Modf{A}$ ou dans $A-\mathrm{mod}$.

  On procède de la même manière pour l'exemple du schéma noethériens $X$, avec l'adjonction~:
  \begin{center}
    % https://q.uiver.app/?q=WzAsMixbMCwwLCJcXGxhbWJkYTpcXG1hdGhjYWx7QX0iXSxbMiwwLCJcXG1hdGhjYWx7Qn06RyJdLFswLDEsIiIsMCx7ImN1cnZlIjotMX1dLFsxLDAsIiIsMCx7ImN1cnZlIjotMX1dLFswLDEsIiIsMSx7InN0eWxlIjp7Im5hbWUiOiJhZGp1bmN0aW9uIn19XV0=
    \begin{tikzcd}
      {j^*:\Ring{X}-\mathrm{q.coh.}} && {U-\mathrm{q.coh.}:j_*}
      \arrow[bend left=12pt, from=1-1, to=1-3]
      \arrow[bend left=12pt, from=1-3, to=1-1]
      \arrow["\dashv"{anchor=center}, draw=none, from=1-1, to=1-3]
    \end{tikzcd}
  \end{center}
  où $\Ring{Y}-\mathrm{q.coh.}$ désigne la catégorie abélienne des modules quasi-cohérents sur le schéma $Y$.
\end{proof}

\begin{theo}[dévissage]\label{theoremedevissage}
  Soit $\A$ une catégorie abélienne, et soit $\mathcal{B}$ une sous-catégorie abélienne pleine de $\A$ close par sous-objet et quotient dans $\mathcal{A}$.
  On suppose que tout objet $M$ de $\A$ admet une filtration finie~:
  $$0=M_0\subseteq\dotsb\subseteq M_n=M$$
  telle que pour tout $j$, $M_j/M_{j-1}$ soit dans $\mathcal{B}$.

  Alors l'inclusion naturelle~:
  $$Q\mathcal{B}\ra Q\mathcal{A}$$
  est une équivalence d'homotopie.
\end{theo}

\begin{proof}
  On note $f:Q\mathcal{B}\ra Q\mathcal{A}$ le foncteur. Soit $M$ un objet de $\A$. Alors la catégorie $f\downarrow M$ est équivalente à
  la catégorie $J(M)$ dont les~:
  \begin{description}
    \item[$\bullet$] objets sont les couches admissibles $(M_0,M_1)$ de $M$ tels que $M_1/M_0$ soit dans $\mathcal{B}$~;
    \item[$\bullet$] les morphismes sont donnés par l'ordre usuel sur les couches, $(M_0,M_1)\leq (M'_0,M'_1)$ si et seulement si
    $M'_0\subseteq M_0\subseteq M_1\subseteq M'_1$. 
  \end{description}
  Il suffit donc de montrer que si $M'\subset M$ est une inclusion dans $\A$ avec $M/M'$ dans $\mathcal{B}$, le foncteur induit~:
  $$i:J(M')\ra J(M)$$
  est une équivalence d'homotopie.

  On définit deux foncteurs~:
  \[
    \begin{array}{llclllcl}
      r: & J(M)      & \ra     & J(M')                   & s: & J(M)      & \ra     & J(M')            \\
         & (M_0,M_1) & \mapsto & (M_0\cap M',M_1\cap M') &    & (M_0,M_1) & \mapsto & (M_0\cap M',M_1)
    \end{array}
  \]
  Ces deux foncteurs sont bien définis car pour une couche $(M_0,M_1)$ de $J(M)$, on a~:
  $$M_1\cap M'/M_0\cap M'\rightarrowtail M_1/M_0\cap M'\rightarrowtail M_1/M_0\times M/M'\in\mathcal{B}$$
  Maintenant, on a~:
  \begin{description}
    \item[$\bullet$] $\id_{J(M')}=ri$~;
    \item[$\bullet$] il existe $ir\Rightarrow s\Leftarrow \id_{J(M)}$ donné par~:
    $$(M_0\cap M',M_1\cap M')\leq (M_0\cap M',M_1)\geq (M_0,M_1)$$  
  \end{description}
  Donc $i$ est une équivalence d'homotopie.
\end{proof}

\begin{coro}\label{corollairedevissageschemas}
  Soit $X$ un schéma noethérien et $i:F\hookrightarrow X$ une immersion fermée.
  Soit $\mathcal{C}$ la catégorie des $\Ring{X}$-modules cohérents à support dans l'image de $i$.
  Alors l'inclusion naturelle $i_*:\Modf{F}\ra\mathcal{C}$ induit une équivalence d'homotopie~:
  $$Q\mathcal{C}\simeq Q\Modf{F}$$
\end{coro}

\begin{proof}
  Soit $\mathcal{F}$ un module cohérent sur $X$ à support dans $F$. Soit $\mathcal{I}$ l'idéal quasi-cohérent noyau de $i^*$.
  Comme $X$ est noethérien, il existe une entier $n\geq 0$ tel que $\mathcal{I}^n\mathcal{F}=0$. On dispose alors d'une filtration de $\mathcal{F}$~:
  $$0=\mathcal{I}^n\mathcal{F}\subseteq\dotsb\subseteq \mathcal{I}^0\mathcal{F}=\mathcal{F}$$
  dont tous les quotients sont des modules sur $\Ring{X}/\mathcal{I}=\Ring{F}$. On conclut avec le \sref{théorème de dévissage}{theoremedevissage}.
\end{proof}

Voici maintenant l'énoncé du théorème de localisation.

\begin{theo}[localisation]\label{theoremelocalisation}
  Soit $\mathcal{C}$ une sous-catégorie épaisse d'une catégorie abélienne $\A$. On note $e:\mathcal{C}\ra \A$ l'inclusion,
  et $s:\A\ra \A/\mathcal{C}$ le quotient. Alors on a une suite exacte longue~:
  $$\dotsb\overset{s_*}{\ra}K_1(\A/\mathcal{C})\ra K_0(\mathcal{C})\overset{e_*}{\ra}K_0(\A)\overset{s_*}{\ra}K_0(\A/\mathcal{C})\ra 0$$
  Et ces suites exactes sont fonctorielles en le couple $(\A,\mathcal{C})$.
\end{theo}

Voir \cite[5.5]{Quil} pour la démonstration.

\begin{coro}
  Soit $X$ un schéma noethérien et soit $i:F\hookrightarrow X$ un sous-schéma fermé. On note $j:U\hookrightarrow X$ l'immersion ouverte complémentaire.
  Alors on a une suite exacte longue en $K'$-théorie~:
  $$\dotsb\ra K'_{i+1}U\ra K'_{i}F\overset{i_*}{\ra}K'_{i}X\overset{j^*}{\ra}K'_{i}U\ra\dotsb$$
\end{coro}

\begin{proof}
  Il suffit d'appliquer le \sref{théorème de localisation}{theoremelocalisation} et le \sref{corollaire}{corollairedevissageschemas}.
\end{proof}

\appendix

\section{Catégories modèles}

\subsection{Argument du petit objet, notations}

Nous ne démontrons pas ici en détail l'argument du petit objet, voir \cite[Chp.1]{Goer} pour cela.
La raison d'être de cette sous-section est de fixer des notations.

Soit $C$ une catégorie cocomplète et $\mathcal{J}=\enstq{u_i:A_i\ra B_i}{i\in I}$ un ensemble, dans $\Omega$, de morphismes dans $C$ tels que
chaque $A_i$ soit $\N$-petit. Pour rappel, $X$ est $\N$-petit si pour tout foncteur $F:\N\ra C$,
on a que $\Hom{C}{\colim_{\N}F}{C}\ra \colim_{n\in\N}\Hom{C}{F(n)}{X}$ est une bijection.

On note $\mathcal{J}-\mathrm{Cell}$ le plus petit ensemble de morphismes
contenant $\mathcal{J}$ et stable par coproduits, colimites, et poussés en avant.
On note $\mathcal{J}^\square$ l'ensemble des morphismes qui ont la propriété de relèvement à droite par rapport
à tous les morphismes de $\mathcal{J}$.

Soit $h:X\ra Y$ un morphisme dans $C$. Nous noterons $G^1(\mathcal{J},h)$ la colimite~:
\begin{center}
  \begin{tikzcd}[column sep = large, row sep = large]
    {\displaystyle\bigsqcup_{D}A_{i_{D}}} & X \\
    {\displaystyle\bigsqcup_{D}B_{i_{D}}} & {G^1(\mathcal{J},h)}
    \arrow["{u_{i_{D}}}", from=1-1, to=2-1]
    \arrow["{f_D}", from=1-1, to=1-2]
    \arrow["h_1", from=1-2, to=2-2]
    \arrow[from=2-1, to=2-2]
    \arrow["\mathlarger{\mathlarger{\mathlarger{\mathlarger{\lrcorner}}}}"{anchor=center, rotate=180, pos=0.05}, draw=none, from=2-2, to=1-1]
  \end{tikzcd}
\end{center}
où l'union porte sur l'ensemble des diagrammes commutatifs $D$ de la forme~:
\begin{center}
  \begin{tikzcd}[column sep = large, row sep = large]
    {A_{i_{D}}} & X \\
    {B_{i_{D}}} & Y
    \arrow["{u_{i_{D}}}", from=1-1, to=2-1]
    \arrow["{f_D}", from=1-1, to=1-2]
    \arrow["h", from=1-2, to=2-2]
    \arrow["g_D", from=2-1, to=2-2]
    \arrow["\mathlarger{\mathlarger{\mathlarger{\mathlarger{\lrcorner}}}}"{anchor=center, rotate=180, pos=0.05}, draw=none, from=2-2, to=1-1]
  \end{tikzcd}
\end{center}
Les $g_D$ induisent un morphisme $p_1:G^1(\mathcal{J},h)\ra Y$.

On pose alors successivement~:
\begin{description}
  \item[] $G^{n+1}(\mathcal{J},h)=G^1(\mathcal{J},p_n)$~;
  \item[] $h_{n+1}: X\ra G^{n+1}(\mathcal{J},h)$~;
  \item[] $p_{n+1}:G^{n+1}(\mathcal{J},h)\ra Y$.
\end{description}
On note~:
\begin{description}
  \item[] $G^{\infty}(\mathcal{J},h):=\colim_n G^n(\mathcal{J},h)$~;
  \item[] $h_\infty:X\ra G^{\infty}(\mathcal{J},h)$~;
  \item[] $p_\infty:G^{\infty}(\mathcal{J},h)\ra Y$.
\end{description}

On a alors $h=p_\infty\circ h_\infty$ avec $h_\infty$ dans $\mathcal{J}-\mathrm{Cell}$ et $p_\infty$ dans $\mathcal{J}^\square$.

Chaque $G^n(\mathcal{J},-)$ et $G^\infty(\mathcal{J},-)$ sont des foncteurs $C^{0\ra 1}\ra C$ et les $h_\infty$ et $p_\infty$
s'assemblent en des transformation naturelles $(p_0:C^{0\ra 1}\ra C)\Rightarrow G^\infty(\mathcal{J},-)$ et
$G^\infty(\mathcal{J},-)\Rightarrow (p_1:C^{0\ra 1}\ra C)$.


\subsection{Carrés homotopiquement cartésiens d'ensembles simpliciaux}
\label{carrehomotopiquementcart}

Le but de cette sous-section est d'introduire la notion de carré homotopiquement cartésien.
Seront admis les notions d'adjonction de Quillen, voir \cite[1.6]{Idri}.

Nous commençons cependant par des remarques valables dans des catégories modèles générales.
Puis nous traiterons le cas particulier de la catégorie des ensembles simpliciaux.

\begin{prop}[structure de Reedy]
  Soit $C$ une catégorie modèle. On note $I$ la catégorie à $3$ objets et $2$ morphismes~:
  \begin{center}
    \begin{tikzcd}
      & 1 \\
      2 & 0
      \arrow[from=1-2, to=2-2]
      \arrow[from=2-1, to=2-2]
    \end{tikzcd}
  \end{center}
  Alors il existe une structure de modèle sur $C^I$ telle que ~:
  \begin{description}
    \item[(W)] les équivalences faibles se vérifient objet par objet~;
    \item[(C)] les équivalences faibles se vérifient objet par objet~;
    \item[(F)] un morphisme $X\ra Y$ est une fibration si et seulement si $X(0)\ra Y(0)$, $X(1)\ra X(0)\times_{Y(0)} Y(1)$
    et $X(2)\ra X(0)\times_{Y(0)} Y(2)$ sont des fibrations dans $C$.
  \end{description}
  On appelle cette structure la structure de modèle de Reedy.
\end{prop}

La démonstration est omise, car assez facile. La définition est faite pour que ça marche. On a immédiatement le corollaire suivant.

\begin{coro}
  Soit $C$ une catégorie modèle et $I$ la catégorie définie comme ci-dessus. Alors la structure de modèle de Reedy sur $C^I$ induit une adjonction
  de Quillen~:
  \begin{center}
    \begin{tikzcd}
      {\mathrm{cst}_I:C} \ar[r,bend left,""{name=A, below}] & {C^I:\lim_I} \ar[l,bend left,""{name=B,above}] \ar[from=A, to=B, phantom,"\perp"]
    \end{tikzcd}
  \end{center}
  Et donc une adjonction entre les catégories homotopiques~:
  \begin{center}
    \begin{tikzcd}
      {\mathbb{L}\:\mathrm{cst}_I:C} \ar[r,bend left,""{name=A, below}] & {\Ho{C^I}:\mathbb{R}\lim_I} \ar[l,bend left,""{name=B,above}] \ar[from=A, to=B, phantom,"\perp"]
    \end{tikzcd}
  \end{center}
\end{coro}

On remarque que si $A$ est un objet de $C$, et $QA\ra A$ un remplacement cofibrant,
alors $\mathrm{cst}_IQA\ra\mathrm{cst}_I A$ est une équivalence. Or $\mathbb{L}\:\mathrm{cst}_I A\simeq\mathrm{cst}_IQA$.
Donc $\mathbb{L}\:\mathrm{cst}_I(A)\simeq \mathrm{cst}_IA$.
Ceci justifie la définition suivante.

\begin{defi}
  Soit $C$ une catégorie modèle et soit~:
  \begin{center}
    \begin{tikzcd}
      X   & Y_1 \\
      Y_2 & Y_0
      \arrow[from=1-2, to=2-2]
      \arrow[from=2-1, to=2-2]
      \arrow[from=1-1, to=2-1]
      \arrow[from=1-1, to=1-2]
    \end{tikzcd}
  \end{center}
  un carré commutatif dans $C$.
  On dit que ce carré est homotopiquement cartésien si le morphisme dans $\Ho{C}$~:
  $$X\ra \mathbb{R}\lim_I Y$$
  adjoint à $\mathbb{L}\:\mathrm{cst}_I(X)=\mathrm{cst}_I(X)\ra Y$, est un isomorphisme (dans $\Ho{C}$).
\end{defi}

\begin{prop}\label{carrecartesienbasique}
  Soit $C$ une catégorie modèle et soit~:
  \begin{center}
    \begin{tikzcd}
      X   & Y_1 \\
      Y_2 & Y_0
      \arrow[two heads, from=1-2, to=2-2]
      \arrow[two heads, from=2-1, to=2-2]
      \arrow[from=1-1, to=2-1]
      \arrow[from=1-1, to=1-2]
      \arrow["\mathlarger{\mathlarger{\mathlarger{\mathlarger{\lrcorner}}}}"{anchor=center, pos=0.05}, draw=none, from=1-1, to=2-2]
    \end{tikzcd}
  \end{center}
  un carré commutatif cartésien dans $C$ avec $Y_0$ fibrant, et $Y_1\twoheadrightarrow Y_0$, $Y_2\twoheadrightarrow Y_0$ des cofibrations.
  Alors ce carré est homotopiquement cartésien.
\end{prop}

\begin{proof}
  Dans ce cas, le diagramme $Y$ est fibrant dans $C^I$, et donc $\lim_I Y\simeq \mathbb{R}\lim_I Y$.
\end{proof}

Nous allons maintenant nous intéresser à des catégories modèles particulières, dites propres à droite.

\begin{defi}
  Soit $C$ une catégorie modèle. On dit que $C$ est propre à droite si le tiré en arrière d'une équivalence le long
  d'une fibration est une équivalence.
\end{defi}

\begin{prop}\label{criterehomocartesienpropreadroite}
  Soit $C$ une catégorie modèle propre à droite. Alors tout carré commutatifs cartésien~:
  \begin{center}
    \begin{tikzcd}
      X   & Y_1 \\
      Y_2 & Y_0
      \arrow[two heads, from=1-2, to=2-2]
      \arrow[from=2-1, to=2-2]
      \arrow[from=1-1, to=2-1]
      \arrow[from=1-1, to=1-2]
      \arrow["\mathlarger{\mathlarger{\mathlarger{\mathlarger{\lrcorner}}}}"{anchor=center, pos=0.05}, draw=none, from=1-1, to=2-2]
    \end{tikzcd}
  \end{center}
  avec $Y_1\ra Y_0$ fibrant, est homotopiquement cartésien.
\end{prop}

\begin{proof}
  On peut choisir un remplacement fibrant $RY_0$ de $Y_0$ puis factoriser $Y_1\twoheadrightarrow Y_0\overset{\sim}{\ra} RY_0$
  en $Y_1\overset{\sim}{\ra} RY_1\twoheadrightarrow RY_0$. On a alors le diagramme suivant~:
  \begin{center}
    \begin{tikzcd}
      X && {Y_1} \\
      & {\tilde{X}} && {\tilde{Y}} & {RY_1} \\
      {Y_2} && {Y_0} \\
      &&&& {RY_0}
      \arrow[two heads, from=2-5, to=4-5]
      \arrow[two heads, from=1-3, to=3-3]
      \arrow["\sim" sloped, from=1-3, to=2-5]
      \arrow["\sim" sloped, from=3-3, to=4-5]
      \arrow[from=3-1, to=3-3]
      \arrow[two heads, from=1-1, to=3-1]
      \arrow[two heads, from=2-4, to=3-3]
      \arrow["\sim" sloped, from=2-4, to=2-5]
      \arrow["\mathlarger{\mathlarger{\mathlarger{\mathlarger{\lrcorner}}}}"{anchor=center, pos=0.05}, draw=none, from=2-4, to=4-5]
      \arrow[two heads, from=2-2, to=3-1]
      \arrow["\sim" sloped, from=1-3, to=2-4]
      \arrow[from=1-1, to=2-2]
      \arrow[from=1-1, to=1-3]
      \arrow[crossing over, from=2-2, to=2-4]
      \arrow["\mathlarger{\mathlarger{\mathlarger{\mathlarger{\lrcorner}}}}"{anchor=center, pos=0.05}, draw=none, from=2-2, to=3-3]
    \end{tikzcd}
  \end{center}
  où $\tilde{Y}$ et $\tilde{X}$ complètent les carrés cartésiens notés par $\lrcorner$. Le morphisme $\tilde{Y}\ra RY_1$
  est acyclique car $C$ est propre à droite.
  Or, dans la catégorie modèle restreinte $C\downarrow Y_0$, par le lemme de Brown (\cite[1.6.6]{Idri}), $Y_2\times_{Y_0}-$ préserve les équivalences faibles
  entre objets fibrants. Donc, comme $Y_1\ra \tilde{Y}$ est une telle équivalence, $X\ra \tilde{X}$ aussi.

  Nous nous sommes donc ramenés au cas où $X=\tilde{X}$, c'est à dire au cas où le carré~:
  \begin{center}
    \begin{tikzcd}
      {Y_1} & {RY_1} \\
      {Y_0} & {RY_0}
      \arrow[two heads, from=1-1, to=2-1]
      \arrow[two heads, from=1-2, to=2-2]
      \arrow[from=1-1, to=1-2]
      \arrow[from=2-1, to=2-2]
    \end{tikzcd}
  \end{center}
  est cartésien.

  Mais dans ce cas, on a le diagramme commutatif à carrés cartésiens suivant~:
  \begin{center}
    \begin{tikzcd}
      X & {Y_2} \\
      {X'} & {RY_2} \\
      {RY_1} & {RY_0}
      \arrow[from=2-1, to=3-1]
      \arrow[two heads, from=2-2, to=3-2]
      \arrow[two heads, from=3-1, to=3-2]
      \arrow[two heads, from=2-1, to=2-2]
      \arrow["\sim" sloped, from=1-2, to=2-2]
      \arrow[from=1-1, to=2-1]
      \arrow[two heads, from=1-1, to=1-2]
      \arrow["\mathlarger{\mathlarger{\mathlarger{\mathlarger{\lrcorner}}}}"{anchor=center, pos=0.05}, draw=none, from=1-1, to=2-2]
      \arrow["\mathlarger{\mathlarger{\mathlarger{\mathlarger{\lrcorner}}}}"{anchor=center, pos=0.05}, draw=none, from=2-1, to=3-2]
    \end{tikzcd}
  \end{center}
  où $Y_2\overset{\sim}{\ra} RY_2\twoheadrightarrow RY_0$ est une factorisation de $Y_2\ra Y_0\overset{\sim}{\ra} RY_0$.
  Comme les flèches horizontales sont des fibrations et $Y_2\ra RY_2$ est acyclique,
  $X\ra X'$ l'est aussi. Or, par la \sref{proposition}{carrecartesienbasique}, le carré du bas est homotopiquement cartésien, et comme les diagrammes
  $Y$ et $RY$ sont équivalents dans $C^I$, $X$ est la limite homotopique de $Y$.
\end{proof}

\begin{prop}\label{homocartesienfibrant}
  Soit $C$ une catégorie modèles et soit~:
  \begin{center}
    \begin{tikzcd}[column sep = small, row sep = small]
      X   && Y_1 \\
      \\
      Y_2 && Y_0 \\
       &  *   &
      \arrow[two heads,from=1-3, to=3-3]
      \arrow["\sim", from=3-1, to=3-3]
      \arrow[from=1-1, to=3-1]
      \arrow[from=1-1, to=1-3]
      \arrow[two heads, from=3-1, to=4-2]
      \arrow[two heads, from=3-3, to=4-2]
      \arrow["\mathlarger{\mathlarger{\mathlarger{\mathlarger{\lrcorner}}}}"{anchor=center, pos=0.05}, draw=none, from=1-1, to=3-3]
    \end{tikzcd}
  \end{center}
  un carré cartésien dans $C$ avec $Y_1\ra Y_0$ une fibration, $Y_2\ra Y_0$ acyclique, et $Y_0$, $Y_2$ fibrants.

  Alors $X\ra Y_1$ est aussi acyclique.

  En d'autres termes, le tiré en arrière d'une équivalence entre objets fibrants le long d'une fibration est une équivalence.
\end{prop}

\begin{rem}
  Il existe évidemment une version duale (prendre $C\op$)~: le poussé en avant d'une équivalence entre objets cofibrants le long
  d'une cofibration est une équivalence.
\end{rem}

La preuve de cette proposition est omise, voir \cite[13.1.2]{Hirs}.

\begin{prop}
  La catégorie des ensembles simpliciaux $\DEns$ est propre à droite.
\end{prop}

\begin{proof}
  La catégorie $\Top$ des espaces topologiques avec la structure de Quillen est propre à droite.
  En effet, comme tous les objets de $\Top$ sont fibrants, c'est une conséquence de la \sref{porposition}{homocartesienfibrant}.

  Or le foncteur réalisation $|\bullet|:\DEns\ra \Top$ préserve les limites finies et les fibrations, et reflète les équivalences
  (voir \cite[Chp.1]{Goer}).
  Donc on en déduit que $\DEns$ est également propre à droite.
\end{proof}

Voici une caractérisation plus concrète des diagrammes homotopiquement cartésiens dans $\DEns$.

\begin{prop}
  On se donne un carré commutatif~:
  \begin{center}
    \begin{tikzcd}
      X   & Y_1 \\
      Y_2 & Y_0
      \arrow[from=1-2, to=2-2]
      \arrow[from=2-1, to=2-2]
      \arrow[from=1-1, to=2-1]
      \arrow[from=1-1, to=1-2]
    \end{tikzcd}
  \end{center}
  Alors ce carré est homotopiquement cartésien si et seulement si le morphisme induit~:
  $$X\ra Y_1\times_{Y_0}Y_0^{\Delta^1}\times_{Y_0} Y_2$$
  est une équivalence faible.
\end{prop}

\begin{proof}
  On dispose d'une factorisation~:
  $$Y_1\overset{\sim}{\hookrightarrow} Y_1\times_{Y_0}Y_0^{\Delta^1} \overset{\mathrm{ev}_1}{\twoheadrightarrow} Y_0$$
  On a donc le diagramme suivant~:
  \begin{center}
    \begin{tikzcd}
      X && {Y_1} \\
      {Y_1\times_{Y_0}Y_0^{\Delta^1}\times_{Y_0}Y_2} && {Y_1\times_{Y_0}Y_0^{\Delta^1}} \\
      {Y_2} && {Y_0}
      \arrow[from=3-1, to=3-3]
      \arrow["\sim" sloped, hook, from=1-3, to=2-3]
      \arrow[two heads, from=2-3, to=3-3]
      \arrow[from=1-1, to=1-3]
      \arrow[from=1-1, to=2-1]
      \arrow[from=2-1, to=3-1]
      \arrow[from=2-1, to=2-3]
      \arrow["\mathlarger{\mathlarger{\mathlarger{\mathlarger{\lrcorner}}}}"{anchor=center, pos=0.05}, draw=none, from=2-1, to=3-3]
    \end{tikzcd}
  \end{center}
  Le carré du bas est cartésien et homotopiquement cartésien par la \sref{proposition}{criterehomocartesienpropreadroite}.
  On note $Y'$ l'élément de $C^I$ associé au carré du bas.
  Le morphisme $X\ra Y_1\times_{Y_0}Y_0^{\Delta^1}\times_{Y_0}Y_2$ dans $C$ induit dans $\Ho{C^I}$ le diagramme commutatif~:
  \begin{center}
    \begin{tikzcd}
      \mathbb{L}\:\mathrm{cst}_I X & \mathbb{L}\:\mathrm{cst}_I Y_1\times_{Y_0}Y_0^{\Delta^1}\times_{Y_0}Y_2 \\
      Y                        & Y'
      \arrow[from=1-2, to=2-2]
      \arrow["\sim", from=2-1, to=2-2]
      \arrow[from=1-1, to=2-1]
      \arrow[from=1-1, to=1-2]
    \end{tikzcd}
  \end{center}
  et, en passant à l'adjoint $\mathbb{R}\lim_I$, le diagramme commutatif dans $\Ho{C}$~:
  \begin{center}
    \begin{tikzcd}
      X                  & Y_1\times_{Y_0}Y_0^{\Delta^1}\times_{Y_0}Y_2 \\
      \mathbb{R}\lim_I Y & \mathbb{R}\lim_I Y'
      \arrow["\sim" sloped, from=1-2, to=2-2]
      \arrow["\sim", from=2-1, to=2-2]
      \arrow[from=1-1, to=2-1]
      \arrow[from=1-1, to=1-2]
    \end{tikzcd}
  \end{center}
  On a donc bien $X\simeq \mathbb{R}\lim_I Y$ si et seulement si $X\simeq Y_1\times_{Y_0}Y_0^{\Delta^1}\times_{Y_0}Y_2$.
\end{proof}

\subsection{Le lemme de collage}
On se place ici dans une catégorie modèle $C$.

\begin{lem}[lemme de collage]\label{lemmedecollage}
  On se donne le diagramme suivant dans $C$~:
  \begin{center}
    \begin{tikzcd}
      {A_1} && {B_1} \\
      & {C_1} && {D_1} \\
      {A_2} && {B_2} \\
      & {C_2} && {D_2}
      \arrow["{f_A}"', from=1-1, to=3-1]
      \arrow["{f_B}"{pos=0.7}, from=1-3, to=3-3]
      \arrow["{j_1}", from=1-1, to=1-3]
      \arrow["{j_2}"{pos=0.3}, from=3-1, to=3-3]
      \arrow[hook, from=1-3, to=2-4]
      \arrow[hook, from=3-3, to=4-4]
      \arrow["{i_2}", hook, from=3-1, to=4-2]
      \arrow["{i_1}", hook, from=1-1, to=2-2]
      \arrow["{f_D}", from=2-4, to=4-4]
      \arrow[from=4-2, to=4-4]
      \arrow[crossing over, from=2-2, to=2-4]
      \arrow[crossing over, "{f_C}"{pos= 0.2}, from=2-2, to=4-2]
      \arrow["\mathlarger{\mathlarger{\mathlarger{\mathlarger{\lrcorner}}}}"{anchor=center, pos=0.1, rotate=200}, draw=none, from=2-4, to=1-1]
      \arrow["\mathlarger{\mathlarger{\mathlarger{\mathlarger{\lrcorner}}}}"{anchor=center, pos=0.1, rotate=200}, draw=none, from=4-4, to=3-1]
    \end{tikzcd}
  \end{center}
  où tous les objets sont cofibrants, $i_1$ et $i_2$ sont des cofibrations, les faces supérieure et inférieure sont cocartésiennes
  et $f_A$, $f_B$ et $f_C$ sont acycliques.

  Alors $f_D$ est acyclique.
\end{lem}

\begin{proof}
  On factorise de façon fonctorielle $j_1=q_1\circ j'_1$ et $j_2=q_2\circ j'_2$ où $j'_1$ et $j'_2$ sont des cofibrations
  et $q_1$ et $q_2$ sont des fibrations acycliques. En coupant le cube en deux, On est alors ramené aux deux cas particuliers suivants~:
  $j_1$ et $j_2$ sont acycliques~;
  $j_1$ et $j_2$ sont des cofibrations.

  Le premier cas est une conséquence de la \sref{proposition}{homocartesienfibrant}. Nous allons maintenant traiter le second cas.
  On note $B'$ (respectivement $D'$) le poussé en avant de $j_1$ et $f_A$ (respectivement $C_1\ra C_2$ et $C_1\ra D_1$).
  On a alors le diagramme suivant~:
  \begin{center}
    \begin{tikzcd}
      {A_1} &&& {B_1} \\
      & {C_1} &&& {D_1} \\
      &&& {B'} \\
      {A_2} &&& {B_2} & {D'} \\
      & {C_2} &&& {D_2}
      \arrow["{i_1}", hook, from=1-1, to=2-2]
      \arrow[hook, from=1-4, to=2-5]
      \arrow["\sim"' sloped, from=1-1, to=4-1]
      \arrow["\sim\;\;\;\;\;\;\;\;\;" sloped, from=1-4, to=3-4]
      \arrow["{n_B}"', "\sim" sloped, from=3-4, to=4-4]
      \arrow[from=2-5, to=4-5]
      \arrow["{n_D}", from=4-5, to=5-5]
      \arrow["\theta", from=3-4, to=4-5]
      \arrow[hook, from=4-4, to=5-5]
      \arrow["{i_2}", hook, from=4-1, to=5-2]
      \arrow[hook, from=5-2, to=5-5]
      \arrow["{j_2}", hook, from=4-1, to=4-4]
      \arrow["{j_1}", hook, from=1-1, to=1-4]
      \arrow[from=4-1, to=3-4]
      \arrow[crossing over, from=5-2, to=4-5]
      \arrow[crossing over, hook, from=2-2, to=2-5]
      \arrow[crossing over, "\sim\;\;\;\;\;\;\;\;\;" sloped, from=2-2, to=5-2]
    \end{tikzcd}
  \end{center}
  Par la \sref{proposition}{homocartesienfibrant}, $B_1\ra B'$ et $D_1\ra D'$ sont des équivalences faibles, donc $n_B$ est acyclique.
  Il reste à montrer que $n_D$ est aussi acyclique.
  Or $\theta$ est le poussé en avant de $i_2$ le long de $A_2\ra B'$, c'est donc une cofibration. Comme $n_D$ est le poussé en avant de $n_B$ le long
  de $\theta$, par la \sref{proposition}{homocartesienfibrant}, $n_D$ est acyclique.
\end{proof}

\section{La correspondance de Dold-Kan}
\label{DoldKan}

Cette sous-section est consacrée à la correspondance de Dold-Kan et à certaines conséquences en homotopie.
Certaines démonstrations sont omises. Des références sont données.

\begin{defi}
  On note $\Chp$ la catégorie des complexes de groupes abéliens concentrés en degrés positifs pour la notation homologique.
  On note $\DAb$ la catégorie des groupes abéliens simpliciaux, ie. des foncteurs $\DCat\ra \Ab$.
\end{defi}

L'objet de la correspondance de Dold-Kan est d'établir une équivalence de catégories abéliennes entre $\Chp$ et $\DAb$.

\begin{rem}
  On peut également voir $\DAb$ comme la catégorie des groupes abéliens dans la catégorie des ensembles simpliciaux $\DEns$.
\end{rem}

\begin{propdefi}\label{definitioncomplexenormalise}
  Soit $A$ un objet de $\DAb$. Le complexe normalisé $(NA_*,\partial_*)$ de $A$ est le complexe de chaîne de $\Chp$ suivant~:
  $$NA_n:=\bigcap_{m=0}^{n-1}\myker{d_m}\;\mid\;\partial_n=(-1)^nd_n:NA_n\ra NA_{n-1}$$
  C'est un sous-complexe du complexe $(CA_*,\partial_*)$ défini par~:
  $$CA_n:=A_n;\partial_n=\sum_{m=0}^n(-1)^md_m:CA_n\ra CA_{n-1}$$
  Le complexe des dégénérescences $DA_*$ de $A$ est le sous complexe de $CA$ défini par~:
  $$DA_n:=\sum_{m=0}^{n-1}s_m(A_{n-1})$$
\end{propdefi}

\begin{proof}
  Le calcul pour vérifier que $CA_*$ est un complexe est classique. Vérifions que $NA_*$ et $DA_*$ sont des sous-complexes.
  Pour $0\leq i\leq n-2$, $d_id_n=d_{n-1}d_i$ annule $NA_n$. Donc $NA_*$ est un sous-complexe.
  Pour $0\leq i\leq n-1$~:
  $$\partial_ns_i=\sum_{m=0}^{n-1}(-1)^md_ms_i=\sum_{m=0}^{i-1}s_{i-1}d_m+(-1)^{i}\id+(-1)^{i+1}\id+\sum_{m=i+2}^{n}s_{i}d_{m+1}$$
  Donc $\partial_ns_i(a)\in DA_{n-1}$ pour $a\in DA_n$.
\end{proof}

\begin{rem}
  Ces constructions étant fonctorielles, on a donc construit $3$ foncteurs $N$, $C$ et $D$ de $\DAb$ dans $\Chp$.
\end{rem}

\begin{lem}\label{lemmeDoldKan}
  L'inclusion naturelle $N\oplus D \ra C$ est un isomorphisme naturel. Donc les foncteurs $N$ et $C/D$ coïncident.
  En particulier, le foncteur $N$ est exact.
\end{lem}

Nous admettons ici ce lemme, qui se démontre par récurrence. Pour plus de détails, voir \cite[III.2.1]{Goer}

\begin{rem}
  On voit ici deux façons de penser au complexe normalisé $NA$~: la définition avec les noyaux donnée dans la \sref{définition}{definitioncomplexenormalise},
  ou le quotient $CA/DA$.
\end{rem}

\begin{propdefi}
  Soit $C$ un complexe de chaîne dans $\Chp$. On note $\Gamma C$ le groupe abélien simplicial définit par~:
  $$\Gamma C_n:=\bigoplus_{\phi:[n]\twoheadrightarrow [r]}C_r$$
  Pour $\chi:[m]\ra [n]$, le morphisme $\chi^*:\Gamma C_n \ra \Gamma C_m$ est construit comme suit. Soit $\phi:[n]\twoheadrightarrow [r]$.
  Alors $\phi\circ \chi$ se factorise uniquement en~:
  \begin{center}
    \begin{tikzcd}
      {[m]} \arrow[r,"\psi", two heads] & {[s]} \arrow[r, "\mu", hook] & {[r]}
    \end{tikzcd}
  \end{center}
  Alors $\chi_*$ est donné sur la coordonnée en $\phi$ par le morphisme $C_r\overset{\mu^*}{\ra} C_s\overset{\psi}{\hookrightarrow}\Gamma C_m$,
  où $\mu^*=0$ sauf si $\mu=d^r$, auquel cas $\mu^*=(-1)^r\partial_r$.
\end{propdefi}

\begin{rem}
  la définition ci-dessus est plus clair si on voit $C$ comme un ensemble pré-simplicial. Un ensemble pré-simplicial est un préfaisceau
  sur la catégorie $\DCat'$ des injections $[u]\hookrightarrow [v]$ de $\DCat$. La structure d'ensemble pré-simplicial est celle décrite dans la définition.
  La construction de $\Gamma$ est alors basée sur celle, classique, de l'adjoint à gauche de l'oubli $[\DCat,\Ab]\ra [\DCat',\Ab]$.
  C'est ce qui est fait dans la démonstration ci-dessous.
\end{rem}

\begin{proof}
  Au vu de la remarque ci-dessus, il suffit de montrer que si $C$ est un ensemble pré-simplicial, alors la construction de $\Gamma C$ est correcte.
  Il s'agit de vérifier la compatibilité de la construction avec les compositions dans $\DCat$.
  Soit $\chi:[m]\ra [n]$ et $\rho:[l]\ra [m]$. Pour $\phi:[n]\twoheadrightarrow [r]$, on a le diagramme commutatif suivant dans $\DCat$~:
  \begin{center}
    % https://q.uiver.app/?q=WzAsNixbMCwwLCJbbF0iXSxbMSwwLCJbbV0iXSxbMiwwLCJbbl0iXSxbMiwxLCJbcl0iXSxbMSwxLCJbc10iXSxbMCwxLCJbdF0iXSxbMCwxLCJcXHJobyJdLFsxLDIsIlxcY2hpIl0sWzIsMywiXFxwaGkiLDAseyJzdHlsZSI6eyJoZWFkIjp7Im5hbWUiOiJlcGkifX19XSxbMSw0LCJcXHBzaSIsMCx7InN0eWxlIjp7ImhlYWQiOnsibmFtZSI6ImVwaSJ9fX1dLFswLDUsIlxcdGhldGEiLDAseyJzdHlsZSI6eyJoZWFkIjp7Im5hbWUiOiJlcGkifX19XSxbNCwzLCJcXG11IiwyLHsic3R5bGUiOnsidGFpbCI6eyJuYW1lIjoiaG9vayIsInNpZGUiOiJ0b3AifX19XSxbNSw0LCJcXG51IiwyLHsic3R5bGUiOnsidGFpbCI6eyJuYW1lIjoiaG9vayIsInNpZGUiOiJ0b3AifX19XV0=
    \begin{tikzcd}
    	{[l]} & {[m]} & {[n]} \\
    	{[t]} & {[s]} & {[r]}
    	\arrow["\rho", from=1-1, to=1-2]
    	\arrow["\chi", from=1-2, to=1-3]
    	\arrow["\phi", two heads, from=1-3, to=2-3]
    	\arrow["\psi", two heads, from=1-2, to=2-2]
    	\arrow["\theta", two heads, from=1-1, to=2-1]
    	\arrow["\mu"', hook, from=2-2, to=2-3]
    	\arrow["\nu"', hook, from=2-1, to=2-2]
    \end{tikzcd}
  \end{center}
  On a donc que sur la coordonnée en $\phi$, $\rho^*\chi^*$ est donné par $\nu^*\mu^*$ et $(\chi\rho)^*$ est donné par $(\mu\nu)^*$.
  Or $\nu^*\mu^*=(\mu\nu)^*$ car $C$ est pré-simplicial.
\end{proof}

\begin{theo}[équivalence de Dold-Kan]
  Les foncteurs $\Gamma:\Chp\ra\DAb$ et $N:\DAb\ra\Chp$ sont inverses et exactes. Ils induisent une équivalence de catégories
  abéliennes $\Chp\simeq\DAb$, appelée équivalence de Dold-Kan.
\end{theo}

\begin{proof}
  Soit $C$ un complexe dans $\Chp$. On cherche à identifier $(N\Gamma C)_n$ comme sous-groupe de~:
  $$\Gamma C_n=\bigoplus_{\phi:[n]\twoheadrightarrow [r]}C_r$$
  Or, un calcul direct montre que $C_n\subseteq (N\Gamma C)_n$ et $\bigoplus_{[n]\twoheadrightarrow [r],r<n}C_r\subseteq (D\Gamma C)_n$.
  Donc, par le \sref{lemme}{lemmeDoldKan}, $C_n=(N\Gamma C)_n$. On a l'isomorphisme souhaité.
  
  Soit $A$ un groupe abélien simplicial. On a un morphisme naturel de groupes pré-simpliciaux $N A\ra A$,
  et donc un morphisme naturel de groupes simpliciaux $\Gamma N A\ra A$. Soit $n\geq 0$. Par récurrence, on suppose que
  $(\Gamma N A)_m\ra A_m$ est un isomorphisme pour $m<n$. Alors l'image de $(\Gamma N A)_n\ra A_n$ contient $NA_n$
  et $DA_n$. Donc par le \sref{lemme}{lemmeDoldKan}, l'application est surjective. Maintenant, soit
  $x=\sum_{\phi} a_{\phi}$ dans le noyau. Soit $r$ maximal tel qu'il existe $\phi:[n]\twoheadrightarrow [r]$ tel que $a_\phi \neq 0$.
  Soit $\phi:[n]\twoheadrightarrow [r]$ maximal pour l'ordre lexicographique tel que $a_\phi \neq 0$. Soit $\mu:[r]\hookrightarrow [n]$
  une section de $\phi$, minimale pour l'ordre lexicographique. Alors $\psi\circ \mu$ n'est pas injectif pour tout
  $\psi:[n]\twoheadrightarrow [r]$ différent de $\phi$ tel que $a_\psi\neq 0$. Donc tous les termes de $\mu^*x$ sont de degré $\leq r$ et le seul terme de degré $r$
  est $\mu^*a_\phi=a_\phi\in NA_r\overset{\id_{[r]}}{\hookrightarrow}(\Gamma N A)_r$. Or, $N A_r\hookrightarrow A_r$. Donc $a_\phi=0$.
  On a donc nécessairement $x=0$. Donc $(\Gamma N A)_n\ra A_n$ est un isomorphisme.
\end{proof}

Nous allons maintenant nous intéresser à une structure de catégorie modèle sur $\DAb$, et la structure correspondante dans $\Chp$.
Nous reprenons \cite[III.2]{Goer}.
Introduisons d'abords les équivalences faibles.

\begin{lem}\label{lemmeN}
  Soit $A$ un groupe abélien simplicial. Alors $DA$ est un complexe acyclique, ie. $H_*(DA)=0$.
  En particulier, $NA\hookrightarrow CA$ et $A\twoheadrightarrow CA/DA$ sont des isomorphismes en homologie.
\end{lem}

Nous admettons ici le lemme, il s'agit de construire par récurrence une homotopie entre $\id_{DA}$ et $0$.
Une autre approche équivalente est donnée dans \cite[II.2.4]{Goer}.

\begin{prop}\label{pinegalHn}
  Soit $n\geq 0$. Il existe un isomorphisme naturel de groupes pour $A$ dans $\DAb$~:
  $$\pi_n(A,0)\simeq H_n(NA)$$
  De même, pour $n\geq 1$, il existe un isomorphisme naturel pour $A\subset B$ paire dans $\DAb$~:
  $$\pi_n(B,A,0)\simeq H_n(NB/NA)$$
  Et ces isomorphismes sont compatibles aux suites exactes longues des paires, ie. le diagramme suivant est commutatif~:
  \begin{center}
    % https://q.uiver.app/?q=WzAsOCxbMCwwLCJcXHBpX3tuKzF9KEIsQSwwKSJdLFsxLDAsIlxccGlfbihBLDApIl0sWzIsMCwiXFxwaV9uKEIsMCkiXSxbMywwLCJcXHBpX24oQixBLDApIl0sWzAsMSwiSF97bisxfShCL0EpIl0sWzEsMSwiSF9uKEEpIl0sWzIsMSwiSF9uKEIpIl0sWzMsMSwiSF9uKEIvQSkiXSxbMyw3LCJcXHNpbSJdLFsyLDYsIlxcc2ltIl0sWzEsNSwiXFxzaW0iXSxbMCw0LCJcXHNpbSJdLFswLDEsIlxccGFydGlhbCJdLFs0LDUsIlxccGFydGlhbCJdLFsxLDJdLFs1LDZdLFsyLDNdLFs2LDddXQ==
    \begin{tikzcd}[column sep = 8ex]
      {\pi_{n+1}(B,A,0)} & {\pi_n(A,0)} & {\pi_n(B,0)} & {\pi_n(B,A,0)} \\
      {H_{n+1}(B/A)} & {H_n(A)} & {H_n(B)} & {H_n(B/A)}
      \arrow["\sim" sloped, from=1-4, to=2-4]
      \arrow["\sim" sloped, from=1-3, to=2-3]
      \arrow["\sim" sloped, from=1-2, to=2-2]
      \arrow["\sim" sloped, from=1-1, to=2-1]
      \arrow["(-1)^{n+1}\partial_\pi", from=1-1, to=1-2]
      \arrow["\partial_H", from=2-1, to=2-2]
      \arrow[from=1-2, to=1-3]
      \arrow[from=2-2, to=2-3]
      \arrow[from=1-3, to=1-4]
      \arrow[from=2-3, to=2-4]
    \end{tikzcd}
  \end{center}
\end{prop}

\begin{proof}
  On utilise ici extensivement que les éléments de $\DAb$ sont des complexes de Kan, voir \cite[I.3.4]{Goer}.
  Soit $A$ dans $\DAb$. On remarque tout d'abord que l'addition sur $A$ induit, par Eckmann–Hilton, la structure de groupe sur $\pi_n(A,0)$.
  Alors, les simplexes $\sigma\in A_n$ qui représentent les éléments de $\pi_n(A,0)$ sont les éléments de $Z_n(NA)$.
  Et un simplexe $\sigma$ représente $0$ si et seulement si il existe $\tau\in NA_{n+1}$ tel que $d_n\tau = \sigma$.
  On a donc bien l'isomorphisme naturel souhaité.

  Soit $A\subseteq B$ paire dans $\DAb$. Alors, les simplexes $\sigma\in NB_n$ qui représentent les éléments
  de $\pi_n(B,A,0)$ sont les éléments dont l'image est dans $Z_n(NB/NA)$. On remarque également que les éléments $\sigma\in A_n$ qui représentent
  $0$ dans $\pi_n(B,A,0)$ sont dans $NA_n$. Maintenant, les éléments $\sigma\in NB_n$ qui représentent $0$ dans $\pi_n(B,A,0)$ sont
  ceux tels qu'il existe $\tau\in B_n$ tel que $\partial\tau=(0,\dotsc,0,\sigma,\mu)$ avec $\mu$ dans $NA_n$.
  Ceci induit l'isomorphisme souhaité entre $\pi_n(B,A,0)$ et $H_n(NB/NA)$.

  Pour la suite exacte, il suffit de remarquer que si $\sigma\in NB_{n+1}$ représente un éléments de $\pi_{n+1}(B,A,0)$,
  alors $\partial_\pi \sigma = [d_{n+1}(\sigma)] = (-1)^{n+1}\partial_H\sigma$.
\end{proof}

Introduisons maintenant la structure de modèle sur $\DAb$.

\begin{prop}
  Il existe une structure de catégorie modèle sur $\DAb$ telle que~:
  \begin{description}
    \item[(W)] les équivalences faibles soient les applications qui induisent des isomorphismes sur tous les groupes d'homotopie~;
    \item[(F)] les fibrations soient les applications dont l'application sous-jacente entre ensembles simpliciaux soit une fibration de Kan. 
  \end{description}
  Cette structure est cofibrement engendré par les cofibrations~:
  $$\mathcal{I}:=\enstq{\Z\partial\Delta^n\ra \Z\Delta^n}{n\geq 0}$$
  et les cofibrations acycliques~:
  $$\mathcal{J}:=\enstq{\Z\Lambda_k^n\ra \Z\Delta^n}{n\geq 1,0\leq k\leq n}$$
\end{prop}

\begin{proof}
  On vérifie aisément que les morphismes de $\mathcal{J}^\square$ sont les fibrations de Kan,
  et les morphismes de $\mathcal{I}^\square$ les fibrations de Kan acycliques.
  On a alors immédiatement que $\mathcal{I}\subset \prescript{\square}{}{(\mathcal{J}^\square)}$ et que
  $\mathcal{J}^\square\cap W=\mathcal{I}^\square$.
  Il reste donc à montrer que $\mathcal{J}\subset W$. Soit $n\geq 1$ et $0\leq k\leq n$. On dispose d'une application
  $\Delta^n\times\Delta^1\ra\Delta^n$ homotopie entre $\id_{\Delta^n}$ et $n$ définie par~:
  \begin{center}
    % https://q.uiver.app/?q=WzAsMTAsWzAsMCwiMCJdLFsyLDAsImstMSJdLFszLDAsImsiXSxbNCwwLCJrKzEiXSxbNiwwLCJuIl0sWzAsMSwibiJdLFsyLDEsIm4iXSxbMywxLCJuIl0sWzQsMSwibiJdLFs2LDEsIm4iXSxbMCwxLCIiLDAseyJzdHlsZSI6eyJib2R5Ijp7Im5hbWUiOiJkb3R0ZWQifX19XSxbNSw2LCIiLDAseyJzdHlsZSI6eyJib2R5Ijp7Im5hbWUiOiJkb3R0ZWQifX19XSxbMyw0LCIiLDAseyJzdHlsZSI6eyJib2R5Ijp7Im5hbWUiOiJkb3R0ZWQifX19XSxbOCw5LCIiLDAseyJzdHlsZSI6eyJib2R5Ijp7Im5hbWUiOiJkb3R0ZWQifX19XSxbMSwyXSxbMiwzXSxbNiw3XSxbNyw4XSxbMCw1XSxbMSw2XSxbMiw3XSxbMyw4XSxbNCw5XV0=
    \begin{tikzcd}
    	0 && {k-1} & k & {k+1} && n \\
    	n && n & n & n && n
    	\arrow[dotted, from=1-1, to=1-3]
    	\arrow[dotted, from=2-1, to=2-3]
    	\arrow[dotted, from=1-5, to=1-7]
    	\arrow[dotted, from=2-5, to=2-7]
    	\arrow[from=1-3, to=1-4]
    	\arrow[from=1-4, to=1-5]
    	\arrow[from=2-3, to=2-4]
    	\arrow[from=2-4, to=2-5]
    	\arrow[from=1-1, to=2-1]
    	\arrow[from=1-3, to=2-3]
    	\arrow[from=1-4, to=2-4]
    	\arrow[from=1-5, to=2-5]
    	\arrow[from=1-7, to=2-7]
    \end{tikzcd}
  \end{center}
  Si $k\neq n$, cette homotopie se restreint en $\Lambda_k^n\times\Delta^1\ra \Lambda_k^n$. On dispose également d'une homotopie entre
  $0$ et $\id_{\Delta^n}$ donnée par~:
  \begin{center}
    % https://q.uiver.app/?q=WzAsMTAsWzAsMCwiMCJdLFsyLDAsIjAiXSxbMywwLCIwIl0sWzQsMCwiMCJdLFs2LDAsIjAiXSxbMCwxLCIwIl0sWzIsMSwiay0xIl0sWzMsMSwiayJdLFs0LDEsImsrMSJdLFs2LDEsIm4iXSxbMCwxLCIiLDAseyJzdHlsZSI6eyJib2R5Ijp7Im5hbWUiOiJkb3R0ZWQifX19XSxbNSw2LCIiLDAseyJzdHlsZSI6eyJib2R5Ijp7Im5hbWUiOiJkb3R0ZWQifX19XSxbMyw0LCIiLDAseyJzdHlsZSI6eyJib2R5Ijp7Im5hbWUiOiJkb3R0ZWQifX19XSxbOCw5LCIiLDAseyJzdHlsZSI6eyJib2R5Ijp7Im5hbWUiOiJkb3R0ZWQifX19XSxbMSwyXSxbMiwzXSxbNiw3XSxbNyw4XSxbMCw1XSxbMSw2XSxbMiw3XSxbMyw4XSxbNCw5XV0=
    \begin{tikzcd}
      0 && 0 & 0 & 0 && 0 \\
      0 && {k-1} & k & {k+1} && n
      \arrow[dotted, from=1-1, to=1-3]
      \arrow[dotted, from=2-1, to=2-3]
      \arrow[dotted, from=1-5, to=1-7]
      \arrow[dotted, from=2-5, to=2-7]
      \arrow[from=1-3, to=1-4]
      \arrow[from=1-4, to=1-5]
      \arrow[from=2-3, to=2-4]
      \arrow[from=2-4, to=2-5]
      \arrow[from=1-1, to=2-1]
      \arrow[from=1-3, to=2-3]
      \arrow[from=1-4, to=2-4]
      \arrow[from=1-5, to=2-5]
      \arrow[from=1-7, to=2-7]
    \end{tikzcd}
  \end{center}
  Si $k\neq 0$, cette homotopie se restreint en $\Lambda_k^n\times\Delta^1\ra \Lambda_k^n$. Maintenant, en passant au foncteur groupe abélien libre,
  on obtient deux applications $\Z\Delta^n\otimes\Z\Delta^1\ra\Z\Delta^n$ et $\Z\Lambda_k^n\otimes\Z\Delta^1\ra\Z\Lambda_k^n$.
  Pour conclure, il suffit de montrer que si $X$ est un ensemble simplicial, alors on a le cône suivant dans $\DEns$~:
  \begin{center}
    \begin{tikzcd}
      \Z X\sqcup \Z X \arrow[r] & \Z X\otimes \Z\Delta^1 \arrow["\sim"',"\mathrm{pr}_1",r] & \Z X
    \end{tikzcd}
  \end{center}
  En effet, on aura alors que $\id_{\Z\Delta^n}$ et $\id_{\Z\Lambda_k^n}$ sont homotopes à des applications constantes.
  On utilise la \sref{proposition}{pinegalHn} et le \sref{lemme}{lemmeN}. On omettra $C$ dans la notation du foncteur $C:\DAb\ra\Chp$.
  On se ramène alors à montrer que $\mathrm{pr}_1$ est un isomorphisme
  en homologie. Tout d'abord, $H_*(\Z\Delta^1)=H_*(N\Z\Delta^1)$. Or $N\Z\Delta^1$ est le complexe~:
  \begin{center}
    % https://q.uiver.app/?q=WzAsNSxbMCwwLCJcXFpcXG9wbHVzIFxcWiJdLFsxLDAsIlxcWiJdLFsyLDAsIjAiXSxbNCwwXSxbMywwXSxbMSwwLCIoMSwtMSkiLDJdLFsyLDFdLFs0LDIsIiIsMix7InN0eWxlIjp7ImJvZHkiOnsibmFtZSI6ImRvdHRlZCJ9fX1dXQ==
    \begin{tikzcd}
    	{\Z\oplus \Z} & \Z & 0 & {} & {}
    	\arrow["{(1,-1)}"', from=1-2, to=1-1]
    	\arrow[from=1-3, to=1-2]
    	\arrow[dotted, from=1-4, to=1-3]
    \end{tikzcd}
  \end{center}
  Donc, $H_*(\Z \Delta^1)$ est $\Z$ concentré en degré $0$. Or, $\Z \Delta^1$ est un complexe de projectifs.
  Donc $\Z \Delta^1\ra \Z$ est une résolution projective de $\Z$.
  Donc il existe $u:\Z\Delta^1\ra \Z[0]$, $v:\Z[0]\ra \Z \Delta^1$ et des homotopies $u\circ v\Rightarrow \id_{\Z[0]}$
  et $v\circ u\Rightarrow \id_{\Z\Delta^1}$.
  Tout cela induit une équivalence d'homotopie entre $\Z X\otimes \Z\Delta^1$ et $\Z X$.
\end{proof}

\begin{lem}\label{lemmeDoldKanHomotopie}
  Soit $f:A\ra B$ une application dans $\DAb$. Alors~:
  \begin{description}
    \item[$(1)$] Si $f$ est surjective, alors $f$ est une fibration~;
    \item[$(2)$] $f$ est surjective si et seulement si $Nf:NA\ra NB$ est surjective~;
    \item[$(3)$] $f$ est une fibration si et seulement si $Nf_n:NA_n\ra NB_n$ est surjective pour tout $n\geq 1$.  
  \end{description}
\end{lem}

\begin{proof}
  \begin{description}
    \item[$(1)$] Soit $u:\Lambda_k^n\ra A$ et $v:\Delta^n\ra B$. Comme $f$ est surjective, il existe $w:\Delta^n\ra A$ relevant $v$.
                On note $u-\partial_kw:\Lambda_k^n\ra \myker{f}$. Or, $\myker{f}$ est une complexe de Kan (\cite[I.3.4]{Goer}).
                Donc il existe $s:\Delta_k^n\ra \myker{f}$ tel que $\partial_ks= u-\partial_kw$. Alors $w+s$ convient.
    \item[$(2)$] Ceci découle du fait que la correspondance de Dold-Kan est exacte et $f=\Gamma Nf$.
    \item[$(3)$] Pour l'implication $(\Rightarrow)$, on se donne, pour $u\in NB_n$, le carré suivant~:
                \begin{center}
                  % https://q.uiver.app/?q=WzAsNCxbMCwwLCJcXExhbWJkYV9uXm4iXSxbMSwwLCJBIl0sWzEsMSwiQiJdLFswLDEsIlxcRGVsdGFebiJdLFswLDMsIlxcc2ltIiwwLHsic3R5bGUiOnsidGFpbCI6eyJuYW1lIjoiaG9vayIsInNpZGUiOiJ0b3AifX19XSxbMSwyLCIiLDAseyJzdHlsZSI6eyJoZWFkIjp7Im5hbWUiOiJlcGkifX19XSxbMCwxLCIwIl0sWzMsMiwidSJdLFszLDEsIlxcZXhpc3RzIHYiLDAseyJzdHlsZSI6eyJib2R5Ijp7Im5hbWUiOiJkYXNoZWQifX19XV0=
                  \begin{tikzcd}
                    {\Lambda_n^n} & A \\
                    {\Delta^n} & B
                    \arrow["\sim" sloped, hook, from=1-1, to=2-1]
                    \arrow[two heads, from=1-2, to=2-2]
                    \arrow["0", from=1-1, to=1-2]
                    \arrow["u", from=2-1, to=2-2]
                    \arrow["{\exists v}" sloped, dashed, from=2-1, to=1-2]
                  \end{tikzcd}
                \end{center}
                Pour l'implication $(\Leftarrow)$, on regarde le diagramme suivant~:
                \begin{center}
                  % https://q.uiver.app/?q=WzAsNSxbMSwxLCJcXHBpXzAoQSlcXHRpbWVzX3tcXHBpXzAoQil9QiJdLFsyLDEsIkIiXSxbMiwyLCJcXHBpXzAoQikiXSxbMSwyLCJcXHBpXzAoQSkiXSxbMCwwLCJBIl0sWzMsMiwiIiwwLHsic3R5bGUiOnsiaGVhZCI6eyJuYW1lIjoiZXBpIn19fV0sWzEsMl0sWzAsMSwiIiwyLHsic3R5bGUiOnsiaGVhZCI6eyJuYW1lIjoiZXBpIn19fV0sWzAsM10sWzAsMiwiIiwxLHsic3R5bGUiOnsibmFtZSI6ImNvcm5lciJ9fV0sWzQsMSwiZiIsMCx7ImN1cnZlIjotMX1dLFs0LDMsIiIsMCx7ImN1cnZlIjoyfV0sWzQsMCwiXFx0aGV0YSJdXQ==
                  \begin{tikzcd}
                    A \\
                    & {\pi_0(A)\times_{\pi_0(B)}B} & B \\
                    & {\pi_0(A)} & {\pi_0(B)}
                    \arrow[two heads, from=3-2, to=3-3]
                    \arrow[from=2-3, to=3-3]
                    \arrow[two heads, "p_2",from=2-2, to=2-3]
                    \arrow[from=2-2, to=3-2]
                    \arrow["\mathlarger{\mathlarger{\mathlarger{\mathlarger{\lrcorner}}}}"{anchor=center, pos=0.1}, draw=none, from=2-2, to=3-3]
                    \arrow["f", bend left = 6pt, from=1-1, to=2-3]
                    \arrow[bend right = 12pt, from=1-1, to=3-2]
                    \arrow["\theta", from=1-1, to=2-2]
                  \end{tikzcd}
                \end{center}
                L'application $\pi_0(A)\ra \pi_0(B)$ est une fibration (comme toute application d'un complexe de Kan dans un groupoïde).
                Comme le carré est cartésien, $p_2$ est aussi une fibration. Or un calcul rapide montre que $N\theta$ est surjective.
                Donc d'après $(2)$, $\theta$ est surjective, et c'est donc une fibration par $(1)$. Donc $f=p_2\circ \theta$ est une fibration.
  \end{description}
\end{proof}

\begin{coro}
  Il existe une structure de catégorie modèle sur $\Chp$ telle que~:
  \begin{description}
    \item[(W)] les équivalences faibles soient les applications qui induisent des isomorphismes en homologie~;
    \item[(F)] les fibrations soient les applications $f:C\ra D$ telles que $f_n:C_n\ra D_n$ soit surjectif pour tout $n\geq 1$~;
    \item[(C)] les cofibrations soient les monomorphisme $f:C\hookrightarrow D$ dont le conoyau $\coker{f}$ soit un complexe de $\Z$-modules projectifs.
  \end{description}
  Cette structure est celle induite par la correspondance de Dold-Kan et la structure de modèle sur $\DAb$.
\end{coro}

\begin{proof}
  Par l'équivalence de Dold-Kan, le \sref{lemme}{lemmeDoldKanHomotopie} et la \sref{proposition}{pinegalHn},
  l'existence de la structure vérifiant (W) et (F) est claire. Il s'agit donc de montrer (C).
  
  Montrons d'abord que les cofibrations sont des monomorphismes. Soit $f:C\rightarrow D$ une cofibration. 
  Pour $M$ un $\Z$-module et $n\geq 0$, on note $D^n(M)$ le complexe formé de $M$ en degrés $n$ et $n+1$, $0$ ailleurs et $d_{n+1}=\id_M$. 
  Alors pour $n\geq 0$, $D^n(C)\ra 0$ est une fibration acyclique. Donc elle a la propriété de relèvement à droite par rapport à $f$.
  On voit tout de suite que cela implique que $C_n\ra D_n$ a une rétraction.

  Soit $f:C\rightarrow D$ une cofibration, $n\geq 0$, $u:M\twoheadrightarrow N$ une surjection entre $\Z$-modules
  et $v:\coker{f}_n\ra N$ un morphisme.
  Le morphisme $u$ induit alors une fibration acyclique $D^n(u):D^n(M)\ra D^n(N)$. On se donne un morphisme $f\Rightarrow u$
  induit par $0:C_n\ra M$ et $D_n\ra \coker{f}_n\ra N$. Comme $f\square u$, il existe un relèvement $D\ra D^n(M)$, ce qui est équivalent
  à l'existence d'un relèvement $\coker{f}_n\ra M$ à $v$. Donc $\coker{f}_n$ est projectif.

  Soit $f:C\hookrightarrow D$ tel que son conoyau $\coker{f}$ soit un complexe de $\Z$-modules projectifs. Soit $u: E\ra F$ une cofibration acyclique,
  et $a,b:f\Rightarrow u$ un morphisme. Montrons, par récurrence sur $n\geq 0$, qu'il existe un relèvement $c:D\ra E$ définit en degrés $m\leq n$
  et compatible aux différentielles. On voit rapidement que $u_0$ est nécessairement une surjection (car $u$ est acyclique).
  Ainsi, pour compléter cette récurrence, il suffit de traiter le cas $n=1$ sachant que $c_0$ existe ($d^2=0$).
  On note $K:=\myker{u}$ et $N:=\coker{f}$. Comme $N_1$ est projectif, on dispose d'un isomorphisme $N_1\oplus C_1 \simeq D_1$.
  On en déduit un morphisme $N_1\ra F_1$ que l'on relève en $N_1\ra E_1$. On en déduit ainsi un morphisme
  $\alpha: D_1\ra E_1$ tel que $b_1=u_1\alpha$ et $a_1=\alpha f_1$. Maintenant, $d\alpha - \alpha d$
  est nul sur $A$ et est à image dans $K_0$. Il induit donc un morphisme $\beta:N_1\ra K_0$.
  Comme $K$ est acyclique, $K_1\ra K_0$ est une surjection. Donc $\beta$ admet un relèvement $N_1\ra K_1$,
  qui induit $\gamma:D_1\ra E_1$. Alors $c_1:=\beta-\gamma$ convient.
\end{proof}

\section{Topologie algébrique}
\label{annexetopologie}

Cette section couvre les notions de topologie utilisées dans le mémoire. Certains points seront admis.


\subsection{Homologie et homologie à coefficients}

Dans ce mémoire, nous déduisons les propriétés élémentaires de l'homologie et de la cohomologie
des structures de catégorie modèle sur $\DEns$, $\DAb$ et $\Chp$, suivant \cite[III.2]{Goer} (que nous citons pour les démonstrations).
Pour une introduction classique à l'homologie et la cohomologie, voir \cite{Hatc}.

Nous commençons par le théorème des coefficients universels (admis, voir \cite[3.6]{Weib2}).

\begin{theo}[Coefficients universels]\label{coeffsuniv}
  Soit $C_*$ un complexe de chaînes de $\Z$-modules libres
  en notations homologiques, et soit $M$ un groupe abélien.
  Alors on a une suite exacte fonctorielle en $C_*$~:
  $$\exa{H_i(C)\otimes M}{H_i(C\otimes M)}{\Tor_1^\Z (H_{i-1}(C),M)}$$
\end{theo}

\begin{theo}\label{coeffsunivcohom}
  Soit $C_*$ un complexe de chaînes de $\Z$-modules libres
  en notations homologiques, et soit $M$ un groupe abélien.
  Alors on a une suite exacte fonctorielle en $C_*$~:
  $$\exa{\mathrm{Ext}^1_\Z(H_{i-1}(C),M)}{H^i(\Hom{}{C}{M})}{\Hom{}{H_{i}(C)}{M}}$$
\end{theo}

\begin{defi}[homologie et cohomologie]
  Soit $X$ un ensemble simplicial et $n\geq 0$. On appelle $n$-ième groupe d'homologie de $X$ le groupe abélien~:
  $$H_n(X):=H_n(\Z X)=H_n(NX)=\pi_n(NX)$$
  Soit $M$ un $\Z$-module. On appelle $n$-ième groupe d'homologie de $X$ à coefficients dans $M$ le groupe abélien~:
  $$H_n(X;M):=H_n(\Z X\otimes M)=H_n(NX\otimes M)$$
  On appelle $n$-ième groupe de cohomologie de $X$ à coefficients dans $M$ le groupe abélien~:
  $$H^n(X;M):=H^n(\Hom{\Z}{\Z X}{M})=H^n(\Hom{\Z}{NX}{M})$$
  Les groupes $H_n(X)$, $H_n(X;M)$ et $H^n(X;M)$ sont fonctorielles en $X$ (de façon contravariante pour le dernier).
\end{defi}

\begin{rem}
  Les égalités $H_n(\Z X\otimes M)=H_n(NX\otimes M)$ et $H_n(\Hom{\Z}{\Z X}{M})=H_n(\Hom{\Z}{NX}{M})$ découlent du théorème des coefficients universels.
\end{rem}

\begin{rem}
  Dans la suite, on notera $H_*(X)$, $H_*(X;M)$ et $H^*(X;M)$ pour leq groupes abéliens $\N$-gradués formé de respectivement
  $H_n(X)$, $H_n(X;M)$ et $H^n(X;M)$ en degré $n$.
\end{rem}

\begin{prop}[invariance par homotopie]
  Le foncteur d'homologie $H_*:\DEns\ra \Ab^\N$ se factorise par $\Ho{\DEns}$. En d'autres termes, l'homologie est invariante par homotopie.
\end{prop}

Pour la démonstration, voir \cite[III.2.14]{Goer}. On a immédiatement, par le théorème des coefficients universels, le corollaire
suivant.

\begin{coro}
  Pour $M$ un groupe abélien, les foncteur $H_*(-;M)$ et $H^*(-;M)$ se factorisent également par $\Ho{\DEns}$.
\end{coro}

\begin{defi}[homologie et cohomologie des paires]
  Soit $A\subset X$ une paire d'ensembles simpliciaux. On définit son homologie par~:
  $$H_*(X,A):=H_*(\Z X/\Z A)$$
  Pour $M$ un groupe abélien, on définit l'homologie à coefficients dans $M$~:
  $$H_*(X,A;M):=H_*(\Z X/\Z A\otimes M)$$
  et sa cohomologie à coefficients dans $M$~:
  $$H^*(X,A;M):=H^*(\Hom{\Z}{\Z X/\Z A}{M})$$
\end{defi}

\begin{rem}
  On déduit facilement des définitions les suites exactes longues des paires.
\end{rem}

Le reste de cette sous-section est consacrée à la définition de l'homologie à coefficients locaux pour les ensembles
simpliciaux et les paires d'ensembles simpliciaux.

\begin{defi}
  Soit $X$ un ensemble simplicial. Un système de coefficients locaux est un foncteur~:
  $$L:\tau_{\leq 1}X\lra \Ab$$
  Un tel système $L$ est dit inversible si $L$ se factorise par $\tau_{\leq 1}X\lra \pi_{\leq 1}X$.
  Soit $f:Y\ra X$ un morphisme d'ensembles simpliciaux et $L$ un système de coefficients locaux sur $X$.
  On note $f^*L$ le système de coefficients locaux sur $Y$ définit par~:
  $$f^*L:\tau_{\leq 1}Y \lra \tau_{\leq 1}X \lra \Ab$$
\end{defi}

\begin{defi}
  Soit $X$ un ensemble simplicial et $L$ un système de coefficients locaux.
  On définit l'homologie de $X$ dans $L$ comme l'homologie du complexe
  $C_*(X;L)$ définit par~:
  $$C_n(X;L):=\bigoplus_{x\in X_n}L(x(0))\cdot x$$
  et tel que $\partial_n$ envoie $l\cdot x$ sur~:
  $$\partial_n l\cdot x = L(x(0)\ra x(1))(l)\cdot d_0(x) +\sum_{i=1}^n l\cdot (-1)^id_i(x)$$
  On notera $H_*(X;L)$ pour l'homologie.
  
  Cette définition est fonctorielle en $(X,L)$ au sens suivant. Si $f:Y\ra X$ est une application
  et $M\ra f^*L$ est un morphisme (transformation naturelle) entre systèmes de coefficients locaux sur $Y$.
  Alors le morphisme~:
  $$C_*(Y;M)\lra C_*(Y;f^*L)\lra C_*(X;L)$$
  induit un morphisme~:
  $$H_*(Y;M)\lra H_*(X;L)$$
\end{defi}

\begin{rem}
  En général, on inclus l'hypothèse inversible dans la définition de système de coefficients locaux. Nous choisissons d'étendre la définition
  pour prendre en compte le cas étudié dans la \sref{sous-section}{soussectioncalculhomologielocalises}.
\end{rem}

\begin{defi}
  Soit $i:A\subset X$ une paire d'ensembles simpliciaux, et $L$ un système de coefficients locaux strict sur $X$.
  Alors $C_*(A;i^*L)$ est une sous-complexe de $C_*(X;L)$. On définit l'homologie de la paire $(X,A)$ à coefficients dans $L$ comme
  l'homologie du complexe quotient~:
  $$H_*(X,A;L):=H_*(C_*(X;L)/C_*(A;i^*L))$$

  Cette définition est fonctorielle en $(X,A,L)$ au sens suivant. Si $f:(Y,B)\ra (X,A)$ est une application entre paires
  et $M\ra f^*L$ est un morphisme entre systèmes de coefficients locaux sur $Y$.
  Alors le morphisme~:
  $$C_*(Y,B;M)\lra C_*(Y,B;f^*L)\lra C_*(X,A;L)$$
  induit un morphisme~:
  $$H_*(Y,B;M)\lra H_*(X,A;L)$$
\end{defi}

\subsection{Le théorème de Van Kampen}

\begin{theo}[Van Kampen]\label{theoremedeVanKampen}
  On se donne le carré cocartésien suivant dans $\DEns$~:
  \begin{center}
    % https://q.uiver.app/?q=WzAsNCxbMCwwLCJBIl0sWzAsMSwiQiJdLFsxLDAsIlgiXSxbMSwxLCJZIl0sWzAsMSwiaSIsMCx7InN0eWxlIjp7InRhaWwiOnsibmFtZSI6Imhvb2siLCJzaWRlIjoidG9wIn19fV0sWzAsMiwiaiJdLFsyLDMsIiIsMCx7InN0eWxlIjp7InRhaWwiOnsibmFtZSI6Imhvb2siLCJzaWRlIjoidG9wIn19fV0sWzEsM10sWzMsMCwiIiwxLHsic3R5bGUiOnsibmFtZSI6ImNvcm5lciJ9fV1d
    \begin{tikzcd}
      A & X \\
      B & Y
      \arrow["i", hook, from=1-1, to=2-1]
      \arrow["j", from=1-1, to=1-2]
      \arrow[hook, from=1-2, to=2-2]
      \arrow[from=2-1, to=2-2]
      \arrow["\mathlarger{\mathlarger{\mathlarger{\mathlarger{\lrcorner}}}}"{anchor=center, pos=0.1, rotate=180}, draw=none, from=2-2, to=1-1]
    \end{tikzcd}
  \end{center}
  où $i$ est une cofibration. Alors le carré induit sur les groupoïdes fondamentaux est cocartésien (dans la catégorie des petits groupoïdes)~:
  \begin{center}
    \begin{tikzcd}
      \pi_{\leq 1}A & \pi_{\leq 1}X \\
      \pi_{\leq 1}B & \pi_{\leq 1}Y
      \arrow["i_*", from=1-1, to=2-1]
      \arrow["j_*", from=1-1, to=1-2]
      \arrow[from=1-2, to=2-2]
      \arrow[from=2-1, to=2-2]
      \arrow["\mathlarger{\mathlarger{\mathlarger{\mathlarger{\lrcorner}}}}"{anchor=center, pos=0.1, rotate=180}, draw=none, from=2-2, to=1-1]
    \end{tikzcd}
  \end{center}
  Si de plus $A$, $B$ et $X$ sont connexes, alors pour tout $x\in A_0$, le carré~:
  \begin{center}
    \begin{tikzcd}
      \pi_1(A,x) & \pi_1(X,x) \\
      \pi_1(B,x) & \pi_1(Y,x)
      \arrow["i_*", from=1-1, to=2-1]
      \arrow["j_*", from=1-1, to=1-2]
      \arrow[from=1-2, to=2-2]
      \arrow[from=2-1, to=2-2]
      \arrow["\mathlarger{\mathlarger{\mathlarger{\mathlarger{\lrcorner}}}}"{anchor=center, pos=0.1, rotate=180}, draw=none, from=2-2, to=1-1]
    \end{tikzcd}
  \end{center}
  est cocartésien dans la catégorie des groupes.
\end{theo}

\begin{proof}
  Pour le premier point, on remarque que les foncteurs $\tau_{\leq 1}:\DEns\ra \Cat$ et $\Cat\ra \mathrm{Groupoides}$ on des adjoints à gauche
  (respectivement le nerf $N$ et l'oubli). Donc leur composition, $\pi_{\leq 1}:\DEns\ra \mathrm{Groupoides}$, commute aux colimites.
  Pour le second point, par le \sref{lemme de collage}{lemmedecollage}, on peut supposer que $j$ est également une cofibration. Alors,
  en choisissant pour chaque $y\in A_0$ (respectivement $B_0\setminus A_0$, respectivement $X_0\setminus A_0$) un chemin de $y$ à $x$,
  $\gamma_y\in\Hom{\pi_{\leq 1}A}{y}{x}$ (respectivement $\Hom{\pi_{\leq 1}B}{y}{x}$, respectivement $\Hom{\pi_{\leq 1}X}{y}{x}$),
  on obtient une rétraction $r$ entre les carrés suivants~:
  \begin{center}
    % https://q.uiver.app/?q=WzAsOCxbMSwxLCJcXHBpX3tcXGxlcSAxfUEiXSxbMSwyLCJcXHBpX3tcXGxlcSAxfUIiXSxbMiwxLCJcXHBpX3tcXGxlcSAxfVgiXSxbMiwyLCJcXHBpX3tcXGxlcSAxfVkiXSxbMCwwLCJcXHBpXzEoQSx4KSJdLFszLDAsIlxccGlfMShYLHgpIl0sWzAsMywiXFxwaV8xKEIseCkiXSxbMywzLCJcXHBpXzEoWSx4KSJdLFswLDEsImlfKiJdLFswLDIsImpfKiJdLFsyLDMsIiIsMCx7InN0eWxlIjp7InRhaWwiOnsibmFtZSI6Imhvb2siLCJzaWRlIjoidG9wIn19fV0sWzEsM10sWzMsMCwiIiwxLHsic3R5bGUiOnsibmFtZSI6ImNvcm5lciJ9fV0sWzQsNV0sWzQsNl0sWzUsN10sWzYsN10sWzEsNiwiciIsMCx7Im9mZnNldCI6LTF9XSxbMCw0LCJyIiwwLHsib2Zmc2V0IjotMX1dLFsyLDUsInIiLDAseyJvZmZzZXQiOi0xfV0sWzMsNywiciIsMCx7Im9mZnNldCI6LTF9XSxbNCwwLCIiLDIseyJvZmZzZXQiOi0xLCJzdHlsZSI6eyJ0YWlsIjp7Im5hbWUiOiJob29rIiwic2lkZSI6InRvcCJ9fX1dLFs1LDIsIiIsMSx7Im9mZnNldCI6LTEsInN0eWxlIjp7InRhaWwiOnsibmFtZSI6Imhvb2siLCJzaWRlIjoidG9wIn19fV0sWzYsMSwiIiwwLHsib2Zmc2V0IjotMSwic3R5bGUiOnsidGFpbCI6eyJuYW1lIjoiaG9vayIsInNpZGUiOiJ0b3AifX19XSxbNywzLCIiLDEseyJvZmZzZXQiOi0xLCJzdHlsZSI6eyJ0YWlsIjp7Im5hbWUiOiJob29rIiwic2lkZSI6InRvcCJ9fX1dXQ==
    \begin{tikzcd}
    	{\pi_1(A,x)} &&& {\pi_1(X,x)} \\
    	& {\pi_{\leq 1}A} & {\pi_{\leq 1}X} \\
    	& {\pi_{\leq 1}B} & {\pi_{\leq 1}Y} \\
    	{\pi_1(B,x)} &&& {\pi_1(Y,x)}
    	\arrow["{i_*}", from=2-2, to=3-2]
    	\arrow["{j_*}", from=2-2, to=2-3]
    	\arrow[from=2-3, to=3-3]
    	\arrow[from=3-2, to=3-3]
    	\arrow["\mathlarger{\mathlarger{\mathlarger{\mathlarger{\lrcorner}}}}"{anchor=center, pos=0.1, rotate=180}, draw=none, from=3-3, to=2-2]
    	\arrow[from=1-1, to=1-4]
    	\arrow[from=1-1, to=4-1]
    	\arrow[from=1-4, to=4-4]
    	\arrow[from=4-1, to=4-4]
    	\arrow["r", shift left=1, from=3-2, to=4-1]
    	\arrow["r", shift left=1, from=2-2, to=1-1]
    	\arrow["r", shift left=1, from=2-3, to=1-4]
    	\arrow["r", shift left=1, from=3-3, to=4-4]
    	\arrow[shift left=1, hook, from=1-1, to=2-2]
    	\arrow[shift left=1, hook, from=1-4, to=2-3]
    	\arrow[shift left=1, hook, from=4-1, to=3-2]
    	\arrow[shift left=1, hook, from=4-4, to=3-3]
    \end{tikzcd}
  \end{center}
  Or, les rétractions préservent les colimites. Donc le carré des $\pi_1(-,x)$ est cocartésien.
\end{proof}

\subsection{Obstruction et tours de Postnikov}

\begin{lem}[Lemme d'extension]\label{lemmedextension}
  Soit $A$ un ensemble simplicial, $a\in A$, $n\geq 1$ et $(\gamma_i)_{i\in I}$ une famille de chemins $\gamma_i:(\Delta^n,\partial\Delta^n)\ra (A,a)$.
  On se donne le carré cocartésien~:
  \begin{center}
    % https://q.uiver.app/?q=WzAsNCxbMCwwLCJcXGJpZ3NxY3VwX0lcXHBhcnRpYWxcXERlbHRhXntuKzF9Il0sWzAsMSwiXFxiaWdzcWN1cF9JXFxEZWx0YV57bisxfSJdLFsxLDAsIkEiXSxbMSwxLCJYIl0sWzMsMCwiIiwwLHsic3R5bGUiOnsibmFtZSI6ImNvcm5lciJ9fV0sWzAsMiwiXFxzcWN1cF9pKGEsXFxkb3RzYyxhLFxcZ2FtbWFfaSkiXSxbMiwzLCJqIiwwLHsic3R5bGUiOnsidGFpbCI6eyJuYW1lIjoiaG9vayIsInNpZGUiOiJ0b3AifX19XSxbMCwxLCIiLDAseyJzdHlsZSI6eyJ0YWlsIjp7Im5hbWUiOiJob29rIiwic2lkZSI6InRvcCJ9fX1dLFsxLDNdXQ==
    \begin{tikzcd}[column sep = 10ex]
    	{\bigsqcup_I\partial\Delta^{n+1}} & A \\
    	{\bigsqcup_I\Delta^{n+1}} & X
    	\arrow["\mathlarger{\mathlarger{\mathlarger{\mathlarger{\lrcorner}}}}"{anchor=center, pos=0.1, rotate=180}, draw=none, from=2-2, to=1-1]
    	\arrow["{\sqcup_i(a,\dotsc,a,\gamma_i)}", from=1-1, to=1-2]
    	\arrow["j", hook, from=1-2, to=2-2]
    	\arrow[hook, from=1-1, to=2-1]
    	\arrow[from=2-1, to=2-2]
    \end{tikzcd}
  \end{center}
  Alors~:
  \begin{description}
    \item[$(a)$] $\pi_m(j):\pi_m(A,a)\ra \pi_m(X,a)$ est un isomorphisme pour $m<n$~;
    \item[$(b)$] $\pi_n(j):\pi_n(A,a)\ra \pi_n(X,a)$ est surjective.
  \end{description}
  Soit $f:A\ra Y$ avec $Y$ fibrant. Alors s'équivalent~:
  \begin{description}
    \item[$(i)$] Il existe $g:X\ra Y$ tel que $f=g\circ j$~;
    \item[$(ii)$] Pour tout $i\in I$, $f_*[\gamma_i]=0$ dans $\pi_n(Y,f(a))$.
  \end{description}
\end{lem}

\begin{proof}
  L'équivalence entre $(i)$ et $(ii)$ est immédiate. Montrons les points $(a)$ et $(b)$. Si $A\overset{\sim}{\hookrightarrow}\tilde{A}\twoheadrightarrow *$,
  on a le diagramme commutatif suivant~:
  \begin{center}
    % https://q.uiver.app/?q=WzAsNixbMCwwLCJcXGJpZ3NxY3VwX0lcXHBhcnRpYWxcXERlbHRhXntuKzF9Il0sWzAsMSwiXFxiaWdzcWN1cF9JXFxEZWx0YV57bisxfSJdLFsxLDAsIkEiXSxbMSwxLCJYIl0sWzIsMCwiXFx0aWxkZXtBfSJdLFsyLDEsIlxcdGlsZGV7WH0iXSxbMywwLCIiLDAseyJzdHlsZSI6eyJuYW1lIjoiY29ybmVyIn19XSxbMCwyLCJcXHNxY3VwX2koYSxcXGRvdHNjLGEsXFxnYW1tYV9pKSJdLFsyLDMsImoiLDAseyJzdHlsZSI6eyJ0YWlsIjp7Im5hbWUiOiJob29rIiwic2lkZSI6InRvcCJ9fX1dLFswLDEsIiIsMCx7InN0eWxlIjp7InRhaWwiOnsibmFtZSI6Imhvb2siLCJzaWRlIjoidG9wIn19fV0sWzEsM10sWzIsNCwiXFxzaW0iLDAseyJzdHlsZSI6eyJ0YWlsIjp7Im5hbWUiOiJob29rIiwic2lkZSI6InRvcCJ9fX1dLFszLDUsIlxcc2ltIiwwLHsic3R5bGUiOnsidGFpbCI6eyJuYW1lIjoiaG9vayIsInNpZGUiOiJ0b3AifX19XSxbNCw1LCIiLDEseyJzdHlsZSI6eyJ0YWlsIjp7Im5hbWUiOiJob29rIiwic2lkZSI6InRvcCJ9fX1dLFs1LDIsIiIsMSx7InN0eWxlIjp7Im5hbWUiOiJjb3JuZXIifX1dXQ==
    \begin{tikzcd}
    	{\bigsqcup_I\partial\Delta^{n+1}} &[5ex] A & {\tilde{A}} \\
    	{\bigsqcup_I\Delta^{n+1}} & X & {\tilde{X}}
      \arrow["\mathlarger{\mathlarger{\mathlarger{\mathlarger{\lrcorner}}}}"{anchor=center, pos=0.1, rotate=180}, draw=none, from=2-2, to=1-1]
    	\arrow["{\sqcup_i(a,\dotsc,a,\gamma_i)}", from=1-1, to=1-2]
    	\arrow["j", hook, from=1-2, to=2-2]
    	\arrow[hook, from=1-1, to=2-1]
    	\arrow[from=2-1, to=2-2]
    	\arrow["\sim", hook, from=1-2, to=1-3]
    	\arrow["\sim", hook, from=2-2, to=2-3]
    	\arrow[hook, from=1-3, to=2-3]
    	\arrow["\mathlarger{\mathlarger{\mathlarger{\mathlarger{\lrcorner}}}}"{anchor=center, pos=0.1, rotate=180}, draw=none, from=2-3, to=1-2]
    \end{tikzcd}
  \end{center}
  Donc on peut supposer sans perte de généralités que $A$ est un complexe de Kan. Or $X$ vérifie alors la propriété d'extension
  relativement aux inclusions $\Lambda_k^m\hookrightarrow \Delta^m$ pour $m\leq n$. En effet $X_m=A_m$ pour $m\leq n$. Donc,
  $X':=G^\infty(\enstq{\Lambda_k^l\hookrightarrow\Delta^l}{l>n},X\twoheadrightarrow *)$ est un remplacement fibrant de $X$.
  Or, $X'_m=A_m$ pour $m\leq n$, donc sur les groupes d'homotopie simpliciaux, $\pi_m(A,a)\ra \pi_m(X',a)$
  est un isomorphisme pour $m<n$ et une surjection pour $m=n$.
\end{proof}

\begin{defi}[tour de Postnikov]\label{defiPostnikov}
  Soit $X$ une ensemble simplicial. Une tour de Postnikov est un diagramme commutatif~:
  \begin{center}
    % https://q.uiver.app/?q=WzAsNixbMCwyLCJYIl0sWzIsMSwiWChuKSJdLFsyLDIsIlgobi0xKSJdLFsyLDMsIlgoMSkiXSxbMiw0LCJYKDApIl0sWzIsMF0sWzAsMSwiaV9uIl0sWzAsMiwiaV97bi0xfSJdLFswLDMsImlfMSJdLFswLDQsImlfMCJdLFszLDQsInFfMSJdLFsyLDMsIiIsMSx7InN0eWxlIjp7ImJvZHkiOnsibmFtZSI6ImRvdHRlZCJ9fX1dLFsxLDIsInFfbiJdLFs1LDEsIiIsMCx7InN0eWxlIjp7ImJvZHkiOnsibmFtZSI6ImRvdHRlZCJ9fX1dXQ==
    \begin{tikzcd}
    	&& {} \\
    	&& {X(n)} \\
    	X && {X(n-1)} \\
    	&& {X(1)} \\
    	&& {X(0)}
    	\arrow["{i_n}", from=3-1, to=2-3]
    	\arrow["{i_{n-1}}", from=3-1, to=3-3]
    	\arrow["{i_1}", from=3-1, to=4-3]
    	\arrow["{i_0}", from=3-1, to=5-3]
    	\arrow["{q_1}", from=4-3, to=5-3]
    	\arrow[dotted, from=3-3, to=4-3]
    	\arrow["{q_n}", from=2-3, to=3-3]
    	\arrow[dotted, from=1-3, to=2-3]
    \end{tikzcd}
  \end{center}
  tel que pour tout $v\in X_0$, $\pi_j(X(n),v)=0$ pour $j>n$ et $(i_n)_*:\pi_j(X,v)\ra\pi_j(X(n),v)$ est un isomorphisme pour $j\leq n$.
\end{defi}

\begin{prop}
  Tout ensemble simplicial $X$ admet une tour de Postnikov.
\end{prop}

\begin{proof}
  On peut supposer sans perte de généralités que $X$ est connexe. Fixons $n\geq 0$. Soit $X\overset{\sim}{\hookrightarrow}\tilde{X}$
  un remplacement fibrant. On pose $X(n,n)=X$ et $\tilde{X}(n,n)=\tilde{X}$. On définit $X(n,n+1)$ comme la somme amalgamée~:
  \begin{center}
    % https://q.uiver.app/?q=WzAsNCxbMSwwLCJcXHRpbGRle1h9Il0sWzEsMSwiWChuLG4rMSkiXSxbMCwwLCJcXGJpZ3NxY3VwX0lcXHBhcnRpYWxcXERlbHRhXntuKzJ9Il0sWzAsMSwiXFxiaWdzcWN1cF9JXFxEZWx0YV57bisyfSJdLFswLDEsIiIsMCx7InN0eWxlIjp7InRhaWwiOnsibmFtZSI6Imhvb2siLCJzaWRlIjoidG9wIn19fV0sWzIsMywiIiwwLHsic3R5bGUiOnsidGFpbCI6eyJuYW1lIjoiaG9vayIsInNpZGUiOiJ0b3AifX19XSxbMiwwLCJmIl0sWzMsMV0sWzEsMiwiIiwwLHsic3R5bGUiOnsibmFtZSI6ImNvcm5lciJ9fV1d
    \begin{tikzcd}
      {\bigsqcup_I\partial\Delta^{n+2}} & {\tilde{X}(n,n)} \\
      {\bigsqcup_I\Delta^{n+2}} & {X(n,n+1)}
      \arrow[hook, from=1-2, to=2-2]
      \arrow[hook, from=1-1, to=2-1]
      \arrow["f", from=1-1, to=1-2]
      \arrow[from=2-1, to=2-2]
      \arrow["\mathlarger{\mathlarger{\mathlarger{\mathlarger{\lrcorner}}}}"{anchor=center, pos=0.1, rotate=180}, draw=none, from=2-2, to=1-1]
    \end{tikzcd}
  \end{center}
  où $f$ est induit par un ensemble $I$ de générateurs de $\pi_{n+1}(\tilde{X}(n,n),v)$. Soit $X(n,n+1)\overset{\sim}{\hookrightarrow}\tilde{X}(n,n+1)$
  un remplacement fibrant. On définit $X(n,n+2)$ de façon similaire à partir de $\tilde{X}(n,n+1)$ avec $I$ un ensemble de générateurs de
  $\pi_{n+2}(\tilde{X}(n,n+1),v)$. On a alors une suite de cofibrations~:
  $$X=X(n,n)\overset{\sim}{\hookrightarrow}\tilde{X}(n,n)\hookrightarrow\dotsb\hookrightarrow X(n,n+k)\overset{\sim}{\hookrightarrow}\tilde{X}(n,n+k)$$
  On définit $X(n)$ comme la colimite de cette suite. On dispose de $i_n:X\ra X(n)$. On remarque également que $X(n)$ est un complexe de Kan.
  On a également les propriétés souhaitée sur $\pi_m(i_n)$ et les $\pi_m(X(n))$.

  Il reste à construire les $q_n$. Pour cela, il suffit
  de construire $q_n:X(n)\ra X(n-1)$ tel que $i_{n-1}=q_n\circ i_n$. Or, on a le diagramme commutatif suivant~:
  \begin{center}
    % https://q.uiver.app/?q=WzAsNixbMiwwLCJYKG4sbitrKSJdLFszLDAsIlxcdGlsZGV7WH0obixuK2spIl0sWzQsMCwiWChuLG4raysxKSJdLFsxLDIsIlgobi0xKSJdLFsxLDAsIlxcdGlsZGV7WH0iXSxbMCwwLCJYIl0sWzAsMSwiXFxzaW0iLDAseyJzdHlsZSI6eyJ0YWlsIjp7Im5hbWUiOiJob29rIiwic2lkZSI6InRvcCJ9fX1dLFsxLDIsIiIsMCx7InN0eWxlIjp7InRhaWwiOnsibmFtZSI6Imhvb2siLCJzaWRlIjoidG9wIn19fV0sWzAsMywiXFxleGlzdHMgcV97bixuK2t9IiwxLHsic3R5bGUiOnsiYm9keSI6eyJuYW1lIjoiZGFzaGVkIn19fV0sWzEsMywiXFxleGlzdHMgXFx0aWxkZXtxfV97bixuK2t9IiwxLHsic3R5bGUiOnsiYm9keSI6eyJuYW1lIjoiZGFzaGVkIn19fV0sWzIsMywiXFxleGlzdHMgcV97bixuK2srMX0iLDEseyJzdHlsZSI6eyJib2R5Ijp7Im5hbWUiOiJkYXNoZWQifX19XSxbNSw0LCJcXHNpbSIsMCx7InN0eWxlIjp7InRhaWwiOnsibmFtZSI6Imhvb2siLCJzaWRlIjoidG9wIn19fV0sWzQsMCwiIiwwLHsic3R5bGUiOnsidGFpbCI6eyJuYW1lIjoiaG9vayIsInNpZGUiOiJ0b3AifSwiYm9keSI6eyJuYW1lIjoiZG90dGVkIn19fV0sWzQsMywiXFxleGlzdHMgXFx0aWxkZXtxfV97bixufSIsMSx7InN0eWxlIjp7ImJvZHkiOnsibmFtZSI6ImRhc2hlZCJ9fX1dLFs1LDMsImlfe24tMX0iLDJdXQ==
    \begin{tikzcd}
    	X & {\tilde{X}} & {X(n,n+k)} & {\tilde{X}(n,n+k)} & {X(n,n+k+1)} \\
    	\\
    	& {X(n-1)}
    	\arrow["\sim", hook, from=1-3, to=1-4]
    	\arrow[hook, from=1-4, to=1-5]
    	\arrow["{\exists q_{n,n+k}}"{description}, dashed, from=1-3, to=3-2]
    	\arrow["{\exists \tilde{q}_{n,n+k}}"{description}, dashed, from=1-4, to=3-2]
    	\arrow["{\exists q_{n,n+k+1}}"{description}, dashed, from=1-5, to=3-2]
    	\arrow["\sim", hook, from=1-1, to=1-2]
    	\arrow[dotted, hook, from=1-2, to=1-3]
    	\arrow["{\exists \tilde{q}_{n,n}}"{description}, dashed, from=1-2, to=3-2]
    	\arrow["{i_{n-1}}"', from=1-1, to=3-2]
    \end{tikzcd}
  \end{center}
  où l'existence des $\tilde{q}_{n,n+k}$ est garantie car $X(n-1)$ est un complexe de Kan, et l'existence
  des $q_{n,n+k}$ est garantie par le \sref{lemme d'extension}{lemmedextension}. En passant à la colimite, on obtient l'application
  $q_n:X(n)\ra X(n-1)$ souhaitée.
\end{proof}

\begin{rem}
  Si $X$ dispose d'une tour de Postnikov, on peut successivement remplacer $X(0)$ par un complexe de Kan, et les $q_n$ par des fibrations,
  On obtient ainsi une tour de Postnikov où tous les $q_n$ sont des fibrations. On appelle une telle tour une tour de Postnikov de fibrations.  
\end{rem}

\begin{lem}\label{lemmePostnikovLimite}
  Soit $n\geq 0$ et $p:X\twoheadrightarrow Y$ une fibration de Kan entre complexes de Kan. Soit $x\in X_0$.
  On suppose que $\pi_{n+1}(X,x)\ra\pi_{n+1}(Y,px)$ est surjectif.
  Soit~:
  \[
    \begin{array}{lcl}
      \gamma:(\Delta^n,\partial\Delta^n)&\ra& (X,x) \\
      \sigma:\Delta^{n+1}&\ra& Y
    \end{array}
  \]
  tels que $[\gamma]=*$ et $\partial\sigma=(px,\dotsc,px,p\gamma)$.

  Alors il existe $\tau:\Delta^{n+1}\ra Y$ tel que $\partial\tau=(x,\dotsc,x,\gamma)$ et $f\tau =\sigma$.
\end{lem}

\begin{proof}
  Soit $h:\Delta^{n}\times\Delta^1\ra X$ homotopie entre $\gamma$ et $*$, on a alors~:
  \begin{center}
    % https://q.uiver.app/?q=WzAsNSxbMCwwLCJcXHBhcnRpYWxcXERlbHRhXntuKzF9XFx0aW1lc1xcRGVsdGFeMSJdLFsxLDAsIlkiXSxbMCwxLCJcXERlbHRhXntuKzF9XFx0aW1lc1xcRGVsdGFeMSJdLFswLDJdLFsxLDEsIioiXSxbMCwxLCIoeCxcXGRvdHNjLHgsaCkiXSxbMSw0LCIiLDAseyJzdHlsZSI6eyJoZWFkIjp7Im5hbWUiOiJlcGkifX19XSxbMiw0XSxbMiwxLCJcXGV4aXN0cyBcXHRoZXRhIiwwLHsic3R5bGUiOnsiYm9keSI6eyJuYW1lIjoiZGFzaGVkIn19fV0sWzAsMiwiIiwwLHsic3R5bGUiOnsidGFpbCI6eyJuYW1lIjoiaG9vayIsInNpZGUiOiJ0b3AifX19XV0=
    \begin{tikzcd}
    	{\partial\Delta^{n+1}\times\Delta^1} &[4ex] Y \\
    	{\Delta^{n+1}\times\Delta^1} & {*} \\
    	\arrow["{(px,\dotsc,px,ph)}", from=1-1, to=1-2]
    	\arrow[two heads, from=1-2, to=2-2]
    	\arrow[from=2-1, to=2-2]
    	\arrow["{\exists \theta}" sloped, dashed, from=2-1, to=1-2]
    	\arrow[hook, from=1-1, to=2-1]
    \end{tikzcd}
  \end{center}
  Maintenant, $\tilde{\sigma}:=\theta\circ (id\times d^0)$ représente un élément de $\pi_{n+1}(Y,px)$. Soit $\tilde{\tau}$ représentant un élément
  de $\pi_{n+1}(X,x)$ tel que $p_*[\tilde{\tau}]=[\tilde{\sigma}]$. On se donne $g:\Delta^{n+1}\times \Delta^{1}\ra Y$ homotopie
  $\mathrm{rel}\;\partial\Delta^{n+1}$ entre $\tilde{\sigma}$ et $p\tilde{\tau}$. On dispose alors de $\beta:\Delta^{n+1}\times\Delta^2\ra Y$ On a alors~:
  \begin{center}
    % https://q.uiver.app/?q=WzAsNSxbMCwwLCJcXHBhcnRpYWxcXERlbHRhXntuKzF9XFx0aW1lc1xcTGFtYmRhXzFeMlxcY3VwXFxEZWx0YV57bisxfVxcdGltZXNcXHsyXFx9Il0sWzEsMCwiWCJdLFswLDEsIlxcRGVsdGFee24rMX1cXHRpbWVzXFxMYW1iZGFfMV4yIl0sWzAsMl0sWzEsMSwiWSJdLFswLDEsIigoeCxcXGRvdHNjLHgsZiksLSx4KSlcXGN1cCBcXHRpbGRle1xcdGF1fSJdLFsxLDQsIiIsMCx7InN0eWxlIjp7ImhlYWQiOnsibmFtZSI6ImVwaSJ9fX1dLFsyLDRdLFsyLDEsIlxcZXhpc3RzIEgiLDAseyJzdHlsZSI6eyJib2R5Ijp7Im5hbWUiOiJkYXNoZWQifX19XSxbMCwyLCIiLDAseyJzdHlsZSI6eyJ0YWlsIjp7Im5hbWUiOiJob29rIiwic2lkZSI6InRvcCJ9fX1dXQ==
    \begin{tikzcd}
    	{\partial\Delta^{n+1}\times\Lambda_1^2\cup\Delta^{n+1}\times\{2\}} &[8ex] X \\
    	{\Delta^{n+1}\times\Lambda_1^2} & Y \\
    	\arrow["{(x,-,(x,\dotsc,x,f))\cup \tilde{\tau}}", from=1-1, to=1-2]
    	\arrow[two heads, from=1-2, to=2-2]
    	\arrow["{(g,-,h)}",from=2-1, to=2-2]
    	\arrow["{\exists H}" sloped, dashed, from=2-1, to=1-2]
    	\arrow[hook, from=1-1, to=2-1]
    \end{tikzcd}
  \end{center}
  Alors, $\tau:=H\circ (id\times (d^1\circ d^2))$ convient.
\end{proof}

\begin{prop}\label{proplimiteinversedefibrations}
  Soit une suite de fibrations~:
  $$\dotsb\twoheadrightarrow X_n\twoheadrightarrow\dotsb\twoheadrightarrow X_1\twoheadrightarrow X_0\twoheadrightarrow *$$
  dans $\DEns$. Soit $i\geq 0$ et $x\in (\lim_n X_n)_0$.
  Alors~:
  $$\lambda:\pi_i(\lim_n X_n,x)\ra \lim_n \pi_i(X_n,x)$$
  est surjectif. Si de plus $\pi_{i+1}(X_n)\ra \pi_{i+1}(X_{n-1})$ est surjectif pour $n$ assez grand,
  $\lambda$ est un isomorphisme.
\end{prop}

\begin{proof}
  Soient $f_n:(\Delta^i,\partial\Delta^i)\ra (X_n,*)$ tels que $[f_{n+1}]\mapsto [f_n]$. Alors, par relèvement des homotopie
  (ie. relativement à $\partial\Delta^{i}\times\Delta^1\cup\Delta^{i}\times {0}\ra\Delta^i\times\Delta^1$), on peut successivement construire
  $f'_n:(\Delta^i,\partial\Delta^i)\ra (X_n,*)$ tels que $[f'_n]=[f_n]$ et $f'_n\mapsto f'_{n+1}$. On a construit $f'\in \pi_i(\lim_n X_n,x)$
  tel que $f'\mapsto (f_n)_n$.

  Supposons maintenant que $\pi_{i+1}(X_n)\ra \pi_{i+1}(X_{n-1})$ est surjectif pour $n$ assez grand. On peut, sans perte de généralité,
  supposer que $\pi_{i+1}(X_n)\ra \pi_{i+1}(X_{n-1})$ est surjectif pour tout $n$. Soit $f$ un représentant d'un élément dans le noyau de $\lambda$,
  et $f_n$ l'image dans $X_n$.
  Alors par le \sref{lemme}{lemmePostnikovLimite}, on peut construire successivement des $\sigma_n:\Delta^{i+1}\ra X_n$
  tels que $\partial\tau=(*,\dotsc,*,f_n)$ et $\tau_n\mapsto \tau_{n-1}$. Ces $(\tau_n)_{n\geq 0}$ forment donc
  un $\tau:\Delta^{i+1}\ra \lim_n X_n$ tel que $\partial\tau=(*,\dotsc,*,f)$. Donc $[f]=*$.
\end{proof}

\begin{coro}\label{limitetourdePostnikov}
  Soit $X$ un ensemble simplicial et $(X(n),i_n,q_n)_n$ une tour de Postnikov de fibrations pour $X$.
  Alors l'application~:
  $$X\ra \lim_{n\in\N}X(n)$$
  est une équivalence d'homotopie faible.
\end{coro}

\begin{proof}
  D'après la \sref{proposition}{proplimiteinversedefibrations}, pour chaque $i\geq 0$ et $x\in X_0$, on a un isomorphisme~:
  $$\pi_i(X,x)\overset{\sim}{\ra}\pi_i(\lim_{n\in\N}X(n),x)$$
  Donc c'est une équivalence d'homotopie faible.
\end{proof}


\subsection{le théorème de Hurewicz}

Nous allons admettre les démonstrations des différentes versions du théorème de Hurewicz, car elles utilisent des outils que nous n'avons pas introduit
(Suite spectrale de Serre ou théorème d'excision en homotopie, voir les remarques).

\begin{defi}[morphisme de Hurewicz]
  Soit $X$ un ensemble simplicial, $n\geq 1$ et $x\in X_0$. L'application naturelle $X\ra \Z X$ induit le diagramme suivant~:
  $$\pi_n(X,x)\lra \pi_n(\Z X/\Z x, 0)\longleftarrow\pi_n(\Z X, 0)$$
  Or, d'après la \sref{proposition}{pinegalHn}, pour $A$ groupe abélien simplicial, on a un isomorphe fonctoriel~:
  $$\pi_n(A,0)\simeq H_n(A)$$
  Or $H_n(\Z x) = 0$. Donc $\pi_n(\Z X, 0)\ra \pi_n(\Z X/\Z x, 0)$ est un isomorphisme, et donc on dispose d'un morphisme~:
  $$\mathcal{H}_n(X,x):\pi_n(X,x)\ra H_n(X)$$
  On appelle ce morphisme le morphisme de Hurewicz. Il est clairement fonctoriel en $X$.
\end{defi}

\begin{theo}[Hurewicz pour le $\pi_1$]\label{theoremeHurewiczpi1}
  Soit $X$ ensemble simplicial connexe et $x\in X_0$. Alors $\mathcal{H}_1(X,x)$ induit un isomorphisme~:
  $$\pi_1(X,x)\ab\lra H_1(X)$$
\end{theo}

\begin{proof}
  On peut supposer que $X$ est un complexe de Kan. Pour $y\in X_0$, on se donne un chemin $\gamma_y:\Delta^1\ra X$ de $x$ à $y$.
  Une réciproque est alors donnée par~:
  $$\sum_i [\delta_i]\mapsto \prod_i[\gamma_{\delta(1)}]^{-1}[\delta_i][\gamma_{\delta(0)}]$$
\end{proof}

\begin{theo}[Hurewicz pour $\pi_n$, $n\geq 2$]\label{theoremeHurewitzpin}
  Soit $X$ un ensemble simplicial connexe, $x\in X_0$ et $n\geq 2$. Si $\pi_m(X,x)=0$ pour tout $1\leq m<n$, alors l'application de Hurewicz~:
  $$\mathcal{H}_n(X,x):\pi_n(X,x)\ra H_n(X)$$
  est un isomorphisme.
\end{theo}

Nous ne démontrons pas ici ce théorème. Une première preuve qui se base sur la suite spectrale de Serre est donnée dans \cite[III.3.7]{Goer}.
En examinant la preuve, on remarque que l'existence d'un isomorphisme entre $\pi_n(X,x)$ et $H_n(X)$ est une conséquence immédiate de la suite spectrale.
Il est plus difficile d'identifier cet isomorphisme avec $\mathcal{H}_n(X,x)$. Une seconde preuve qui se base sur le théorème d'excision en homotopie
est donnée dans \cite[4.37]{Hatc} (prendre $A=x$).

\begin{defi}[morphisme de Hurewicz pour les paires]\label{theoremeHurewiczpaire}
  Soit $X$ un ensemble simplicial, $A$ un sous-ensemble simplicial de $X$, et $x\in A_0$. Soit $n\geq 1$.
  Le morphisme naturel $(X,A)\ra (\Z X/\Z x,\Z A/\Z x)$ induit~:
  $$\pi_n(X,A,x)\lra \pi_n(\Z X/\Z x,\Z A/\Z x,0)\longleftarrow \pi_n(\Z X,\Z A,0)$$
  Or, d'après \sref{proposition}{pinegalHn}, pour $(U,V)$ paire des groupe abélien simplicial, on a un isomorphe fonctoriel~:
  $$\pi_n(U,V,0)\simeq H_n(U/V)$$
  Donc $\pi_n(\Z X,\Z A,0)\ra \pi_n(\Z X/\Z x,\Z A/\Z x,0)$ est un isomorphisme, et donc on dispose d'un morphisme~:
  $$\mathcal{H}_n(X,A,x):\pi_n(X,A,x)\ra H_n(X,A)$$
  On appelle ce morphisme le morphisme de Hurewicz pour les paires. Il est clairement fonctoriel en $(X,A)$.
\end{defi}

\begin{rem}
  De la compatibilité aux suites exactes longues énoncée dans la \sref{proposition}{pinegalHn}, on déduit que les morphismes de Hurewicz sont compatibles
  aux suites exactes longues des paires. Plus précisément, si $x\in A\subseteq X$, alors le diagramme suivant commute pour $n\geq 1$~:
  \begin{center}
    % https://q.uiver.app/?q=WzAsOCxbMSwwLCJcXHBpX24oQSx4KSJdLFsyLDAsIlxccGlfbihYLHgpIl0sWzAsMCwiXFxwaV97bisxfShYLEEseCkiXSxbMCwxLCJIX3tuKzF9KFgsQSkiXSxbMywwLCJcXHBpX3tufShYLEEseCkiXSxbMywxLCJIX3tufShYLEEpIl0sWzEsMSwiSF97bn0oQSkiXSxbMiwxLCJIX3tufShYKSJdLFszLDYsIlxccGFydGlhbCJdLFsyLDAsIlxccGFydGlhbCJdLFswLDFdLFs2LDddLFsxLDRdLFs3LDVdLFsyLDMsIlxcbWF0aGNhbHtIfV97bisxfShYLEEseCkiLDFdLFs0LDUsIlxcbWF0aGNhbHtIfV97bn0oWCxBLHgpIiwxXSxbMSw3LCJcXG1hdGhjYWx7SH1fe259KFgseCkiLDFdLFswLDYsIlxcbWF0aGNhbHtIfV97bn0oQSx4KSIsMV1d
    \begin{tikzcd}[column sep = large, row sep = large]
    	{\pi_{n+1}(X,A,x)} & {\pi_n(A,x)} & {\pi_n(X,x)} & {\pi_{n}(X,A,x)} \\
    	{H_{n+1}(X,A)} & {H_{n}(A)} & {H_{n}(X)} & {H_{n}(X,A)}
    	\arrow["\partial", from=2-1, to=2-2]
    	\arrow["\partial", from=1-1, to=1-2]
    	\arrow[from=1-2, to=1-3]
    	\arrow[from=2-2, to=2-3]
    	\arrow[from=1-3, to=1-4]
    	\arrow[from=2-3, to=2-4]
    	\arrow["{\mathcal{H}_{n+1}(X,A,x)}"{description}, from=1-1, to=2-1]
    	\arrow["{\mathcal{H}_{n}(X,A,x)}"{description}, from=1-4, to=2-4]
    	\arrow["{\mathcal{H}_{n}(X,x)}"{description}, from=1-3, to=2-3]
    	\arrow["{\mathcal{H}_{n}(A,x)}"{description}, from=1-2, to=2-2]
    \end{tikzcd}
  \end{center}
\end{rem}

\begin{theo}[Hurewicz pour les paires]
  Soit $n\geq 2$ et $(X,A)$ une paire d'ensembles simpliciaux tels que $\pi_1(A,x)$ agisse trivialement sur $\pi_n(X,A,x)$.
  Alors le morphisme de Hurewicz~:
  $$\mathcal{H}_n(X,A,x):\pi_n(X,A,x)\ra H_n(X,A)$$
  est un isomorphisme.
\end{theo}

Nous ne démontrons pas ici ce théorème. Voir \cite[4.37]{Hatc} pour une preuve.

\subsection{Espaces d'Eilenberg Mac-Lane et tours de Postnikov de fibrations principales}

La première partie de cette sous-section est consacrée à la définition des espaces d'Eilenberg-MacLane et la seconde aux tours de Postnikov
de fibrations principales. La première utilise beaucoup la correspondance de Dold-Kan énoncée dans la \sref{sous-section}{DoldKan}.
La seconde utilise le théorème de Hurewicz pour les paires pour démontrer l'existence de tours de Postnikov de fibrations principales pour
les espaces simples. Ce résultat est central pour l'unicité de la construction $+$.

\begin{defi}
  Soit $n\geq 1$ et $\Pi$ un groupe qui soit abélien si $n\geq 2$. Un espace d'Eilenberg-MacLane pour $n$ et $\Pi$ est un ensemble simplicial
  connexe $K(\Pi,n)$ tel que~:
  $$\pi_m(K(\Pi,n))=\begin{cases}\Pi\text{ si }m=n\\ 0\text{ sinon}\end{cases}$$
\end{defi}

\begin{rem}
  La donnée de $K(\Pi,n)$ comprend un isomorphisme $\pi_m(K(\Pi,n))\simeq \Pi$. Si $n=1$ et $\Pi$ n'est pas abélien, il faut choisir un point base pour
  $K(\Pi,n)$. En pratique, on se restreint ici aux espaces simples.
\end{rem}

\begin{rem}
  Dans la suite de la section et le reste du mémoire, $K(\Pi, n)$ désigne toujours un espace d'Eilenberg-MacLane pour $n$ et $\Pi$.
\end{rem}

\begin{prop}
  Soit $n\geq 1$ et $\Pi$ un groupe qui soit abélien si $n\geq 2$. Alors il existe un espace d'Eilenberg-MacLane pour $n$ et $\Pi$.
  Plus précisément, on peut utiliser les deux constructions suivantes.

  Si $n=1$, $N\Pi$ est un espace d'Eilenberg-MacLane pour $n$ et $\Pi$.

  Si $n\geq 1$ et $\Pi$ est abélien, $C_*$ le complexe de chaîne défini par~:
  $$C_n=\begin{cases}\Pi\text{ si }m=n\\ 0\text{ sinon}\end{cases}$$
  On note $\Gamma C_*$ le groupe abélien simplicial associé (voir le \sref{théorème}{DoldKan}).
  Alors $\Gamma C_*$ est un espace d'Eilenberg-MacLane pour $n$ et $\Pi$.
\end{prop}

\begin{proof}
  Si $n=1$, $N\Pi$ convient. On se place maintenant dans le second cas. Par le \sref{théorème}{DoldKan}, on a pour $m\geq 0$~:
  $$\pi_m(\Gamma C_*,0)=H_n(C_ *)=\begin{cases}\Pi\text{ si }m=n\\ 0\text{ sinon}\end{cases}$$
\end{proof}

Ci-dessous, on ne démontre l'unicité que dans le cas abélien, pour éviter de devoir passer par les espaces pointés.

\begin{prop}
  Soit $n\geq 1$, $\Pi$ un groupe abélien, et $K(\Pi,n)$ un complexe de Kan et un espace d'Eilenberg-MacLane pour $n$ et $\Pi$.
  Si $K'$ en est un autre, alors il existe une
  équivalence d'homotopie faible $\theta:K(\Pi,n)\ra K'$ telle que $\pi_n(\theta)$ d'identifie à $\id_\Pi$.
\end{prop}

\begin{proof}
  Soit $I$ un ensemble de générateurs de $\Pi$, $X_{n-1}:=\bigvee_I\Delta^n/\partial\Delta^n$ et $\theta_{n-1}:X_{n-1}\ra K'$ le morphisme
  associé à l'inclusion $I\ra \pi_n(K,*)$. Soit $J_n$ un ensemble de générateurs du noyau de $\Z^{(I)}\ra \Pi$. On définit $X_n$ comme la somme amalgamée~:
  \begin{center}
    \begin{tikzcd}[column sep = 10ex]
    	{\bigsqcup_{J_n}\partial\Delta^{n+1}} & X_{n-1} \\
    	{\bigsqcup_{J_n}\Delta^{n+1}} & X_n
    	\arrow["\mathlarger{\mathlarger{\mathlarger{\mathlarger{\lrcorner}}}}"{anchor=center, pos=0.1, rotate=180}, draw=none, from=2-2, to=1-1]
    	\arrow["{\sqcup_j(a,\dotsc,a,\gamma_j)}", from=1-1, to=1-2]
    	\arrow[hook, from=1-2, to=2-2]
    	\arrow[hook, from=1-1, to=2-1]
    	\arrow[from=2-1, to=2-2]
    \end{tikzcd}
  \end{center}
  On a alors, par le \sref{lemme d'extension}{lemmedextension}, une factorisation $\theta_n:X_n\ra K'$ de $\theta_{n-1}$. De plus,
  on a maintenant nécessairement que $\pi_n(\theta_n)$ est un isomorphisme.
  Nous pouvons maintenant définir par récurrence $X_m$ pour $m\geq n$ de la façon suivante. Soit $J_m$ un ensemble de générateurs
  de $\pi_m(X_m)$. On définit alors $X_{m+1}$ comme la somme amalgamée~:
  \begin{center}
    \begin{tikzcd}[column sep = 10ex]
    	{\bigsqcup_{J_m}\partial\Delta^{n+1}} & X_{m} \\
    	{\bigsqcup_{J_m}\Delta^{n+1}} & X_{m+1}
    	\arrow["\mathlarger{\mathlarger{\mathlarger{\mathlarger{\lrcorner}}}}"{anchor=center, pos=0.1, rotate=180}, draw=none, from=2-2, to=1-1]
    	\arrow["{\sqcup_j(a,\dotsc,a,\gamma_j)}", from=1-1, to=1-2]
    	\arrow[hook, from=1-2, to=2-2]
    	\arrow[hook, from=1-1, to=2-1]
    	\arrow[from=2-1, to=2-2]
    \end{tikzcd}
  \end{center}
  Maintenant, par le \sref{lemme d'extension}{lemmedextension}, on peut étendre $\theta_n$ en $\theta_m:X_m\ra K'$ pour chaque $m\geq 0$.
  Quitte à le faire dans l'ordre, on peut choisir une famille compatible de $(\theta_m)_m$. Alors on dispose d'un morphisme~:
  $$\theta:X:=\colim_m X_m\ra K'$$
  qui est un isomorphisme sur chaque $\pi_m$, et donc une équivalence faible.

  Cependant, on remarque que la construction de $X$ est indépendante de $K'$, en effet elle ne dépend que du choix de $I$ et des $J_m$
  pour $m\geq n$. Donc chaque espace d'Eilenberg-MacLane pour $n$ et $\Pi$ est homotopiquement équivalent à $X$. Ce qui conclut. 
\end{proof}

\begin{theo}\label{EMfoncteur}
  Soit $n\geq 1$ et $\Pi$ un groupe abélien. Alors, si on note $[-,-]$ pour $\Hom{\Ho{\DEns}}{-}{-}$, on a, pour $K(\Pi,n)$ un espace
  d'Eilenberg-MacLane pour $n$ et $\Pi$, un isomorphisme de foncteurs~:
  $$[-,K(\Pi,n)]\simeq H^n(-;\Pi)$$
\end{theo}

\begin{proof}
  Il suffit de le montrer pour une construction. On choisit donc $K(\Pi,n):=\Gamma C_*$, avec $C_*$ le complexe de chaîne définit par~:
  $$C_n=\begin{cases}\Pi\text{ si }m=n\\ 0\text{ sinon}\end{cases}$$
  On utilise maintenant les structures de modèles sur $\DEns$ et $\DAb$. Dans $\DEns$, on a le cône fonctoriel en $X$ suivant~:
  \begin{center}
    % https://q.uiver.app/?q=WzAsMyxbMCwwLCJYXFxzcWN1cCBYIl0sWzEsMCwiWFxcdGltZXNcXERlbHRhXjEiXSxbMiwwLCJYIl0sWzAsMSwiIiwwLHsic3R5bGUiOnsidGFpbCI6eyJuYW1lIjoiaG9vayIsInNpZGUiOiJ0b3AifX19XSxbMSwyLCJcXHNpbSIsMCx7InN0eWxlIjp7ImhlYWQiOnsibmFtZSI6ImVwaSJ9fX1dXQ==
    \begin{tikzcd}
    	{X\sqcup X} & {X\times\Delta^1} & X
    	\arrow[hook, from=1-1, to=1-2]
    	\arrow["\sim", two heads, from=1-2, to=1-3]
    \end{tikzcd}
  \end{center}
  Dans $\DAb$, on a le cône fonctoriel en $X\in \DEns$ suivant~:
  \begin{center}
    % https://q.uiver.app/?q=WzAsMyxbMCwwLCJcXFogWFxcc3FjdXAgXFxaIFgiXSxbMSwwLCJcXFogWFxcb3RpbWVzXFxaIFxcRGVsdGFeMSJdLFsyLDAsIlxcWiBYIl0sWzAsMSwiIiwwLHsic3R5bGUiOnsidGFpbCI6eyJuYW1lIjoiaG9vayIsInNpZGUiOiJ0b3AifX19XSxbMSwyLCJcXHNpbSIsMCx7InN0eWxlIjp7ImhlYWQiOnsibmFtZSI6ImVwaSJ9fX1dXQ==
    \begin{tikzcd}
    	{\Z X\oplus \Z X} & {\Z X\otimes\Z \Delta^1} & {\Z X}
    	\arrow[hook, from=1-1, to=1-2]
    	\arrow["\sim", two heads, from=1-2, to=1-3]
    \end{tikzcd}
  \end{center}
  Or, pour $A$ un groupe abélien simplicial, les morphismes dans $\DEns$ de $X\times \Delta^1$ dans $A$ s'identifient aux morphismes
  dans $\DAb$ de $\Z X\otimes\Z \Delta^1$ dans $A$. Or, comme les groupes abéliens dans $\DEns$ sont fibrants,
  on a~:
  $$[X,A]_{\DEns}\simeq [\Z X,A]_{\DAb}\simeq [N\Z X,N A]_{\Chp}$$
  Or, $N\Z X\ra \Z X$ est une équivalence en homologie d'après le \sref{théorème}{DoldKan}. Et comme $\Z X$ est cofibrant,
  on a donc un isomorphisme fonctoriel~:
  $$[X,K(\Pi,n)]_{\DEns}\simeq [\Z X,C_*]_{\Chp}\simeq \mathrm{Ext}_\Z^n(\Z X,\Pi)\simeq H^n(X;\Pi)$$
\end{proof}

Nous introduisons maintenant une structure supplémentaire sur certaines tours de Postnikov.

\begin{defi}
  Soit $X$ un ensemble simplicial simple. Une tour de Postnikov de fibrations principales pour $X$ est un diagramme commutatif~:
  \begin{center}
    \begin{tikzcd}
      && {X(n)} \\
      {X} && {X(n-1)} & {K(n+1,\pi_n(X))} \\
      && {X(1)} & {K(3,\pi_2(X))} \\
      && {X(0)\simeq *} & {K(2,\pi_1(X))}
      \arrow["q_n", two heads, from=1-3, to=2-3]
      \arrow["{u_{n-1}}", from=2-3, to=2-4]
      \arrow[dotted, two heads, from=2-3, to=3-3]
      \arrow["{u_1}", from=3-3, to=3-4]
      \arrow["q_1", two heads, from=3-3, to=4-3]
      \arrow["{u_0}", from=4-3, to=4-4]
      \arrow["i_n", from=2-1, to=1-3]
      \arrow["i_{n-1}", from=2-1, to=2-3]
      \arrow["i_1", from=2-1, to=3-3]
      \arrow["i_0", from=2-1, to=4-3]
    \end{tikzcd}
  \end{center}
  Où les $X(n)$, $i_n$ et $q_n$ forment une tour de Postnikov, et tels que les suites~:
  $$X(n)\lra X(n-1)\lra K(n+1,\pi_n(X))$$
  soient des suites fibres pour tout $n\geq 1$.
\end{defi}

\begin{lem}\label{lemmePostnikovfibrationsprincipales}
  Soit $A\subset X$ une paire d'ensembles simpliciaux connexes. Soit $n\geq 1$ et $x\in A$. On suppose que~:
  $$\pi_m(X,A,x)=\begin{cases}\Pi\text{ si }m=n \\0\text{ sinon} \end{cases}$$
  avec $\Pi$ un groupe abélien.
  Alors, si $\pi_1(A,x)$ agit trivialement sur $\pi_n(X,A,x)$, il existe $\theta:X\ra K(\Pi,n)$ tel que la suite~:
  $$A\lra X\lra K(\Pi,n)$$
  soit une suite fibre.
\end{lem}

\begin{proof}
  Par le \sref{théorème de Hurewicz pour les paires}{theoremeHurewiczpaire}, on a un isomorphismes~:
  $$\mathcal{H}_n(X,A,x):\pi_n(X,A,x)\simeq H_n(X,A)$$
  De plus, comme $\pi_m(X,A,x)=0$ pour $m<n$, $\pi_m(A)\ra\pi_m(X)$ est une surjection pour $m<n$.
  Donc on peut supposer sans perte de généralité que $A_m=X_m$ pour $m<n$. Ainsi, $\pi_m(X/A,A/A)=0$ pour $m<n$.
  Donc, par le \sref{théorème de Hurewitz}{theoremeHurewitzpin}, on a également un isomorphisme~:
  $$\mathcal{H}_n(X/A,A/A,A/A):\pi_n(X/A,A/A,A/A)\ra H_n(X/A,A/A)$$
  Or le morphisme naturel $(X,A) \ra (X/A,A/A)$ induit un isomorphisme en homologie. Donc, par fonctorialité du morphisme de Hurewicz
  $\mathcal{H}_n$, $\pi_n(X,A,x)\ra \pi_n(X/A,A/A)$ est un isomorphisme.
  À l'aide du \sref{lemme d'extension}{lemmedextension},
  nous pouvons construire un morphisme~:
  $$u:X/A\ra K(\Pi,n)$$
  qui soit un isomorphisme sur les $\pi_m$ pour $m\leq n$. On note $F\ra X$ la fibre homotopique du morphisme induit $\theta:X\ra K(\Pi,n)$.
  Comme $A\ra K(\Pi,n)$ est contractile, on a un morphisme de paires $(A,X)\ra (F,X)$. C'est un isomorphisme sur les groupes d'homotopie des paires
  pour tout $m\geq 1$. Donc, par la suite exacte longue en homotopie, $A\ra F$ est une équivalence d'homotopie. Donc la suite~:
  $$A\lra X\lra K(\Pi,n)$$
  est une suite fibre.
\end{proof}

\begin{theo}\label{Postnikovfibrationsprincipales}
  Soit $X$ un ensemble simplicial simple. Alors il existe une tour de Postnikov de fibrations principales pour $X$.
\end{theo}

\begin{proof}
  On se donne des $X(n)$, $i_n:X\ra X(n)$ et $q_n:X(n)\ra X(n-1)$ qui forment une tour de Postnikov pour $X$.
  Alors, la paire $(X(n),X(n-1))$ vérifie les hypothèses du \sref{lemme}{lemmePostnikovfibrationsprincipales}.
  En effet, $\pi_1(X(n-1),x)=\pi_1(X,x)$ agit trivialement sur les $\pi_m(X(n),x)$ et les $\pi_m(X(n-1),x)$, donc également
  sur les groupes $\pi_m(X(n),X(n-1),x)$ par la suite exacte longue en homotopie.
  Ceci conclut.
\end{proof}

\section*{Remerciements} Je tiens à remercier mon directeur de stage Frédéric Déglise pour ses explications et le temps qu'il m'a accordé.

\bibliographystyle{alpha}
\bibliography{references}

\end{document}