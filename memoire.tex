\documentclass{amsart}
% package
\usepackage[utf8]{inputenc}
\usepackage[T1]{fontenc}
\usepackage[french]{babel}
%\usepackage[backend=bibtex,style=verbose-trad2]{biblatex}
\usepackage{cite}
\usepackage{amsmath,amssymb}
\usepackage{amsfonts}
\usepackage{amsthm}
\usepackage{calc}
\usepackage{mathtools}
\usepackage{relsize}

\usepackage{tikz}
\usetikzlibrary{positioning}
\usepackage{tikz-cd}


\tikzcdset{arrow style=tikz}

% Pour les liens hypertexte
\usepackage[hidelinks]{hyperref}
\newcommand{\sref}[2]{\hyperref[#2]{#1 \ref*{#2}}}

% Numérotation des théorèmes, propositions, définitions, lemmes, remarques consécutives par subsection
\theoremstyle{plain}
\newtheorem{theo}{Théorème}[section]
\newtheorem{theodefi}[theo]{Théorème\,-\,définition}
\newtheorem{prop}[theo]{Proposition}
\newtheorem{propdefi}[theo]{Proposition\,-\,définition}
\newtheorem{coro}[theo]{Corollaire}
\newtheorem{lem}[theo]{Lemme}
\newtheorem{lemdefi}[theo]{Lemme\,-\,définition}
\newtheorem{conj}[theo]{Conjecture}

\theoremstyle{definition}
\newtheorem{defi}[theo]{Définition}
\newtheorem{ex}[theo]{Exemple}
\newtheorem{cstr}[theo]{Construction}

\theoremstyle{remark}
\newtheorem{rem}[theo]{Remarque}

\renewcommand{\proofname}{Démonstration}

\renewcommand{\descriptionlabel}[1]{%
  \hspace\labelsep \upshape\bfseries #1%
}

% Commande ensemble de nombres
\newcommand{\ensemblenombre }[1]{\mathbb{#1}}
\newcommand{\N}{\ensemblenombre{N}}
\newcommand{\Np}{\ensemblenombre{N}\setminus\{0\}}
\newcommand{\Z}{\ensemblenombre{Z}}
\newcommand{\Q}{\ensemblenombre{Q}}

\newcommand{\M}{\mathcal{M}}
\newcommand{\A}{\mathcal{A}}
\newcommand{\E}{\mathcal{E}}
\newcommand{\Ring}[1]{\mathcal{O}_{#1}}

\newcommand{\Ob}[1]{\mathrm{Ob}\:#1}
\newcommand{\Hom}[3]{\mathrm{Hom}_{#1}(#2,#3)}
\newcommand{\Iso}[3]{\mathrm{Iso}_{#1}(#2,#3)}
\newcommand{\Isos}[1]{\mathrm{Isos}(#1)}
\newcommand{\Aut}[1]{\mathrm{Aut}({#1})}
\newcommand{\id}{\mathrm{id}}
\newcommand{\op}{^\mathrm{op}}
\newcommand{\ab}{^\mathrm{ab}}
\newcommand{\Fon}[2]{\mathrm{Fon}(#1,#2)}
\newcommand{\Ens}{\mathrm{Ens}}
\newcommand{\DEns}{\widehat{\mathbf{\Delta}}}
\newcommand{\DCat}{\mathbf{\Delta}}
\newcommand{\Chp}{\mathrm{\mathrm{Ch}_+}}
\newcommand{\DAb}{\mathbf{\Delta}\mathrm{Ab}}
\newcommand{\Cat}{\mathrm{Cat}}
\newcommand{\CatEx}{\mathrm{CatEx}}
\newcommand{\Ab}{\mathrm{Ab}}
\newcommand{\mylim}{\mathrm{lim}}
\newcommand{\colim}{\mathrm{colim}}
\newcommand{\hocolim}[2]{\mathrm{hocolim}_{#1}\:{#2}}
\newcommand{\Top}{\mathrm{Top}}
\newcommand{\Grp}{\mathrm{Grp}}
\newcommand{\Supp}[1]{\mathrm{Supp}({#1})}

\newcommand{\sk}[2]{\mathrm{sk}_{#1}\;{#2}}

\newcommand{\Ho}[1]{\mathrm{Ho}({#1})}
\newcommand{\Sing}{\mathrm{Sing}}
\newcommand{\xlongrightrightarrows}[2]{\mathop{\rightrightarrows}_{#1}^{#2}}


\newcommand{\coend}[2]{\bigsqcup_{\phi:[n]\ra [m]}{#1}_m\times {#2}^n\xlongrightrightarrows{\sqcup\phi^*\times\id}{\sqcup\id\times\phi_*}\bigsqcup_{[n]}{#1}_n\times {#2}^n}

\newcommand{\GL}[2]{\mathrm{GL}_{#1}(#2)}
\newcommand{\End}[2]{\mathrm{End}_{#1}(#2)}
\newcommand{\EGL}[2]{\mathrm{E}_{#1}(#2)}

\newcommand{\Proj}[1]{\mathrm{P}({#1})}
\newcommand{\Pgr}[1]{\mathrm{Pgr}({#1})}
\newcommand{\Modf}[1]{\mathrm{Modf}({#1})}
\newcommand{\Tor}{\mathrm{Tor}}
\newcommand{\fTor}[4]{\mathrm{Tor}_{#1}^{#2}({#3},{#4})}

\newcommand{\xrightarrowtail}[1]{\overset{#1}{\rightarrowtail}}
\newcommand{\xtwoheadrightarrow}[1]{\overset{#1}{\twoheadrightarrow}}
\newcommand{\xtwoheadleftarrow}[1]{\overset{#1}{\twoheadleftarrow}}
\newcommand{\ra}{\rightarrow}
\newcommand{\lra}{\longrightarrow}
\newcommand{\mono}{\rightarrowtail}
\newcommand{\epi}{\twoheadrightarrow}
\newcommand{\exa}[3]{0\ra {#1}\ra {#2}\ra {#3}\ra 0}
\newcommand{\exac}[3]{{#1}\rightarrowtail {#2}\twoheadrightarrow {#3}}
\newcommand{\exaname}[5]{0\ra {#1}\xrightarrow{#2} {#3}\xrightarrow{#4} {#5}\ra 0}
\newcommand{\exacname}[5]{{#1}\xrightarrowtail{#2} {#3}\xtwoheadrightarrow{#4} {#5}}

\newcommand{\myker}[1]{\mathrm{ker}({#1})}
\newcommand{\coker}[1]{\mathrm{coker}({#1})}
\newcommand{\im}[1]{\mathrm({#1})}

\newcommand{\Spec}[1]{\mathrm{Spec}({#1})}
\newcommand{\Open}[1]{\mathrm{D}({#1})}

% Commande "ensemble tel que"
\newcommand{\enstq}[2]{\left\{#1\,\middle|\,#2\right\}}


\title{K-théorie algébrique}
\author{Pierre Godfard}
\date{}


\begin{document}
\renewcommand{\abstractname}{Introduction}
\begin{abstract}
  [TBD]
\end{abstract}

\maketitle

\tableofcontents

\section{\texorpdfstring{Les groupes $K_0$, $K_1$ et $K_2$}{Les groupes KO, K1 et K2}}\label{sectionK0}

Dans cette section, nous allons définir et étudier les groupes $K_0$, $K_1$ et $K_2$ associés à un anneau. Nous définirons également le $K_0$ d'une catégorie exacte.

\subsection{\texorpdfstring{Les groupes $K_0$  et $K'_0$ d'un anneau}{Les groupes K0  et K'0 d'un anneau}}

Pour $A$ un anneau, nous noterons $\Proj{A}$ la catégorie des modules projectifs de type fini sur $A$ et $\Modf{A}$ la catégorie des modules de type fini sur $A$.
Si $A$ n'est pas commutatif, les modules sur $A$ seront toujours par défaut des $A$-modules à gauche.

\begin{defi}[$K_0$  et $K'_0$]
  Le groupe abélien $K_0(A)$ est le groupe de Grothendieck de $\Proj{A}$~: le quotient du groupe abélien libre $\bigoplus_P \Z\cdot [P]$ engendré par les objets de $\Proj{A}$,
  par les relations $[R] = [P] + [Q]$ pour chaque suite exacte $0\ra P\ra R\ra Q\ra 0$.

  On définit de même le groupe $K'_0(A)$ comme le groupe de Grothendieck de $\Modf{A}$.
\end{defi}

On appelle $K_0(A)$ le groupe de $K$-théorie de $A$ et $K'_0(A)$ le groupe de $K'$-théorie de $A$. En général, pour utiliser $K'_0(A)$,
on préfère se restreindre au cas où $A$ est noethérien, de telle sorte que $\Modf{A}$ soit une catégorie abélienne.

\begin{rem}
  Les catégories $\Proj{A}$ et $\Modf{A}$ ont un ensemble de classes d'isomorphisme. Or si $f:M\ra M'$ est un isomorphisme, les suites exactes $\exa{0}{0}{0}$ et $\exa{M}{M'}{0}$,
  induisent $0=[0]$ et $[M]=[M']$ dans les groupes de Grothendieck. Donc les éléments des groupes $K_0(A)$  et $K'_0(A)$ forment des ensembles.
\end{rem}

Le foncteur $\Proj{A}\times \Proj{A}\lra \Proj{A},\;(P,Q)\mapsto P\oplus Q$ induit un morphisme $\oplus:K_0(A)\times K_0(A)\lra K_0(A)$.
Or pour tout $M$ et $N$ $A$-modules, on a une suite exacte $\exa{M}{M\oplus N}{N}$. Donc $\oplus$ est l'addition. De même avec $\Modf{A}$ et $K'_0$.

Si $A$ est commutatif, le foncteur $\Proj{A}\times \Proj{A}\lra \Proj{A},\;(P,Q)\mapsto P\otimes Q$ passe également aux quotients en un morphisme
$\otimes:K_0(A)\times K_0(A)\lra K_0(A)$ (les modules projectifs sont plats).

De même, $\Proj{A}\times \Modf{A}\lra \Modf{A},\;(P,M)\mapsto P\otimes M$ induit $\otimes:K_0(A)\times K'_0(A)\lra K'_0(A)$.
On vérifie aisément que~:
\begin{prop}
  Si $A$ est commutatif, l'application $\otimes:K_0(A)\times K_0(A)\lra K_0(A)$ est bilinéaire et induit sur $K_0(A)$ une structure d'anneau de neutre $[0]$ et d'unité $[A]$.

  Dans ce cas, $\otimes:K_0(A)\times K'_0(A)\lra K'_0(A)$ fait de $K_0'(A)$ un $K_0(A)$-module.
\end{prop}

Ces definitions sont fonctorielles en $A$. En effet, si $f:A\ra B$ est un morphisme d'anneaux, le foncteur $-\otimes_A B:\Proj{A}\ra \Proj{B}$
induit un morphisme $f^*:K_0(A)\ra K_0(B)$, et on a, si $g:B\ra C$, $(gf)^*=g^*f^*$. Le morphisme $f^*$ est un morphisme d'anneau si $A$ et $B$ sont commutatifs.
De même, si $f:A\ra B$ est un morphisme plat, $-\otimes_A B:\Modf{A}\ra \Modf{B}$ induit un morphisme $f^*:K'_0(A)\ra K'_0(B)$ (de $K_0(A)$-modules si $A$ et $B$ sont commutatifs).
La notation "contravariante" $f^*$ s'explique par la definition de la $K$-théorie d'un schema.

Si $f:A\ra B$ est un morphisme projectif et fini (c'est-à-dire que $B$ est un $A$ module projectif de type fini),
alors le foncteur d'oubli $(-)_A:\Proj{B}\ra \Proj{A}, M\mapsto M$ induit un morphisme \textbf{de groupe} $f_*:K_0(B)\ra K_0(A)$.
De même, si $f:A\lra B$ est un morphisme fini, le foncteur d'oubli $(-)_A:\Modf{B}\ra \Modf{A}$ induit un morphisme \textbf{de groupe} $f_*:K'_0(B)\ra K'_0(A)$.

Si $f_*$ et $f^*$ sont définis pour $K_0$, et si $A$ et $B$ sont commutatifs, l'isomorphisme entre les foncteurs
$(P,N)\mapsto (P\otimes_AN)\otimes B$ et $(P,N)\mapsto P\otimes_B(N\otimes B)$
induit la formule de projection suivante. Pour tout $a$ dans $K_0(A)$ et $b$ dans $K_0(B)$~:
$$f_*(b\cdot f^*(a))=f_*(b)\cdot a\text{ dans }K_0(A)$$
De même, si $f_*$ est défini pour $K_0$ et $K'_0$, et $f^*$ est défini pour $K'_0$, alors pour tout $a$ dans $K'_0(A)$ et $b$ dans $K_0(B)$~:
$$f_*(b\cdot f^*(a))=f_*(b)\cdot a\text{ dans }K'_0(A)$$

On peut étendre le cadre où $f_*$ en $K$-théorie et $f^*$ en $K'$-théorie sont définis.
Pour les definitions de la dimension projective et de la dimension $\Tor$, voir le \cite[Chp.4]{Weib2}.

\begin{defi}
  Un morphisme d'anneau $f:A\ra B$ est dit de dimension $\Tor$ finie si $B$ est un $A$-module de dimension $\Tor$ finie.
\end{defi}

Pour étendre $f^*$, nous utilisons le résultat suivant.

\begin{prop}\label{KprimeTordimfinie}
  Soit $f:A\ra B$ de dimension $\Tor$ finie avec $A$ noethérien.
  Soit $\mathcal{M}$ la sous-catégorie pleine de $\Modf{A}$ des modules $M$ tels que $\fTor{i}{A}{B}{M}=0$ pour tout $i\geq 1$. On note $K'_0(\mathcal{M})$
  son groupe de Grothendieck. Alors l'inclusion $\mathcal{M}\ra \Modf{A}$ induit un isomorphisme $K'_0(\mathcal{M})\ra K'_0(A)$.
\end{prop}

\begin{proof}
  Nous allons décrire un morphisme inverse $K'_0(A)\ra K'_0(\mathcal{M})$. On note $n<\infty$ la dimension $\Tor$ de $B$ sur $A$. Soit $M$ un $A$-module de type fini.
  On se donne $P_*\ra M$ une résolution projective de $M$ par des $A$-modules de type fini (possible car $A$ est noethérien).
  On pose $\tilde{M} := \coker{d:P_{n+1}\ra P_n}$. Par décalage, $\tilde{M}$ est un objet de $\mathcal{M}$.
  On obtient donc une résolution finie $0\ra \tilde{M}\ra P_{n-1}\ra \dotsb\ra P_0\ra M\ra 0$ de $M$ par des objets de $\mathcal{M}$, que l'on note $Q_*\ra M$.
  On pose alors~:
  $$u(M) := \sum_{i=0}^n (-1)^i[Q_i]\in K'_0(\mathcal{M})$$
  Montrons que $u(M)$ est indépendant de la résolution $P_*\ra M$. Soit $P'_*\ra M$ une autre résolution. On montre facilement qu'il existe une troisième
  résolution $P''_*\ra M$ et deux morphismes $a:P''_*\ra P_*$ et $a':P''_*\ra P_*$ surjectifs en chaque degrés, par exemple en prenant une résolution
  projective du complexe $P\times_{M[0]}P'$ par des modules de type fini.
  Alors $\myker{a}_*\ra 0$ est une résolution de $0$ par des objets de $\mathcal{M}$.
  On note $\tilde{M}:= \coker{d:P_{n+1}\ra P_n}$, $\tilde{M''}:= \coker{d:P''_{n+1}\ra P''_n}$ et $\tilde{K}:= \coker{d:\myker{a}_{n+1}\ra \myker{a}_n}$.
  Alors $\exa{\tilde{K}}{\tilde{M''}}{\tilde{M}}$ est exacte. En effet la suite est exacte à droite par le lemme du serpent et exacte à gauche car $\tilde{K}$ et
  $\tilde{M''}$ s'injectent dans $P''_{n-1}$. Si on note $Q''_*\ra M$ la résolution finie associée à $P''_*$ et $K_*\ra 0$ celle associée à $\myker{a}_*\ra 0$,
  on a~:
  $$\sum_{i=0}^n (-1)^i[Q_i]=\sum_{i=0}^n (-1)^i[Q''_i]-\sum_{i=0}^n (-1)^i[K_i]=\sum_{i=0}^n (-1)^i[Q''_i]\text{ dans }K'_0(\mathcal{M})$$
  Montrons maintenant que si $\exa{M'}{M}{M''}$ est une suite exacte dans $\Modf{M}$, alors $u(M) = u(M'')+u(M')$.
  Par le lemme du fer à cheval, on dispose de résolutions respectant la suite exacte~:
  \begin{center}
    \begin{tikzcd}
      0\rar & P'_* \rar\dar & P_* \rar\dar & P''_* \rar\dar & 0\\
      0\rar & M'   \rar     & M   \rar     & M''   \rar     & 0
    \end{tikzcd}
  \end{center}
  Par le même argument que ci-dessus $\exa{\tilde{M'}}{\tilde{M}}{\tilde{M''}}$ est exacte, et donc $\exa{Q'_*}{Q_*}{Q''_*}$ est exacte.
  Il est alors clair que $u(M) = u(M'')+u(M')$.

  On vérifie immédiatement que $u$ est inverse à droite et à gauche de $K'_0(\mathcal{M})\ra K'_0(A)$.
\end{proof}

Ainsi, avec les hypothèses et notations de la proposition, le foncteur $-\otimes_A B:\mathcal{M}\ra \Modf{B}$ induit un morphisme $K'_0(\mathcal{M})\ra K'_0(B)$
et donc un morphisme $f^*:K'_0(A)\ra K'_0(B)$. Si $B$ est plat sur $A$, on retrouve la définition plus haut.

Pour étendre $f_*$, on donne une formule explicite.

\begin{propdefi}
  Soit $f:A\ra B$ fini et de dimension $\Tor$ finie avec $A$ noethérien. Alors tout $B$ module projectif $P$ admet
  une $A$-résolution projective finie par des modules de type fini $Q_*\ra P$. On pose~:
  $$f_*([P]) := \sum_{i} (-1)^i[Q_i]\in K_0(A)$$
  Alors $f_*:K_0(B)\ra K_0(A)$ est un morphisme de groupe.
\end{propdefi}

On remarque que la définition ci-dessus étend la définition plus haut dans le cas où $f$ est projectif de type fini.

\begin{proof}
  Soit $n$ la dimension $\Tor$ de $B$ sur $A$. Comme $A$ est noethérien, pour tout module de type fini $M$ sur $A$, sa dimension projective
  est égale à sa dimension $\Tor$. De plus, pour tout $B$-module projectif de type fini $P$, $\Tor\dim(P)=n$. Donc tout tel module $P$ admet une
  $A$-résolution projective finie par des modules de type fini $Q_*\ra P$. 

  Montrons que $\sum_{i} (-1)^i[Q_i]$ est indépendant de la résolution $Q_*\ra P$. Comme dans la \sref{proposition}{KprimeTordimfinie}, on peut
  supposer qu'il existe $u:Q'_*\ra Q_*$ surjection entre les deux résolutions considérées. Alors,
  $K:=\myker{u}_*$ est un complexe fini acyclique de modules projectifs de type fini. Donc~:
  $$\sum_{i} (-1)^i[Q'_i]=\sum_{i} (-1)^i[Q_i]+\sum_{i} (-1)^i[K_i]=\sum_{i} (-1)^i[Q_i]$$

  Montrons maintenant que si $\exa{P'}{P}{P''}$ est une suite exacte dans $\Proj{B}$, alors $f_*[P] = f_*[P'']+f_*[P']$.
  Par le lemme du fer à cheval, on dispose de résolutions finies respectant la suite exacte~:
  \begin{center}
    \begin{tikzcd}
      0\rar & Q'_* \rar\dar & Q_* \rar\dar & Q''_* \rar\dar & 0\\
      0\rar & P'   \rar     & P   \rar     & P''   \rar     & 0
    \end{tikzcd}
  \end{center}
  On a alors clairement $f_*[P] = f_*[P'']+f_*[P']$.  
\end{proof}

On peut maintenant étendre la formule de projection.

\begin{prop}[formule de projection]
  Soit $f:A\ra B$ fini et de dimension $\Tor$ finie, avec $A$ et $B$ commutatifs et $A$ noethérien.
  Alors pour tout $a$ dans $K_0'(A)$ et $b$ dans $K_0(B)$~:
  $$f_*(b\cdot f^*(a))=f_*(b)\cdot a\text{ dans }K'_0(A)$$
  De même, pour tout $a$ dans $K_0(A)$ et $b$ dans $K_0(B)$~:
  $$f_*(b\cdot f^*(a))=f_*(b)\cdot a\text{ dans }K_0(A)$$
\end{prop}

\begin{proof}
  On reprend les notations de la \sref{proposition}{KprimeTordimfinie}. On démontre la première formule, la seconde se démontre de façon similaire.
  On se ramène au cas où $a=[M]$ avec $M$ dans $\mathcal{M}$ et $b=[P]$ avec $P$ dans $\Proj{B}$. On se donne $Q_*\ra P$ une résolution
  finie de $P$ par des $A$-modules projectifs de type fini. Or $M\otimes_A Q_*\ra M\otimes_A P$ est exacte
  car pour tout $i>0$, $\fTor{i}{A}{M}{P} = 0$. Donc~:
  \begin{align*}
    f_*(b\cdot f^*(a)) &= [P\otimes_B (B\otimes_A M)] \\
                       &= [P\otimes_A M] \\
                       &= \sum_i(-1)^i[Q_i\otimes_A M] \\
                       &= (\sum_i(-1)^i[Q_i])\cdot [M] \\
                       &= f_*(b)\cdot a
  \end{align*}
\end{proof}


Nous allons maintenant nous intéresser au cas particulier des anneaux réguliers. Nous verrons que si $A$ est régulier, alors $K_0(A)$ et $K'_0(A)$ coïncident.

\begin{defi}
  Un anneau noethérien (à gauche) $A$ est dit proj-régulier si tout $A$-module de type fini est de dimension projective finie.
\end{defi}

Dans le cas commutatif, cette definition coïncide avec la definition usuelle d'anneau régulier (cf. le \sref{corollaire}{regulierprojregulier} ci-dessous).

\begin{theo}\label{thmanneauxreguliers}
  Soit $(A,\mathfrak{m})$ un anneau local commutatif noethérien. S'équivalent~:
  \begin{description}
    \item[(i)] $A$ est de dimension projective finie~;
    \item[(ii)] $A$ est local régulier ($\mathrm{Krull}\dim A = \dim_{A/\mathfrak{m}} \mathfrak{m}/\mathfrak{m}^2$)~;
    \item[(iii)] $k:=A/\mathfrak{m}$ est de dimension projective finie sur $A$.
  \end{description} 
  Et dans ce cas, la dimension projective de $A$ est la dimension projective de $k:=A/\mathfrak{m}$ comme $A$-module.
\end{theo}

Nous renvoyons au \cite[Chp.4]{Weib2} pour la démonstration et les différentes propriétés élémentaires de la dimension projective qui seront utilisées par la suite.

\begin{coro}\label{regulierprojregulier}
  Un anneau commutatif noethérien $A$ est proj-régulier si et seulement si il est régulier (ie. pour tout $\mathfrak{m}$ idéal maximal,
  $\mathrm{ht}\:\mathfrak{m} = \dim_{A/\mathfrak{m}} \mathfrak{m}/\mathfrak{m}^2$).
\end{coro}

\begin{proof}
  $\mathbf{(\Leftarrow)}$ Soit $P_*\ra M$ une résolution projective par des modules de type fini d'un $A$-module de type fini $M$. On note $Z_n:=\coker{d:P_n\ra P_{n-1}}$.
  Comme pour tout $\mathfrak{m}$ idéal maximal, $M_\mathfrak{m}$ est de dimension projective finie, il existe $n\geq 0$ tel que $(Z_n)_\mathfrak{m}$ soit projectif.
  Mais, comme $A$ est noethérien, $Z_n$ est de présentation finie. Donc $(Z_n)_\mathfrak{m}$ est un $A_\mathfrak{m}$-module libre de type fini.
  Donc, il existe $f_\mathfrak{m}\in A\backslash\mathfrak{m}$ tel que $(Z_n)_{f_\mathfrak{m}}$ soit libre, et donc projectif. Or si $(Z_n)_{f_\mathfrak{m}}$
  est projectif, $(Z_m)_{f_\mathfrak{m}}$ est aussi projectif pour tout $m\geq n$. Donc, comme $\Spec{A}$ est quasi-compact, il existe $n\geq 0$ et $f_1,\dotsc,f_l$
  tels que $\Spec{A}=\bigcup_i\Open{f_i}$ et $(Z_n)_{f_i}$ soit projectif de type fini pour tout $i$. Alors $Z_n$ est projectif,
  et $M$ est de dimension projective finie sur $A$.

  $\mathbf{(\Rightarrow)}$ Soit $\mathfrak{m}$ un idéal maximal. Alors $A/\mathfrak{m}$ est de dimension projective finie sur $A$ et donc aussi sur $A_\mathfrak{m}$.
  Donc, d'après le \sref{théorème}{thmanneauxreguliers}, $A_\mathfrak{m}$ vérifie $\mathrm{Krull}\dim A_\mathfrak{m} = \dim_{A/\mathfrak{m}} \mathfrak{m}/\mathfrak{m}^2$.
  On a donc bien $\mathrm{ht}\:\mathfrak{m} = \dim_{A/\mathfrak{m}} \mathfrak{m}/\mathfrak{m}^2$.
\end{proof}


\begin{theo}\label{KegalKprimeAnneauxReguliers}
  Soit $A$ un anneau commutatif noethérien régulier. Alors le foncteur naturel $\Proj{A}\ra\Modf{A}$ induit un isomorphisme $K_0(A)\ra K'_0(A)$.
\end{theo}

\begin{proof}
  Nous allons décrire un inverse $u:K'_0(A)\ra K_0(A)$. Soit $M$ un $A$-module de type fini, et soit $P_*\ra M$ une résolution finie par des $A$-modules projectifs.
  On pose $u(M):=\sum_i(-1)^i[P_i]$.

  Montrons que $u(M)$ ne dépend pas de la résolution. Comme dans la preuve de la \sref{proposition}{KprimeTordimfinie}, il suffit
  de se restreindre au cas où il existe $a:P'_*\ra P_*$ surjection entre deux résolutions de $M$. Alors $\myker{a}_*\ra 0$ est une résolution projective.
  On a alors $\sum_i(-1)^i[P_i]=\sum_i(-1)^i[P'_i]-\sum_i(-1)^i[\myker{a}_i]=\sum_i(-1)^i[P'_i]$.

  Montrons maintenant que si $\exa{M'}{M}{M''}$ est une suite exacte dans $\Modf{M}$, alors $u(M) = u(M'')+u(M')$.
  Par le lemme du fer à cheval, on dispose de résolutions finies respectant la suite exacte~:
  \begin{center}
    \begin{tikzcd}
      0\rar & P'_* \rar\dar & P_* \rar\dar & P''_* \rar\dar & 0\\
      0\rar & M'   \rar     & M   \rar     & M''   \rar     & 0
    \end{tikzcd}
  \end{center}
  On a alors clairement $u(M) = u(M'')+u(M')$.
\end{proof}

On remarque que ce théorème ressemble beaucoup à la \sref{proposition}{KprimeTordimfinie}. Ces deux résultats sont encore vraie en $K$-théorie supérieure,
et sont des applications d'un même résultat (voir [lien vers plus loin]).


\subsection{\texorpdfstring{Les groupes $K_0$ et $K'_0$ d'un schéma}{Les groupes K0 et K'0 d'un schéma}}

On peut étendre la definition des groupes de $K$ et $K'$-théorie des anneaux commutatifs aux schémas.
Pour $X$ un schéma, nous noterons $\Proj{X}$ la catégorie des modules projectifs localement de type fini sur $X$
et $\Modf{X}$ la catégorie des modules quasi-cohérents localement de type fini sur $X$.

\begin{defi}[$K_0$  et $K'_0$]
  Soit $X$ un schéma.
  Le groupe abélien $K_0(X)$ est le groupe de Grothendieck de $\Proj{X}$~: le quotient du groupe abélien libre $\bigoplus_\mathcal{P} \Z\cdot [\mathcal{P}]$
  engendré par les objets de $\Proj{X}$,
  par les relation $[\mathcal{R}] = [\mathcal{P}] + [\mathcal{Q}]$ pour chaque suite exacte $0\ra \mathcal{P}\ra \mathcal{R}\ra \mathcal{Q}\ra 0$.

  On définit de même le groupe $K'_0(X)$ comme le groupe de Grothendieck de $\Modf{X}$.
\end{defi}

De même que pour les anneaux, nous avons les propriétés suivantes~:
\begin{description}
  \item[(i)] $(K_0(X),\oplus,\otimes)$ est un anneau et le produit tensoriel fait de $K'_0(X)$ un $K_0(X)$-module~;
  \item[(ii)] Si $f:X\ra Y$ est un morphisme de schéma, le foncteur $f^*:\Proj{Y}\ra\Proj{X}$ induit un morphisme d'anneau~:
    $$f^*:K_0(Y)\lra K_0(X)$$
  \item[(ii')] Si $f:X\ra Y$ est un morphisme plat, le foncteur $f^*:\Modf{Y}\ra\Modf{X}$ induit un morphisme d'anneau~:
    $$f^*:K'_0(Y)\lra K'_0(X)$$
  \item[(iii)] Si $f:X\ra Y$ est un morphisme fini localement libre, le foncteur $f_*:\Proj{X}\ra\Proj{Y}$ induit un morphisme \textbf{de groupe}~:
    $$f_*:K_0(X)\lra K_0(Y)$$
    Et on a la formule de projection, pour tout $x$ dans $K_0(X)$ et $y$ dans $K_0(Y)$~:
    $$f_*(x\cdot f^*(y))=f_*(x)\cdot y\text{ dans }K_0(Y)$$
  \item[(iii')] Si $f:X\ra Y$ est un morphisme fini, le foncteur $f_*:\Modf{X}\ra\Modf{Y}$ induit un morphisme~:
    $$f_*:K'_0(X)\lra K'_0(Y)$$
    Et si $f$ est localement libre, on a la formule de projection, pour tout $x$ dans $K_0(Y)$ et $y$ dans $K'_0(Y)$~:
    $$f_*(x\cdot f^*(y))=f_*(x)\cdot y\text{ dans }K'_0(Y)$$
\end{description}


[à terminer, ajouter les morphismes propres et les morphismes de Tor dimension finie]

Nous allons maintenant aborder le cas des schémas réguliers. Si les résultats sont assez similaires au cas des anneaux,
les schémas posent le problème de l'existence de résolutions par des fibrés vectoriels.

\begin{defi}
  Un schéma $X$ est dit régulier si l'anneau local $\Ring{X,x}$ est régulier pour tout point $x$ dans $X$.
\end{defi}

\begin{rem}
  Un schéma, et même un anneau, dont les anneaux locaux sont réguliers n'est pas nécessairement noethérien. Par exemple, c'est le cas du localisé
  de $k[X_1,X_2,X_3,\dotsc]$ par le complémentaire de l'union $\bigcup_n(X_{n^2},\dotsc,X_{(n+1)^2-1})$ (contre-exemple dû à Nagata).
\end{rem}

\begin{lem}\label{RegulierSepareQuotient}
  Soit $X$ un schéma noethérien, régulier et séparé. Alors tout faisceau cohérent sur $X$ est quotient d'un fibré vectoriel.
\end{lem}

La démonstration ci-dessous utilise des propriétés des diviseurs de Cartier et de Weil. Pour plus d'information, voir \cite[chp. 11]{Gort}.

\begin{proof}
  \begin{description}
    \item[(a)] Si $(A,\mathfrak{m})$ est local noethérien et normal, alors~:
    $$\dim(A)\leq 1 \Leftrightarrow \Spec{A}\backslash \{\mathfrak{m}\}\text{ est affine}$$
    En effet, si $\dim(A)=0$, c'est clair. Si $\dim(A)=1$, il existe $f\in \mathfrak{m}$ qui évite les idéaux premiers minimaux de $A$.
    On a alors $\Spec{A}\backslash \{\mathfrak{m}\}=\Spec{A_f}$. Si $\dim(A)\geq 2$, par le lemme de Hartogs (\cite[6.45]{Gort}),
    $$\Ring{\Spec{A}}(\Spec{A})\ra \Ring{\Spec{A}}(\Spec{A}\backslash \{\mathfrak{m}\})$$
    est un isomorphisme, donc $\Spec{A}\backslash \{\mathfrak{m}\}$ ne peut être affine.
    
    \item[(b)] Si $X$ est localement noethérien et normal, et si $U\hookrightarrow X$ est une immersion ouverte affine, alors toutes les composantes
    irréductibles de $X\backslash U$ sont de codimension $\leq 1$ dans $X$.
    
    En effet, si $\xi\in X\backslash U$ est le point générique d'une composante irréductible, on pose 
    $V=\Spec{\Ring{X,\xi}}\backslash\{\mathfrak{m}_{X,\xi}\}$. Par l'hypothèse, c'est un ouvert affine et $\Ring{X,\xi}$ est noethérien normal.
    Donc par (a), $\dim(\Ring{X,\xi})\leq 1$.
    
    \item[(c)] Soit $U\subseteq X$ affine avec $X$ noethérien, régulier et séparé. Alors il existe un fibré en droite $\mathcal{L}$ et $f\in \mathcal{L}(X)$
    tels que $U=\Open{f}$.
    
    Pour démontrer ce point nous pouvons supposer $X$ intègre et $U$ non vide. Comme $X$ est séparé, $U\hookrightarrow X$ est affine.
    Ainsi, $X\backslash U$ est de codimension pure $1$. Donc, par définition, il existe un diviseur de Weil effectif $D$ tel que
    $X\backslash U=\Supp{D}$. Comme $X$ est régulier, il existe un diviseur de Cartier effectif $\tilde{D}$ dont le diviseur
    de Weil associé est $D$. Alors avec $f:=1_{\tilde{D}}\in \Ring{\tilde{D}}(X)$ la section associée à $1\in\mathrm{K}(X)$ et 
    $\mathcal{L}:=\Ring{\tilde{D}}(X)$, on a $U=\Open{f}$.

    \item[(d)] Conclusion~:

    On pose $X=\bigcup_{i\in I} U_i$ avec $I$ fini et chaque $U_i$ affine. Pour chaque $i$, à l'aide de (c), on se donne $\mathcal{L}_i$
    et $f_i\in\mathcal{L}_i(X)$ tels que $U_i=\Open{f_i}$.

    Soit $\mathcal{F}$ un faisceau cohérent sur $X$. Pour chaque $i\in I$, on note $a_{i1},\dotsc,a_{in_i}$ une famille de générateurs de
    $\mathcal{F}(U_i)$. Il existe alors, pour chaque $i$, $m_i\geq 0$ tel que pour tout $j$, $a_{ij}\otimes f_i^{m_i}=b_{ij\mid\Open{f_i}}$
    avec $b_{ij}\in \mathcal{F}\otimes \mathcal{L}^{\otimes m_i}(X)$.

    Les $b_{ij}$ induisent des surjections $\Ring{X}^{n_i}\lra \mathcal{F}\otimes\mathcal{L}^{\otimes m_i}$ sur $U_i$ et donc une surjection~:
    $$\phi:\bigoplus_{i\in I}\Ring{X}^{n_i}\otimes\mathcal{L}^{\otimes -m_i}\lra \mathcal{F}$$
  \end{description}
\end{proof}

\begin{lem}\label{resolutionsVB}
  Soit $X$ un schéma noethérien, régulier et séparé. Alors tout complexe fini de modules cohérents sur $X$ 
  admet une résolution finie par des fibrés vectoriels.  
\end{lem}

\begin{proof}
  Soit $\mathcal{F}_*$ un complexe de modules cohérents sur $X$, concentré en degrés $\{k,\dotsc,l-1\}$. Alors, d'après le 
  \sref{lemme}{RegulierSepareQuotient}, ce complexe admet une résolution $\mathcal{E}_*\ra \mathcal{F}_*$ par des fibrés vectoriels.
  Soit $X=\bigcup_{i\in I} U_i$ un recouvrement fini par des ouverts affines.
  Alors, sur chaque ouvert $U_i$, on a une résolution $\mathcal{E}(U_i)_*\ra \mathcal{F}(U_i)_*$ du complexe de
  $\Ring{X}(U_i)$-modules de type fini $\mathcal{F}(U_i)_*$. Or comme $\Ring{X}(U_i)$ est régulier, $\im{\mathcal{E}(U_i)_{l+1}\ra \mathcal{E}_l(U_i))}$
  est de dimension projective finie. Donc, il existe $n_i\geq 0$ tel que pour tout 
  $m\geq n_i$, $\coker{\mathcal{E}_{m+1}(U_i)\ra \mathcal{E}_m(U_i)}$ soit projectif.
  Si on choisi $n\geq n_i$ pour tout $i$, alors 
  $\coker{\mathcal{E}_{n+1}(U_i)\ra \mathcal{E}_n(U_i)}\ra \mathcal{E}_{n-1}(U_i)\ra\dotsb\ra \mathcal{E}_0(U_i)\ra \mathcal{F}_*$ est une 
  résolution finie de $\mathcal{F}$ par des fibrés vectoriels.
\end{proof}


\begin{lem}\label{ferAChevalVB}
  Soit $\exa{\mathcal{F}'}{\mathcal{F}}{\mathcal{F}''}$ une suite exacte de modules cohérents sur un schéma $X$ noethérien régulier séparé.
  Alors il existe des résolutions finies par des fibrés vectoriels respectant la suite exacte~:
  \begin{center}
    \begin{tikzcd}
      0\rar & \mathcal{P}'_* \rar\arrow[d, "u'"] & \mathcal{P}_* \rar\arrow[d, "u"] & \mathcal{P}''_* \rar\arrow[d, "u''"] & 0\\
      0\rar & \mathcal{F}'   \rar                & \mathcal{F}   \rar               & \mathcal{F}''   \rar                 & 0
    \end{tikzcd}
  \end{center}
\end{lem}

\begin{proof}
  Montrons d'abord le résultat suivante~:
  
  (*)~: Soit $\exa{\mathcal{F}'}{\mathcal{F}}{\mathcal{F}''}$ une suite exacte de modules cohérents. Alors, on peut la compléter en un diagramme~:
  \begin{center}
    \begin{tikzcd}
      0\rar & \mathcal{P}' \rar\arrow[d, "u'"] & \mathcal{P} \rar\arrow[d, "u"] & \mathcal{P}'' \rar\arrow[d, "u''"] & 0\\
      0\rar & \mathcal{F}' \rar                & \mathcal{F} \rar               & \mathcal{F}'' \rar                 & 0
    \end{tikzcd}
  \end{center}
  où les flèches verticales sont surjectives et les modules $\mathcal{P}'$,$\mathcal{P}$ et $\mathcal{P}''$ sont des fibrés vectoriels.
  
  Montrons l'existence d'un tel diagramme. D'après le \sref{lemme}{RegulierSepareQuotient}, il existe des surjections $v:\mathcal{P}''\ra\mathcal{F}$
  et $u':\mathcal{P}'\ra\mathcal{F}'$ avec $\mathcal{P}''$ et $\mathcal{P}'$ fibré vectoriels.
  On pose alors $\mathcal{P}:=\mathcal{P}'\oplus\mathcal{P}''$ et on complète le diagramme de manière évidente avec
  $\exa{\mathcal{P}'}{\mathcal{P}'\oplus\mathcal{P}''}{\mathcal{P}''}$ induit par le scindage, $u:=u'\oplus v$ et 
  $u'':=(\mathcal{F}\ra\mathcal{F}'')\circ v$.

  Maintenant, en appliquant (*) successivement aux la suite exacte $\exa{\myker{u'}}{\myker{u}}{\myker{u''}}$, on obtient des résolutions 
  respectant la suite exacte $\exa{\mathcal{F}'}{\mathcal{F}}{\mathcal{F}''}$. D'après la preuve du \sref{lemme}{resolutionsVB}, nous
  pouvons tronquer ces résolutions à un range $n\geq 0$. Il reste à montrer que la suite~:
  $$\exa{\coker{\mathcal{P}'_{n+1}\ra\mathcal{P}'_{n}}}{\coker{\mathcal{P}_{n+1}\ra\mathcal{P}_{n}}}{\coker{\mathcal{P}''_{n+1}\ra\mathcal{P}''_{n}}}$$
  est exacte. Elle est exacte à droite par le lemme du serpent, et à gauche car les deux premiers termes s'injectent dans $\mathcal{P}_{n-1}$.
\end{proof}

\begin{theo}
  Soit $X$ un schéma noethérien, régulier et séparé. Alors le foncteur $\Proj{X}\ra \Modf{X}$ induit un isomorphisme de groupes
  $K_0(X)\simeq K'_0(X)$.
\end{theo}

\begin{proof}
  la preuve est essentiellement la même que pour le \sref{théorème}{KegalKprimeAnneauxReguliers}. Il faut pour appliquer la preuve montrer deux
  lemmes dans le cadre des modules sur le schéma $X$~: l'existence de résolutions finies pour les complexes finis de modules cohérents, c'est le
  \sref{lemme}{resolutionsVB} ci-dessus~; l'existence de résolutions finies respectant les suites exactes, 
  c'est le \sref{lemme}{ferAChevalVB} ci-dessus.
\end{proof}


\subsection{\texorpdfstring{Categories exactes et leurs $K_0$}{Categories exactes et leurs K0}}

Dans les deux sous-sections précédentes, nous avons défini les groupes de Grothendieck de différentes catégories~:
$\Proj{A}$, $\Modf{A}$, $\Proj{X}$, $\Modf{X}$, ou encore $\mathcal{M}$ de la \sref{proposition}{KprimeTordimfinie}.
Ces catégories ont en commun d'être des sous-catégories d'une catégorie abélienne admettant des extension.

\begin{defi}
  Une catégorie exacte plongée est une sous-catégorie pleine $\M$ d'une catégorie abélienne $\A$ admettant des extensions. C'est-à-dire, pour 
  toute suite exacte $\exa{M'}{M}{M''}$ dans $\A$ telle que $M'$ et $M''$ soient dans $\M$, il existe un objet $\tilde{M}$ de $\M$ isomorphe à $M$.
\end{defi}

Soit $\M\hookrightarrow\A$ une catégorie exacte plongée. On note $\E$ la classe des suites $\exa{M'}{M}{M''}$ d'objets de $\M$ exactes dans $\A$.
On appelle monomorphisme admissible un morphisme $M'\lra M$ dans $\M$ qui apparaît comme le premier morphisme d'un élément $\exa{M'}{M}{M''}$ de $\E$.
On le notera alors $M'\mono M$.
On appelle épimorphisme admissible un morphisme $M\lra M''$ dans $\M$ qui apparaît comme le second morphisme d'un élément $\exa{M'}{M}{M''}$ de $\E$.
On le notera alors $M\epi M''$.
Alors $\M$ est additive et $\M$ et $\E$ vérifient les propriétés suivantes.
\begin{description}
  \item[(a1)] Si $\exa{M'}{M}{M''}$ est une suite dans $\M$ isomorphe à une suite exacte de $\E$, alors $\exa{M'}{M}{M''}$ est dans $\E$.
  \item[(a2)] Pour tous $M'$ et $M''$ dans $\M$, la suite naturelle $\exa{M'}{M'\oplus M''}{M''}$ est dans $\E$.
  \item[(a3)] Pour tout $\exaname{M'}{i}{M}{j}{M''}$ dans $\E$, $i$ est le noyau de $j$ dans $\M$ et $j$ est le conoyau de $i$ dans $\M$.
  \item[(b1)] Les monomorphismes admissibles sont stables par poussé en avant par un morphisme quelconque de $\mathcal{M}$. Les épimorphismes admissibles sont
              stables par tiré en arrière par un morphisme quelconque de $\mathcal{M}$.
  \item[(b2)] Les monomorphismes admissibles sont stables par composition. Les épimorphismes admissibles sont stables par composition. 
\end{description}

\begin{proof}
  Les points (a1)-(a3) sont immédiats. Soit $j:M\ra M''$ un épimorphisme admissible et $i:M'\ra M$ son noyau. Soit $f:N\ra M''$ un morphisme.
  Alors on a le diagramme suivant avec des lignes exactes dans $\A$ et le carré de droite est cartésien.
  \begin{center}
    \begin{tikzcd}
      0\rar & M' \rar\dar & M\times_{M''}N \rar\dar & N   \rar\dar & 0\\
      0\rar & M' \rar     & M              \rar     & M'' \rar     & 0
      \arrow["\mathlarger{\mathlarger{\mathlarger{\mathlarger{\lrcorner}}}}"{anchor=center, pos=0.125}, draw=none, from=1-3, to=2-4]
    \end{tikzcd}
  \end{center}
  Or, $N$ et $M'$ sont des objets de $\M$. Donc il existe $P$ dans $\M$ isomorphe à $M\times_{M''}N$. Ceci démontre (b1).

  Soit $j:M\ra M''$ et $p:M''\ra M'''$ deux épimorphismes admissibles. On note $i:N\ra M''$ le noyau de $p$.
  On a alors à nouveau le diagramme ci-dessus. On vérifie immédiatement que $M\times_{M''}N\ra M$ est le noyau de $p\circ j$.
\end{proof}

\begin{defi}[catégorie exacte]
  Une catégorie exacte est une catégorie additive $\M$ muni d'une classe $\E$ de suites $\exa{M'}{M}{M''}$ dans $\M$ vérifiant les axiomes
  (a1),(a2),(a3),(b1) et (b2).
  Un foncteur $F:\M\ra \M'$ entre catégories exactes est un foncteur additif préservant les suites exactes (ie. $F$ envoie $\E$ dans $\E'$).
\end{defi}

\begin{ex}
  Les catégories suivantes sont exactes~:
  \begin{itemize}
    \item une catégorie abélienne $\A$ munie de ses suites exactes~;
    \item une catégorie additive $\mathcal{N}$ munie des suites exactes scindés~;
    \item si $\M$ est une catégorie exacte, la catégorie $\M\op$ est exacte~;
    \item $\Proj{A}$ ou $\Modf{A}$ pour $A$ un anneau~;
    \item $\Proj{X}$ ou $\Modf{X}$ pour $X$ un schéma.
  \end{itemize}
\end{ex}

\begin{prop}
  Soit $\M$ une catégorie exacte et $\E$ son ensemble de suites exactes. Alors~:
  \begin{description}
    \item[(c1)] Si $f:M\ra M''$ a un noyau dans $\M$ et si $N\xrightarrow{u} M\xrightarrow{f} M''$ est un épimorphisme admissible,
                alors $f$ est un épimorphisme admissible.
    \item[(c2)] Si $f:M'\ra M$ a un conoyau dans $\M$ et si $M'\xrightarrow{f} M\xrightarrow{u} N$ est un monomorphisme admissible,
    alors $f$ est un monomorphisme admissible.
  \end{description}
\end{prop}

Nous ne démontrons pas ici cette proposition. La démonstration est élémentaire et relativement courte. Elle est rédigée dans \cite[A.1]{Kell}.
[Frédéric~: mettre la preuve en annexe~?]

\begin{theo}[plongement]\label{plongementExacte}
  Pour toute catégorie exacte $\M$, il existe une catégorie abélienne $\A$ et un foncteur $i:\M\ra \A$ additif, exact et pleinement fidèle.
\end{theo}

Ce théorème sera admis ici, car sa démonstration a peu de rapport avec le reste du mémoire. Il permet d'utiliser les propriétés des catégories
abéliennes quand on travaille avec des catégories exactes (lemme du serpent par exemple).
[Frédéric~: voir un moyen de se débarrasser du thm ou de le démontrer facilement~?]

Un autre exemple important de catégories exactes est le suivant.

\begin{propdefi}
  Soit $\M$ une catégorie exacte et $\E$ la classe de ses suites exactes. Alors $\E$ a une structure naturelle de catégorie additive où les
  morphismes de $\exac{M'}{M}{M''}$ vers $\exac{N'}{N}{N''}$ sont les diagrammes commutatifs~:
  \begin{center}
    \begin{tikzcd}
      M' \arrow[r,tail]\dar & M \arrow[r,two heads]\dar & M'' \dar \\
      N' \arrow[r,tail]     & N \arrow[r,two heads]     & N''
    \end{tikzcd}
  \end{center}
  On a alors $3$ foncteur $s,t,q:\E\ra \M$ donnés par~:
  \[
    \begin{array}{llcl}
      s:&\exac{M'}{M}{M''}&\mapsto& M' \\
      t:&\exac{M'}{M}{M''}&\mapsto& M  \\
      q:&\exac{M'}{M}{M''}&\mapsto& M''
    \end{array}
  \]
  On note $\mathcal{F}$ l'ensemble des suites $S=\exa{E'}{E}{E''}$ dans $\E$ tels que $s(S)$, $t(S)$
  et $q(S)$ soient exactes.
  
  Alors $\mathcal{F}$ fait de $\E$ une catégorie exacte et les foncteurs $s,t,q:\E\ra \M$ sont exactes.
\end{propdefi}

\begin{proof}
  Les propriétés (a1), (a2) et (a3) sont faciles à vérifier.
  Par le lemme "3x3" des catégories abéliennes et le \sref{théorème de plongement}{plongementExacte},
  $i:E'\ra E$ est un épimorphisme admissible dans $\E$ si et seulement si $s(i)$, $t(i)$ et $q(i)$ sont des épimorphismes 
  admissibles dans $\M$. De même pour les monomorphismes admissibles. Or la composition, le tiré en arrière et le poussé en avant se calculent
  termes à termes. Ceci montre (b1) et (b2). 
\end{proof}

Nous pouvons maintenant définir le $K_0$ d'une catégorie exacte.

\begin{defi}[$K_0$ d'une catégorie exacte]
  Soit $\M$ une catégorie exacte de suites exactes $\E$ telle que les classes d'isomorphisme de $\M$ forment un ensemble.
  On définit le groupe de $K$-théorie de $\M$, $K_0(\M)$, comme le quotient du groupe libre $\bigoplus_M\Z\cdot [M]$ sur les objets de $\M$, par les relations
  $[M]=[M']+[M'']$ pour chaque suite $\exac{M'}{M}{M''}$ dans $\E$.
\end{defi}

\begin{rem}
  La exacte $\exac{0}{0}{0}$ implique $0=[0]$ dans $K_0(\M)$. Pour chaque isomorphisme $\eta:M\ra\tilde{M}$,
  la suite $\exacname{0}{}{M}{\eta}{\tilde{M}}$ est isomorphe à $\exacname{0}{}{0\oplus M}{0\oplus\id}{M}$
  qui est dans $\E$. Ceci montre que $[M]=[\tilde{M}]$ dans $K_0(\M)$. Ainsi, si les classes d'isomorphisme de $\M$ forment un ensemble,
  $K_0(\M)$ est un ensemble.
\end{rem}

\begin{prop}\label{CatExColimites}
  Soit $\M_{(-)}:I\ra \CatEx$, $i\mapsto \M_i$ un foncteur d'une catégorie filtrante $I$ dans la catégorie des petites catégories exactes
  et des foncteurs exactes. Alors la colimite $\M:=\colim_i\M_i$ existe. La catégorie sous-jacente est la colimite dans la catégorie des catégories
  et l'ensemble $\E$ des suites exactes est donnée par la colimite $\colim_i\E_i$, où $\E_i$ est l'ensembles des suites exactes dans $\M_i$.
  En d'autres termes $\exacname{M'}{k}{M}{p}{M''}$ est dans $\E$ si et seulement si seulement si pour un certain
  rang $i$ dans $\Ob{I}$, il existe $M'_i$,$M_i$,$M''_i$,$k_i$ et $p_i$  induisant respectivement $M'$,$M$,$M''$,$k$ et $p$ dans $\M$,
  tels que $\exacname{M'_i}{k_i}{M_i}{p_i}{M''_i}$ soit exacte dans $\M_i$.
\end{prop}
\begin{proof}
  On pose $\M$ la colimite de $\M_{(-)}$ dans la catégorie des petites catégories. On pose $\E$ la colimite $\colim_i\E_i$.
  Montrons que $\M$ est exacte. Il sera alors immédiat que $\M$ est la colimite dans $\CatEx$.
  Comme tout diagramme fini est réalisé à un rang $i$ dans $\Ob{i}$, les axiomes (a1),(a2) et (a3) passent à la colimite.
  L'axiome (b2) passe de même clairement à la colimite. L'axiome (b1) est vérifié par le \sref{lemme}{tireEnArriereExact} suivant. 
\end{proof}

\begin{lem}\label{tireEnArriereExact}
  Soit $F:\M\ra\M'$ un foncteur exact entre catégories exactes. Soit $j:M\twoheadrightarrow M''$ un épimorphisme admissible dans $\M$ et $f:N\ra M''$
  un morphisme. Alors le carré~:
  \begin{center}
    \begin{tikzcd}
      F(M\times_{M''}N) \arrow[two heads, r] \arrow[d] & F(N) \arrow[d, "F(f)"] \\
      F(M)              \arrow[two heads, r, "F(j)"]           & F(M'') 
    \end{tikzcd}
  \end{center}
  est cartésien. En d'autres termes, les foncteurs exactes préservent les tirés en arrière des épimorphismes admissibles.
\end{lem}

\begin{proof}
  On note $i:M'\ra M$ le noyau de $j$.
  On note $\alpha: F(M\times_{M''}N)\ra F(M)\times_{F(M'')}F(N)$ le morphisme canonique. Montrons que $\alpha$ a noyau nul.
  Soit $u:P\ra F(M\times_{M''}N)$ tel que $\alpha\circ u=0$. Alors comme $\exac{F(M')}{F(M\times_{M''}N)}{F(N)}$ est exacte,
  $u$ se factorise en $v: P\ra F(M')$ par $F(M')\rightarrowtail F(M\times_{M''}N)$. Or, $F(i)\circ v=0$, donc $v=0$ car
  $F(i)$ est un noyau.
  Or la composition de $\alpha$ avec $F(M)\times_{F(M'')}F(N)\ra F(N)$ est un épimorphisme admissible. Donc par (c1), 
  $\alpha$ est un épimorphisme admissible de noyau nul. Par (a3) c'est donc un isomorphisme.
\end{proof}

\begin{prop}\label{CatExColimitesK0}
  Soit $\M_{(-)}:I\ra \CatEx$, $i\mapsto \M_i$ un foncteur d'une catégorie filtrante $I$ dans la catégorie des petites catégories exactes.
  On note $\M$ sa colimite. Alors l'application induite $\colim_i K_0(\M_i)\ra K_0(\M)$ est un isomorphisme.
\end{prop}

\begin{proof}
  On a $\bigoplus_{M\in\Ob{\M}} \Z = \colim_i \bigoplus_{M\in\Ob{\M_i}}\Z$ et le sous-groupe $<[M']+[M'']-[M]>_{\exac{M'}{M}{M''}}$
  de $\bigoplus_{M\in\Ob{\M}}\Z$ est colimite des sous-groupes corresponds des $\bigoplus_{M\in\Ob{\M_i}}\Z$.
\end{proof}

Nous allons maintenant appliquer ce résultat de colimite pour effectuer un calcul.

\begin{defi}
  Soit $A=A_0\oplus A_1\oplus\dotsb$ un anneau gradué en degrés positifs. On note $\Pgr{A}$ la catégorie des $A$-modules $\Z$-gradués de type fini
  et projectifs
  (en tant que modules gradués\footnote{ce qui est équivalent à projectif en tant que $A$-module, voir \cite[pp.636-637]{Bass}. Ce ne sera pas utile ici}).
  C'est une catégorie exacte.
  On notera $t:\Pgr{A}\ra \Pgr{A}$, $(P_n)_n\mapsto (P_{n-1})_n$ le foncteur exact de décalage. On notera également par $t$ l'automorphisme induit
  sur $K_0(\Pgr{A})$.
\end{defi}

\begin{prop}
  Soit $A=A_0\oplus A_1\oplus\dotsb$ un anneau gradué en degrés positifs.
  L'automorphisme $t$ fait de $K_0(\Pgr{A})$ un $\Z[t,t^{-1}]$-module. On a un isomorphisme de $\Z[t,t^{-1}]$-modules~:
  \[
  \begin{array}{llll}
    \phi:&\Z[t,t^{-1}]\otimes_{\Z}K_0(A_0) &\ra    & K_0(\Pgr{A}) \\
         &1\otimes x                       &\mapsto& (A\otimes_{A_0}-)_*x
  \end{array}
  \]
\end{prop}

\begin{proof}
  Pour $k$ dans $\Z$, on note $F_k$ le foncteur $\Pgr{A}\ra\Pgr{A}$, $P\mapsto <P_n>_{n\leq q}$, où $<P_n>_{n\leq q}$ est le sous-module de $P$
  engendré par les éléments homogènes de degrés $n\leq q$.
  Pour $q\geq 0$, on note $\Pgr{A}_q$ la sous-catégorie pleine de $\Pgr{A}$ des $P$ tels que $F_{-q-1}P=0$ et $F_qP=P$.
  On définit le foncteur exact $T:\Pgr{A}\ra \Pgr{A_0}$, $P\mapsto A_0\otimes_A P$. Pour tout $P$ et tout $n$, nous allons vérifier que le morphisme
  canonique suivant est un isomorphisme~:
  \[
  \begin{array}{lll}
    A[-n]\otimes_{A_0}T(P)_n &\ra& F_nP/F_{n-1}P \\
    a_m\otimes \overline{b_n}&\mapsto &\overline{a_m\cdot b_n}
  \end{array}
  \]
  pour $a_m$ et $b_n$ homogènes de degrés $m$ et $n$. On dispose d'un épimorphisme naturel de $A_0$-modules gradués $f:P\ra T(P)$.
  Or $T(P)$ est un $A_0$-module gradué projectif.
  Soit $g:T(P)\ra P$ une section. On note $h:T(P)\otimes_{A_0}A\ra P$ le morphisme de $A$-modules gradués de type fini induit.
  Le morphisme $T(h)$ s'identifie à $\id_{T(P)}$. Donc $T(\coker{h})$ est nul. Or $\coker{h}$ est de type fini et donc inférieurement borné.
  Donc $\coker{h}=0$. Donc $h$ est surjectif. Comme $P$ est projectif, on a $T(P)\otimes_{A_0}A\simeq \myker{h}\oplus P$. Mais par cet isomorphisme,
  $T(T(P)\otimes_{A_0}A)$ est isomorphe à $T(P)$. Donc $T(\myker{h})=0$. Or, comme quotient de $P$, $\myker{h}$ est de type fini et donc borné
  inférieurement. Donc $\myker{h}=0$. Ainsi on a un isomorphisme non-canonique $T(P)\otimes_{A_0}A\simeq P$. Via cet isomorphisme,
  $F_nP\simeq \oplus_{q\leq n} A[-q]\otimes_{A_0} T(P)_q$. On a donc bien $A[-n]\otimes_{A_0}T(P)_n\simeq F_nP/F_{n-1}P$, et ce morphisme correspond
  à celui défini plus haut car $h$ est induit par une section de $P\ra T(P)$.

  Nous pouvons maintenant calculer $K_0(\Pgr{A}_q)$. Soit $P$ un objet de $\Pgr{A}_q$. On a une filtration de $P$~:
  $$0=F_{-q-1}P\subseteq F_{-q}\subseteq \dotsb \subseteq F_qP=P$$
  Et donc, au vu de l'isomorphisme ci-dessus~:
  $$[P]=\sum_{n=-q}^q [F_nP/F_{n-1}P]=\sum_{n=-q}^q [A[-n]\otimes_{A_0}T(P)_n]=\sum_{n=-q}^q [\phi(t^n\otimes T(P)_n)]$$
  Ainsi, si on pose $\chi_q: K_0(\Pgr{A}_q)\ra \bigoplus_{n=-q}^q K_0(A_0)$, $P\mapsto \oplus_n T(P)_n$, 
  on a que $\phi_q$ et $\chi_q$ sont réciproques, où $\phi_q$ est la restriction de $\phi$ à $\bigoplus_{n=-q}^q \Z\cdot t^n\otimes K_0(A_0)$.
  Donc $\phi_q$ est un isomorphisme. Par la \sref{proposition}{CatExColimitesK0}, $\phi$ est un isomorphisme.
\end{proof}

\subsection{\texorpdfstring{Les groupes $K_1$ et $K_2$ d'un anneau}{Les groupes K1 et K2 d'un anneau}}

Dans cette section, nous fixons un anneau $A$.

On note $\GL{n}{A}$ le groupe des matrices $n\times n$ inversibles à coéfficients dans $A$.
On fait de $n\mapsto \GL{n}{A}$ un foncteur $(\N,\leq)\ra \Grp$ via les inclusions~:
\[
  \begin{array}{lcl}
    \GL{n}{A} &\ra     & \GL{n+1}{A} \\
    M         &\mapsto & \begin{pmatrix} M & 0 \\ 0 & 1 \end{pmatrix}
  \end{array}
\]
et on note $\GL{}{A}:=\colim_{n\in\N} \GL{n}{A}$.  

\begin{defi}[$K_1$ d'un anneau]
  Pour $A$ un anneau, on défini le $1^{\mathrm{er}}$ groupe de $K$-théorie de $A$, $K_1(A)$, par~:
  $$K_1(A):=\GL{}{A}\ab$$
\end{defi}

Nous allons maintenant donner une description du groupe des commutateurs $[\GL{}{A},\GL{}{A}]$.

\begin{defi}
  Pour $1\leq i,j \leq n$, $i\neq j$ et $a\in A$, on pose~:
  $$e_{ij}(a):= \id+aE_{ij}\in \GL{n}{A}$$
  On note $\EGL{n}{A}$ le sous-groupe de $\GL{n}{A}$ engendré par les $e_{ij}(a)$ pour $i,j >0$, $i\neq j$ et $a\in A$.
  On note $\EGL{}{A}:=\bigcup_n\EGL{n}{A}\subset\GL{}{A}$.
\end{defi}

\begin{ex}
  \begin{enumerate}
    \item Toutes les matrices de permutation paires sont dans $\EGL{}{A}$~;
    \item Si $M\in \GL{n}{A}$, alors~:
         $$\begin{pmatrix} M & 0 \\ 0 & M^{-1} \end{pmatrix}\in \EGL{2n}{A}$$
         En effet, on a la formule~:
         $$\begin{pmatrix} M & 0  \\ 0       & M^{-1} \end{pmatrix}=%
           \begin{pmatrix} 1 & M  \\ 0       & 1      \end{pmatrix}%
           \begin{pmatrix} 1 & 0  \\ -M^{-1} & 0      \end{pmatrix}%
           \begin{pmatrix} 1 & M  \\ 0       & 1      \end{pmatrix}%
           \begin{pmatrix} 0 & -1 \\ 1       & 0      \end{pmatrix}$$
  \end{enumerate}
\end{ex}

\begin{prop}[lemme de Whitehead]
  Pour $A$ un anneau, $\EGL{}{A}$ et parfait, et on a l'égalité~:
  $$\EGL{}{A}=[\GL{}{A},\GL{}{A}]$$
  Et donc~:
  $$K_1(A)=\GL{}{A}/\EGL{}{A}$$
\end{prop}

\begin{proof}
  \begin{description}
    \item[$\subseteq$:] on a $e_{ij}(a)=[e_{ik}(a),e_{kj}(a)]$ pour $i$,$j$ et $k$ distincts. Ceci montre également que $\EGL{}{A}$ et parfait.
    \item[$\supseteq$:] Pour $M,N\in \GL{n}{A}$, on a, dans $\GL{2n}{A}$~:
                      $$[M,N]=\begin{pmatrix} M         & 0  \\ 0 & M^{-1} \end{pmatrix}%
                              \begin{pmatrix} N         & 0  \\ 0 & N^{-1} \end{pmatrix}%
                              \begin{pmatrix} (NM)^{-1} & 0  \\ 0 & NM     \end{pmatrix}\in\EGL{2n}{A}$$ 
  \end{description}
\end{proof}

[Frédéric: ajouter $f^*$ et $f_*$? (nécessite une interprétation cohomologique)]

[ajouter $K_2$ d'un anneau]

\section{\texorpdfstring{Définition de la $K$-théorie supérieure}{Définition de la K-théorie supérieure}}

Cette section est dédiée à la construction $Q$ de la $K$-théorie supérieure, dûe à Daniel Quillen. La référence principale est l'article
original de Quillen \cite{Quil}.

\subsection{\texorpdfstring{La construction $Q$ de Quillen}{La construction Q de Quillen}}

Dans cette sous-section, $\M$ désigne une petite catégorie exacte.

\begin{propdefi}\label{propBicartesien}
  Un diagramme dans $\M$ de la forme~:
  \begin{center}
    \begin{tikzcd}[column sep = large, row sep = large]
      A \arrow[d, two heads, "p'"] \arrow[r, tail, "i"] & B \arrow[d, two heads, "p"] \\
      B \arrow[r, tail, "i'"]                           & C
    \end{tikzcd}
  \end{center}
  est cartésien si et seulement si il est cocartésien. Si c'est le cas, $p$ et $p'$ ont même noyau, et $i$ et $i'$ ont même conoyau.
  
  On appellera un tel diagramme bicartésien.

  De plus, tout diagramme cartésien de la forme $(1)$ ci-dessous ou cocartésien de la forme $(2)$ ci-dessous est bicartésien.
  \begin{center}
    \begin{tikzcd}[column sep = small, row sep = small]
      \cdot \arrow[dd] \arrow[rr] &     & \cdot \arrow[dd, two heads] & \cdot \arrow[dd, two heads] \arrow[rr, tail] &     & \cdot \arrow[dd] \\
                                  & (1) &                             &                                              & (2) &                  \\
      \cdot \arrow[rr, tail]      &     & \cdot                       & \cdot \arrow[rr]                             &     & \cdot
      \arrow["\mathlarger{\mathlarger{\mathlarger{\mathlarger{\lrcorner}}}}"{anchor=center, pos=0.125}, draw=none, from=1-1, to=3-3]
      \arrow["\mathlarger{\mathlarger{\mathlarger{\mathlarger{\lrcorner}}}}"{anchor=center, pos=0.125, rotate=180}, draw=none, from=3-6, to=1-4]
    \end{tikzcd}
  \end{center}
\end{propdefi}

\begin{proof}
  Supposons seulement que le diagramme soit cartésien, que $i'$ soit un monomorphisme admissible et $p$ un épimorphisme admissible (cas $(1)$).
  On note $k:N\rightarrowtail B$ le noyau de $p$. Un morphisme $u:P\ra A$ qui vérifie
  $(P\xrightarrow{u} A\ra C)=0$ s'identifie à un morphisme $P\ra B$ qui vérifie $(P\ra B\ra D)=0$, c'est à dire un morphisme $P\ra N$.
  Donc $p$ et $p'$ ont même noyau. De plus, par l'axiome (b1), $p'$ est un épimorphisme admissible.

  Maintenant, un couple de morphismes $u:C\ra P$, $v:B\ra P$ s'identifie à un morphisme $v:B\ra P$ tel que $A\ra B\ra P$ se factorise
  par $C$. C'est-à-dire, comme $p$ et $p'$ ont même noyau, un morphisme de $w:D\ra P$. Ainsi, le diagramme est cocartésien.
  
  Montrons que $i$ est un monomorphisme admissible. Soit $j':D\twoheadrightarrow Q$ le conoyau de $i'$. Un morphisme $u:P\ra B$
  vérifie $j'\circ p\circ u=0$ si et seulement si $p\circ u$ se factorise par $i':C\ra D$ si et seulement si $u$ se factorise par $i$.
  Donc $i$ est le noyau de $j'\circ p$, qui est un épimorphisme admissible. Donc $i$ est un monomorphisme admissible.
  
  Les autres énoncés sont duaux de ceux démontrés.
\end{proof}

%Or les deux premières colonnes sont exactes, donc par le \sref{théorème de plongement}{plongementExacte} et le lemme "$3\times 3$" des catégories
%abéliennes, on a que $v$ est un isomorphisme. Ainsi, $i$ et $i'$ on même conoyau.

%\begin{center}
%  \begin{tikzcd}
%    N \arrow[d, tail, "k'"]      \arrow[r, equal]     & N \arrow[d, tail, "k"]      \arrow[r]                 & 0 \arrow[d]      \\
%    A \arrow[d, two heads, "p'"] \arrow[r, tail, "i"] & B \arrow[d, two heads, "p"] \arrow[r, two heads,"j"]  & M \arrow[d, "v"] \\
%    B \arrow[r, tail, "i'"]                           & C                           \arrow[r, two heads,"j'"] & M'
%  \end{tikzcd}
%\end{center}

\begin{defi}[catégorie $Q\M$]
  On définit la catégorie $Q\M$ comme la catégorie dont~:
  \begin{enumerate}
    \item les objets sont les objets de $\M$~;
    \item l'ensemble $\Hom{Q\M}{M}{M'}$ est l'ensemble des diagrammes~:
          $$M\twoheadleftarrow N \rightarrowtail M'$$
          à isomorphisme près, où un isomorphisme entre $M\twoheadleftarrow N \rightarrowtail M'$ et $M\twoheadleftarrow N' \rightarrowtail M'$
          est la donnée d'un isomorphisme $\eta:N\ra N'$ dans $\M$ faisant commuter le diagramme~:
          \begin{center}
            \begin{tikzcd}[row sep = small]
                & N \arrow[dd, "\eta"] \arrow[dl, two heads] \arrow[dr, tail] &    \\
              M &                                                             & M' \\
                & N'                   \arrow[ul, two heads] \arrow[ur, tail] &    \\
            \end{tikzcd}
          \end{center}
    \item la composition de $M\twoheadleftarrow N \rightarrowtail M'$ et $M'\twoheadleftarrow N' \rightarrowtail M''$ est donnée par le diagramme~:
    \begin{center}
      \begin{tikzcd}
          &                                          & N\times_{M'}N'\arrow[dl, two heads] \arrow[dr, tail] &                                        &     \\
        M & N \arrow[l, two heads] \arrow[r, tail]   & M'                                                   & N'\arrow[l, two heads] \arrow[r, tail] & M'' \\
      \end{tikzcd}
    \end{center}
  \end{enumerate}
\end{defi}

\begin{rem}
  La composition dans $Q\M$ est bien définie par fonctorialité des limites. L'associativité est facile à vérifier.
\end{rem}

\begin{defi}
  Pour $i:M\rightarrowtail M'$, on note $i_!:M\ra M'$ le morphisme associé dans $Q\M$. Un tel morphisme sera appelé une injection.

  Pour $j:M\twoheadrightarrow M''$, on note $j^!:M''\ra M$ le morphisme associé dans $Q\M$. Un tel morphisme sera appelé une surjection.
\end{defi}

\begin{prop}
  Tout morphisme $u$ de $Q\M$ se factorise en $u=i_!j^!$ uniquement à unique isomorphisme près. De même,
  $u$ se factorise uniquement sous la forme $u=j^!i_!$ à unique isomorphisme près.

  Un morphisme $u$ est un isomorphe si et seulement si c'est une injection et une surjection.

  De plus, on a $\Iso{Q\M}{M}{M'}\simeq \Iso{\M}{M}{M'}$.
\end{prop}

\begin{proof}
  Le premier point est la définition. Le second point découle de la bijection entre factorisations $u=j^!i_!$ et $u=i_!j^!$ induite par les propriétés
  des carrés bicartésiens (\sref{proposition}{propBicartesien}).

  La caractérisation des isomorphismes est immédiate dès que l'on remarque~:
  si $u'=\tilde{i}_!\tilde{j}^!$ est inverse de $u=i_!j^!$, on pose $a_!b^!=\tilde{j}^!i_!$. Alors $\id=u'u= (\tilde{i}a)_!(bj)^!$~;
  donc $\tilde{i}a$ et $bj$ sont des isomorphismes dans $\M$, donc $\tilde{i}$ et $j$ également.
\end{proof}

Une conséquence immédiate de la proposition ci-dessus et de la \sref{proposition}{propBicartesien} est la propriété universelle
vérifiée par $Q\M$ dans $\Cat$.

\begin{prop}[Propriété universelle de $Q\M$]\label{propUnivQM}
  La donnée d'un foncteur $F:Q\M\ra D$ avec $D$ une petite catégorie est équivalente à la donnée~:
  \begin{enumerate}
    \item d'une application $F:\Ob\M\ra \Ob D$~;
    \item de morphismes $Fi_!:FM'\ra FM$ pour chaque $i:M\rightarrowtail M'$~;
    \item de morphismes $Fj^!:FM''\ra FM$ pour chaque $j:M\twoheadrightarrow M''$~;
    \item tels que $F(ii')_!=Fi_!\circ Fi'_!$ et $F(jj')^!=Fj'^!\circ Fj^!$~;
    \item et tels que pour chaque diagramme bicartésien~:
    \begin{center}
      \begin{tikzcd}[column sep = large, row sep = large]
        \cdot \arrow[d, two heads, "j'"] \arrow[r, tail, "i'"] & \cdot \arrow[d, two heads, "j"] \\
        \cdot \arrow[r, tail, "i"]                             & \cdot
      \end{tikzcd}
    \end{center}
    On ai $Fj^!\circ Fi_!=Fi'_!\circ Fj'^!$.
  \end{enumerate}
\end{prop}

Nous allons maintenant donner une autre interprétation des morphismes.

\begin{propdefi}
  On appelle sous-objet de $M\in \Ob \M$ une classe d'isomorphisme de monomorphismes admissibles $M'\rightarrowtail M$, et quotient de $M$
  une classe d'isomorphisme d'épimorphismes admissibles $M\twoheadrightarrow M''$. Les sous-objets sont en bijection avec les quotients via les
  suites exactes.

  On définit une relation d'ordre sur les sous-objets de $M$ par $M_0\leq M_1$ si et seulement si il existe $M_0\rightarrowtail M_1$ monomorphisme
  admissible au-dessus  de $M$.
  
  L'ensemble partiellement ordonné des niveaux de $M$ est définie par~:
  \begin{itemize}
    \item[$\bullet$] Objets~: couples $(M_0,M_1)$ de sous objets avec $M_0\leq M_1$~;
    \item[$\bullet$] Morphismes~: $(M_0,M_1)\leq (M'_0,M'_1)$ si et seulement si $M'_0\leq M_0\leq M_1\leq M'_1$. 
  \end{itemize}

  On a une équivalence entre $Q\M\downarrow M$ et la catégorie des niveaux donnée par~:
  $$(M_0,M_1)\mapsto (M_1/M_0\twoheadleftarrow M_1\rightarrowtail M)$$
\end{propdefi}

La démonstration est immédiate.

\begin{prop}
  On a un isomorphisme naturel de catégories~:
  \[
    \begin{array}{ccc}
      Q\M\op                                        &\simeq & Q\M \\
      M                                             &\mapsto& M   \\
      M\xtwoheadleftarrow{p} N\xrightarrowtail{i} M' &\mapsto& M \xrightarrowtail{p\op} N\xtwoheadleftarrow{i\op}M'
    \end{array}
  \]
\end{prop}

Nous pouvons maintenant donner une définition de la $K$-théorie supérieure. Pour cela, nous allons considérer la catégorie
$Q\M$ comme un ensemble simplicial. Pour des rappels sur ce point et sur les groupes d'homotopie des ensembles simpliciaux, voir le début de
la \sref{sous-section suivante}{thmAetB}.

\begin{defi}[$K$-théorie supérieure]
  Soit $\M$ une catégorie exacte et $O$ un objet nul de $\M$. On définit, pour $n\geq 0$, le $n$-ème groupe de $K$-théorie de $\M$ comme~:
  $$K_n(\M):=\pi_n(Q\M,O)$$
\end{defi}

\begin{rem}
  Comme $Q\M$ est connexe, la classe d'isomorphisme de $K_n(\M)$ est indépendante de $O$. Il faut cependant en dire un tout petit peu plus
  pour faire de $K_n$ un foncteur de $\CatEx$ dans $\Ab$, ce sera fait dans la \sref{sous-section}{premieresprops}.
  Il reste également à vérifier que la nouvelle définition de $K_0$ coïncide avec celle donnée dans la \sref{section}{sectionK0}. C'est l'objet
  du théorème ci-dessous.
\end{rem}

Nous allons maintenant étudier le groupoïde associé à $Q\M$. Soit $O\in \Ob\M$ un objet nul.

Nous notons, pour $M$ objet de $Q\M$, $i_M:O\rightarrowtail M$ et $j_M:M\twoheadrightarrow O$ les uniques morphismes.
La démonstration du théorème utilise la \sref{proposition}{pi1categorie}.
\begin{theo}
  L'application~:
  $$
  \begin{array}{lcl}
    K_0(\M) &\ra     & \pi_1(Q\M,O) \\
    {[{M}]}     &\mapsto & [i_{M!}]^{-1}[j_M^!]
  \end{array}
  $$
  induit un isomorphisme de groupes.
\end{theo}

\begin{proof}
  L'idée de la preuve est de montrer qu'on dispose d'une équivalence de catégorie~:
  $$[K_0(\M),\Isos{\Ens}]\simeq [Q\M,\Isos{\Ens}]$$
  Où $\Isos{\Ens}$ est la catégorie des bijections entre ensembles.
  Pour cela, on remarque dans un premier temps que $[Q\M,\Isos{\Ens}]$ est équivalente à sa sous-catégorie pleine $\mathcal{F}$,
  formée des foncteurs $F$ tels que~:
  $$\forall M,\; F(M)=F(O)\text{ et }F(i_{M!})=\id_{F(O)}$$
  En effet, un foncteur réciproque à l'inclusion est donné par~·
  $$\gamma:F\mapsto (u:M\ra M'\mapsto F(i_{M'!})^{-1}\circ F(u)\circ F(i_{M!}))$$
  Étudions les éléments de $\mathcal{F}$. Si $i:M\rightarrowtail M'$, $ii_{M'}=i_M$, et donc $F(i_!)=id_{F(O)}$. 
  Soit $\exacname{M'}{i}{M}{j}{M''}$ une suite exacte. On a un carré bicartésien~:
  \begin{center}
    \begin{tikzcd}[column sep = large, row sep = large]
      M' \arrow[d, two heads, "j_M"] \arrow[r, tail, "i"] & M \arrow[d, two heads, "j"] \\
      0  \arrow[r, tail, "i_{M''}"]                       & M''
    \end{tikzcd}
  \end{center}
  On a donc $j^!i_{M''!}=i_!j_{M'}^!$, puis $F(j^!)=F(j_{M'}^!)$. De plus $j_M^!=j^!j_{M''}^!$. Donc $F(j_M^!)=F(j_{M'}^!)F(j_{M''}^!)$.
  Nous disposons donc d'un morphisme naturel en $F$~:
  $$
    \begin{array}{lcl}
      K_0(\M) &\ra     & \Aut{F(O)} \\
      {[{M}]} &\mapsto & F(j_M^!)
    \end{array}
  $$
  Ainsi, nous avons un foncteur:
  $$ \alpha : \mathcal{F}\ra [K_0(\M),\Ens]$$
  On remarque également que les éléments $F$ de $\mathcal{F}$ sont entièrement déterminés par $F(O)$ et leurs valeurs sur les $j_M^!$.
  Nous allons maintenant décrire un morphisme réciproque. 
  
  Soit $S$ un $K_0(\M)$-ensemble. On pose~:
  \[
    \begin{array}{cccc}
      F_S: & Q\M                                       &\ra    & \Isos{\Ens} \\
           & M                                         &\mapsto& S   \\
           & i_!                                       &\mapsto& id_S \\
           & j^!                                       &\mapsto& [\myker{j}]\cdot(-)
    \end{array}
  \]
  Vérifions que $F_S$ est un foncteur. Si $\cdot\xtwoheadrightarrow{j}\cdot\xtwoheadrightarrow{j'}\cdot$, la suite
  $\exacname{\myker{j}}{}{\myker{j'j}}{j}{\myker{j'}}$
  est exacte, donc $F((jj')^!)=F(j^!)F(j'^!)$. Si on a un carré bicartésien~:
  \begin{center}
    \begin{tikzcd}[column sep = large, row sep = large]
      \cdot \arrow[d, two heads, "j'"] \arrow[r, tail, "i'"] & \cdot \arrow[d, two heads, "j"] \\
      \cdot \arrow[r, tail, "i"]                             & \cdot
    \end{tikzcd}
  \end{center}
  On a $\myker{j}\simeq\myker{j'}$ et donc $F(i'_!j'^!)=F(j^!i_!)$. Ainsi, par la propriété universelle de $Q\M$ énoncée dans la
  \sref{proposition}{propUnivQM}, $F_S$ est un foncteur.
  On dispose donc d'un foncteur~:
  $$\beta: [K_0(\M),\Ens]\ra \mathcal{F}$$
  Or on voit que $\alpha$ et $\beta$ sont réciproques. En effet, on a un isomorphisme naturel $\alpha(\beta(S))\simeq S$ 
  induit par $\myker{j_M}\simeq M$.
  De même, on a un isomorphisme naturel $\beta(\alpha(F))\simeq F$ induit par les $F(j^!)=F(j^!_{M'})$ et $\myker{j}\simeq M'$
  pour $\exacname{M'}{i}{M}{j}{M''}$ exacte.

  On note $\delta:[Q\M,\Isos{\Ens}] \ra [\pi_1(Q\M,O),\Ens]$ l'équivalence naturelle. L'équivalence~:
  $$\delta\circ(\mathcal{F}\ra [Q\M,\Isos{\Ens}])\circ\beta:[K_0(\M),\Ens]\ra [\pi_1(Q\M,O),\Ens]$$
  Est induite par un isomorphe de groupe $\phi: \pi_1(Q\M,O)\ra K_0(\M)$ (résultat classique sur les actions de groupe).
  On vérifie, en appliquant l'équivalence au $K_0(\M)$-ensemble $K_0(\M)$, que $\phi$ envoie $[i_{M!}]^{-1}[j_M^!]$ sur ${[{M}]}$.
  Ce qui conclut.
\end{proof}

\subsection{Les théorèmes A et B de Quillen}\label{thmAetB}

Comme la définition de la $K$-théorie supérieure repose sur de la topologie, cette section donne quelques rappels sans démonstrations
et prouve $2$ théorèmes (dits A et B) sur lesquels reposeront la majorité des résultats sur la construction $Q$.

On note, pour $C$ une catégorie et $x$ un objet de $C$, $C\downarrow x$ la catégorie au-dessus de $x$ des couples $(y,u:y\ra x)$.
De même, on note $C\uparrow x$ la catégorie au-dessous de $x$ des couples $(y,u:x\ra y)$.

Pour $f:C\ra D$ un foncteur et $d$ un objet de $D$, on note $f\downarrow d$ la catégorie des couples $(c,u:fc\ra d)$. La catégorie
$f\downarrow d$ est le tiré en arrière de $D\downarrow d$ par $f$. De même pour $f\uparrow d$. On note parfois
$C\downarrow d$ pour $f\downarrow d$, quand le foncteur $f$ est évident, par exemple quand $C$ est une sous-catégorie pleine de $D$.

Pour $f:C\ra D$ un foncteur et $d$ un objet de $D$, on note $f^{-1}d$ la sous-catégorie de $C$ formée des objets $c$ tels que $fc=d$ et des
morphismes $u:c\ra c'$ tels que $fu=\id_d$.

\begin{propdefi}\label{definitionnerf}
  Le foncteur nerf $N:\Cat\ra \DEns$ est défini par
  $$NC_\bullet:=\Hom{\Cat}{[\bullet]}{C}$$
  où $[\bullet]$ désigne la catégorie cosimpliciale
  $\DCat\ra \Cat,[n]\mapsto (0\ra 1\ra\dotsb\ra n)$.
  Ce foncteur est pleinement fidèle et admet un adjoint à gauche $\tau_{\leq 1}:\DEns\ra\Cat$.
\end{propdefi}

Pour la démonstration, voir le \cite[chp. 1]{Goer}.

Dans la suite, comme $N$ est pleinement fidèle, nous l'omettrons et notérons $C$ pour l'ensemble simplicial associé à une catégorie $C$.

Nous admettrons les structures de modèles de Kan-Quillen sur $\DEns$ et $\Top$.

Nous noterons $|\bullet|:\DEns\ra \Top$ le foncteur réalisation et $\Sing:\Top\ra \DEns$ le foncteur complexe singulier.
Pour simplifier des démonstrations, nous utiliserons l'équivalence de Quillen~:
\begin{center}
  \begin{tikzcd}
    {|\bullet|:\Ho{\DEns}} \ar[r,bend left,""{name=A, below}] & {\Ho{\Top}:\Sing} \ar[l,bend left,""{name=B,above}] \ar[from=A, to=B, phantom,"\perp"]
  \end{tikzcd}
\end{center}

Pour $X$ ensemble simplicial et $x\in X_0$, nous noterons $\pi_n(X,x)$ pour $\pi_n(|X|,x)$. Nous utiliserons également les groupes d'homotopie
simpliciaux sur les complexes de Kan.

Voir \cite[chp. 1]{Goer} pour plus d'informations sur ces sujets.

Nous allons maintenant nous intéresser aux propriétés homotopiques des catégories.

\begin{prop}
  \begin{description}
    \item[(1)] Une transformation naturelle $\theta:f\Rightarrow g$ entre $f$ et $g:C\ra D$ induit une homotopie~:
              $$C\times \Delta^1\ra D$$ 
    \item[(2)] Si $f:C\ra D$ a un adjoint à droite ou à gauche, alors $f$ est une équivalence d'homotopie~;
    \item[(3)] Une catégorie $C$ avec un objet initial ou final est contractile. 
  \end{description}
\end{prop}

\begin{proof}
  \begin{description}
    \item[(1)] $N:\Cat\ra \DEns$ préserve les produits et la donnée de $\theta$ est équivalente à celle d'un
              foncteur $C\times (0\ra 1) \ra D$~;
    \item[(2)] On note $f':D\ra C$ l'adjoint. On a alors par (1) des homotopies $ff'\simeq \id_D$ et $f'f\simeq \id_C$~;
    \item[(3)] Le foncteur $C\ra (*)$ a alors un adjoint à droite ou à gauche. 
  \end{description}
\end{proof}

\begin{prop}
  \begin{description}
    \item[(1)] Si $C_{(-)}:I\ra \Cat$ est un foncteur avec $I$ petite catégorie filtrante et $C$ sa colimite, l'application naturelle~·
              $$\colim_I NC_{(-)}\lra NC\text{ est un isomorphisme.}$$ 
    \item[(2)] Dans ce même contexte, si on se donne des objets compatibles $(x_i\in C_i)_{i\in I}$ et $x\in C$ associé, les applications naturelles~:
              $$\colim_{i\in I} \pi_n(C_{i},x_i)\lra \pi_n(C,x)\text{ sont des isomorphismes.}$$
    \item[(3)] Une catégorie $I$ filtrante est contractile. 
  \end{description}
\end{prop}

\begin{proof}
  \begin{description}
    \item[(1)] Il suffit de montrer que $\colim_I NC_{(-)}$ est une catégorie, c'est à dire qu'il a la propriété d'unique extension le long
               des $\Lambda_k^n\ra\Delta^n$. Or, si $f:\Lambda_k^n\ra \colim_I NC_{(-)}$ est une application, comme $\Lambda_k^n$ est un objet
               compacte de $\DEns$, $f$ proviens de $f_i:\Lambda_k^n\ra C_i$ pour un certain $i\in \Ob I$. On peut alors l'étendre en
               $g_i:\Delta^n\ra C_i$, lequel induit $g:\Delta^n\ra \colim_I NC_{(-)}$. Si $g$ et $g'$ conviennent, alors ils proviennent
               respectivement de $g_j:\Delta^n\ra C_j$ et $g'_j:\Delta^n\ra C_j$, pour un même $j$ car $I$ est filtrante.
               Quitte à changer $j$, on peut supposer que $g_{j|\Lambda_k^n}=g'_{j'|\Lambda_k^n}$. Or $C_j$ est une catégorie, 
               donc $g_j=g_j'$ et $g=g'$.
    \item[(2)] Le foncteur remplacement fibrant $R:\DEns\ra \DEns$ induit par l'application de l'argument du petit objet aux inclusions
               $\Lambda_k^n\ra\Delta^n$ commute aux colimites filtrantes. En effet, comme on l'a vu ci-dessus~:
               $$\Hom{\DEns}{\Lambda_k^n}{C}\simeq \colim_i\Hom{\DEns}{\Lambda_k^n}{C_i}$$
               Donc, avec $\mathcal{J}=\{\Lambda_k^n\ra\Delta^n\}$, et $G^*$ la construction du petit objet,
               on a $G^1(\mathcal{J},C\ra *)=\colim_i G^1(\mathcal{J},C_i\ra *)$, et donc par induction et passage à la colimite~:
               $RC = \colim_i RC_i$.

               Nous nous sommes donc ramené à montrer que les groupes d'homotopie simpliciaux commutent aux colimites filtrantes de complexes de Kan.
               Ce qui se montre de manière équivalente au (1).
    \item[(3)] On pose $F$ le foncteur $I\ra \Cat,\;i\mapsto I\downarrow i$. On a $I=\colim_I F$. Or chaque $I\downarrow i$ a un objet final et 
               est donc contractile. Donc par (2), $I$ est contractile.
  \end{description}
\end{proof}

Il existe une autre manière de décrire le $\pi_1$ d'une catégorie. On note $\pi:\Cat \ra \mathrm{Groupoides}$ l'adjoint à gauche de l'oubli.

\begin{prop}\label{pi1categorie}
  Soit $C$ une catégorie et $\pi C$ le groupoïde engendré. pour $x$ un objet de $C$, on a un isomorphisme~:
  $$\mathrm{Aut}_{\pi C}(x)\simeq \pi_1(C,x)$$
\end{prop}

\begin{proof}
  On note $X$ l'ensemble simplicial obtenu en ajoutant un $2$-simplexe $\sigma_a$ à $C$ le long de chaque application $(a,\id,-):\Lambda_2^2\ra C$,
  pour $a$ morphisme de $C$. On a alors $\tau_{\leq 1}X=\pi C$. On utilise alors le résultat suivant~:
  
  $(*)$ Si $X$ est un ensemble simplicial tel que $\tau_{\leq 1} X$ est un groupoïde, si et $f:\Lambda_k^2\ra X$ une application,
  on note $g:X\ra Y$ le poussé en avant de $\Lambda_k^2\ra \Delta^2$ le long de $f$. Alors $\tau_{\leq 1}g$ est un isomorphisme.
  
  Par $(*)$, on dispose d'un remplacement fibrant $K$ de $X$, et donc de $C$, tel que $\tau_{\leq 1} K=\pi C$. Or c'est un résultat classique sur
  les complexes de Kan que $\tau_{\leq 1} K= \pi_{\leq 1} K$. Donc, on a~:
  $$\pi_1(C,x)=\pi_1(K,x)=\mathrm{Aut}_{\pi_{\leq 1} K}(x)=\mathrm{Aut}_{\pi C}(x)$$

  Il reste à montrer $(*)$. Pour cela, par exemple si $k=2$, on note $f=(a,b,-)$ et $(a,b,c):\Delta^2\ra Y$ l'extension.
  On a $\tau_{\leq 1}Y=(\tau_{\leq 1}X)[c]/(ac=b)$. Or, dans $\tau_{\leq 1}Y$, $[c] = [a]^{-1}[b]\in \tau_{\leq 1}X$.
  Donc $\tau_{\leq 1}Y=\tau_{\leq 1}X$.
\end{proof}

Pour la suites, nous aurons besoins de deux résultats sur les ensembles bisimpliciaux.

\begin{defi}
  Un ensemble bisimplicial est un foncteur~:
  $$\DCat\op\times\DCat\op\lra \Ens$$
  Pour $X_{\bullet\bullet}$ un ensemble bisimplicial, on appelle sa réalisation $|X|$ le coégalisateur suivant dans $\DEns$~:
  $$\coend{X}{\Delta}$$
  Où $X_n$ désigne l'ensemble simplicial $X_{n,\bullet}$.
  
  Sa diagonale est l'ensemble simplicial $d(X):[n]\mapsto X_{n,n}$
\end{defi}

\begin{prop}
  Soit $X$ un ensemble bisimplicial. Alors les applications~:
  $$\alpha_n:X_n\times \Delta^n\ra d(X),\;(x_m,\phi:[m]\ra [n])\mapsto (\phi\times \id_{[m]})^*(x_m)$$
  induisent un isomorphisme $|X|\ra d(X)$.
\end{prop}

\begin{proof}
  Soit $\phi:[n]\ra [m]$ et $(x,\chi:[r]\ra [n])\in X_{m,r}\times \Delta^n_r$. Alors~:
  $$\alpha_n(\phi^*x,\chi)=((\phi\chi)\times \id)^*x=\alpha_m(x,\phi\chi)$$
  Donc les $\alpha_n$ induisent bien une application $|X|\ra d(X)$. Cette dernière est clairement fonctorielle en $X$.
  Or $X\mapsto |X|$ et $X\mapsto d(X)$ commutent aux colimites, et dans la catégorie $[\DCat\op\times\DCat\op,\Ens]$
  des ensembles bisimpliciaux, tout ensemble $X$ est colimite de $\DCat\times\DCat\downarrow X$, où l'inclusion
  $\DCat\times\DCat\ra [\DCat\op\times\DCat\op,\Ens]$ est donnée par Yoneda.
  Donc, nous sommes ramenées à $X_{\bullet_1\bullet_2}=\Delta^r_{\bullet_1}\times\Delta^s_{\bullet_2}$. Or~:
  $$d(\Delta^r_{\bullet_1}\times\Delta^s_{\bullet_2})=\Delta^r\times\Delta^s$$
  Et~:
  \begin{align*}
    |\Delta^r_{\bullet_1}\times\Delta^s_{\bullet_2}|_l&=\mathrm{coeq}\Big(\bigsqcup_{\phi:[n]\ra [m]}\Delta^r_m\times\Delta^s_l\times\Delta^n_l
    \xlongrightrightarrows{\sqcup\phi^*\times\id}{\sqcup\id\times\phi_*}
    \bigsqcup_{[n]}\Delta^r_n\times\Delta^s_l\times\Delta^n_l\Big)\\
    &=\mathrm{coeq}\Big(\bigsqcup_{\phi:[n]\ra [m]}\Delta^r_m\times\Delta^n_l
     \xlongrightrightarrows{\sqcup\phi^*\times\id}{\sqcup\id\times\phi_*}
     \bigsqcup_{[n]}\Delta^r_n\times\Delta^n_l\Big)\times \Delta^s_l\\
     &=\mathrm{coeq}\Big(\bigsqcup_{\phi:[n]\ra [m]}\Delta_m^r\times\Delta^n
     \xlongrightrightarrows{\sqcup\phi^*\times\id}{\sqcup\id\times\phi_*}
     \bigsqcup_{[n]}\Delta_n^r\times\Delta^n\Big)_l\times \Delta^s_l\\
     &=\Delta^r_l\times \Delta^s_l
  \end{align*}
  Où la dernière égalité est une égalité classique sur les extensions de Yoneda. Plus précisément, on utilise que si $C$ est une petite
  catégorie et $X$ un objet de $[C\op,\Ens]$, $Y:C\ra [C\op,\Ens]$ le plongement de Yoneda, et $\Pi:C\downarrow X\ra C$
  l'oubli, alors~:
  $$X=\colim_C Y\circ\Pi=\mathrm{coeq}\Big(\bigsqcup_{\phi:c\ra d} X(d)\times Y(c)
  \xlongrightrightarrows{\sqcup X(\phi)\times\id}{\sqcup\id\times Y(\phi)}
  \bigsqcup_{c} X(c)\times Y(c) \Big)$$
  On vérifie facilement que $|\Delta^r_{\bullet_1}\times\Delta^s_{\bullet_2}|\ra d(\Delta^r_{\bullet_1}\times\Delta^s_{\bullet_2})$
  est l'identité via ces deux identifications.
\end{proof}

\begin{lem}\label{diagonalequiv}
  Soit $\phi:X_{\bullet\bullet}\ra Y_{\bullet\bullet}$ un morphisme entre ensembles bisimpliciaux. Si pour tout $n\geq 0$,
  $\phi_n:X_{n\bullet}\ra Y_{n\bullet}$ est une équivalence d'homotopie faible, alors $d(\phi):d(X)\ra d(Y)$ est une équivalence d'homotopie faible.
\end{lem}

\begin{proof}\cite[IV.1.7]{Goer}
  Introduisons d'abord quelques notations.
  Soit $Z$ un ensemble bisimplicial. Pour $p\geq 0$ on note $d(Z)^{(p)}$ l'image de $\bigsqcup_ {n\leq p} Z_n\times\Delta^n$ dans $d(Z)$.
  Pour $r\geq 0$ et $p\geq 0$, on note $s_{[r]}Z_p=\bigcup_{i\leq r}s_i(Z_p)\subset Z_{p+1}$. On a alors les $3$ diagrammes cocartésiens suivants~:
  \begin{center}
    \begin{tikzcd}[column sep=0.2ex,row sep=scriptsize]
      s_{[r]}Z_{p-1} \arrow[dd, hook] \arrow[rr, "s_{r+1}"] &   &s_{[r]}Z_p \arrow[dd, hook] \\
                                                            &(*)& \\
      Z_p \arrow[rr, "s_{r+1}"]                             &   &s_{[r+1]}Z_p
      \arrow["\mathlarger{\mathlarger{\mathlarger{\mathlarger{\lrcorner}}}}"{anchor=center, pos=0.125, rotate=180}, draw=none, from=3-3, to=1-1]\\

      (s_{[p]}Z_p\times\Delta^{p+1})\cup (Z_{p+1}\times\partial\Delta^{p+1}) \arrow[dd, hook] \arrow[rr] & &d(Z)^{(p)} \arrow[dd, hook] \\
      &(**)&\\
      Z_{p+1}\times \Delta^{p+1} \arrow[rr]                                                           & &d(Z)^{(p+1)}
      \arrow["\mathlarger{\mathlarger{\mathlarger{\mathlarger{\lrcorner}}}}"{anchor=center, pos=0.125, rotate=180}, draw=none, from=6-3, to=4-1]\\

      s_{[p]}Z_p\times\partial \Delta^{p+1} \arrow[dd, hook] \arrow[rr] & &Z_{p+1}\times\partial\Delta^{p+1} \arrow[dd, hook] \\
      &(***)&\\
      s_{[p]}Z_p\times\Delta^{p+1} \arrow[rr]                                                           & &(s_{[p]}Z_p\times\Delta^{p+1})\cup (Z_{p+1}\times\partial\Delta^{p+1})
      \arrow["\mathlarger{\mathlarger{\mathlarger{\mathlarger{\lrcorner}}}}"{anchor=center, pos=0.125, rotate=180}, draw=none, from=9-3, to=7-1]\\
    \end{tikzcd}
  \end{center}
  Montrons par récurrence sur $p$ que $d(X)^{(p)}\ra d(Y)^{(p)}$ est une équivalence d'homotopie faible. Cela conclura, car les $d(X)^{(p)}$
  filtrent $d(X)$.

  Pour initialiser, on remarque que $d(X)^{(p)}=d(\sk{p}{X})$, et en particulier, $d(X)^{(0)}=X_0$.

  Soit $p\geq 0$, et supposons que $d(X^{(i)})\ra d(Y^{(i)})$ soit une équivalence d'homotopie faible pour tout $i\leq p$.
  \begin{description}
    \item[(1)] En appliquant successivement $(*)$ et le \sref{lemme de collage}{lemmedecollage}, on a que $s_{[r]}X_p\ra s_{[r]}Y_p$
              est une équivalence d'homotopie faible pour tout $r\leq p$~;
    \item[(2)] En utilisant $(***)$ et le \sref{lemme de collage}{lemmedecollage}, on a que~:
              $$(s_{[p]}X_p\times\Delta^{p+1})\cup (X_{p+1}\times\partial\Delta^{p+1})\ra (s_{[p]}Y_p\times\Delta^{p+1})\cup (Y_{p+1}\times\partial\Delta^{p+1})$$
              est une équivalence faible~;
    \item[(3)] Par $(**)$ et le \sref{lemme de collage}{lemmedecollage},
    $$d(X^{(p+1)})\ra d(Y^{(p+1)})$$
    est une équivalence faible.
  \end{description}
  Ceci conclut la récurrence.
\end{proof}

Nous faisons ici quelques rappels sur les foncteurs (pré)(co)fibrés.

\begin{defi}[foncteur (pré)fibré]
  Un foncteur $f:C\ra D$ est préfibré si pour tout $y$ objet de $D$, le foncteur pleinement fidèle~:
  $$f^{-1}(y)\ra f\uparrow y,\; x\mapsto (x,\id_y)$$
  admet un adjoint à droite, alors noté $(x,v:y\ra fx)\mapsto v^*x$.

  Alors, si $v:y\ra y'$, on peut restreindre cet adjoint à $f^{-1}(y')$, et on obtient un foncteur~:
  $$v^*:f^{-1}(y')\lra f^{-1}(y)$$
  que l'on appelle changement de base de $y'$ à $y$.

  Un foncteur $f:C\ra D$ est fibré s'il est préfibré, et si pour tout $u$ et $v$ composables dans $D$,
  l'application naturelle $u^*v^*\ra (vu)^*$ est un isomorphisme.
\end{defi}

On ajoute ci-dessous la définition complètement duale de foncteur (pré)cofibré.

\begin{defi}[foncteur (pré)cofibré]
  Un foncteur $f:C\ra D$ est précofibré si pour tout $y$ objet de $D$, le foncteur pleinement fidèle~:
  $$f^{-1}(y)\ra f\downarrow y,\; x\mapsto (x,\id_y)$$
  admet un adjoint à gauche, alors noté $(x,v:fx\ra y)\mapsto v_*x$.

  Alors, si $v:y'\ra y$, on peut restreindre cet adjoint à $f^{-1}(y')$, et on obtient un foncteur~:
  $$v_*:f^{-1}(y')\lra f^{-1}(y)$$
  que l'on appelle changement de cobase de $y'$ à $y$.

  Un foncteur $f:C\ra D$ est cofibré s'il est précofibré, et si pour tout $u$ et $v$ composables dans $D$,
  l'application naturelle $(vu)_*\ra v_*u_*$ est un isomorphisme.
\end{defi}

\begin{rem}
  Pour $f:C\ra D$ (pré)(co)cofibré, on dit souvent que $C$ est une catégorie (pré)(co)cofibrée au-dessus de $D$.
\end{rem}

\begin{rem}
  Les définitions ci-dessus sont celles de \cite[§1]{Quil}. Ces définitions ont l'avantage de faire directement apparaître le résultat suivant
  (et son dual évident)~:

  $(*)$ Si $f:C\ra D$ est préfibré, alors pour tout $y$ objet de $D$, $f^{-1}y\ra f\uparrow y$ est une équivalence d'homotopie.

  Pour comprendre l'existence des applications naturelles
  $u^*v^*\ra (vu)^*$, il peut être commode de voir l'adjonction à l'aide des morphismes (pré)cartésiens. Soit $f:C\ra D$.
  
  Un morphisme $\xi:\tilde{z}\ra z$ dans $C$ est $f$-précartésien si pour tout $w:x\ra z$ avec $f(w)=f(\xi)$, il existe
  un unique $u:x\ra \tilde{z}$ tel que $w=\xi u$ et $f(u)=\id_{f\tilde{z}}$.

  \begin{center}
    \begin{tikzcd}[row sep=large,column sep=large]
      x & {\tilde{z}} & z \\
      {f\tilde{z}} & {f\tilde{z}} & fz
      \arrow[maps to, from=1-1, to=2-1]
      \arrow[maps to, from=1-2, to=2-2]
      \arrow[maps to, from=1-3, to=2-3]
      \arrow["w", bend left=20, from=1-1, to=1-3]
      \arrow["\xi"', from=1-2, to=1-3]
      \arrow["\id"', from=2-1, to=2-2]
      \arrow["f\xi"', from=2-2, to=2-3]
      \arrow["{\exists ! u}"', dashed, from=1-1, to=1-2]
    \end{tikzcd}
  \end{center}
  
  Un morphisme $\xi:\tilde{z}\ra z$ dans $C$ est $f$-cartésien si pour tout $w:x\ra z$ et $a:fx\ra f\tilde{z}$ avec $f(w)=f(\xi)a$, il existe
  un unique $u:x\ra \tilde{z}$ tel que $w=\xi u$ et $f(u)=a$.

  \begin{center}
    \begin{tikzcd}[row sep=large,column sep=large]
      x & {\tilde{z}} & z \\
      {fx} & {f\tilde{z}} & fz
      \arrow[maps to, from=1-1, to=2-1]
      \arrow[maps to, from=1-2, to=2-2]
      \arrow[maps to, from=1-3, to=2-3]
      \arrow["w", bend left=20, from=1-1, to=1-3]
      \arrow["\xi"', from=1-2, to=1-3]
      \arrow["a"', from=2-1, to=2-2]
      \arrow["f\xi"', from=2-2, to=2-3]
      \arrow["{\exists ! u}"', dashed, from=1-1, to=1-2]
    \end{tikzcd}
  \end{center}

  Soit maintenant $y$ un objet de $D$.
  Alors $f^{-1}(y)\ra f\uparrow y$ admet un adjoint à droite si et seulement si pour tout $z$ objet de $C$ et $v:y\ra fz$,
  il existe $\xi:v^* z\ra z$ $f$-précartésien avec $f(\xi)=v$.

  \begin{center}
    \begin{tikzcd}[row sep=large,column sep=large]
      {\exists v^*z} & z \\
      y & fz
      \arrow["v", from=2-1, to=2-2]
      \arrow[maps to, from=1-2, to=2-2]
      \arrow[maps to, from=1-1, to=2-1]
      \arrow["\xi", from=1-1, to=1-2]
    \end{tikzcd}
  \end{center}

  Il est maintenant aisé de vérifier que $f$ est fibré si et seulement si pour tout $y$ objet de $D$, $z$ objet de $C$ et $v:y\ra fz$,
  il existe $\xi:v^* z\ra z$ $f$-cartésien avec $f(\xi)=v$.
\end{rem}

\begin{propdefi}[catégorie (co)fibrée associée à un préfaisceau]
  Soit $p:C\op\ra\Ens$ un foncteur avec $C$ une petite catégorie (souvent appelé préfaisceau sur $C$). On définit la catégorie fibrée
  $$f:F(p)\ra C$$
  dont les objets sont les couples $(c,x)$ avec $x\in p(C)$, et les morphismes de $(c,x)$ dans $(d,y)$ sont les morphismes
  $u:c\ra d$ dans $C$ tels que $p(u)(y)=x$.

  On appelle $f:F(p)\ra C$ la catégorie fibrée associée à $p$.

  De même, à partir d'un foncteur $q:C\ra \Ens$, on construit la catégorie cofibrée
  $$g:G(q)\ra C$$
  dont les objets sont les couples $(c,x)$ avec $x\in q(C)$, et les morphismes de $(c,x)$ dans $(d,y)$ sont les morphismes
  $u:c\ra d$ dans $C$ tels que $p(u)(x)=y$.

  On appelle $g:G(q)\ra C$ la catégorie cofibrée associée à $q$.
\end{propdefi}

Nous pouvons maintenant énoncé un premier résultat sur le type d'homotopie de catégories.

\begin{theo}[Théorème A de Quillen]\label{theoremeA}
  Soit $f:C\ra D$ un foncteur entre petites catégories. Si pour tout objet $y$ de $D$, $f\uparrow y$ est une catégorie contractile,
  alors $f$ est une équivalence d'homotopie.
\end{theo}

\begin{proof}
  On note $S(f)$ la catégorie cofibrée au dessus de $D\op\times C$ définie par le foncteur~:
  \begin{align*}
    D\op\times C &\ra     \Ens \\
    (y,x)        &\mapsto \Hom{D}{y}{fx}
  \end{align*}
  On dispose de foncteurs~:
  $$D\op\overset{p_2}{\longleftarrow} S(f) \overset{p_1}{\lra} C$$
  L'ensemble simplicial $S(f)$ est la diagonale de l'ensemble bisimplicial $T(f)$ défini par~:
  $$T(f)_{pq}:=\{(y_p\ra\dotsb\ra y_0\ra fx_0,\;x_0\ra\dotsb\ra x_q)\}$$
  
  On dispose d'un morphisme d'ensembles bisimpliciaux $m:T(f)_{\bullet_1\bullet_2}\ra C_{\bullet_2}$, où $(C_{\bullet_2})_{pq}=NC_q$.
  Alors, $d(m)$ s'identifie à $p_1:S(f)\ra C$. Or, on a, pour $q\geq 0$~:
  $$m_{\bullet q}:\bigsqcup_{x_0\ra\dotsb\ra x_q} (D\downarrow f(x_0))\op \ra \bigsqcup_{x_0\ra\dotsb\ra x_q} *$$
  C'est une équivalence d'homotopie, car chaque $(D\downarrow f(x_0))\op$ a un objet initial et est donc contractile.
  Ainsi, par le \sref{lemme}{diagonalequiv}, $d(m)=p_1:S(f)\ra C$ est une équivalence d'homotopie.

  De même, on dispose d'un morphisme d'ensembles bisimpliciaux $n:T(f)_{\bullet_1\bullet_2}\ra D\op_{\bullet_1}$. Et pour tout $p\geq 0$~:
  $$n_{p\bullet}:\bigsqcup_{y_p\ra\dotsb\ra y_0} f\uparrow y_0 \ra \bigsqcup_{y_p\ra\dotsb\ra y_0} *$$
  Or, par hypothèse, chaque $f\uparrow y_0$ est contractile. Donc $n_{p\bullet}$ est une équivalence faible.
  Donc, par le \sref{lemme}{diagonalequiv}, $d(n)=p_2:S(f)\ra D\op$ est une équivalence d'homotopie.

  Or, la factorisation~:
  $$D\op\times C \overset{\id\times f}{\lra} D\op\times D\overset{\Hom{D}{-}{-}}{\lra} \Ens$$
  induit le diagramme commutatif suivant dans $\Cat$~:
  \begin{center}
    \begin{tikzcd}[row sep=large,column sep=large]
      D\op & {S(f)} & C \\
      D\op & {S(\id_D)} & D
      \arrow[equal, from=1-1, to=2-1]
      \arrow["f", from=1-3, to=2-3]
      \arrow["{p_1}","\sim"', from=1-2, to=1-3]
      \arrow["{p_2}"',"\sim", from=1-2, to=1-1]
      \arrow["\sim"',from=2-2, to=2-1]
      \arrow["\sim",from=2-2, to=2-3]
      \arrow["{f_*}", from=1-2, to=2-2]
    \end{tikzcd}
  \end{center}
  Par les résultats ci-dessus, les flèches avec des $\sim$ sont des équivalences d'homotopie faibles.
  Donc $f$ également.
\end{proof}

\begin{rem}
  Le théorème A admet la forme duale suivante~:

  Si $f:C\ra D$ un foncteur entre petites catégories et si pour tout objet $y$ de $D$, $f\downarrow y$ est une catégorie contractile,
  alors $f$ est une équivalence d'homotopie.
\end{rem}

\begin{coro}
  Si $f:C\ra D$ un foncteur pré(co)fibré entre petites catégories et si pour tout objet $y$ de $D$, $f^{-1}(y)$ est une catégorie contractile,
  alors $f$ est une équivalence d'homotopie.
\end{coro}
\begin{proof}
  Dans ce cas $f^{-1}(y)\ra f\uparrow y$ ou $f^{-1}(y)\ra f\downarrow y$ est une équivalence faible d'homotopie. On peut donc utiliser
  le \sref{théorème}{theoremeA}.
\end{proof}

Pour énoncé le théorème B de Quillen, nous aurons besoin du formalisme des carrés homotopiquement cartésiens dans $\DEns$.
Nous avons fait le choix de le mettre dans \sref{l'annexe}{carrehomotopiquementcart}.

\begin{defi}
  Soit $Z:I\ra \DEns$ un foncteur avec $I$ petite catégorie. On définit l'ensemble bisimplicial~:
  $$\mathrm{BE}_IZ_{m,n}:=\bigsqcup_{i_0\ra\dotsb\ra i_m}Z(i_0)_n$$
  Et on note sa diagonale~:
  $$\hocolim{I}{Z}:=d(\mathrm{BE}_IZ)$$
\end{defi}

\begin{lem}\label{lemmetheoremeB}
  Soit $X:I\ra \DEns$ un foncteur, avec $I$ une petite catégorie, tel que pour tout $\alpha:i\ra j$ dans $I$,
  $X(\alpha): X(i)\ra X(j)$ soit une équivalence d'homotopie faible. Alors pour tout $j$ objet de $I$, le diagramme cartésien~:
  \begin{center}
    \begin{tikzcd}[column sep=0.2ex,row sep=small]
      {X(j)} && {\hocolim{I}{X}} \\
      & {(D)} \\
      {*} && I
      \arrow["j", from=3-1, to=3-3]
      \arrow[from=1-1, to=3-1]
      \arrow["\pi", from=1-3, to=3-3]
      \arrow[from=1-1, to=1-3]
      \arrow["\mathlarger{\mathlarger{\mathlarger{\mathlarger{\lrcorner}}}}"{anchor=center, pos=0.05}, draw=none, from=1-1, to=3-3]
    \end{tikzcd}
  \end{center}
  est homotopiquement cartésien.
\end{lem}

\begin{rem}
  Le diagramme $(D)$ est la diagonale du diagramme cartésien suivant d'ensembles bisimpliciaux~:
  \begin{center}
    \begin{tikzcd}[column sep=large,row sep=large]
      {X(j)_n} & {\Big(\displaystyle\bigsqcup_{i_0\ra\dotsb\ra i_n}X(i_0)\Big)_n} \\
      {*} & {\Big(\displaystyle\bigsqcup_{i_0\ra\dotsb\ra i_n}*\Big)_n}
      \arrow["j\ra\dotsb\ra j", from=2-1, to=2-2]
      \arrow[from=1-1, to=2-1]
      \arrow["\pi", from=1-2, to=2-2]
      \arrow["j\ra\dotsb\ra j",from=1-1, to=1-2]
      \arrow["\mathlarger{\mathlarger{\mathlarger{\mathlarger{\lrcorner}}}}"{anchor=center, pos=0.05}, draw=none, from=1-1, to=2-2]
    \end{tikzcd}
  \end{center}
\end{rem}

\begin{proof}
  \cite[IV.5.7]{Goer}. On remarque d'abord le fait suivant~:

  $(*)$ le tiré en arrière par $\pi$ commute aux colimites dans $\DEns\downarrow I$.
  
  En effet, il suffit, pour montrer $(*)$, de vérifier que $-\times_x y$ commute aux sommes disjointes et coégalisateurs dans $\Ens\downarrow x$,
  ce qui est facile.

  On factorise $j$ à l'aide de l'argument du petit objet appliqué à $\mathcal{J}=\{\Lambda_k^n\ra\Delta^n\}$ en
  \begin{tikzcd}
    {*} & U & I
    \arrow["i", "\sim"', hook, from=1-1, to=1-2]
    \arrow["p", two heads, from=1-2, to=1-3]
  \end{tikzcd}.
  Notre objectif est de montrer que l'application induite $X(j)\ra U\times_I\hocolim{I}{X}$ est acyclique.
  Par la \sref{proposition}{criterehomocartesienpropreadroite}, ceci conclura.

  Or $i$ est construit comme un élément de $\mathcal{J}-\mathrm{Cell}$ colimite de poussés en avant de la forme~:
  \begin{center}
    \begin{tikzcd}[column sep=large,row sep=large]
      {\bigsqcup\Lambda_k^n} & {G^m(\mathcal{J},j)} \\
      {\bigsqcup\Delta^n} & {G^{m+1}(\mathcal{J},j)}
      \arrow["\sim" sloped, hook, from=1-1, to=2-1]
      \arrow["\sim" sloped, hook, from=1-2, to=2-2]
      \arrow[from=1-1, to=1-2]
      \arrow[from=2-1, to=2-2]
      \arrow["\mathlarger{\mathlarger{\mathlarger{\mathlarger{\lrcorner}}}}"{anchor=center, pos=0.05, rotate=180}, draw=none, from=2-2, to=1-1]
    \end{tikzcd}  
  \end{center}
  Et donc, par $(*)$, en tirant en arrière par $\pi$~:
  \begin{center}
    \begin{tikzcd}[column sep=large,row sep=large]
      {\bigsqcup\Lambda_k^n\times_I\hocolim{I}{X}} & {G^m(\mathcal{J},j)\times_I\hocolim{I}{X}} \\
      {\bigsqcup\Delta^n\times_I\hocolim{I}{X}} & {G^{m+1}(\mathcal{J},j)\times_I\hocolim{I}{X}}
      \arrow["u", hook, from=1-1, to=2-1]
      \arrow["v", hook, from=1-2, to=2-2]
      \arrow[from=1-1, to=1-2]
      \arrow[from=2-1, to=2-2]
      \arrow["\mathlarger{\mathlarger{\mathlarger{\mathlarger{\lrcorner}}}}"{anchor=center, pos=0.05, rotate=180}, draw=none, from=2-2, to=1-1]
    \end{tikzcd}  
  \end{center}
  Or, le produit fibré dans $\Ens$, et donc également dans $\DEns$, préserve les injections. Donc $u$ est une cofibration.
  Il nous reste à montrer que $u$ est acyclique. Car alors $v$ sera également une cofibration acyclique, et par passage à la colimite,
  $i_*:X(j)\ra U\times_I\hocolim{I}{X}$ également.

  Nous sommes donc ramené au problème suivant~:

  $(**)$ si 
  \begin{tikzcd}
    {\Lambda^k_n} & {\Delta^n} & I
    \arrow["w", "\sim" ',hook, from=1-1, to=1-2]
    \arrow["\sigma", from=1-2, to=1-3]
  \end{tikzcd}
  alors $w_*:\Lambda^k_n\times_I\hocolim{I}{X}\ra \Delta^n\times_I\hocolim{I}{X}$ est une cofibration acyclique.

  Or $\sigma: \Delta^n\ra I$ correspond à un foncteur $\sigma: [n]\ra I$. De plus, le carré cartésien suivant d'ensembles bisimpliciaux
  \begin{center}
    \begin{tikzcd}[column sep=large,row sep=large]
      {\Big(\displaystyle\bigsqcup_{u_0\ra\dotsb\ra u_n}X(\sigma(u_0))\Big)_n} & {\Big(\displaystyle\bigsqcup_{i_0\ra\dotsb\ra i_n}X(i_0)\Big)_n} \\
      {\Big(\displaystyle\bigsqcup_{u_0\ra\dotsb\ra u_n}*\Big)_n} & {\Big(\displaystyle\bigsqcup_{i_0\ra\dotsb\ra i_n}*\Big)_n}
      \arrow["\sigma", from=2-1, to=2-2]
      \arrow[from=1-1, to=2-1]
      \arrow["\pi", from=1-2, to=2-2]
      \arrow["\sigma",from=1-1, to=1-2]
      \arrow["\mathlarger{\mathlarger{\mathlarger{\mathlarger{\lrcorner}}}}"{anchor=center, pos=0.05}, draw=none, from=1-1, to=2-2]
    \end{tikzcd}
  \end{center}
  induit un isomorphisme $\Delta^n\times_I\hocolim{I}{X}\simeq \hocolim{[n]}{X\circ \sigma}$. On a alors le diagramme suivant d'ensembles bisimpliciaux
  \begin{center}
    \begin{tikzcd}[column sep=large,row sep=large]
      {\displaystyle\bigsqcup_{u_0\ra\dotsb\ra u_r\in\Lambda_k^n}X(\sigma(0))} & {\displaystyle\bigsqcup_{u_0\ra\dotsb\ra u_r\in\Delta^n}X(\sigma(0))} \\
      {\displaystyle\bigsqcup_{u_0\ra\dotsb\ra u_r\in\Lambda_k^n}X(\sigma(u_0))} & {\displaystyle\bigsqcup_{u_0\ra\dotsb\ra u_r\in\Delta^n}X(\sigma(u_0))} \\
      {\displaystyle\bigsqcup_{u_0\ra\dotsb\ra u_r\in\Lambda_k^n}*} & {\displaystyle\bigsqcup_{u_0\ra\dotsb\ra u_r\in\Delta^n}*}
      \arrow[from=2-1, to=3-1]
      \arrow[from=2-2, to=3-2]
      \arrow["w", from=3-1, to=3-2]
      \arrow["{w_*}", from=2-1, to=2-2]
      \arrow["{\theta_*}", from=1-1, to=2-1]
      \arrow["{\theta_*}", from=1-2, to=2-2]
      \arrow["j", from=1-1, to=1-2]
      \arrow["\mathlarger{\mathlarger{\mathlarger{\mathlarger{\lrcorner}}}}"{anchor=center, pos=0.05}, draw=none, from=2-1, to=3-2]
    \end{tikzcd}
  \end{center}
  Où $\theta:X(\sigma(0))\ra X\circ\sigma$ est transformation naturelle évidente. Alors $\theta_*$ est une équivalence faible terme à terme par hypothèse.
  D'après le \sref{lemme}{diagonalequiv}, $d(\theta_*)$ est une équivalence faible. Or $d(j)$ s'identifie à
  $\Lambda_k^n\times X(\sigma(0))\ra \Delta^n\times X(\sigma(0))$, qui est une équivalence faible.
  Donc $d(w_*)$ est une équivalence faible, ce qui conclut.
\end{proof}

\begin{theo}[Théorème B de Quillen]\label{theoremeB}
  Soit $f:C\ra D$ un foncteur entre petites catégories. Si pour tout $u:y\ra y'$ dans $D$, $u^*:f\uparrow y'\ra f\uparrow y$
  est une équivalence faible d'homotopie, alors pour tout objet $y$ de $D$, le diagramme cartésien~:
  \begin{center}
    \begin{tikzcd}[column sep=large,row sep=large]
      {f\uparrow y} & {C} \\
      {D\uparrow y} & D
      \arrow["j'", from=2-1, to=2-2]
      \arrow["f_*", from=1-1, to=2-1]
      \arrow["f", from=1-2, to=2-2]
      \arrow["j", from=1-1, to=1-2]
      \arrow["\mathlarger{\mathlarger{\mathlarger{\mathlarger{\lrcorner}}}}"{anchor=center, pos=0.05}, draw=none, from=1-1, to=2-2]
    \end{tikzcd}
  \end{center}
  est homotopiquement cartésien.

  En particulier, on dispose, pour tout $x$ objet de $f^{-1}(y)$, d'une suite exacte~:
  $$\dotsb\ra \pi_{i+1}(D,y)\ra \pi_{i}(f\uparrow y,\bar{x})\overset{j_*}{\ra} \pi_{i}(C,x)\overset{f_*}{\ra} \pi_{i}(D,y)\ra \dotsb$$
  où $\bar{x}=(x,\id_y)$.
\end{theo}

\begin{rem}
  Le théorème A découle du théorème B via la suite exacte. La suite exacte est une conséquence directe du diagramme homotopiquement
  cartésien car $D\uparrow y$ est contractile.
\end{rem}

\begin{proof}
  On reprend le cadre de la démonstration du \sref{théorème}{theoremeA}~: $S(f)$, $T(f)$ et le diagramme suivant.
  \begin{center}
    \begin{tikzcd}[row sep=large,column sep=large]
      D\op & {S(f)} & C \\
      D\op & {S(\id_D)} & D
      \arrow[equal, from=1-1, to=2-1]
      \arrow["f", from=1-3, to=2-3]
      \arrow["{p_1}","\sim"', from=1-2, to=1-3]
      \arrow["{p_2}"', from=1-2, to=1-1]
      \arrow["\sim"',from=2-2, to=2-1]
      \arrow["\sim",from=2-2, to=2-3]
      \arrow["{f_*}", from=1-2, to=2-2]
    \end{tikzcd}
  \end{center}
  On rappel que $p_2=d(n:T(f)\ra D\op)$ où $n$ est donné par~:
  $$n_{p\bullet}:\bigsqcup_{y_p\ra\dotsb\ra y_0} f\uparrow y_0 \ra \bigsqcup_{y_p\ra\dotsb\ra y_0} *$$
  Donc $p_2=\hocolim{D\op}{N(f\uparrow -)}\ra D\op$. Donc, par le \sref{lemme}{lemmetheoremeB}, le carré
  \begin{center}
    \begin{tikzcd}[column sep=large,row sep=large]
      {f\uparrow y} & {S(f)} \\
      {*} & D\op
      \arrow["y", from=2-1, to=2-2]
      \arrow[from=1-1, to=2-1]
      \arrow["p_2", from=1-2, to=2-2]
      \arrow[from=1-1, to=1-2]
      \arrow["\mathlarger{\mathlarger{\mathlarger{\mathlarger{\lrcorner}}}}"{anchor=center, pos=0.05}, draw=none, from=1-1, to=2-2]
    \end{tikzcd}
  \end{center}
  est homotopiquement cartésien.
  Or, on a le diagramme commutatif suivant~:
  \begin{center}
    \begin{tikzcd}[column sep=tiny,row sep=tiny]
      {f\uparrow y} && {S(f)} && C \\
      &(1)&& (2)&\\
      {D\uparrow y} && {S(\id_D)} && D \\
      &(3)&&& \\
      {*} && D\op
      \arrow["\sim", from=1-3, to=1-5]
      \arrow["\sim", from=3-3, to=3-5]
      \arrow["f", from=1-5, to=3-5]
      \arrow["{f_*}", from=1-3, to=3-3]
      \arrow["\mathlarger{\mathlarger{\mathlarger{\mathlarger{\lrcorner}}}}"{anchor=center, pos=0.05}, draw=none, from=1-3, to=3-5]
      \arrow[from=1-1, to=1-3]
      \arrow[from=3-1, to=3-3]
      \arrow[from=1-1, to=3-1]
      \arrow["\sim" sloped, from=3-1, to=5-1]
      \arrow["\sim" sloped, from=3-3, to=5-3]
      \arrow["y", from=5-1, to=5-3]
      \arrow["\mathlarger{\mathlarger{\mathlarger{\mathlarger{\lrcorner}}}}"{anchor=center, pos=0.05}, draw=none, from=3-1, to=5-3]
      \arrow["\mathlarger{\mathlarger{\mathlarger{\mathlarger{\lrcorner}}}}"{anchor=center, pos=0.05}, draw=none, from=1-1, to=3-3]
    \end{tikzcd}
  \end{center}
  où les flèches notées $\sim$ sont des équivalences faibles. Le carré formé par $(1)$ et $(3)$ est homotopiquement cartésien,
  donc celui formé par $(1)$ également, et donc celui formé par $(1)$ et $(2)$ aussi. C'est le résultat recherché.
\end{proof}

\begin{rem}
  Le théorème B admet la forme duale suivante~:

  Soit $f:C\ra D$ un foncteur entre petites catégories. Si pour tout $u:y\ra y'$ dans $D$, $u_*:f\downarrow y\ra f\downarrow y'$
  est une équivalence faible d'homotopie, alors pour tout objet $y$ de $D$, le diagramme cartésien~:
  \begin{center}
    \begin{tikzcd}[column sep=large,row sep=large]
      {f\downarrow y} & {C} \\
      {D\downarrow y} & D
      \arrow["j'", from=2-1, to=2-2]
      \arrow["f_*", from=1-1, to=2-1]
      \arrow["f", from=1-2, to=2-2]
      \arrow["j", from=1-1, to=1-2]
      \arrow["\mathlarger{\mathlarger{\mathlarger{\mathlarger{\lrcorner}}}}"{anchor=center, pos=0.05}, draw=none, from=1-1, to=2-2]
    \end{tikzcd}
  \end{center}
  est homotopiquement cartésien.
\end{rem}

\begin{coro}
  Si $f:C\ra D$ un foncteur préfibré (respectivement précofibré) entre petites catégories et si pour tout $u:y\ra y'$ dans $D$,
  $u^*:f^{-1} y'\ra f^{-1} y$ (respectivement $u_*:f^{-1} y\ra f^{-1} y'$)
  est une équivalence faible d'homotopie, alors pour tout objet $y$ de $D$, $f^{-1}(y)$ est la fibre homotopique de $f$ au dessus de $y$.
  On a alors, pour tout $x$ objet de $f^{-1}(y)$, une suite exacte~:
  $$\dotsb\ra \pi_{i+1}(D,y)\ra \pi_{i}(f^{-1} y,x)\overset{j_*}{\ra} \pi_{i}(C,x)\overset{f_*}{\ra} \pi_{i}(D,y)\ra \dotsb$$
\end{coro}
\begin{proof}
  Dans ce cas $f^{-1}(y)\ra f\uparrow y$ ou $f^{-1}(y)\ra f\downarrow y$ est une équivalence faible d'homotopie. Plaçons nous dans 
  le premier cas, le second est similaire. On a un carré~:
  \begin{center}
    \begin{tikzcd}[column sep=large,row sep=large]
      {f^{-1}y'} & {f\uparrow y'} \\
      {f^{-1}y}  & {f\uparrow y}
      \arrow["\sim", from=2-1, to=2-2]
      \arrow["u^*","\sim"' sloped, from=1-1, to=2-1]
      \arrow["u^*", from=1-2, to=2-2]
      \arrow["\sim", from=1-1, to=1-2]
    \end{tikzcd}
  \end{center}
  Les deux foncteurs induits de $f^{-1}y'$ dans $f\uparrow y$ sont~:
  \[
    \begin{array}{lcl}
      F:x'&\mapsto& (x',u: y\ra y'=fx)\\
      G:x'&\mapsto& (u^*x',\id_y)             
    \end{array}
  \]
  Or on dispose d'une transformation naturelle $G\Rightarrow F,\; u^*x'\ra x'$ induite par l'adjonction.
  Donc le carré commute à homotopie près. Donc $u^*:f\uparrow y'\ra f\uparrow y$ est une équivalence d'homotopie. On peut maintenant appliquer le théorème B.
\end{proof}

\subsection{Premières propriétés}\label{premieresprops}

\section{\texorpdfstring{Construction $+$ et théorème "$+=Q$"}{Construction + et théorème "+=Q"}}

\subsection{\texorpdfstring{La construction $+$ en topologie}{La construction + en topologie}}

Dans cette section, nous définissons une construction $+$ en topologie. Nous utilisons un certain nombre de résultats
de topologie, notamment l'homologie à coefficients et les tours de Postnikov.
Voir \sref{l'annexe}{annexetopologie} pour plus d'information.

\begin{defi}\label{definitionplustopologie}
  Soit $X$ connexe dans $\DEns$, $x\in X_0$, et $P\triangleleft \pi_1(X,x)$ un sous-groupe normal parfait.
  Un morphisme $f:X\ra X^+$ est une construction $+$ pour $P$ si~:
  \begin{description}
    \item[$(1)$] $0\ra P\ra \pi_1(X,x) \ra \pi_1(X^+,fx)\ra 0$ est exacte~;
    \item[$(2)$] Pour tout $L$ système de coefficients sur $X^+$, l'application $H_*(X,f^*L)\ra H_*(X^+,L)$ est un isomorphisme.
  \end{description}
\end{defi}

\begin{rem}
  Comme $X$ est supposé connexe, le point $(1)$ ne dépend pas de $x\in X_0$.
\end{rem}

\begin{prop}\label{constructionpluscasfacile}
  Soit $X$ connexe dans $\DEns$, $x\in X_0$, tel que $\pi_1(X,x)$ soit un groupe parfait. Alors une construction $+$ existe pour $\pi_1(X,x)$.
\end{prop}

\begin{proof}
  Quitte à changer $X$, on peut supposer que $X$ est un complexe de Kan.
  On se donne $I$ un ensemble de générateurs de $\pi_1(X,x)$. On défini $Y$ comme la somme amalgamée~:
  \begin{center}
    \begin{tikzcd}[column sep = large, row sep = large]
      {\bigvee_{\gamma\in I}\partial\Delta^2} & X \\
      {\bigvee_{\gamma\in I}\Delta^2} & Y
      \arrow[hook, from=1-1, to=2-1]
      \arrow[from=1-2, to=2-2]
      \arrow["{\vee(\gamma,*,*)}", from=1-1, to=1-2]
      \arrow[from=2-1, to=2-2]
      \arrow["\mathlarger{\mathlarger{\mathlarger{\mathlarger{\lrcorner}}}}"{anchor=center, rotate=180, pos=0.05}, draw=none, from=2-2, to=1-1]
    \end{tikzcd}
  \end{center}
  Alors, par le [lemme d'extension] $\pi_1(Y)=0$.
  Maintenant, comme $\bigvee_{\gamma\in I}\Delta^2$ est contractile,
  ${\bigvee_{\gamma\in I}\partial\Delta^2}\ra X\ra Y$ est une suite cofibre. On a donc les suites exactes suivantes en homologie sur $\Z$~:
  $$0\ra H_i(X)\ra H_i(Y)\ra 0\text{ pour }i\geq 3$$
  $$H_2({\bigvee_{\gamma\in I}\partial\Delta^2})\ra H_2(X)\ra H_2(Y)\ra H_1({\bigvee_{\gamma\in I}\partial\Delta^2})\ra H_1(X)$$
  Or, par [Hurewicz pi1], $H_1(X)=0$. De plus $H_2({\bigvee_{\gamma\in I}\partial\Delta^2})=0$
  et $H_1({\bigvee_{\gamma\in I}\partial\Delta^2})$ est $\Z$-libre.
  Donc, la suite se scinde et on peut choisir un isomorphisme~:
  $$H_2(Y)\simeq H_2(X)\oplus \bigoplus_{\gamma\in I}\Z\cdot [\gamma]$$
  Or, $\pi_0(Y)=*$ et $\pi_1(Y)=0$, donc par [Hurewicz pi2], on a l'isomorphisme de Hurewicz $\mathcal{H}_2:\pi_2(Y,fx)\simeq H_2(Y)$.
  On se donne $\tilde{Y}$ complexe de Kan équivalent à $Y$
  et, pour chaque $\gamma\in I$, $s_\gamma\in \pi_2(\tilde{Y},fx)$ tel que $\mathcal{H}_2(s_\gamma)=[\gamma]$.
  On définit alors $X^+$ comme la somme amalgamée~:
  \begin{center}
    \begin{tikzcd}[column sep = large, row sep = large]
      {\bigvee_{\gamma\in I}\partial\Delta^3} & \tilde{Y} \\
      {\bigvee_{\gamma\in I}\Delta^3} & X^+
      \arrow[hook, from=1-1, to=2-1]
      \arrow[from=1-2, to=2-2]
      \arrow["{\vee(s_\gamma,*,*,*)}", from=1-1, to=1-2]
      \arrow[from=2-1, to=2-2]
      \arrow["\mathlarger{\mathlarger{\mathlarger{\mathlarger{\lrcorner}}}}"{anchor=center, rotate=180, pos=0.05}, draw=none, from=2-2, to=1-1]
    \end{tikzcd}
  \end{center}
  Par le [lemme d'extension], $\pi_1(X^+)=0$. Or la suite ${\bigvee_{\gamma\in I}\partial\Delta^3}\ra \tilde{Y}\ra X^+$ est une suite cofibre.
  Comme $H_i(\bigvee_{\gamma\in I}\partial\Delta^3)=0$ pour $i\geq 3$, et $H_2(\bigvee_{\gamma\in I}\partial\Delta^3)\hookrightarrow H_2(\tilde{Y})$
  est une injection, on a $H_i(\tilde{Y})\simeq H_i(X^+)$ et donc $H_i(X)\simeq H_i(X^+)$ pour $i\geq 3$. C'est également clair pour $H_1$ et $H_0$.
  Pour $H_2$, on a le diagramme suivant~:
  \begin{center}
    \begin{tikzcd}[column sep = large, row sep = large]
      & {H_1(\bigvee_I\partial\Delta^2)} \\
      {H_2(\bigvee_{I}\partial\Delta^3)} & {H_2(\tilde{Y})} & {H_2(X^+)} & {} \\
      & {H_2(X)}
      \arrow[hook, from=2-1, to=2-2]
      \arrow[two heads, from=2-2, to=2-3]
      \arrow[hook, from=3-2, to=2-2]
      \arrow[from=3-2, to=2-3]
      \arrow[two heads, from=2-2, to=1-2]
      \arrow["\sim" sloped, from=2-1, to=1-2]
    \end{tikzcd}
  \end{center}
  La flèche $H_2(\bigvee_{I}\partial\Delta^3)\ra H_1(\bigvee_I\partial\Delta^2)$ est un isomorphisme par construction de
  $\bigvee_I\partial\Delta^3\ra \tilde{Y}$. Comme la ligne et la colonne sont exactes, ceci implique que la flèche
  $H_2(X)\ra H_2(X^+)$ est aussi un isomorphisme.

  On a donc que $H_*(-;\Z)$ envoie $X\ra X^+$ sur un isomorphisme. Ainsi, par le [théorème des coefficients universelles],
  $H_*(-;M)$ envoie $X\ra X^+$ sur un isomorphisme pour tout $M$ $\Z$-module. Or, comme $\pi_1(X^+)=0$, un système de coéfficients
  sur $X^+$ n'est autre qu'un $\Z$-module. Donc $X\ra X^+$ est une construction $+$ pour $\pi_1(X,x)$.
\end{proof}

\begin{prop}
  Soit $X$ connexe dans $\DEns$, $x\in X_0$ et $P\triangleleft \pi_1(X,x)$ sous-groupe normal parfait. Alors une construction $+$
  pour $P$ existe.
\end{prop}

\begin{proof}
  On note $p:\tilde{X}\ra X$ le revêtement connexe correspondant à $\pi_1(X,x)/P$. On se donne $\tilde{x}\in \tilde{X}$ au-dessus de $x$.
  On a alors $\pi_1(\tilde{X},\tilde{x})=P$.

  On pose $\tilde{f}:\tilde{X}\hookrightarrow \tilde{X}^+$ une construction $+$ pour $\pi_1(\tilde{X},\tilde{x})$ qui soit une cofibration.
  On définit $X^+$ comme la somme amalgamée suivante~:
  \begin{center}
    \begin{tikzcd}[column sep = large, row sep = large]
      {\tilde{X}} & \tilde{X}^+ \\
      {X} & X^+
      \arrow["p", from=1-1, to=2-1]
      \arrow["p'", from=1-2, to=2-2]
      \arrow[hook, "\tilde{f}", from=1-1, to=1-2]
      \arrow[hook, "f", from=2-1, to=2-2]
      \arrow["\mathlarger{\mathlarger{\mathlarger{\mathlarger{\lrcorner}}}}"{anchor=center, rotate=180, pos=0.05}, draw=none, from=2-2, to=1-1]
    \end{tikzcd}
  \end{center}
  Comme tous les objets de $\DEns$ sont cofibrants, par le dual de la \sref{proposition}{homocartesienfibrant}, 
  $\DEns$ est propre à gauche.
  Ainsi, par le dual de la \sref{proposition}{criterehomocartesienpropreadroite}, le carré ci-dessus est homotopiquement cocartésien.
  Par le [théorème de Van-Kampen dans $\DEns$], la suite~:
  $$\exa{P}{\pi_1(X,x)}{\pi_1(X^+,fx)}\text{ est exacte}$$
  Comme le carré est homotopiquement cocartésien, et comme $\tilde{X}\hookrightarrow\tilde{X}\times\Delta^1\sqcup_{X}X\overset{\sim}{\twoheadrightarrow}X$,
  on a le diagramme commutatif suivant a carré cocartésiens~:
  \begin{center}
    \begin{tikzcd}[column sep = large, row sep = large]
      {\tilde{X}} & \tilde{X}^+ \\
      {\tilde{X}\times\Delta^1\sqcup_{X}X} & {X'^+} \\
      {X} & {X^+}                            
      \arrow[hook, from=1-1, to=2-1]
      \arrow["p''",hook, from=1-2, to=2-2]
      \arrow[hook, from=1-1, to=1-2]
      \arrow[hook, from=2-1, to=2-2]
      \arrow[two heads, "\sim" sloped,from=2-1, to=3-1]
      \arrow[hook, from=3-1, to=3-2]
      \arrow["\sim" sloped, from=2-2, to=3-2]
      \arrow["\mathlarger{\mathlarger{\mathlarger{\mathlarger{\lrcorner}}}}"{anchor=center, rotate=180, pos=0.05}, draw=none, from=2-2, to=1-1]
      \arrow["\mathlarger{\mathlarger{\mathlarger{\mathlarger{\lrcorner}}}}"{anchor=center, rotate=180, pos=0.05}, draw=none, from=3-2, to=2-1]
    \end{tikzcd}
  \end{center}
  Où les applications marqué de $\sim$ sont des équivalences.
  Maintenant, soit $L$ un système de coefficients locaux sur $X^+$. Comme on peut identifier $\tilde{X}^+\setminus \tilde{X}$
  et $X'^+\setminus \tilde{X}\times\Delta^1\sqcup_{X}X$, on a~:
  $$C_*(\tilde{X}^+,\tilde{X};p''^*L)\simeq C_*(X'^+, \tilde{X}\times\Delta^1\sqcup_{X}X;L)$$
  Et donc~: $$H_*(\tilde{X}^+,\tilde{X};p'^*L)\lra H_*(X^+,X;L)$$ est un isomorphisme.
  Or, comme $H_*(\tilde{X};p'^*L)\ra H_*(\tilde{X}^+;\tilde{f}^*p'^*L)$ est un isomorphisme, $H_*(\tilde{X}^+,\tilde{X};p'^*L)=0$.
  Donc $H_*(X;f^*L)\ra H_*(X^+;L)$ est un isomorphisme.
\end{proof}

Nous allons maintenant énoncer un résultat d'unicité.

\begin{prop}\label{topologiepluspropuniv}
  Soit $X$ connexe dans $\DEns$, $x\in X_0$ et $P\triangleleft \pi_1(X,x)$ un sous-groupe normal parfait. Soit $f:X\hookrightarrow X^+$ une construction $+$
  pour $P$. Alors pour tout $g:X\ra Y$ avec $P\subseteq \myker{\pi_1(g,x)}$ et $Y$ complexe de Kan, il existe $h:X^+\ra Y$ tel que $h\circ f=g$.
  \begin{center}
    \begin{tikzcd}[column sep = small, row sep = large]
      X && Y \\
      & {X^+}
      \arrow["f",hook, from=1-1, to=2-2]
      \arrow["{\exists h}", dashed, from=2-2, to=1-3]
      \arrow["g", from=1-1, to=1-3]
    \end{tikzcd}
  \end{center}
  De plus, tout autre $h'$ a le même type d'homotopie que $h$ dans $\Ho{\DEns\uparrow X}$. 
\end{prop}

\begin{rem}
  La dernière affirmation est équivalente à~: $h'$ est homotope à $h$ relativement à $X$. 
\end{rem}

\begin{proof}
  Sans perte de généralité, on peut supposer que $Y$ est connexe.
  On se donne $\tilde{X}\ra X$ revêtement associé à $\pi_1(X,x)/P$ et $\tilde{Y}\ra Y$ revêtement universels. Alors~:
  \begin{center}
    \begin{tikzcd}[column sep = large, row sep = large]
      {\tilde{X}} & {\tilde{Y}} \\
      X & Y
      \arrow["{\exists \tilde{g}}", dashed, from=1-1, to=1-2]
      \arrow[two heads, from=1-1, to=2-1]
      \arrow[two heads, from=1-2, to=2-2]
      \arrow["g", from=2-1, to=2-2]
    \end{tikzcd}
  \end{center}
  Maintenant, $\tilde{Y}$ est simplement connexe et par [mettre ref], il dispose donc d'une tour de Postnikov de fibrations
  principales.
  \begin{center}
    \begin{tikzcd}
      && {\tilde{Y}(n)} \\
      {\tilde{Y}} && {\tilde{Y}(n-1)} & {K(n+1,\pi_n(\tilde{Y}))} \\
      && {\tilde{Y}(2)} & {K(4,\pi_3(\tilde{Y}))} \\
      && {\tilde{Y}(1)\simeq *} & {K(3,\pi_2(\tilde{Y}))}
      \arrow[two heads, from=1-3, to=2-3]
      \arrow["{u_{n-1}}", from=2-3, to=2-4]
      \arrow[dotted, two heads, from=2-3, to=3-3]
      \arrow["{u_2}", from=3-3, to=3-4]
      \arrow[two heads, from=3-3, to=4-3]
      \arrow["{u_1}", from=4-3, to=4-4]
      \arrow[from=2-1, to=1-3]
      \arrow[from=2-1, to=2-3]
      \arrow[from=2-1, to=3-3]
      \arrow[from=2-1, to=4-3]
    \end{tikzcd}
  \end{center}
  On choisit des espaces d'Eilenberg-MacLane qui soient fibrants.
  Supposons que l'on ai défini des applications compatibles $\tilde{h}(i):\tilde{X}^+\ra \tilde{Y}(i)$, pour $1\leq i\leq n-1$, tels que pour chaque $i$~:
  \begin{center}
    \begin{tikzcd}
      {\tilde{X}} & {\tilde{Y}} & {\tilde{Y}(i)} \\
      & {\tilde{X}^+}
      \arrow[hook, from=1-1, to=2-2]
      \arrow["{\tilde{h}(i)}", from=2-2, to=1-3]
      \arrow["{\tilde{g}}", from=1-1, to=1-2]
      \arrow[from=1-2, to=1-3]
    \end{tikzcd}
  \end{center}
  On a alors le diagramme commutatif suivant~:
  \begin{center}
    \begin{tikzcd}
      {\tilde{X}}   & {\tilde{Y}(n)}   & \mathrm{F}(u_{n-1})\\
      {\tilde{X}^+} & {\tilde{Y}(n-1)} & \mathrm{Path}(u_{n-1}) & {K(n+1,\pi_n(\tilde{Y}))}
      \arrow[from=1-2, to=2-2]
      \arrow[hook, from=1-3, to=2-3]
      \arrow["\sim", from=1-2, to=1-3]
      \arrow["\sim", from=2-2, to=2-3]
      \arrow[two heads, "p", from=2-3, to=2-4]
      \arrow["{\tilde{h}(n-1)}", from=2-1, to=2-2]
      \arrow[from=1-1, to=1-2]
      \arrow[from=1-1, to=2-1]
      \arrow["\sim", from=1-2, to=1-3]
    \end{tikzcd}
  \end{center}
  Où $\mathrm{Path}(u_{n-1})={\tilde{Y}(n-1)\times_{K(n+1,\pi_n(\tilde{Y}))} K(n+1,\pi_n(\tilde{Y}))^{\Delta^1}}$ et $\mathrm{F}(u_{n-1})$
  est la fibre de $p$.
  Comme le L à droite est une suite fibre, ce diagramme nous donne un morphisme $\tilde{X}^+\cup C\tilde{X}\ra K(n+1,\pi_n(\tilde{Y}))$
  où $C\tilde{X}:=\tilde{X}\times\Delta^1 / \tilde{X}\times\{1\}$ est le cône de $\tilde{X}$.
  La classe d'homotopie de ce morphisme correspond à un élément de
  $H^{n+1}(\tilde{X}^+\cup C\tilde{X};\pi_2(\tilde{Y}))=H^{n+1}(\tilde{X}^+,\tilde{X};\pi_2(\tilde{Y}))=0$
  (par le [corollaire au théorème des coefficients universels]). Donc $\tilde{X}^+\cup C\tilde{X}\ra K(n+1,\pi_n(\tilde{Y}))$
  est homotope à une application constante et on peut l'étendre en $C\tilde{X}^+\ra K(n+1,\pi_n(\tilde{Y}))$,
  c'est-à-dire prolonger $\tilde{X}\ra \mathrm{F}(u_{n-1})$ à $\tilde{X}^+$. On peut donc trouver $\tilde{h}(n):\tilde{X}^+\ra \tilde{Y}(n)$
  qui complète le diagramme~:
  \begin{center}
    \begin{tikzcd}
      {\tilde{X}} && {\tilde{Y}(n)} \\
      {\tilde{X}^+} && {\tilde{Y}(n-1)}
      \arrow[two heads, from=1-3, to=2-3]
      \arrow["{\tilde{h}(n-1)}"', from=2-1, to=2-3]
      \arrow[from=1-1, to=1-3]
      \arrow[from=1-1, to=2-1]
      \arrow["{\tilde{h}(n)}", from=2-1, to=1-3]
    \end{tikzcd}
  \end{center}
  Ainsi, nous avons construit des des applications compatibles $\tilde{h}(i):\tilde{X}^+\ra \tilde{Y}(i)$, pour $i\geq 1$.
  Or par la [référence à mettre] $\lim_{i\in \N}\tilde{Y}(i)=\tilde{Y}$. Donc on a construit $\tilde{h}:\tilde{X}^+\ra \tilde{Y}$.
  Donc~:
  \begin{center}
    \begin{tikzcd}[column sep = large, row sep = large]
      {\tilde{X}} & {\tilde{X}^+} & {\tilde{Y}} \\
      X & {X^+} & Y
      \arrow["{\tilde{h}}", from=1-2, to=1-3]
      \arrow[hook, from=1-1, to=1-2]
      \arrow[two heads, from=1-1, to=2-1]
      \arrow[two heads, from=1-3, to=2-3]
      \arrow[hook, from=2-1, to=2-2]
      \arrow["{\exists h}", dashed, from=2-2, to=2-3]
      \arrow["g"', bend right=20, from=2-1, to=2-3]
      \arrow[two heads, from=1-2, to=2-2]
      \arrow["\mathlarger{\mathlarger{\mathlarger{\mathlarger{\lrcorner}}}}"{anchor=center, rotate=180, pos=0.05}, draw=none, from=2-2, to=1-1]
    \end{tikzcd}
  \end{center}
  Montrons maintenant l'unicité à homotopie près. Soit $h'$ une autre telle application. Par propriété de relèvement le long des revêtement, on
  peut trouver $\tilde{h}$ et $\tilde{h'}$ qui complètent le digramme~:
  \begin{center}
    \begin{tikzcd}[column sep = large, row sep = large]
      {\tilde{X}} & {\tilde{X}^+} & {\tilde{Y}} \\
      X & {X^+} & Y
      \arrow["{\tilde{h}}", shift left=1, from=1-2, to=1-3]
      \arrow[hook, from=1-1, to=1-2]
      \arrow[two heads, from=1-1, to=2-1]
      \arrow[two heads, from=1-3, to=2-3]
      \arrow[hook, from=2-1, to=2-2]
      \arrow["h", shift left=1, from=2-2, to=2-3]
      \arrow[two heads, from=1-2, to=2-2]
      \arrow["\lrcorner"{anchor=center, pos=0.125, rotate=180}, draw=none, from=2-2, to=1-1]
      \arrow["{h'}"', shift right=1, from=2-2, to=2-3]
      \arrow["{\tilde{h'}}"', shift right=1, from=1-2, to=1-3]
    \end{tikzcd}
  \end{center}
  Il suffit donc de montrer que $\tilde{h}$ et $\tilde{h'}$ sont homotopes relativement à $\tilde{X}$.
  Pour cela, on se donne une homotopie~:
  $$H(1):\tilde{h}(1)\Rightarrow\tilde{h'}(1):\tilde{X}^+\times\Delta^1\ra \tilde{Y}(0)$$
  C'est toujours possible car $\tilde{Y}(0)$ est contractile.
  Or, on remarque que 
  $$\tilde{X}\times\Delta^1\cup\tilde{X}^+\times\partial\Delta^1\hookrightarrow\tilde{X}^+\times\Delta^1$$
  est une équivalence en homologie, et donc aussi en cohomologie.
  En effet, la suite exacte de complexes~:
  \begin{center}
    \begin{tikzcd}[row sep = small, column sep = small]
      0 \ra C_*(\tilde{X}\times\Delta^1) \arrow[r]
      & C_*(\tilde{X}\times \Delta^1\cup\tilde{X}^+\times \{0\}\oplus C_*(\tilde{X}\times \Delta^1\cup\tilde{X}^+\times \{1\})
      \arrow[d, phantom, ""{coordinate, name=Z}]
      \arrow[d,
      rounded corners,
      to path={ -- ([xshift=2ex]\tikztostart.east)
      |- (Z) [near end]\tikztonodes
      -| ([xshift=-2ex]\tikztotarget.west)
      -- (\tikztotarget)}] \\
      &C_*(\tilde{X}\times\Delta^1\cup\tilde{X}^+\times\partial\Delta^1) \ra 0
      \end{tikzcd}
  \end{center}
  induit, car $\tilde{X}\times \Delta^1\cup\tilde{X}^+\times \{-\}\simeq \tilde{X}^+$, des suites exactes~:
  $$H_n(\tilde{X}^+)\overset{(1,-1)}{\hookrightarrow}H_n(\tilde{X}^+)\oplus H_n(\tilde{X}^+)\twoheadrightarrow%
    H_n(\tilde{X}\times\Delta^1\cup\tilde{X}^+\times\partial\Delta^1)$$
  On a donc bien que $\tilde{X}^+\ra \tilde{X}\times\Delta^1\cup\tilde{X}^+\times\partial\Delta^1$ est un isomorphisme en homologie.
  Or $\tilde{X}^+ \simeq \tilde{X}^+ \times \Delta^1$.
  Ainsi, on peut, comme ci-dessus, relever successivement l'homotopie à
  chaque $\tilde{Y}(n)$, et en passant à la limite, à $\tilde{Y}$.
\end{proof}

On a directement le corollaire suivant.

\begin{coro}\label{uniciteplustopologie}
  Soit $X$ connexe dans $\DEns$, $x\in X_0$ et $P\triangleleft \pi_1(X,x)$ un sous-groupe normal parfait.
  Alors une construction $+$ pour $P$ est unique à équivalence d'homotopie sous $X$ près.
\end{coro}

Plus loin, nous auront besoin d'une version améliorée de ces résultats. On rappel qu'un espace $X$ est simple si en chaque point $x\in X_0$ et chaque
$n\geq 1$, $\pi_1(X,x)$ agit trivialement sur $\pi_n(X,x)$. En particulier, $\pi_1(X,x)$ est abélien pour tout $x\in X_0$.

\begin{lem}
  Soit $X$ connexe dans $\DEns$, $x\in X_0$ tel que le sous-groupe $C:=[\pi_1(X,x),\pi_1(X,x)]$ soit parfait.
  Soit $f:X\ra X^+$ une construction $+$ pour $C$ telle que $X^+$ soit simple.
  Alors, si $g:X\ra Y$ vérifie~:
  \begin{description}
    \item[$(1)$] $Y$ est simple et connexe~;
    \item[$(2)$] $g_*:H_*(X,\Z)\ra H_*(Y,\Z)$ est un isomorphisme. 
  \end{description}
  Alors $g$ est aussi une construction $+$ pour $C$. En particulier $Y$ et $X$ sont équivalents relativement à $X$.
\end{lem}

\begin{rem}
  La preuve de ce lemme repose sur la [mettre ref à l'annexe], laquelle admet la version générale du théorème de Hurewicz pour les paires.
  Voir la référence donnée dans la preuve de cette proposition pour plus d'information.
\end{rem}

\begin{proof}
  Soit $h:X\ra Z$ avec $Z$ simple. Alors par la [mettre ref à l'annexe], $Z$ admet une tour de Postnikov de fibrations principales~:
  \begin{center}
    \begin{tikzcd}
      && {\tilde{Y}(n)} \\
      {\tilde{Y}} && {\tilde{Y}(n-1)} & {K(n+1,\pi_n(\tilde{Y}))} \\
      && {\tilde{Y}(1)} & {K(3,\pi_2(\tilde{Y}))} \\
      && {\tilde{Y}(0)\simeq *} & {K(2,\pi_1(\tilde{Y}))}
      \arrow[two heads, from=1-3, to=2-3]
      \arrow["{u_{n-1}}", from=2-3, to=2-4]
      \arrow[dotted, two heads, from=2-3, to=3-3]
      \arrow["{u_1}", from=3-3, to=3-4]
      \arrow[two heads, from=3-3, to=4-3]
      \arrow["{u_0}", from=4-3, to=4-4]
      \arrow[from=2-1, to=1-3]
      \arrow[from=2-1, to=2-3]
      \arrow[from=2-1, to=3-3]
      \arrow[from=2-1, to=4-3]
    \end{tikzcd}
  \end{center}
  Comme $g_*:H_*(X,\Z)\ra H_*(Y,\Z)$ est un isomorphisme, c'est également le cas de $g^*:H^*(Y,M)\ra H^*(X,M)$ pour tout $M$ $\Z$-module
  par le [corollaire du théorème des coefficients universels]. Donc, comme dans la preuve de la \sref{proposition}{topologiepluspropuniv},
  on dispose de $h_Y:Y\ra Z$ qui complète le diagramme commutatif~:
  \begin{center}
    \begin{tikzcd}[column sep = small, row sep = large]
      X && Z \\
      & Y
      \arrow["h", from=1-1, to=1-3]
      \arrow["g"', from=1-1, to=2-2]
      \arrow["{h_Y}"', from=2-2, to=1-3]
    \end{tikzcd}
  \end{center}
  Comme $X\ra X^+$ vérifie également $(1)$ et $(2)$, on a également un morphisme $h_{X^+}:X^+\ra Z$ qui complète le diagramme~:
  \begin{center}
    \begin{tikzcd}[column sep = small, row sep = large]
      X && Z \\
      & {X^+}
      \arrow["h", from=1-1, to=1-3]
      \arrow["f"', from=1-1, to=2-2]
      \arrow["{h_{X^+}}"', from=2-2, to=1-3]
    \end{tikzcd}
  \end{center}
  Maintenant, on pose $h=f$, $Z=X^+$ pour obtenir un morphisme $f_Y:Y\ra X^+$, et $h=g$, $Z=Y$ pour obtenir un morphisme $g_{X^+}:X^+\ra Y$.
  Alors, $g_{X^+}\circ f_Y$ et $\id_{X^+}$ sont tout deux des morphismes $X^+\ra X^+$ dans $\DEns\uparrow X$. Or, $X^+$ étant simple,
  il admet une tour de Postnikov de fibrations principales. Donc, comme dans la preuve de l'unicité de la \sref{proposition}{topologiepluspropuniv},
  $g_{X^+}\circ f_Y$ et $\id_{X^+}$ sont homotopes sous $X$. De même pour $f_Y\circ g_{X^+}$ et $\id_{Y}$.
  Ainsi, $X^+$ et $Y$ sont homotopes sous $X$, et $g:X\ra Y$ est une construction $+$ pour $C$.
\end{proof}

\subsection{\texorpdfstring{La construction $+$ de la $K$-théorie}{La construction + de la K-théorie}}

Dans cette section, nous verrons certains groupes comme des ensembles simpliciaux via \sref{le foncteur nerf}{definitionnerf}.

\begin{defi}
  Soit $A$ un anneau. On note $f_A:\GL{}{A}\ra\GL{}{A}^+$ une construction $+$ pour $\EGL{}{A}$ (\sref{définition}{definitionplustopologie}).
  Pour $n\geq 1$, on pose~:
  $$K_n(A):=\pi_{n}(\GL{}{A},f_A*)$$
  où $*\in \GL{}{A}_0$ est l'unique point.

  Si $u:A\ra B$ est un morphisme d'anneau, par la \sref{proposition}{topologiepluspropuniv}, il existe $u^*$ complétant le carré~:
  \begin{center}
    % https://q.uiver.app/?q=WzAsNCxbMCwwLCJcXEdMe317QX0iXSxbMSwwLCJcXEdMe317Qn0iXSxbMCwxLCJcXEdMe317QX1eKyJdLFsxLDEsIlxcR0x7fXtCfV4rIl0sWzAsMSwiXFxHTHt9e3V9Il0sWzIsMywidV8qIl0sWzAsMiwiZl9BIiwyXSxbMSwzLCJmX0IiXV0=
    \begin{tikzcd}[column sep = large, row sep = large]
    	{\GL{}{A}} & {\GL{}{B}} \\
    	{\GL{}{A}^+} & {\GL{}{B}^+}
    	\arrow["{\GL{}{u}}", from=1-1, to=1-2]
    	\arrow["{u_*}", from=2-1, to=2-2]
    	\arrow["{f_A}"', from=1-1, to=2-1]
    	\arrow["{f_B}", from=1-2, to=2-2]
    \end{tikzcd}
  \end{center}
  qui induit des morphismes de groupe $u^*:K_n(A)\ra K_n(B)$ pour $n\geq 1$.
\end{defi}

\begin{rem}
  Cette définition ne définit bien les groupes $K_n(A)$, et pas seulement à conjugaison près. En effet, l'unicité de la construction $+$ donnée par le
  \sref{corollaire}{uniciteplustopologie} est à homotopie sous $\GL{}{A}$ près. De même pour les morphismes $u^*$.
\end{rem}

Nous auront besoin, pour démontrer le théorème $+=Q$ [mettre ref], de montrer que $\GL{}{A}^+$ est un espace simple.
C'est l'objet de la fin de cette sous-section.

\begin{lem}\label{lemmeplussimple}
  Soit $G$ un groupe tel que $[G,G]$ soit parfait. On note $f:G\hookrightarrow G^+$ une construction $+$ pour $G$ telle que
  $G^+$ soit un complexe de Kan. On choisit $\gamma\in [G,G]$, et on note $c_\gamma$ la conjugaison par $\gamma$ dans $G$.
  Alors il existe une homotopie \textbf{pointée} $H:G\times\Delta^1\ra G^+$ entre $f$ et $f\circ c_\gamma$.
\end{lem}

\begin{proof}
  On remarque d'abord que $\gamma$ induit une transformation naturelle entre $\id_G$ et $c_\gamma$~:
  $$h:G\times(0\ra 1)\ra G,\;*\overset{\gamma}{\ra} *$$
  Mais $h$ n'est pas pointée (cela fera toute la différence dans la \sref{proposition}{GLAplussimple}).
  Comme $f_*[\gamma]=0$, on dispose de $\tau:\Delta^2\ra G^+$ tel que~:
  \begin{center}
    \begin{tikzcd}[row sep = 0.50em, column sep = 1em]
      & 1 \\
      & {\tau} \\[-0.30em]
      0 && 2
      \arrow["{s_0(*)}", from=3-1, to=1-2]
      \arrow["{s_0(*)}", from=1-2, to=3-3]
      \arrow["\gamma"', from=3-1, to=3-3]
    \end{tikzcd}
  \end{center}
  Maintenant, on peut compléter le diagramme~:

  \begin{minipage}{0.70\textwidth}
    \begin{center}
      % https://q.uiver.app/?q=WzAsMyxbMCwwLCJHXFx0aW1lc1xcTGFtYmRhXzJeMlxcY3VwXFx7KlxcfVxcdGltZXNcXERlbHRhXjIiXSxbMSwwLCJHXisiXSxbMCwxLCJHXFx0aW1lc1xcRGVsdGFeMiJdLFswLDIsIiIsMCx7InN0eWxlIjp7InRhaWwiOnsibmFtZSI6Imhvb2siLCJzaWRlIjoidG9wIn19fV0sWzAsMSwiKHNfMChmKSxmXFxjaXJjIGgsLSlcXGN1cCBcXHRhdSJdLFsyLDEsIlxcZXhpc3RzIFxcdGhldGEiLDIseyJzdHlsZSI6eyJib2R5Ijp7Im5hbWUiOiJkYXNoZWQifX19XV0=
      \begin{tikzcd}[column sep = 6em]
        {G\times\Lambda_0^2\cup\{*\}\times\Delta^2} & {G^+} \\
        {G\times\Delta^2}
        \arrow[hook, from=1-1, to=2-1]
        \arrow["{(-,f\circ h,s_0(f))\cup \tau}", from=1-1, to=1-2]
        \arrow["{\exists \theta}"', dashed, from=2-1, to=1-2]
      \end{tikzcd}
    \end{center}
  \end{minipage}\hfill
  \begin{minipage}{0.26\textwidth}
    \begin{center}
      % https://q.uiver.app/?q=WzAsNCxbMSwxLCJcXGV4aXN0c1xcdGhldGEiXSxbMSwwLCIxIl0sWzAsMiwiMCJdLFsyLDIsIjIiXSxbMiwzLCJmXFxjaXJjIGgiLDJdLFsxLDMsInNfMChmKSJdLFsyLDEsIiIsMix7InN0eWxlIjp7ImJvZHkiOnsibmFtZSI6ImRhc2hlZCJ9fX1dXQ==
      \begin{tikzcd}[row sep = 0.50em, column sep = 1em]
        & 1 \\
        & \exists\theta \\[-0.30em]
        0 && 2
        \arrow["{f\circ h}"', from=3-1, to=3-3]
        \arrow["{s_0(f)}", from=3-1, to=1-2]
        \arrow["H", dashed, from=1-2, to=3-3]
      \end{tikzcd}
    \end{center}
  \end{minipage}

  Alors $H:=\theta\circ (\id_G\times d^0)$ convient.
\end{proof}

\begin{prop}\label{GLAplussimple}
  Soit $A$ un anneau. Alors $\GL{}{A}^+$ est simple.
\end{prop}

\begin{proof}
  \begin{description}
    \item[(A)] On commence par construire une application~:
    $$\mu^+: \GL{}{A}^+\times\GL{}{A}^+\ra\GL{}{A}^+$$
    Pour cela,
    on pose $\mu:\GL{}{A}\times\GL{}{A}\ra \GL{}{A}$ l'application $(M,N)\mapsto P$, avec $P_{2n+1,2m+1}=M_{n,m}$
    $P_{2n,2m}=N_{n,m}$ et $P_{i,j}=0$ sinon. Or, en utilisant [ref homologie du produit], on remarque que~:
    $$\GL{}{A}\times \GL{}{A}\ra \GL{}{A}^+\times\GL{}{A}^+$$
    est une construction $+$. On peut donc étendre $\mu$ en $\mu^+:\GL{}{A}^+\times\GL{}{A}^+\ra\GL{}{A}^+$ comme souhaité.
    \item[(B)] De la chaîne de cofibrations~:
    $$\GL{1}{A}\hookrightarrow \GL{2}{A}\hookrightarrow\dotsb\hookrightarrow\GL{n}{A}\hookrightarrow\dotsb$$
    On peut déduire, par la \sref{proposition}{topologiepluspropuniv}, une chaîne de cofibrations entre complexes de Kan~:
    $$\GL{1}{A}^+\hookrightarrow \GL{2}{A}^+\hookrightarrow\dotsb\hookrightarrow\GL{n}{A}^+\hookrightarrow\dotsb$$
    On note $G$ sa colimite. Alors par fonctorialité des colimites, on a une application $u:\GL{}{A}\ra G$.
    De plus~:
    $$\pi_1(\GL{}{A})\ra \pi_1(G)=\GL{}{A}\ra\colim_{n}\pi_1(\GL{n}{A}^+)$$
    Et~:
    $$H_*(\GL{}{A},u^*L)\ra H_*(G,L)=\colim_{n}(H_*(\GL{n}{A},u^*L)\ra H_*(\GL{n}{A}^+,L))$$
    Donc $u:\GL{}{A}\ra G$ est une construction $+$ pour $[\GL{}{A},\GL{}{A}]$. Ainsi, on peut supposer que~:
    $$\GL{}{A}^+ =\bigcup_n\GL{n}{A}^+$$
    \item[(C)] Fixons $n\geq 1$. On note $\mu_n:\GL{n}{A}\times\GL{n}{A}\ra\GL{}{A}^+$ la restriction de $\mu$,
    et $i_n:\GL{n}{A}\ra\GL{}{A}^+$ l'application naturelle. Alors, $\mu_n(*,-)$ et $i_n$ sont conjugués par une permutation $\gamma\in \EGL{2n}{A}$.
    Ainsi, on dispose, par le \sref{lemme}{lemmeplussimple}, d'une homotopie pointée $h:\GL{}{A}\times\Delta^1\ra \GL{}{A}^+$,
    telle que sa restriction $H:\GL{n}{A}\times\Delta^1\ra \GL{}{A}^+$ soit une homotopie pointée entre $\mu_n(*,-)$ et $i_n$. Or, l'application~:
    $$\GL{n}{A}\times\Delta^1\cup\GL{n}{A}^+\times\partial\Delta^1\ra\GL{n}{A}^+\times\Delta^1$$
    est un isomorphisme sur tous les groupes d'homologie (voir la fin de la démonstration de la \sref{proposition}{topologiepluspropuniv}).
    Donc on peut étendre $H$ en~:
    $$H_n^+:\GL{n}{A}^+\times\Delta^1\ra \GL{}{A}^+$$
    homotopie entre $\mu_n^+(*,-)$ et $i_n^+$ (extensions données par la \sref{proposition}{topologiepluspropuniv}). On pourrait faire de même
    avec $\mu_n^+(-,*)$ et $i_n^+$.
    \item[(D)] Maintenant, nous fixons $m\geq 1$, $\eta\in \pi_1(\GL{}{A}^+,*)$ et $\delta\in\pi_m(\GL{}{A}^+,*)$. Par (B), il existe
    $n\geq 1$ tel que $\delta$ et $\eta$ soient supportés dans $\GL{n}{A}^+$. Alors~:
    $$\mu^+(\eta,\delta):\Delta^1\times\Delta^m\ra \GL{}{A}^+,\;\mu^+(*,-)\simeq i_n^+\;\mathrm{rel}\: *\text{ et }\mu^+(-,*)\simeq i_n^+\;\mathrm{rel}\: *\text{ (C)}$$
    induisent une homotopie entre $\eta\cdot\delta$ (action) et $\mu(\eta(1),\delta)=\mu(*,\delta)$. Or, $\mu(*,\delta)\simeq \delta\;\mathrm{rel}\: *$.
    Donc $[\eta\cdot\delta]=[\delta]$.
  \end{description}
\end{proof}

\subsection{\texorpdfstring{La $K$-théorie des corps finis}{La K-théorie des corps finis}}

\section{\texorpdfstring{Le théorème "$+=Q$"}{Le théorème "+=Q"}}

Cette section reprends la démonstration du théorème $+=Q$ donnée dans \cite{Gray}.



\section{\texorpdfstring{Propriétés de la $K$-théorie supérieure}{Propriétés de la K-théorie supérieure}}

\section{\texorpdfstring{La $K$-théorie des schémas}{La K-théorie des schémas}}

\appendix

\section{Catégories modèles}

\subsection{Argument du petit objet, notations}

Nous ne démontrons pas ici en détail l'argument du petit objet, voir \cite[Chp.1]{Goer} pour cela.
La raison d'être de cette sous-section est de fixer des notations.

Soit $C$ une catégorie cocomplète et $\mathcal{J}=\enstq{u_i:A_i\ra B_i}{i\in I}$ un ensemble de morphismes dans $C$ tels que
chaque $A_i$ soit $\N$-petit.

Soit $h:X\ra Y$ un morphisme dans $C$. Nous noterons $G^1(\mathcal{J},h)$ la colimite~:
\begin{center}
  \begin{tikzcd}[column sep = large, row sep = large]
    {\displaystyle\bigsqcup_{D}A_{i_{D}}} & X \\
    {\displaystyle\bigsqcup_{D}B_{i_{D}}} & {G^1(\mathcal{J},h)}
    \arrow["{u_{i_{D}}}", from=1-1, to=2-1]
    \arrow["{f_D}", from=1-1, to=1-2]
    \arrow["h_1", from=1-2, to=2-2]
    \arrow[from=2-1, to=2-2]
    \arrow["\mathlarger{\mathlarger{\mathlarger{\mathlarger{\lrcorner}}}}"{anchor=center, rotate=180, pos=0.05}, draw=none, from=2-2, to=1-1]
  \end{tikzcd}
\end{center}
où l'union porte sur l'ensemble des diagrammes commutatifs $D$ de la forme~:
\begin{center}
  \begin{tikzcd}[column sep = large, row sep = large]
    {A_{i_{D}}} & X \\
    {B_{i_{D}}} & Y
    \arrow["{u_{i_{D}}}", from=1-1, to=2-1]
    \arrow["{f_D}", from=1-1, to=1-2]
    \arrow["h", from=1-2, to=2-2]
    \arrow["g_D", from=2-1, to=2-2]
    \arrow["\mathlarger{\mathlarger{\mathlarger{\mathlarger{\lrcorner}}}}"{anchor=center, rotate=180, pos=0.05}, draw=none, from=2-2, to=1-1]
  \end{tikzcd}
\end{center}
Les $g_D$ induisent un morphisme $p_1:G^1(\mathcal{J},h)\ra Y$.

On pose alors successivement~:
\begin{description}
  \item[] $G^{n+1}(\mathcal{J},h)=G^1(\mathcal{J},h^n)$~;
  \item[] $h_{n+1}: X\ra G^{n+1}(\mathcal{J},h)$~;
  \item[] $p_{n+1}:G^{n+1}(\mathcal{J},h)\ra Y$.
\end{description}
On note~:
\begin{description}
  \item[] $G^{\infty}(\mathcal{J},h):=\colim_n G^n(\mathcal{J},h)$~;
  \item[] $h_\infty:X\ra G^{\infty}(\mathcal{J},h)$~;
  \item[] $p_\infty:G^{\infty}(\mathcal{J},h)\ra Y$.
\end{description}

On a alors $h=p_\infty\circ h_\infty$ avec $h_\infty$ dans $\mathcal{J}-\mathrm{Cell}$ et $p_\infty$ dans $\mathcal{J}^\square$.

Chaque $G^n(\mathcal{J},-)$ et $G^\infty(\mathcal{J},-)$ sont des foncteurs $C^{0\ra 1}\ra C$ et les $h_\infty$ et $p_\infty$
s'assemblent en des transformation naturelles $(p_0:C^{0\ra 1}\ra C)\Rightarrow G^\infty(\mathcal{J},-)$ et
$G^\infty(\mathcal{J},-)\Rightarrow (p_1:C^{0\ra 1}\ra C)$.


\subsection{Carrés homotopiquement cartésiens d'ensembles simpliciaux}
\label{carrehomotopiquementcart}

Le but de cette sous-section est d'introduire la notion de carré homotopiquement cartésien.
Seront admis les notions d'adjonction de Quillen, voir \cite[1.6]{Idri}.

Nous commençons cependant par des remarques valables dans des catégories modèles générales.
Puis nous traiterons le cas particulier de la catégorie des ensembles simpliciaux.

\begin{prop}[structure de Reedy]
  Soit $C$ une catégorie modèle. On note $I$ la catégorie à $3$ objets et $2$ morphismes~:
  \begin{center}
    \begin{tikzcd}
      & 1 \\
      2 & 0
      \arrow[from=1-2, to=2-2]
      \arrow[from=2-1, to=2-2]
    \end{tikzcd}
  \end{center}
  Alors il existe une structure de modèle sur $C^I$ telle que ~:
  \begin{description}
    \item[(W)] les équivalences faibles se vérifient objet par objet~;
    \item[(C)] les équivalences faibles se vérifient objet par objet~;
    \item[(F)] un morphisme $X\ra Y$ est une fibration si et seulement si $X(0)\ra Y(0)$, $X(1)\ra X(0)\times_{Y(0)} Y(1)$
    et $X(2)\ra X(0)\times_{Y(0)} Y(2)$ sont des fibrations dans $C$.
  \end{description}
  On appelle cette structure la structure de modèle de Reedy.
\end{prop}

La démonstration est omise, car assez facile. La définition est faite pour que ça marche. On a immédiatement le corollaire suivant.

\begin{coro}
  Soit $C$ une catégorie modèle et $I$ la catégorie définie comme ci-dessus. Alors la structure de modèle de Reedy sur $C^I$ induit une adjonction
  de Quillen~:
  \begin{center}
    \begin{tikzcd}
      {\mathrm{cst}_I:C} \ar[r,bend left,""{name=A, below}] & {C^I:\lim_I} \ar[l,bend left,""{name=B,above}] \ar[from=A, to=B, phantom,"\perp"]
    \end{tikzcd}
  \end{center}
  Et donc une adjonction entre les catégories homotopiques~:
  \begin{center}
    \begin{tikzcd}
      {\mathbb{L}\:\mathrm{cst}_I:C} \ar[r,bend left,""{name=A, below}] & {\Ho{C^I}:\mathbb{R}\lim_I} \ar[l,bend left,""{name=B,above}] \ar[from=A, to=B, phantom,"\perp"]
    \end{tikzcd}
  \end{center}
\end{coro}

On remarque que si $A$ est un objet de $C$, et $QA\ra A$ un remplacement cofibrant,
alors $\mathrm{cst}_IQA\ra\mathrm{cst}_I A$ est une équivalence. Or $\mathbb{L}\:\mathrm{cst}_I A\simeq\mathrm{cst}_IQA$.
Donc $\mathbb{L}\:\mathrm{cst}_I(A)\simeq \mathrm{cst}_IA$.
Ceci justifie la définition suivante.

\begin{defi}
  Soit $C$ une catégorie modèle et soit~:
  \begin{center}
    \begin{tikzcd}
      X   & Y_1 \\
      Y_2 & Y_0
      \arrow[from=1-2, to=2-2]
      \arrow[from=2-1, to=2-2]
      \arrow[from=1-1, to=2-1]
      \arrow[from=1-1, to=1-2]
    \end{tikzcd}
  \end{center}
  un carré commutatif dans $C$.
  On dit que ce carré est homotopiquement cartésien si le morphisme dans $\Ho{C}$~:
  $$X\ra \mathbb{R}\lim_I Y$$
  adjoint à $\mathbb{L}\:\mathrm{cst}_I(X)=\mathrm{cst}_I(X)\ra Y$, est un isomorphisme (dans $\Ho{C}$).
\end{defi}

\begin{prop}\label{carrecartesienbasique}
  Soit $C$ une catégorie modèle et soit~:
  \begin{center}
    \begin{tikzcd}
      X   & Y_1 \\
      Y_2 & Y_0
      \arrow[two heads, from=1-2, to=2-2]
      \arrow[two heads, from=2-1, to=2-2]
      \arrow[from=1-1, to=2-1]
      \arrow[from=1-1, to=1-2]
      \arrow["\mathlarger{\mathlarger{\mathlarger{\mathlarger{\lrcorner}}}}"{anchor=center, pos=0.05}, draw=none, from=1-1, to=2-2]
    \end{tikzcd}
  \end{center}
  un carré commutatif cartésien dans $C$ avec $Y_0$ fibrant, et $Y_1\twoheadrightarrow Y_0$, $Y_1\twoheadrightarrow Y_0$ des cofibrations.
  Alors ce carré est homotopiquement cartésien.
\end{prop}

\begin{proof}
  Dans ce cas, le diagramme $Y$ est fibrant dans $C^I$, et donc $\lim_I Y\simeq \mathbb{R}\lim_I Y$.
\end{proof}

Nous allons maintenant nous intéresser à des catégories modèles particulières, dites propres à droite.

\begin{defi}
  Soit $C$ une catégorie de modèles. On dit que $C$ est propre à droite si le tiré en arrière d'une équivalence le long
  d'une fibration est une équivalence.
\end{defi}

\begin{prop}\label{criterehomocartesienpropreadroite}
  Soit $C$ une catégorie modèle propre à droite. Alors tout carré commutatifs cartésien~:
  \begin{center}
    \begin{tikzcd}
      X   & Y_1 \\
      Y_2 & Y_0
      \arrow[two heads, from=1-2, to=2-2]
      \arrow[from=2-1, to=2-2]
      \arrow[from=1-1, to=2-1]
      \arrow[from=1-1, to=1-2]
      \arrow["\mathlarger{\mathlarger{\mathlarger{\mathlarger{\lrcorner}}}}"{anchor=center, pos=0.05}, draw=none, from=1-1, to=2-2]
    \end{tikzcd}
  \end{center}
  avec $Y_1\ra Y_0$ fibrant, est homotopiquement cartésien.
\end{prop}

\begin{proof}
  On peut choisir un remplacement fibrant $RY_0$ de $Y_0$ puis factoriser $Y_1\twoheadrightarrow Y_0\overset{\sim}{\ra} RY_0$
  en $Y_1\overset{\sim}{\ra} RY_1\twoheadrightarrow RY_0$. On a alors le diagramme suivant~:
  \begin{center}
    \begin{tikzcd}
      X && {Y_1} \\
      & {\tilde{X}} && {\tilde{Y}} & {RY_1} \\
      {Y_2} && {Y_0} \\
      &&&& {RY_0}
      \arrow[two heads, from=2-5, to=4-5]
      \arrow[two heads, from=1-3, to=3-3]
      \arrow["\sim" sloped, from=1-3, to=2-5]
      \arrow["\sim" sloped, from=3-3, to=4-5]
      \arrow[from=3-1, to=3-3]
      \arrow[two heads, from=1-1, to=3-1]
      \arrow[two heads, from=2-4, to=3-3]
      \arrow["\sim" sloped, from=2-4, to=2-5]
      \arrow["\mathlarger{\mathlarger{\mathlarger{\mathlarger{\lrcorner}}}}"{anchor=center, pos=0.05}, draw=none, from=2-4, to=4-5]
      \arrow[two heads, from=2-2, to=3-1]
      \arrow["\sim" sloped, from=1-3, to=2-4]
      \arrow[from=1-1, to=2-2]
      \arrow[from=1-1, to=1-3]
      \arrow[crossing over, from=2-2, to=2-4]
      \arrow["\mathlarger{\mathlarger{\mathlarger{\mathlarger{\lrcorner}}}}"{anchor=center, pos=0.05}, draw=none, from=2-2, to=3-3]
    \end{tikzcd}
  \end{center}
  où $\tilde{Y}$ et $\tilde{X}$ complètent les carrés cartésiens notés par $\lrcorner$. Le morphisme $\tilde{Y}\ra RY_1$
  est acyclique car $C$ est propre à droite.
  Or, dans la catégorie modèle restreinte $C\downarrow Y_0$, par le lemme de Brown (\cite[1.6.6]{Idri}), $Y_2\times_{Y_0}-$ préserve les équivalences faibles
  entre objets fibrants. Donc, comme $Y_1\ra \tilde{Y}$ est une telle équivalence, $X\ra \tilde{X}$ aussi.

  Nous nous sommes donc ramené au cas où $X=\tilde{X}$, c'est à dire au cas où le carré~:
  \begin{center}
    \begin{tikzcd}
      {Y_1} & {RY_1} \\
      {Y_0} & {RY_0}
      \arrow[two heads, from=1-1, to=2-1]
      \arrow[two heads, from=1-2, to=2-2]
      \arrow[from=1-1, to=1-2]
      \arrow[from=2-1, to=2-2]
    \end{tikzcd}
  \end{center}
  est cartésien.

  Mais dans ce cas, on a le diagramme commutatif à carrés cartésiens suivant~:
  \begin{center}
    \begin{tikzcd}
      X & {Y_2} \\
      {X'} & {RY_2} \\
      {RY_1} & {RY_0}
      \arrow[from=2-1, to=3-1]
      \arrow[two heads, from=2-2, to=3-2]
      \arrow[two heads, from=3-1, to=3-2]
      \arrow[two heads, from=2-1, to=2-2]
      \arrow["\sim" sloped, from=1-2, to=2-2]
      \arrow[from=1-1, to=2-1]
      \arrow[two heads, from=1-1, to=1-2]
      \arrow["\mathlarger{\mathlarger{\mathlarger{\mathlarger{\lrcorner}}}}"{anchor=center, pos=0.05}, draw=none, from=1-1, to=2-2]
      \arrow["\mathlarger{\mathlarger{\mathlarger{\mathlarger{\lrcorner}}}}"{anchor=center, pos=0.05}, draw=none, from=2-1, to=3-2]
    \end{tikzcd}
  \end{center}
  où $Y_2\overset{\sim}{\ra} RY_2\twoheadrightarrow RY_0$ est une factorisation de $Y_2\ra Y_0\overset{\sim}{\ra} RY_0$.
  Comme les flèches horizontales sont des fibrations et $Y_2\ra RY_2$ est acyclique,
  $X\ra X'$ l'est aussi. Or, par la \sref{proposition}{carrecartesienbasique}, le carré du bas est cartésien, et comme les diagrammes
  $Y$ et $RY$ sont équivalents dans $C^I$, $X$ est la limite homotopique de $Y$.
\end{proof}

\begin{prop}\label{homocartesienfibrant}
  Soit $C$ une catégorie modèles et soit~:
  \begin{center}
    \begin{tikzcd}[column sep = small, row sep = small]
      X   && Y_1 \\
      \\
      Y_2 && Y_0 \\
       &  *   &
      \arrow[two heads,from=1-3, to=3-3]
      \arrow["\sim", from=3-1, to=3-3]
      \arrow[from=1-1, to=3-1]
      \arrow[from=1-1, to=1-3]
      \arrow[two heads, from=3-1, to=4-2]
      \arrow[two heads, from=3-3, to=4-2]
      \arrow["\mathlarger{\mathlarger{\mathlarger{\mathlarger{\lrcorner}}}}"{anchor=center, pos=0.05}, draw=none, from=1-1, to=3-3]
    \end{tikzcd}
  \end{center}
  un carré cartésien dans $C$ avec $Y_1\ra Y_0$ une fibration, $Y_2\ra Y_0$ acyclique, et $Y_0$, $Y_2$ fibrants.

  Alors $X\ra Y_1$ est aussi acyclique.

  En d'autres termes, le tiré en arrière d'une équivalence entre objets fibrants le long d'une fibration est une équivalence.
\end{prop}

\begin{rem}
  Il existe évidemment une version duale (prendre $C\op$)~: le poussé en avant d'une équivalence entre objets cofibrants le long
  d'une cofibration est une équivalence.
\end{rem}

La preuve de cette proposition est omise, voir \cite[13.1.2]{Hirs}.

\begin{prop}
  La catégorie des ensembles simpliciaux $\DEns$ est propre à droite.
\end{prop}

\begin{proof}
  La catégorie $\Top$ des espaces topologiques avec la structure de Quillen est propre à droite.
  En effet, comme tous les objets de $\Top$ sont fibrants, c'est une conséquence de la \sref{porposition}{homocartesienfibrant}.

  Or le foncteur réalisation $|\bullet|:\DEns\ra \Top$ préserve les limites finies et les fibrations, et reflète les équivalences
  (voir \cite[Chp.1]{Goer}).
  Donc on en déduit que $\DEns$ est également propre à droite.
\end{proof}

Voici une caractérisation plus concrète des diagrammes homotopiquement cartésiens dans $\DEns$.

\begin{prop}
  On se donne un carré commutatif~:
  \begin{center}
    \begin{tikzcd}
      X   & Y_1 \\
      Y_2 & Y_0
      \arrow[from=1-2, to=2-2]
      \arrow[from=2-1, to=2-2]
      \arrow[from=1-1, to=2-1]
      \arrow[from=1-1, to=1-2]
    \end{tikzcd}
  \end{center}
  Alors ce carré est homotopiquement cartésien si et seulement si le morphisme induit~:
  $$X\ra Y_1\times_{Y_0}Y_0^{\Delta^1}\times_{Y_0} Y_2$$
  est une équivalence faible.
\end{prop}

\begin{proof}
  On dispose d'une factorisation~:
  $$Y_1\overset{\sim}{\hookrightarrow} Y_1\times_{Y_0}Y_0^{\Delta^1} \overset{\mathrm{ev}_1}{\twoheadrightarrow} Y_0$$
  On a donc le diagramme suivant~:
  \begin{center}
    \begin{tikzcd}
      X && {Y_1} \\
      {Y_1\times_{Y_0}Y_0^{\Delta^1}\times_{Y_0}Y_2} && {Y_1\times_{Y_0}Y_0^{\Delta^1}} \\
      {Y_2} && {Y_0}
      \arrow[from=3-1, to=3-3]
      \arrow["\sim" sloped, hook, from=1-3, to=2-3]
      \arrow[two heads, from=2-3, to=3-3]
      \arrow[from=1-1, to=1-3]
      \arrow[from=1-1, to=2-1]
      \arrow[from=2-1, to=3-1]
      \arrow[from=2-1, to=2-3]
      \arrow["\mathlarger{\mathlarger{\mathlarger{\mathlarger{\lrcorner}}}}"{anchor=center, pos=0.05}, draw=none, from=2-1, to=3-3]
    \end{tikzcd}
  \end{center}
  Le carré du bas est cartésien et homotopiquement cartésien par la \sref{proposition}{criterehomocartesienpropreadroite}.
  On note $Y'$ l'élément de $C^I$ associé au carré du bas.
  Le morphisme $X\ra Y_1\times_{Y_0}Y_0^{\Delta^1}\times_{Y_0}Y_2$ dans $C$ induit dans $\Ho{C^I}$ le diagramme commutatif~:
  \begin{center}
    \begin{tikzcd}
      \mathbb{L}\:\mathrm{cst}_I X & \mathbb{L}\:\mathrm{cst}_I Y_1\times_{Y_0}Y_0^{\Delta^1}\times_{Y_0}Y_2 \\
      Y                        & Y'
      \arrow[from=1-2, to=2-2]
      \arrow["\sim", from=2-1, to=2-2]
      \arrow[from=1-1, to=2-1]
      \arrow[from=1-1, to=1-2]
    \end{tikzcd}
  \end{center}
  et, en passant à l'adjoint $\mathbb{R}\lim_I$, le diagramme commutatif dans $\Ho{C}$~:
  \begin{center}
    \begin{tikzcd}
      X                  & Y_1\times_{Y_0}Y_0^{\Delta^1}\times_{Y_0}Y_2 \\
      \mathbb{R}\lim_I Y & \mathbb{R}\lim_I Y'
      \arrow["\sim" sloped, from=1-2, to=2-2]
      \arrow["\sim", from=2-1, to=2-2]
      \arrow[from=1-1, to=2-1]
      \arrow[from=1-1, to=1-2]
    \end{tikzcd}
  \end{center}
  On a donc bien $X\simeq \mathbb{R}\lim_I Y$ si et seulement si $X\simeq Y_1\times_{Y_0}Y_0^{\Delta^1}\times_{Y_0}Y_2$.
\end{proof}

\subsection{Le lemme de collage}
On se place ici dans une catégorie de modèle $C$.

\begin{lem}[lemme de collage]\label{lemmedecollage}
  On se donne le diagramme suivant dans $C$~:
  \begin{center}
    \begin{tikzcd}
      {A_1} && {B_1} \\
      & {C_1} && {D_1} \\
      {A_2} && {B_2} \\
      & {C_2} && {D_2}
      \arrow["{f_A}"', from=1-1, to=3-1]
      \arrow["{f_B}", from=1-3, to=3-3]
      \arrow["{j_1}", from=1-1, to=1-3]
      \arrow["{j_2}"{pos=0.3}, from=3-1, to=3-3]
      \arrow[hook, from=1-3, to=2-4]
      \arrow[hook, from=3-3, to=4-4]
      \arrow["{i_2}", hook, from=3-1, to=4-2]
      \arrow["{i_1}", hook, from=1-1, to=2-2]
      \arrow["{f_D}", from=2-4, to=4-4]
      \arrow[from=4-2, to=4-4]
      \arrow[crossing over, from=2-2, to=2-4]
      \arrow[crossing over, "{f_C}"{pos= 0.2}, from=2-2, to=4-2]
      \arrow["\mathlarger{\mathlarger{\mathlarger{\mathlarger{\lrcorner}}}}"{anchor=center, pos=0.1, rotate=200}, draw=none, from=2-4, to=1-1]
      \arrow["\mathlarger{\mathlarger{\mathlarger{\mathlarger{\lrcorner}}}}"{anchor=center, pos=0.1, rotate=200}, draw=none, from=4-4, to=3-1]
    \end{tikzcd}
  \end{center}
  où tous les objets sont cofibrants, $i_1$ et $i_2$ sont des cofibrations, les faces supérieure et inférieure sont cocartésiennes
  et $f_A$, $f_B$ et $f_C$ sont acycliques.

  Alors $f_D$ est acyclique.
\end{lem}

\begin{proof}
  On factorise de façon fonctorielle $j_1=q_1\circ j'_1$ et $j_2=q_2\circ j'_2$ où $j'_1$ et $j'_2$ sont des cofibrations
  et $q_1$ et $q_2$ sont des fibrations acycliques. En coupant le cube en deux, On est alors ramené aux deux cas particuliers suivants~:
  $j_1$ et $j_2$ sont acycliques~;
  $j_1$ et $j_2$ sont des cofibrations.

  Le premier cas est une conséquence de la \sref{proposition}{homocartesienfibrant}. Nous allons maintenant traiter le second cas.
  On note $B'$ (respectivement $D'$) le poussé en avant de $j_1$ et $f_A$ (respectivement $C_1\ra C_2$ et $C_1\ra D_1$).
  On a alors le diagramme suivant~:
  \begin{center}
    \begin{tikzcd}
      {A_1} &&& {B_1} \\
      & {C_1} &&& {D_1} \\
      &&& {B'} \\
      {A_2} &&& {B_2} & {D'} \\
      & {C_2} &&& {D_2}
      \arrow["{i_1}", hook, from=1-1, to=2-2]
      \arrow[hook, from=1-4, to=2-5]
      \arrow["\sim"' sloped, from=1-1, to=4-1]
      \arrow["\sim\;\;\;\;\;\;\;\;\;" sloped, from=1-4, to=3-4]
      \arrow["{n_B}"', "\sim" sloped, from=3-4, to=4-4]
      \arrow[from=2-5, to=4-5]
      \arrow["{n_D}", from=4-5, to=5-5]
      \arrow["\theta", from=3-4, to=4-5]
      \arrow[hook, from=4-4, to=5-5]
      \arrow["{i_2}", hook, from=4-1, to=5-2]
      \arrow[hook, from=5-2, to=5-5]
      \arrow["{j_2}", hook, from=4-1, to=4-4]
      \arrow["{j_1}", hook, from=1-1, to=1-4]
      \arrow[from=4-1, to=3-4]
      \arrow[crossing over, from=5-2, to=4-5]
      \arrow[crossing over, hook, from=2-2, to=2-5]
      \arrow[crossing over, "\sim\;\;\;\;\;\;\;\;\;" sloped, from=2-2, to=5-2]
    \end{tikzcd}
  \end{center}
  Par la \sref{proposition}{homocartesienfibrant}, $B_1\ra B'$ et $D_1\ra D'$ sont des équivalences faibles, donc $n_B$ est acyclique.
  Il reste à montrer que $n_D$ est aussi acyclique.
  Or $\theta$ est le poussé en avant de $i_2$ le long de $A_2\ra B'$, c'est donc une cofibration. Comme $n_D$ est le poussé en avant de $n_B$ le long
  de $\theta$, par la \sref{proposition}{homocartesienfibrant}, $n_D$ est acyclique.
\end{proof}


\section{Catégories exactes}

\section{Homologie et suites spectrales}

\subsection{La correspondance de Dold-Kan}
\label{DoldKan}

Cette sous-section est consacrée à la correspondance de Dold-Kan et a certaines conséquences en homotopie.
Certaines démonstrations sont omises. Des références sont données.

\begin{defi}
  On note $\Chp$ la catégorie des complexes de groupes abéliens concentrés en degré positifs pour la notation homologique.
  On note $\DAb$ la catégorie des groupes abéliens simpliciaux, ie. des foncteurs $\DCat\ra \Ab$.
\end{defi}

L'objet de la correspondance de Dold-Kan est d'établir une équivalence de catégories abéliennes entre $\Chp$ et $\DAb$.

\begin{rem}
  On peut également voir $\DAb$ comme la catégorie des groupes abéliens dans la catégorie des ensembles simpliciaux $\DEns$.
\end{rem}

\begin{propdefi}\label{definitioncomplexenormalise}
  Soit $A$ un objet de $\DAb$. Le complexe normalisé $(NA_*,\partial_*)$ de $A$ est le complexe de chaîne de $\Chp$ suivant~:
  $$NA_n:=\bigcap_{m=0^{n-1}}\myker{d_m}\;\partial_n=(-1)^nd_n:NA_n\ra NA_{n-1}$$
  C'est un sous-complexe du complexe $CA_*,\partial_*$ définit par~:
  $$CA_n:=A_n;\partial_n=\sum_{m=0}^n(-1)^md_m:CA_n\ra CA_{n-1}$$
  Le complexe des dégénérescences $DA_*$ de $A$ est le sous complexe de $CA$ définit par~:
  $$DA_n:=\sum_{m=0}^{n}s_m(A_{n+1})$$
\end{propdefi}

\begin{proof}
  Le calcul pour vérifier que $CA_*$ est un complexe est classique. Vérifions que $NA_*$ et $DA_*$ sont des sous-complexes.
  Pour $0\leq i\leq n-2$, on a $d_id_n=d_{n-1}d_i$ annule $NA_n$. Donc $NA_*$ est un sous-complexe.
  Pour $0\leq i\leq n$~:
  $$\sum_{m=0}(-1)^md_ms_i=\sum_{m=0}^{i-1}s_{i-1}d_m+(-1)^{i}\id+(-1)^{i+1}\id+\sum_{m=i+2}^{n}s_{i}d_{m+1}$$
  Donc $\partial_ns_i(a)\in DA_{n-1}$ pour $a\in DA_n$.
\end{proof}

\begin{rem}
  Ces constructions étant fonctorielles, on a donc construit $3$ foncteurs $N$, $C$ et $D$ de $\DAb$ dans $\Chp$.
\end{rem}

\begin{lem}\label{lemmeDoldKan}
  L'inclusion naturelle $N\oplus D \ra C$ est un isomorphisme naturel. Donc les foncteurs $N$ et $C/D$ coïncident.
  En particulier, le foncteur $N$ est exact.
\end{lem}

Nous admettons ici ce lemme, qui se démontre par récurrence. Pour plus de détails, voir \cite[III.2.1]{Goer}

\begin{rem}
  On voit ici deux façons de penser au complexe normalisé $NA$~: la définition avec les noyaux donnée dans la \sref{définition}{definitioncomplexenormalise},
  ou le quotient $CA/DA$.
\end{rem}

\begin{propdefi}
  Soit $C$ un complexe de chaîne dans $\Chp$. On note $\Gamma C$ le groupe abélien simplicial définit par~:
  $$\Gamma C_n:=\bigoplus_{\phi:[n]\twoheadrightarrow [r]}C_r$$
  Pour $\chi:[m]\ra [n]$, le morphisme $\chi^*:\Gamma C_n \ra \Gamma C_m$ est construit comme suit. Soit $\phi:[n]\twoheadrightarrow [r]$.
  Alors $\phi\circ \chi$ se factorise uniquement en~:
  \begin{center}
    \begin{tikzcd}
      {[m]} \arrow[r,"\psi", two heads] & {[s]} \arrow[r, "\mu", hook] & {[r]}
    \end{tikzcd}
  \end{center}
  Alors $\chi_*$ est donné sur la coordonnée en $\phi$ par le morphisme $C_r\overset{\mu^*}{\ra} C_s\overset{\psi}{\hookrightarrow}\Gamma C_m$,
  où $\mu^*=0$ sauf si $\mu=d^r$, auquel cas $\mu^*=(-1)^r\partial_r$.
\end{propdefi}

\begin{rem}
  la définition ci-dessus est plus clair si on voit $C$ comme un ensemble pré-simplicial. Un ensemble pré-simplicial est un préfaisceau
  sur la catégorie $\DCat'$ des injections de $\DCat$. La structure d'ensemble pré-simplicial est celle décrite dans la définition.
  La construction de $\Gamma$ est alors basée sur celle, classique, de l'adjoint à gauche de l'oubli $[\DCat,\Ab]\ra [\DCat',\Ab]$.
  C'est ce qui est fait dans la démonstration ci-dessous.
\end{rem}

\begin{proof}
  Au vu de la remarque ci-dessus, il suffit de montrer que si $C$ est un ensemble pré-simplicial, alors la construction de $\Gamma C$ est correcte.
  Il s'agit de vérifier la compatibilité de la construction avec les compositions dans $\DCat$.
  Soit $\chi:[m]\ra [n]$ et $\rho:[l]\ra [m]$. Pour $\phi:[n]\twoheadrightarrow [r]$, on a le diagramme commutatif suivant dans $\DCat$~:
  \begin{center}
    % https://q.uiver.app/?q=WzAsNixbMCwwLCJbbF0iXSxbMSwwLCJbbV0iXSxbMiwwLCJbbl0iXSxbMiwxLCJbcl0iXSxbMSwxLCJbc10iXSxbMCwxLCJbdF0iXSxbMCwxLCJcXHJobyJdLFsxLDIsIlxcY2hpIl0sWzIsMywiXFxwaGkiLDAseyJzdHlsZSI6eyJoZWFkIjp7Im5hbWUiOiJlcGkifX19XSxbMSw0LCJcXHBzaSIsMCx7InN0eWxlIjp7ImhlYWQiOnsibmFtZSI6ImVwaSJ9fX1dLFswLDUsIlxcdGhldGEiLDAseyJzdHlsZSI6eyJoZWFkIjp7Im5hbWUiOiJlcGkifX19XSxbNCwzLCJcXG11IiwyLHsic3R5bGUiOnsidGFpbCI6eyJuYW1lIjoiaG9vayIsInNpZGUiOiJ0b3AifX19XSxbNSw0LCJcXG51IiwyLHsic3R5bGUiOnsidGFpbCI6eyJuYW1lIjoiaG9vayIsInNpZGUiOiJ0b3AifX19XV0=
    \begin{tikzcd}
    	{[l]} & {[m]} & {[n]} \\
    	{[t]} & {[s]} & {[r]}
    	\arrow["\rho", from=1-1, to=1-2]
    	\arrow["\chi", from=1-2, to=1-3]
    	\arrow["\phi", two heads, from=1-3, to=2-3]
    	\arrow["\psi", two heads, from=1-2, to=2-2]
    	\arrow["\theta", two heads, from=1-1, to=2-1]
    	\arrow["\mu"', hook, from=2-2, to=2-3]
    	\arrow["\nu"', hook, from=2-1, to=2-2]
    \end{tikzcd}
  \end{center}
  On a donc que sur la coordonnée en $\phi$, $\rho^*\chi^*$ est donné par $\nu^*\mu^*$ et $(\chi\rho)^*$ est donné par $(\mu\nu)^*$.
  Or $\nu^*\mu^*=(\mu\nu)^*$ car $C$ est pré-simplicial.
\end{proof}

\begin{theo}[équivalence de Dold-Kan]
  Les foncteurs $\Gamma:\Chp\ra\DAb$ et $N:\DAb\ra\Chp$ sont quasi-inverses et exactes. Ils induisent une équivalence de catégories
  abéliennes $\Chp\simeq\DAb$, appelée équivalence de Dold-Kan.
\end{theo}

\begin{proof}
  Soit $C$ un complexe dans $\Chp$. On cherche à identifier $(N\Gamma C)_n$ comme sous-groupe de~:
  $$\Gamma C_n=\bigoplus_{\phi:[n]\twoheadrightarrow [r]}C_r$$
  Or, un calcul direct montre que $C_n\subseteq (N\Gamma C)_n$ et $\bigoplus_{[n]\twoheadrightarrow [r],r<n}C_r\subseteq (D\Gamma C)_n$.
  Donc, par le \sref{lemme}{lemmeDoldKan}, $C_n=(N\Gamma C)_n$. On a l'isomorphisme souhaité.
  Soit $A$ un groupe abélien simplicial. On a un morphisme naturel de groupes pré-simpliciaux $N A\ra A$,
  et donc un morphisme naturel d'ensembles simpliciaux $\Gamma N A\ra A$. Soit $n\geq 0$. Par récurrence, on suppose que
  $(\Gamma N A)_m\ra A_m$ est un isomorphisme pour $m<n$. Alors l'image de $(\Gamma N A)_n\ra A_n$ contient $NA_n$
  et $DA_n$. Donc par le \sref{lemme}{lemmeDoldKan}, l'application est surjective. Maintenant, soit
  $x=\sum_{\phi} a_{\phi}$ dans le noyau. Soit $r$ maximal tel qu'il existe $\phi:[n]\twoheadrightarrow [r]$ tel que $a_\phi \neq 0$.
  Soit $\phi:[n]\twoheadrightarrow [r]$ maximal pour l'ordre lexicographique tel que $a_\phi \neq 0$. Soit $\mu:[r]\hookrightarrow [n]$
  une section de $\phi$, minimale pour l'ordre lexicographique. Alors $\psi\circ \mu$ n'est pas injectif pour tout
  $\psi:[n]\twoheadrightarrow [r]$ différent de $\phi$ tel que $a_\psi\neq 0$. Donc tous les termes de $\mu^*x$ sont de degré $\leq r$ et le seul terme de degré $r$
  est $\mu^*a_\phi=a_\phi\in NA_r\overset{\id_{[r]}}{\hookrightarrow}(\Gamma N A)_r$. Or, $N A_r\hookrightarrow A_r$. Donc $a_\phi=0$.
  On a donc nécessairement $x=0$. Donc $(\Gamma N A)_n\ra A_n$ est un isomorphisme.
\end{proof}

Nous allons maintenant nous intéresser à une structure de catégorie modèle sur $\DAb$, et la correspondante dans $\Chp$.
Introduisons d'abords les équivalences faibles.

\begin{lem}
  Soit $A$ un groupe abélien simplicial. Alors $DA$ est un complexe acyclique, ie. $H_*(DA)=0$.
  En particulier, $NA\hookrightarrow CA$ et $A\twoheadrightarrow CA/DA$ sont des isomorphismes en homologie.
\end{lem}

Nous admettons ici le lemme, il s'agit de construire par récurrence une homotopie entre $\id_{DA}$ et $0$.
Une autre approche équivalente est donnée dans \cite[II.2.4]{Goer}.

\begin{prop}
  Soit $n\geq 0$, il existe un isomorphisme naturel de groupes pour $A$ dans $\DAb$~:
  $$\pi_n(A,0)\simeq H_n(NA)$$
\end{prop}

\begin{proof}
  On remarque tout d'abord que l'addition sur $A$ induit, par Eckmann–Hilton, la structure de groupe sur $\pi_n(A,0)$.
  Alors, les simplexes $\sigma\in A_n$ qui représentent les éléments de $\pi_n(A,0)$ sont les éléments de $Z_n(NA)$.
  Et un simplexe $\sigma$ représente $0$ si et seulement si il existe $\tau\in NA_{n+1}$ tel que $d_n\tau = \sigma$.
  On a donc bien l'isomorphisme naturel souhaité.
\end{proof}

Introduisons maintenant la structure de modèle sur $\DAb$.

\begin{prop}
  Il existe une structure de catégorie modèle sur $\DAb$ telle que~:
  \begin{description}
    \item[(W)] les équivalences faibles soient les applications qui induisent des isomorphismes sur tous les groupes d'homotopie~;
    \item[(F)] les fibrations soient les applications dont l'application sous-jacente entre ensembles simpliciaux soit une fibrations de Kan. 
  \end{description}
  Cette structure est cofibrement engendré par les cofibrations~:
  $$\mathcal{I}:=\enstq{\Z\partial\Delta^n\ra \Z\Delta^n}{n\geq 0}$$
  et les cofibrations acycliques~:
  $$\mathcal{I}:=\enstq{\Z\Lambda_k^n\ra \Z\Delta^n}{n\geq 0,0\leq k\leq n}$$.
\end{prop}

\begin{proof}
  (à faire)
\end{proof}

\begin{prop}
  Soit $f:A\ra B$ une application dans $\DAb$. ALors~:
  \begin{description}
    \item[$(1)$] Si $f$ est surjective, alors $f$ est un fibration~;
    \item[$(2)$] $f$ est surjective si et seulement si $Nf:NA\ra NB$ est surjective~;
    \item[$(3)$] $f$ est un cofibration si et seulement si $Nf_n:NA_n\ra NB_n$ est surjective pour tout $n\geq 1$.  
  \end{description}
\end{prop}

\begin{proof}
  (à faire)
\end{proof}

\subsection{Homologie et homologie à coefficients}

\subsection{Le morphisme d'Eilenberg-Zilber}

\subsection{Suite spectrale d'une filtration}

\section{Topologie algébrique}
\label{annexetopologie}

Cette section couvre les notions de topologie utilisées dans le mémoire. Certains points seront admis.

\subsection{Le théorème de Van Kampen}

\begin{theo}[Van Kampen]\label{theoremedeVanKampen}
  On se donne le carré cocartésien suivant dans $\DEns$~:
  \begin{center}
    % https://q.uiver.app/?q=WzAsNCxbMCwwLCJBIl0sWzAsMSwiQiJdLFsxLDAsIlgiXSxbMSwxLCJZIl0sWzAsMSwiaSIsMCx7InN0eWxlIjp7InRhaWwiOnsibmFtZSI6Imhvb2siLCJzaWRlIjoidG9wIn19fV0sWzAsMiwiaiJdLFsyLDMsIiIsMCx7InN0eWxlIjp7InRhaWwiOnsibmFtZSI6Imhvb2siLCJzaWRlIjoidG9wIn19fV0sWzEsM10sWzMsMCwiIiwxLHsic3R5bGUiOnsibmFtZSI6ImNvcm5lciJ9fV1d
    \begin{tikzcd}
      A & X \\
      B & Y
      \arrow["i", hook, from=1-1, to=2-1]
      \arrow["j", from=1-1, to=1-2]
      \arrow[hook, from=1-2, to=2-2]
      \arrow[from=2-1, to=2-2]
      \arrow["\mathlarger{\mathlarger{\mathlarger{\mathlarger{\lrcorner}}}}"{anchor=center, pos=0.1, rotate=180}, draw=none, from=2-2, to=1-1]
    \end{tikzcd}
  \end{center}
  où $i$ est une cofibration. Alors le carré induit sur les groupoïdes fondamentaux est cocartésien (dans la catégorie des petits groupoïdes)~:
  \begin{center}
    \begin{tikzcd}
      \pi_{\leq 1}A & \pi_{\leq 1}X \\
      \pi_{\leq 1}B & \pi_{\leq 1}Y
      \arrow["i_*", from=1-1, to=2-1]
      \arrow["j_*", from=1-1, to=1-2]
      \arrow[from=1-2, to=2-2]
      \arrow[from=2-1, to=2-2]
      \arrow["\mathlarger{\mathlarger{\mathlarger{\mathlarger{\lrcorner}}}}"{anchor=center, pos=0.1, rotate=180}, draw=none, from=2-2, to=1-1]
    \end{tikzcd}
  \end{center}
  Si de plus $A$, $B$ et $X$ sont connexes, alors pour tout $x\in A_0$, le carré~:
  \begin{center}
    \begin{tikzcd}
      \pi_1(A,x) & \pi_1(X,x) \\
      \pi_1(B,x) & \pi_1(Y,x)
      \arrow["i_*", from=1-1, to=2-1]
      \arrow["j_*", from=1-1, to=1-2]
      \arrow[from=1-2, to=2-2]
      \arrow[from=2-1, to=2-2]
      \arrow["\mathlarger{\mathlarger{\mathlarger{\mathlarger{\lrcorner}}}}"{anchor=center, pos=0.1, rotate=180}, draw=none, from=2-2, to=1-1]
    \end{tikzcd}
  \end{center}
  est cocartésien dans la catégorie des groupes.
\end{theo}

\begin{proof}
  Pour le premier point, on remarque que les foncteurs $\tau_{\leq 1}:\DEns\ra \Cat$ et $\Cat\ra \mathrm{Groupoides}$ on des adjoints à gauche
  (respectivement le nerf $N$ et l'oubli). Donc leur composition, $\pi_{\leq 1}:\DEns\ra \mathrm{Groupoides}$, commute aux colimites.
  Pour le second point, par le \sref{lemme de collage}{lemmedecollage}, on peut supposer que $j$ est également une cofibration. Alors,
  en choisissant pour chaque $y\in A_0$ (respectivement $B_0\setminus A_0$, respectivement $X_0\setminus A_0$) un chemin de $y$ à $x$,
  $\gamma_y\in\Hom{\pi_{\leq 1}A}{y}{x}$ (respectivement $\Hom{\pi_{\leq 1}B}{y}{x}$, respectivement $\Hom{\pi_{\leq 1}X}{y}{x}$),
  on obtient une rétraction $r$ entre les carrés suivants~:
  \begin{center}
    % https://q.uiver.app/?q=WzAsOCxbMSwxLCJcXHBpX3tcXGxlcSAxfUEiXSxbMSwyLCJcXHBpX3tcXGxlcSAxfUIiXSxbMiwxLCJcXHBpX3tcXGxlcSAxfVgiXSxbMiwyLCJcXHBpX3tcXGxlcSAxfVkiXSxbMCwwLCJcXHBpXzEoQSx4KSJdLFszLDAsIlxccGlfMShYLHgpIl0sWzAsMywiXFxwaV8xKEIseCkiXSxbMywzLCJcXHBpXzEoWSx4KSJdLFswLDEsImlfKiJdLFswLDIsImpfKiJdLFsyLDMsIiIsMCx7InN0eWxlIjp7InRhaWwiOnsibmFtZSI6Imhvb2siLCJzaWRlIjoidG9wIn19fV0sWzEsM10sWzMsMCwiIiwxLHsic3R5bGUiOnsibmFtZSI6ImNvcm5lciJ9fV0sWzQsNV0sWzQsNl0sWzUsN10sWzYsN10sWzEsNiwiciIsMCx7Im9mZnNldCI6LTF9XSxbMCw0LCJyIiwwLHsib2Zmc2V0IjotMX1dLFsyLDUsInIiLDAseyJvZmZzZXQiOi0xfV0sWzMsNywiciIsMCx7Im9mZnNldCI6LTF9XSxbNCwwLCIiLDIseyJvZmZzZXQiOi0xLCJzdHlsZSI6eyJ0YWlsIjp7Im5hbWUiOiJob29rIiwic2lkZSI6InRvcCJ9fX1dLFs1LDIsIiIsMSx7Im9mZnNldCI6LTEsInN0eWxlIjp7InRhaWwiOnsibmFtZSI6Imhvb2siLCJzaWRlIjoidG9wIn19fV0sWzYsMSwiIiwwLHsib2Zmc2V0IjotMSwic3R5bGUiOnsidGFpbCI6eyJuYW1lIjoiaG9vayIsInNpZGUiOiJ0b3AifX19XSxbNywzLCIiLDEseyJvZmZzZXQiOi0xLCJzdHlsZSI6eyJ0YWlsIjp7Im5hbWUiOiJob29rIiwic2lkZSI6InRvcCJ9fX1dXQ==
    \begin{tikzcd}
    	{\pi_1(A,x)} &&& {\pi_1(X,x)} \\
    	& {\pi_{\leq 1}A} & {\pi_{\leq 1}X} \\
    	& {\pi_{\leq 1}B} & {\pi_{\leq 1}Y} \\
    	{\pi_1(B,x)} &&& {\pi_1(Y,x)}
    	\arrow["{i_*}", from=2-2, to=3-2]
    	\arrow["{j_*}", from=2-2, to=2-3]
    	\arrow[from=2-3, to=3-3]
    	\arrow[from=3-2, to=3-3]
    	\arrow["\mathlarger{\mathlarger{\mathlarger{\mathlarger{\lrcorner}}}}"{anchor=center, pos=0.1, rotate=180}, draw=none, from=3-3, to=2-2]
    	\arrow[from=1-1, to=1-4]
    	\arrow[from=1-1, to=4-1]
    	\arrow[from=1-4, to=4-4]
    	\arrow[from=4-1, to=4-4]
    	\arrow["r", shift left=1, from=3-2, to=4-1]
    	\arrow["r", shift left=1, from=2-2, to=1-1]
    	\arrow["r", shift left=1, from=2-3, to=1-4]
    	\arrow["r", shift left=1, from=3-3, to=4-4]
    	\arrow[shift left=1, hook, from=1-1, to=2-2]
    	\arrow[shift left=1, hook, from=1-4, to=2-3]
    	\arrow[shift left=1, hook, from=4-1, to=3-2]
    	\arrow[shift left=1, hook, from=4-4, to=3-3]
    \end{tikzcd}
  \end{center}
  Or, les rétractions préservent les colimites. Donc le carré des $\pi_1(-,x)$ est cocartésien.
\end{proof}

\subsection{Obstruction et tours de Postnikov}

\begin{lem}[Lemme d'extension]\label{lemmedextension}
  Soit $A$ un ensemble simplicial, $a\in A$, $n\geq 1$ et $(\gamma_i)_{i\in I}$ une famille de chemins $\gamma_i:(\Delta^n,\partial\Delta^n)\ra (A,a)$.
  On se donne le carré cocartésien~:
  \begin{center}
    % https://q.uiver.app/?q=WzAsNCxbMCwwLCJcXGJpZ3NxY3VwX0lcXHBhcnRpYWxcXERlbHRhXntuKzF9Il0sWzAsMSwiXFxiaWdzcWN1cF9JXFxEZWx0YV57bisxfSJdLFsxLDAsIkEiXSxbMSwxLCJYIl0sWzMsMCwiIiwwLHsic3R5bGUiOnsibmFtZSI6ImNvcm5lciJ9fV0sWzAsMiwiXFxzcWN1cF9pKGEsXFxkb3RzYyxhLFxcZ2FtbWFfaSkiXSxbMiwzLCJqIiwwLHsic3R5bGUiOnsidGFpbCI6eyJuYW1lIjoiaG9vayIsInNpZGUiOiJ0b3AifX19XSxbMCwxLCIiLDAseyJzdHlsZSI6eyJ0YWlsIjp7Im5hbWUiOiJob29rIiwic2lkZSI6InRvcCJ9fX1dLFsxLDNdXQ==
    \begin{tikzcd}[column sep = 8ex]
    	{\bigsqcup_I\partial\Delta^{n+1}} & A \\
    	{\bigsqcup_I\Delta^{n+1}} & X
    	\arrow["\mathlarger{\mathlarger{\mathlarger{\mathlarger{\lrcorner}}}}"{anchor=center, pos=0.1, rotate=180}, draw=none, from=2-2, to=1-1]
    	\arrow["{\sqcup_i(a,\dotsc,a,\gamma_i)}", from=1-1, to=1-2]
    	\arrow["j", hook, from=1-2, to=2-2]
    	\arrow[hook, from=1-1, to=2-1]
    	\arrow[from=2-1, to=2-2]
    \end{tikzcd}
  \end{center}
  Alors~:
  \begin{description}
    \item[$(a)$] $\pi_m(j):\pi_m(A,a)\ra \pi_m(X,a)$ est un isomorphisme pour $m<n$~;
    \item[$(b)$] $\pi_n(j):\pi_n(A,a)\ra \pi_n(X,a)$ est surjective.
  \end{description}
  Soit $f:A\ra Y$ avec $Y$ fibrant. Alors s'équivalent~:
  \begin{description}
    \item[$(i)$] Il existe $g:X\ra Y$ tel que $f=g\circ j$~;
    \item[$(ii)$] Pour tout $i\in I$, $f_*[\gamma_i]=0$ dans $\pi_n(Y,f(a))$.
  \end{description}
\end{lem}

\begin{proof}
  L'équivalence entre $(i)$ et $(ii)$ est immédiate. Montrons les points $(a)$ et $(b)$. Si $A\overset{\sim}{\hookrightarrow}\tilde{A}\twoheadrightarrow *$,
  on a le diagramme commutatif suivant~:
  \begin{center}
    % https://q.uiver.app/?q=WzAsNixbMCwwLCJcXGJpZ3NxY3VwX0lcXHBhcnRpYWxcXERlbHRhXntuKzF9Il0sWzAsMSwiXFxiaWdzcWN1cF9JXFxEZWx0YV57bisxfSJdLFsxLDAsIkEiXSxbMSwxLCJYIl0sWzIsMCwiXFx0aWxkZXtBfSJdLFsyLDEsIlxcdGlsZGV7WH0iXSxbMywwLCIiLDAseyJzdHlsZSI6eyJuYW1lIjoiY29ybmVyIn19XSxbMCwyLCJcXHNxY3VwX2koYSxcXGRvdHNjLGEsXFxnYW1tYV9pKSJdLFsyLDMsImoiLDAseyJzdHlsZSI6eyJ0YWlsIjp7Im5hbWUiOiJob29rIiwic2lkZSI6InRvcCJ9fX1dLFswLDEsIiIsMCx7InN0eWxlIjp7InRhaWwiOnsibmFtZSI6Imhvb2siLCJzaWRlIjoidG9wIn19fV0sWzEsM10sWzIsNCwiXFxzaW0iLDAseyJzdHlsZSI6eyJ0YWlsIjp7Im5hbWUiOiJob29rIiwic2lkZSI6InRvcCJ9fX1dLFszLDUsIlxcc2ltIiwwLHsic3R5bGUiOnsidGFpbCI6eyJuYW1lIjoiaG9vayIsInNpZGUiOiJ0b3AifX19XSxbNCw1LCIiLDEseyJzdHlsZSI6eyJ0YWlsIjp7Im5hbWUiOiJob29rIiwic2lkZSI6InRvcCJ9fX1dLFs1LDIsIiIsMSx7InN0eWxlIjp7Im5hbWUiOiJjb3JuZXIifX1dXQ==
    \begin{tikzcd}
    	{\bigsqcup_I\partial\Delta^{n+1}} &[3ex] A & {\tilde{A}} \\
    	{\bigsqcup_I\Delta^{n+1}} & X & {\tilde{X}}
      \arrow["\mathlarger{\mathlarger{\mathlarger{\mathlarger{\lrcorner}}}}"{anchor=center, pos=0.1, rotate=180}, draw=none, from=2-2, to=1-1]
    	\arrow["{\sqcup_i(a,\dotsc,a,\gamma_i)}", from=1-1, to=1-2]
    	\arrow["j", hook, from=1-2, to=2-2]
    	\arrow[hook, from=1-1, to=2-1]
    	\arrow[from=2-1, to=2-2]
    	\arrow["\sim", hook, from=1-2, to=1-3]
    	\arrow["\sim", hook, from=2-2, to=2-3]
    	\arrow[hook, from=1-3, to=2-3]
    	\arrow["\mathlarger{\mathlarger{\mathlarger{\mathlarger{\lrcorner}}}}"{anchor=center, pos=0.1, rotate=180}, draw=none, from=2-3, to=1-2]
    \end{tikzcd}
  \end{center}
  Donc on peut supposer sans perte de généralités que $A$ est un complexe de Kan. Or $X$ vérifie alors la propriété d'extension
  relativement aux inclusions $\Lambda_k^m\hookrightarrow \Delta^m$ pour $m\leq n$. En effet $X_m=A_m$ pour $m\leq n$. Donc,
  $X':=G^\infty(\enstq{\Lambda_k^l\hookrightarrow\Delta^l}{l>n},X\twoheadrightarrow *)$ est un remplacement fibrant de $X$.
  Or, $X'_m=A_m$ pour $m\leq n$, donc sur les groupes d'homotopie simpliciaux, $\pi_m(A,a)\ra \pi_m(X',a)$
  est un isomorphisme pour $m<n$ et une surjection pour $m=n$.
\end{proof}

\begin{defi}[tour de Postnikov]\label{defiPostnikov}
  Soit $X$ une ensemble simplicial. Une tour de Postnikov est un diagramme commutatif~:
  \begin{center}
    % https://q.uiver.app/?q=WzAsNixbMCwyLCJYIl0sWzIsMSwiWChuKSJdLFsyLDIsIlgobi0xKSJdLFsyLDMsIlgoMSkiXSxbMiw0LCJYKDApIl0sWzIsMF0sWzAsMSwiaV9uIl0sWzAsMiwiaV97bi0xfSJdLFswLDMsImlfMSJdLFswLDQsImlfMCJdLFszLDQsInFfMSJdLFsyLDMsIiIsMSx7InN0eWxlIjp7ImJvZHkiOnsibmFtZSI6ImRvdHRlZCJ9fX1dLFsxLDIsInFfbiJdLFs1LDEsIiIsMCx7InN0eWxlIjp7ImJvZHkiOnsibmFtZSI6ImRvdHRlZCJ9fX1dXQ==
    \begin{tikzcd}
    	&& {} \\
    	&& {X(n)} \\
    	X && {X(n-1)} \\
    	&& {X(1)} \\
    	&& {X(0)}
    	\arrow["{i_n}", from=3-1, to=2-3]
    	\arrow["{i_{n-1}}", from=3-1, to=3-3]
    	\arrow["{i_1}", from=3-1, to=4-3]
    	\arrow["{i_0}", from=3-1, to=5-3]
    	\arrow["{q_1}", from=4-3, to=5-3]
    	\arrow[dotted, from=3-3, to=4-3]
    	\arrow["{q_n}", from=2-3, to=3-3]
    	\arrow[dotted, from=1-3, to=2-3]
    \end{tikzcd}
  \end{center}
  tel que pour tout $v\in X_0$, $\pi_j(X(n),v)=0$ pour $j>n$ et $(i_n)_*:\pi_j(X,v)\ra\pi_j(X(n),v)$ est un isomorphisme pur $j\leq n$.
\end{defi}

\begin{prop}
  Tout ensemble simplicial $X$ admet une tour de Postnikov.
\end{prop}

\begin{proof}
  On peut supposer sans perte de généralités que $X$ est connexe. Fixons $n\geq 0$. Soit $X\overset{\sim}{\hookrightarrow}\tilde{X}$
  un remplacement fibrant. On pose $X(n,n)=X$ et $\tilde{X}(n,n)=\tilde{X}$. On définit $X(n,n+1)$ comme la somme amalgamée~:
  \begin{center}
    % https://q.uiver.app/?q=WzAsNCxbMSwwLCJcXHRpbGRle1h9Il0sWzEsMSwiWChuLG4rMSkiXSxbMCwwLCJcXGJpZ3NxY3VwX0lcXHBhcnRpYWxcXERlbHRhXntuKzJ9Il0sWzAsMSwiXFxiaWdzcWN1cF9JXFxEZWx0YV57bisyfSJdLFswLDEsIiIsMCx7InN0eWxlIjp7InRhaWwiOnsibmFtZSI6Imhvb2siLCJzaWRlIjoidG9wIn19fV0sWzIsMywiIiwwLHsic3R5bGUiOnsidGFpbCI6eyJuYW1lIjoiaG9vayIsInNpZGUiOiJ0b3AifX19XSxbMiwwLCJmIl0sWzMsMV0sWzEsMiwiIiwwLHsic3R5bGUiOnsibmFtZSI6ImNvcm5lciJ9fV1d
    \begin{tikzcd}
      {\bigsqcup_I\partial\Delta^{n+2}} & {\tilde{X}(n,n)} \\
      {\bigsqcup_I\Delta^{n+2}} & {X(n,n+1)}
      \arrow[hook, from=1-2, to=2-2]
      \arrow[hook, from=1-1, to=2-1]
      \arrow["f", from=1-1, to=1-2]
      \arrow[from=2-1, to=2-2]
      \arrow["\mathlarger{\mathlarger{\mathlarger{\mathlarger{\lrcorner}}}}"{anchor=center, pos=0.1, rotate=180}, draw=none, from=2-2, to=1-1]
    \end{tikzcd}
  \end{center}
  où $f$ est induit par un ensemble $I$ de générateurs de $\pi_{n+1}(\tilde{X}(n,n),v)$. Soit $X(n,n+1)\overset{\sim}{\hookrightarrow}\tilde{X}(n,n+1)$
  un remplacement fibrant. On définit $X(n,n+2)$ de façon similaire à partir de $\tilde{X}(n,n+1)$ avec $I$ un ensemble de générateurs de
  $\pi_{n+2}(\tilde{X}(n,n+1),v)$. On a alors une suite de cofibrations~:
  $$X=X(n,n)\overset{\sim}{\hookrightarrow}\tilde{X}(n,n)\hookrightarrow\dotsb\hookrightarrow X(n,n+k)\hookrightarrow\tilde{X}(n,n+k)$$
  On définit $X(n)$ comme la colimite de cette suite. On dispose de $i_n:X\ra X(n)$. On remarque également que $X(n)$ est un complexe de Kan.
  On a également les propriétés souhaitée sur $\pi_m(i_n)$ et les $\pi_m(X(n))$.

  Il reste à construire les $q_n$. On les construit par récurrence sur $n$. Pour compléter la récurrence, il suffit
  de construire $q_n:X(n)\ra X(n-1)$ tel que $i_{n-1}=q_n\circ i_n$. Or, on a le diagramme commutatif suivant~:
  \begin{center}
    % https://q.uiver.app/?q=WzAsNixbMiwwLCJYKG4sbitrKSJdLFszLDAsIlxcdGlsZGV7WH0obixuK2spIl0sWzQsMCwiWChuLG4raysxKSJdLFsxLDIsIlgobi0xKSJdLFsxLDAsIlxcdGlsZGV7WH0iXSxbMCwwLCJYIl0sWzAsMSwiXFxzaW0iLDAseyJzdHlsZSI6eyJ0YWlsIjp7Im5hbWUiOiJob29rIiwic2lkZSI6InRvcCJ9fX1dLFsxLDIsIiIsMCx7InN0eWxlIjp7InRhaWwiOnsibmFtZSI6Imhvb2siLCJzaWRlIjoidG9wIn19fV0sWzAsMywiXFxleGlzdHMgcV97bixuK2t9IiwxLHsic3R5bGUiOnsiYm9keSI6eyJuYW1lIjoiZGFzaGVkIn19fV0sWzEsMywiXFxleGlzdHMgXFx0aWxkZXtxfV97bixuK2t9IiwxLHsic3R5bGUiOnsiYm9keSI6eyJuYW1lIjoiZGFzaGVkIn19fV0sWzIsMywiXFxleGlzdHMgcV97bixuK2srMX0iLDEseyJzdHlsZSI6eyJib2R5Ijp7Im5hbWUiOiJkYXNoZWQifX19XSxbNSw0LCJcXHNpbSIsMCx7InN0eWxlIjp7InRhaWwiOnsibmFtZSI6Imhvb2siLCJzaWRlIjoidG9wIn19fV0sWzQsMCwiIiwwLHsic3R5bGUiOnsidGFpbCI6eyJuYW1lIjoiaG9vayIsInNpZGUiOiJ0b3AifSwiYm9keSI6eyJuYW1lIjoiZG90dGVkIn19fV0sWzQsMywiXFxleGlzdHMgXFx0aWxkZXtxfV97bixufSIsMSx7InN0eWxlIjp7ImJvZHkiOnsibmFtZSI6ImRhc2hlZCJ9fX1dLFs1LDMsImlfe24tMX0iLDJdXQ==
    \begin{tikzcd}
    	X & {\tilde{X}} & {X(n,n+k)} & {\tilde{X}(n,n+k)} & {X(n,n+k+1)} \\
    	\\
    	& {X(n-1)}
    	\arrow["\sim", hook, from=1-3, to=1-4]
    	\arrow[hook, from=1-4, to=1-5]
    	\arrow["{\exists q_{n,n+k}}"{description}, dashed, from=1-3, to=3-2]
    	\arrow["{\exists \tilde{q}_{n,n+k}}"{description}, dashed, from=1-4, to=3-2]
    	\arrow["{\exists q_{n,n+k+1}}"{description}, dashed, from=1-5, to=3-2]
    	\arrow["\sim", hook, from=1-1, to=1-2]
    	\arrow[dotted, hook, from=1-2, to=1-3]
    	\arrow["{\exists \tilde{q}_{n,n}}"{description}, dashed, from=1-2, to=3-2]
    	\arrow["{i_{n-1}}"', from=1-1, to=3-2]
    \end{tikzcd}
  \end{center}
  où l'existence des $\tilde{q}_{n,n+k}$ est garantie car $X(n-1)$ est un complexe de Kan, et l'existence
  des $q_{n,n+k}$ est garantie par le \sref{lemme d'extension}{lemmedextension}. En passant à la colimite, on obtient l'application
  $q_n:X(n)\ra X(n-1)$ souhaitée.
\end{proof}

\begin{rem}
  Si $X$ dispose d'une tour de Postnikov, on peut successivement remplacer $X(0)$ par un complexe de Kan, et les $q_n$ par des fibrations,
  On obtient ainsi une tour de Postnikov où tous les $q_n$ sont des fibrations. On appelle une telle tour une tour de Postnikov de fibrations.  
\end{rem}

\begin{lem}\label{lemmePostnikovLimite}
  Soit $n\geq 0$ et $p:X\twoheadrightarrow Y$ une fibration de Kan entre complexes de Kan. Soit $x\in X_0$.
  On suppose que $\pi_{n+1}(X,x)\ra\pi_{n+1}(Y,px)$ est surjectif.
  Soit~:
  \[
    \begin{array}{lcl}
      \gamma:(\Delta^n,\partial\Delta^n)&\ra& (X,x) \\
      \sigma:\Delta^{n+1}&\ra& Y
    \end{array}
  \]
  tels que $[\gamma]=*$ et $\partial\sigma=(px,\dotsc,px,p\gamma)$.

  Alors il existe $\tau:\Delta^{n+1}\ra Y$ tel que $\partial\tau=(x,\dotsc,x,\gamma)$ et $f\tau =\sigma$.
\end{lem}

\begin{proof}
  Soit $h:\Delta^{n}\times\Delta^1\ra X$ homotopie entre $\gamma$ et $*$, on a alors~:
  \begin{center}
    % https://q.uiver.app/?q=WzAsNSxbMCwwLCJcXHBhcnRpYWxcXERlbHRhXntuKzF9XFx0aW1lc1xcRGVsdGFeMSJdLFsxLDAsIlkiXSxbMCwxLCJcXERlbHRhXntuKzF9XFx0aW1lc1xcRGVsdGFeMSJdLFswLDJdLFsxLDEsIioiXSxbMCwxLCIoeCxcXGRvdHNjLHgsaCkiXSxbMSw0LCIiLDAseyJzdHlsZSI6eyJoZWFkIjp7Im5hbWUiOiJlcGkifX19XSxbMiw0XSxbMiwxLCJcXGV4aXN0cyBcXHRoZXRhIiwwLHsic3R5bGUiOnsiYm9keSI6eyJuYW1lIjoiZGFzaGVkIn19fV0sWzAsMiwiIiwwLHsic3R5bGUiOnsidGFpbCI6eyJuYW1lIjoiaG9vayIsInNpZGUiOiJ0b3AifX19XV0=
    \begin{tikzcd}
    	{\partial\Delta^{n+1}\times\Delta^1} &[4ex] Y \\
    	{\Delta^{n+1}\times\Delta^1} & {*} \\
    	\arrow["{(px,\dotsc,px,ph)}", from=1-1, to=1-2]
    	\arrow[two heads, from=1-2, to=2-2]
    	\arrow[from=2-1, to=2-2]
    	\arrow["{\exists \theta}" sloped, dashed, from=2-1, to=1-2]
    	\arrow[hook, from=1-1, to=2-1]
    \end{tikzcd}
  \end{center}
  Maintenant, $\tilde{\sigma}:=\theta\circ (id\times d^0)$ représente un élément de $\pi_{n+1}(Y,px)$. Soit $\tilde{\tau}$ représentant un élément
  de $\pi_{n+1}(X,x)$ tel que $p_*[\tilde{\tau}]=[\tilde{\sigma}]$. On se donne $g:\Delta^{n+1}\times \Delta^{1}\ra Y$ homotopie
  $\mathrm{rel}\;\partial\Delta^{n+1}$ entre $\tilde{\sigma}$ et $p\tilde{\tau}$. On dispose alors de $\beta:\Delta^{n+1}\times\Delta^2\ra Y$ On a alors~:
  \begin{center}
    % https://q.uiver.app/?q=WzAsNSxbMCwwLCJcXHBhcnRpYWxcXERlbHRhXntuKzF9XFx0aW1lc1xcTGFtYmRhXzFeMlxcY3VwXFxEZWx0YV57bisxfVxcdGltZXNcXHsyXFx9Il0sWzEsMCwiWCJdLFswLDEsIlxcRGVsdGFee24rMX1cXHRpbWVzXFxMYW1iZGFfMV4yIl0sWzAsMl0sWzEsMSwiWSJdLFswLDEsIigoeCxcXGRvdHNjLHgsZiksLSx4KSlcXGN1cCBcXHRpbGRle1xcdGF1fSJdLFsxLDQsIiIsMCx7InN0eWxlIjp7ImhlYWQiOnsibmFtZSI6ImVwaSJ9fX1dLFsyLDRdLFsyLDEsIlxcZXhpc3RzIEgiLDAseyJzdHlsZSI6eyJib2R5Ijp7Im5hbWUiOiJkYXNoZWQifX19XSxbMCwyLCIiLDAseyJzdHlsZSI6eyJ0YWlsIjp7Im5hbWUiOiJob29rIiwic2lkZSI6InRvcCJ9fX1dXQ==
    \begin{tikzcd}
    	{\partial\Delta^{n+1}\times\Lambda_1^2\cup\Delta^{n+1}\times\{2\}} &[8ex] X \\
    	{\Delta^{n+1}\times\Lambda_1^2} & Y \\
    	\arrow["{(x,-,(x,\dotsc,x,f))\cup \tilde{\tau}}", from=1-1, to=1-2]
    	\arrow[two heads, from=1-2, to=2-2]
    	\arrow["{(g,-,h)}",from=2-1, to=2-2]
    	\arrow["{\exists H}" sloped, dashed, from=2-1, to=1-2]
    	\arrow[hook, from=1-1, to=2-1]
    \end{tikzcd}
  \end{center}
  Alors, $\tau:=H\circ (id\times (d^1\circ d^2))$ convient.
\end{proof}

\begin{prop}\label{proplimiteinversedefibrations}
  Soit une suite de fibrations~:
  $$\dotsb\twoheadrightarrow X_n\twoheadrightarrow\dotsb\twoheadrightarrow X_1\twoheadrightarrow X_0\twoheadrightarrow *$$
  dans $\DEns$. Soit $i\geq 0$ et $x\in (\lim_n X_n)_0$.
  Alors~:
  $$\lambda:\pi_i(\lim_n X_n,x)\ra \lim_n \pi_i(X_n,x)$$
  est surjectif. Si de plus $\pi_{i+1}(X_n)\ra \pi_{i+1}(X_{n-1})$ est surjectif pour $n$ assez grand,
  $\lambda$ est un isomorphisme.
\end{prop}

\begin{proof}
  Soient $f_n:(\Delta^i,\partial\Delta^i)\ra (X_n,*)$ tels que $[f_{n+1}]\mapsto [f_n]$. Alors, par relèvement des homotopie
  (ie. relativement à $\partial\Delta^{i}\times\Delta^1\cup\Delta^{i}\times {0}\ra\Delta^i\times\Delta^1$), on peut successivement construire
  $f'_n:(\Delta^i,\partial\Delta^i)\ra (X_n,*)$ tels que $[f'_n]=[f_n]$ et $f'_n\mapsto f'_{n+1}$. On a construit $f'\in \pi_i(\lim_n X_n,x)$
  tel que $f'\mapsto (f_n)_n$.

  Supposons maintenant que $\pi_{i+1}(X_n)\ra \pi_{i+1}(X_{n-1})$ est surjectif pour $n$ assez grand. On peut, sans perte de généralité,
  supposer que $\pi_{i+1}(X_n)\ra \pi_{i+1}(X_{n-1})$ est surjectif pour tout $n$. Soit $f$ un représentant d'un élément dans le noyau de $\lambda$,
  et $f_n$ l'image dans $X_n$.
  Alors par le \sref{lemme}{lemmePostnikovLimite}, on peut construire successivement des $\sigma_n:\Delta^{i+1}\ra X_n$
  tels que $\partial\tau=(*,\dotsc,*,f_n)$ et $\tau_n\mapsto \tau_{n-1}$. Ces $(\tau_n)_{n\geq 0}$ forment donc
  un $\tau:\Delta^{i+1}\ra \lim_n X_n$ tel que $\partial\tau=(*,\dotsc,*,f)$. Donc $[f]=*$.
\end{proof}

\begin{coro}
  Soit $X$ un ensemble simplicial et $(X(n),i_n,q_n)_n$ une tour de Postnikov de fibrations pour $X$.
  Alors l'application~:
  $$X\ra \lim_{n\in\N}X(n)$$
  est une équivalence d'homotopie faible.
\end{coro}

\begin{proof}
  D'après la \sref{proposition}{proplimiteinversedefibrations}, pour chaque $i\geq 0$ et $x\in X_0$, on a un isomorphisme~:
  $$\pi_i(X,x)\overset{\sim}{\ra}\pi_i(\lim_{n\in\N}X(n),x)$$
  Donc c'est une équivalence d'homotopie faible.
\end{proof}


\subsection{le théorème de Hurewicz}

Nous allons admettre les démonstrations des différentes versions du théorème de Hurewicz, car elles utilisent des outils que nous n'avons pas introduit
(Suite spectrale de Serre ou théorème d'excision en homotopie, voir les remarques).

\begin{defi}[morphisme de Hurewicz]
  Soit $X$ un ensemble simplicial, $n\geq 1$ et $x\in X_0$. L'application naturelle $X\ra \Z X$ induit le diagramme suivant~:
  $$\pi_n(X,x)\lra \pi_n(\Z X/\Z x, 0)\longleftarrow\pi_n(\Z X, 0)$$
  Or, d'après [mettre ref Dold-Kan], pour $A$ groupe abélien simplicial, on a un isomorphe fonctoriel~:
  $$\pi_n(A,0)\simeq H_n(A)$$
  Or $H_n(\Z x) = 0$. Donc $\pi_n(\Z X, 0)\ra \pi_n(\Z X/\Z x, 0)$ est un isomorphisme, et donc on dispose d'un morphisme~:
  $$\mathcal{H}_n(X,x):\pi_n(X,x)\ra H_n(X)$$
  On appelle ce morphisme le morphisme de Hurewicz. Il est clairement fonctoriel en $X$.
\end{defi}

\begin{theo}[Hurewicz pour le $\pi_1$]\label{theoremeHurewiczpi1}
  Soit $X$ ensemble simplicial connexe et $x\in X_0$. Alors $\mathcal{H}_1(X,x)$ induit un isomorphisme~:
  $$\pi_1(X,x)\ab\lra H_1(X)$$
\end{theo}

\begin{proof}
  On peut supposer que $X$ est un complexe de Kan. Pour $y\in X_0$, on se donne un chemin $\gamma_y:\Delta^1\ra X$ de $x$ à $y$.
  Une réciproque est alors donnée par~:
  $$\sum_i [\delta_i]\mapsto \prod_i[\gamma_{\delta(1)}]^{-1}[\delta_i][\gamma_{\delta(0)}]$$
\end{proof}

\begin{theo}[Hurewicz pour $pi_n$, $n\geq 2$]\label{theoremeHurewitzpin}
  Soit $X$ un ensemble simplicial connexe, $x\in X_0$ et $n\geq 2$. Si $\pi_m(X,x)=0$ pour tout $1\leq m<n$, alors l'application de Hurewicz~:
  $$\mathcal{H}_n(X,x):\pi_n(X,x)\ra H_n(X)$$
  est un isomorphisme.
\end{theo}

Nous ne démontrons pas ici ce théorème. Une première preuve qui se base sur la suite spectrale de Serre est donnée dans \cite[III.3.7]{Goer}.
En examinant la preuve, on remarque que l'existence d'un isomorphisme entre $\pi_n(X,x)$ et $H_n(X)$ est une conséquence immédiate de la suite spectrale.
Il est plus difficile d'identifier cet isomorphisme avec $\mathcal{H}_n(X,x)$. Une seconde preuve qui se base sur le théorème d'excision en homotopie
est donnée dans \cite[4.37]{Hatc} (prendre $A=x$).

\begin{defi}[morphisme de Hurewicz pour les paires]\label{theoremeHurewiczpaire}
  Soit $X$ un ensemble simplicial, $A$ un sous-ensemble simplicial de $X$, et $x\in A_0$. Soit $n\geq 1$.
  Le morphisme naturel $(X,A)\ra (\Z X/\Z x,\Z A/\Z x)$ induit~:
  $$\pi_n(X,A,x)\lra \pi_n(\Z X/\Z x,\Z A/\Z x,0)\longleftarrow \pi_n(\Z X,\Z A,0)$$
  Or, d'après [mettre ref Dold-Kan], pour $(U,V)$ paire des groupe abélien simplicial, on a un isomorphe fonctoriel~:
  $$\pi_n(U,V,0)\simeq H_n(U/V)$$
  Donc $\pi_n(\Z X,\Z A,0)\ra \pi_n(\Z X/\Z x,\Z A/\Z x,0)$ est un isomorphisme, et donc on dispose d'un morphisme~:
  $$\mathcal{H}_n(X,A,x):\pi_n(X,A,x)\ra H_n(X,A)$$
  On appelle ce morphisme le morphisme de Hurewicz pour les paires. Il est clairement fonctoriel en $(X,A)$.
\end{defi}

\begin{rem}
  De la compatibilité aux suites exactes longues énoncée dans la [mettre ref Dold-Kan], on déduit que les morphismes de Hurewicz sont compatibles
  aux suites exactes longues des paires. Plus précisément, si $x\in A\subseteq X$, alors le digramme suivant commute pour $n\geq 1$~:
  \begin{center}
    % https://q.uiver.app/?q=WzAsOCxbMSwwLCJcXHBpX24oQSx4KSJdLFsyLDAsIlxccGlfbihYLHgpIl0sWzAsMCwiXFxwaV97bisxfShYLEEseCkiXSxbMCwxLCJIX3tuKzF9KFgsQSkiXSxbMywwLCJcXHBpX3tufShYLEEseCkiXSxbMywxLCJIX3tufShYLEEpIl0sWzEsMSwiSF97bn0oQSkiXSxbMiwxLCJIX3tufShYKSJdLFszLDYsIlxccGFydGlhbCJdLFsyLDAsIlxccGFydGlhbCJdLFswLDFdLFs2LDddLFsxLDRdLFs3LDVdLFsyLDMsIlxcbWF0aGNhbHtIfV97bisxfShYLEEseCkiLDFdLFs0LDUsIlxcbWF0aGNhbHtIfV97bn0oWCxBLHgpIiwxXSxbMSw3LCJcXG1hdGhjYWx7SH1fe259KFgseCkiLDFdLFswLDYsIlxcbWF0aGNhbHtIfV97bn0oQSx4KSIsMV1d
    \begin{tikzcd}[column sep = large, row sep = large]
    	{\pi_{n+1}(X,A,x)} & {\pi_n(A,x)} & {\pi_n(X,x)} & {\pi_{n}(X,A,x)} \\
    	{H_{n+1}(X,A)} & {H_{n}(A)} & {H_{n}(X)} & {H_{n}(X,A)}
    	\arrow["\partial", from=2-1, to=2-2]
    	\arrow["\partial", from=1-1, to=1-2]
    	\arrow[from=1-2, to=1-3]
    	\arrow[from=2-2, to=2-3]
    	\arrow[from=1-3, to=1-4]
    	\arrow[from=2-3, to=2-4]
    	\arrow["{\mathcal{H}_{n+1}(X,A,x)}"{description}, from=1-1, to=2-1]
    	\arrow["{\mathcal{H}_{n}(X,A,x)}"{description}, from=1-4, to=2-4]
    	\arrow["{\mathcal{H}_{n}(X,x)}"{description}, from=1-3, to=2-3]
    	\arrow["{\mathcal{H}_{n}(A,x)}"{description}, from=1-2, to=2-2]
    \end{tikzcd}
  \end{center}
\end{rem}

\begin{theo}[Hurewicz pour les paires]
  Soit $n\geq 2$ et $(X,A)$ une paire d'ensembles simpliciaux tels que $\pi_1(A,x)$ agisse trivialement sur $\pi_n(X,A,x)$.
  Alors le morphisme de Hurewicz~:
  $$\mathcal{H}_n(X,A,x):\pi_n(X,A,x)\ra H_n(X,A)$$
  est un isomorphisme.
\end{theo}

Nous ne démontrons pas ici ce théorème. Voir \cite[4.37]{Hatc} pour une preuve.

\subsection{Espaces d'Eilenberg Mac-Lane et tours de Postnikov de fibrations principales}

La première partie de cette sous-section est consacrée à la définition des espaces d'Eilenberg-MacLane et la seconde aux tours de Postnikov
de fibrations principales. La première utilise beaucoup la correspondance de Dold-Kan énoncée dans la \sref{sous-section}{DoldKan}.
La seconde utilise le théorème de Hurewicz pour les paires pour démontrer l'existence de tours de Postnikov de fibrations principales pour
les espaces simples. Ce résultat est central pour l'unicité de la construction $+$.

\begin{defi}
  Soit $n\geq 1$ et $\Pi$ un groupe qui soit abélien si $n\geq 2$. Un espace d'Eilenberg-MacLane pour $n$ et $\Pi$ est un ensemble simplicial
  connexe $K(\Pi,n)$ tel que~:
  $$\pi_m(K(\Pi,n))=\begin{cases}\Pi\text{ si }m=n\\ 0\text{ sinon}\end{cases}$$
\end{defi}

\begin{rem}
  La donnée de $K(\Pi,n)$ comprend un isomorphisme $\pi_m(K(\Pi,n))\simeq \Pi$. Si $n=1$ et $\Pi$ n'est pas abélien, il faut choisir un point base pour
  $K(\Pi,n)$. En pratique, on se restreint ici aux espaces simples.
\end{rem}

\begin{rem}
  Dans la suite de la section et le reste du mémoire, $K(\Pi, n)$ désigne toujours un espace d'Eilenberg-MacLane pour $n$ et $\Pi$.
\end{rem}

\begin{prop}
  Soit $n\geq 1$ et $\Pi$ un groupe qui soit abélien si $n\geq 2$. Alors il existe un espace d'Eilenberg-MacLane pour $n$ et $\Pi$.
  Plus précisément, on peut utiliser les deux constructions suivantes.

  Si $n=1$, $N\Pi$ est un espace d'Eilenberg-MacLane pour $n$ et $\Pi$.

  Si $n\geq 2$, $C_*$ le complexe de chaîne défini par~:
  $$C_n=\begin{cases}\Pi\text{ si }m=n\\ 0\text{ sinon}\end{cases}$$
  On note $\Gamma C_*$ le groupe abélien simplicial associé (voir [Dold-Kan]).
  Alors $\Gamma C_*$ est un espace d'Eilenberg-MacLane pour $n$ et $\Pi$.
\end{prop}

\begin{proof}
  Si $n=1$, $N\Pi$ convient. On se place maintenant dans le cas $n\geq 2$. Par la [Dold-Kan], on a pour $m\geq 0$~:
  $$\pi_m(\Gamma C_*,0)=H_n(C_ *)=\begin{cases}\Pi\text{ si }m=n\\ 0\text{ sinon}\end{cases}$$
\end{proof}

Ci-dessous, on ne démontre l'unicité que dans le cas abélien, pour éviter de devoir passer par les espaces pointés.

\begin{prop}
  Soit $n\geq 1$, $\Pi$ un groupe abélien, et $K(\Pi,n)$ un complexe de Kan et un espace d'Eilenberg-MacLane pour $n$ et $\Pi$.
  Si $K'$ en est un autre, alors il existe une
  équivalence d'homotopie faible $\theta:K(\Pi,n)\ra K'$ telle que $\pi_n(\theta)$ d'identifie à $\id_\Pi$.
\end{prop}

\begin{proof}
  Soit $I$ un ensemble de générateurs de $\Pi$, $X_{n-1}:=\bigvee_I\Delta^n/\partial\Delta^n$ et $\theta_{n-1}:X_{n-1}\ra K'$ le morphisme
  associé à l'inclusion $I\ra pi_n(K,*)$. Soit $J_n$ un ensemble de générateurs du noyau de $\Z^{(I)}\ra \Pi$. On définit $X_n$ comme la somme amalgamée~:
  \begin{center}
    \begin{tikzcd}[column sep = 9ex]
    	{\bigsqcup_{J_n}\partial\Delta^{n+1}} & X_{n-1} \\
    	{\bigsqcup_{J_n}\Delta^{n+1}} & X_n
    	\arrow["\mathlarger{\mathlarger{\mathlarger{\mathlarger{\lrcorner}}}}"{anchor=center, pos=0.1, rotate=180}, draw=none, from=2-2, to=1-1]
    	\arrow["{\sqcup_j(a,\dotsc,a,\gamma_j)}", from=1-1, to=1-2]
    	\arrow[hook, from=1-2, to=2-2]
    	\arrow[hook, from=1-1, to=2-1]
    	\arrow[from=2-1, to=2-2]
    \end{tikzcd}
  \end{center}
  On a alors, par le \sref{lemme d'extension}{lemmedextension}, une factorisation $\theta_n:X_n\ra K'$ de $\theta_{n-1}$. De plus,
  on a maintenant nécessairement que $\pi_n(\theta_n)$ est un isomorphisme.
  Nous pouvons maintenant définir par récurrence $X_m$ pour $m\geq n$ de la façon suivante. Soit $J_m$ un ensemble de générateurs
  de $\pi_m(X_m)$. On définit alors $X_{m+1}$ comme la somme amalgamée~:
  \begin{center}
    \begin{tikzcd}[column sep = 9ex]
    	{\bigsqcup_{J_m}\partial\Delta^{n+1}} & X_{m} \\
    	{\bigsqcup_{J_m}\Delta^{n+1}} & X_{m+1}
    	\arrow["\mathlarger{\mathlarger{\mathlarger{\mathlarger{\lrcorner}}}}"{anchor=center, pos=0.1, rotate=180}, draw=none, from=2-2, to=1-1]
    	\arrow["{\sqcup_j(a,\dotsc,a,\gamma_j)}", from=1-1, to=1-2]
    	\arrow[hook, from=1-2, to=2-2]
    	\arrow[hook, from=1-1, to=2-1]
    	\arrow[from=2-1, to=2-2]
    \end{tikzcd}
  \end{center}
  Maintenant, par le \sref{lemme d'extension}{lemmedextension}, on peut étendre $\theta_n$ en $\theta_m:X_m\ra K'$ pour chaque $m\geq 0$.
  Quitte à le faire dans l'ordre, on peut choisir une famille compatible de $(\theta_m)_m$. Alors on dispose d'un morphisme~:
  $$\theta:X:=\colim_m X_m\ra K'$$
  qui est un isomorphisme sur chaque $\pi_m$, et donc une équivalence faible.

  Cependant, on remarque que la construction de $X$ est indépendante de $K'$, en effet elle ne dépend que du choix de $I$ et des $J_m$
  pour $m\geq n$. Donc chaque espace d'Eilenberg-MacLane pour $n$ et $\Pi$ est homotopiquement équivalent à $X$. Ce qui conclut. 
\end{proof}

\begin{theo}
  Soit $n\geq 1$ et $\Pi$ un groupe abélien. Alors si on note $[-,-]$ pour $\Hom{\Ho{\DEns}}{-}{-}$, on a, pour $K(\Pi,n)$ un espace
  d'Eilenberg-MacLane pour $n$ et $\Pi$, un isomorphisme de foncteurs~:
  $$[-,K(\Pi,n)]\simeq H^n(-;\Pi)$$
\end{theo}

\begin{proof}
  Il suffit de le montrer pour une construction. On choisit donc $K(\Pi,n):=\Gamma C_*$, avec $C_*$ le complexe de chaîne définit par~:
  $$C_n=\begin{cases}\Pi\text{ si }m=n\\ 0\text{ sinon}\end{cases}$$
  On utilise maintenant les structures de modèles sur $\DEns$ et $\DAb$. Dans $\DEns$, on a le cône fonctoriel en $X$ suivant~:
  \begin{center}
    % https://q.uiver.app/?q=WzAsMyxbMCwwLCJYXFxzcWN1cCBYIl0sWzEsMCwiWFxcdGltZXNcXERlbHRhXjEiXSxbMiwwLCJYIl0sWzAsMSwiIiwwLHsic3R5bGUiOnsidGFpbCI6eyJuYW1lIjoiaG9vayIsInNpZGUiOiJ0b3AifX19XSxbMSwyLCJcXHNpbSIsMCx7InN0eWxlIjp7ImhlYWQiOnsibmFtZSI6ImVwaSJ9fX1dXQ==
    \begin{tikzcd}
    	{X\sqcup X} & {X\times\Delta^1} & X
    	\arrow[hook, from=1-1, to=1-2]
    	\arrow["\sim", two heads, from=1-2, to=1-3]
    \end{tikzcd}
  \end{center}
  Dans $\DAb$, on a le cône fonctoriel en $X\in \DEns$ suivant~:
  \begin{center}
    % https://q.uiver.app/?q=WzAsMyxbMCwwLCJcXFogWFxcc3FjdXAgXFxaIFgiXSxbMSwwLCJcXFogWFxcb3RpbWVzXFxaIFxcRGVsdGFeMSJdLFsyLDAsIlxcWiBYIl0sWzAsMSwiIiwwLHsic3R5bGUiOnsidGFpbCI6eyJuYW1lIjoiaG9vayIsInNpZGUiOiJ0b3AifX19XSxbMSwyLCJcXHNpbSIsMCx7InN0eWxlIjp7ImhlYWQiOnsibmFtZSI6ImVwaSJ9fX1dXQ==
    \begin{tikzcd}
    	{\Z X\sqcup \Z X} & {\Z X\otimes\Z \Delta^1} & {\Z X}
    	\arrow[hook, from=1-1, to=1-2]
    	\arrow["\sim", two heads, from=1-2, to=1-3]
    \end{tikzcd}
  \end{center}
  Or, pour $A$ un groupe abélien simplicial, les morphismes dans $\DEns$ de $X\times \Delta^1$ dans $A$ s'identifient aux morphismes
  dans $\DAb$ de $\Z X\otimes\Z \Delta^1$ dans $A$. Or, comme les groupes abéliens dans $\DEns$ sont fibrants,
  on a~:
  $$[X,A]_{\DEns}\simeq [\Z X,A]_{\DAb}\simeq [N\Z X,N A]_{\Chp}$$
  Or, $N\Z X\ra \Z X$ est une équivalence en homologie d'après [Dold-Kan]. Donc on a un isomorphisme fonctoriel~:
  $$[X,K(\Pi,n)]_{\DEns}\simeq [\Z X,C_*]_{\Chp}\simeq \mathrm{Ext}_\Z^n(\Z X,\Pi)\simeq H^n(X;\Pi)$$
\end{proof}

Nous introduisons maintenant une structure supplémentaire sur certaines tours de Postnikov.

\begin{defi}
  Soit $X$ un ensemble simplicial simple. Une tour de Postnikov de fibrations principales pour $X$ est un diagramme commutatif~:
  \begin{center}
    \begin{tikzcd}
      && {X(n)} \\
      {X} && {X(n-1)} & {K(n+1,\pi_n(X))} \\
      && {X(1)} & {K(3,\pi_2(X))} \\
      && {X(0)\simeq *} & {K(2,\pi_1(X))}
      \arrow["q_n", two heads, from=1-3, to=2-3]
      \arrow["{u_{n-1}}", from=2-3, to=2-4]
      \arrow[dotted, two heads, from=2-3, to=3-3]
      \arrow["{u_1}", from=3-3, to=3-4]
      \arrow["q_1", two heads, from=3-3, to=4-3]
      \arrow["{u_0}", from=4-3, to=4-4]
      \arrow["i_n", from=2-1, to=1-3]
      \arrow["i_{n-1}", from=2-1, to=2-3]
      \arrow["i_1", from=2-1, to=3-3]
      \arrow["i_0", from=2-1, to=4-3]
    \end{tikzcd}
  \end{center}
  Où les $X(n)$, $i_n$ et $q_n$ forment une tour de Postnikov, et tels que les suites~:
  $$X(n)\lra X(n-1)\lra K(n+1,\pi_n(X))$$
  soient des suites fibres pour tout $n\geq 1$.
\end{defi}

\begin{lem}\label{lemmePostnikovfibrationsprincipales}
  Soit $A\subset X$ une paire d'ensembles simpliciaux connexes. Soit $n\geq 1$ et $x\in A$. On suppose que~:
  $$\pi_m(X,A,x)=\begin{cases}\Pi\text{ si }m=n \\0\text{ sinon} \end{cases}$$
  avec $\Pi$ un groupe abélien.
  Alors, si $\pi_1(A,x)$ agit trivialement sur $\pi_n(X,A,x)$, il existe $\theta:X\ra K(\Pi,n)$ tel que la suite~:
  $$A\lra X\lra K(\Pi,n)$$
  soit une suite fibre.
\end{lem}

\begin{proof}
  Par le \sref{théorème de Hurewicz pour les paires}{theoremeHurewiczpaire}, on a un isomorphismes~:
  $$\mathcal{H}_n(X,A,x):\pi_n(X,A,x)\simeq H_n(X,A)$$
  De plus, comme $\pi_m(X,A,x)=0$ pour $m<n$, $\pi_m(A)\ra\pi_m(X)$ est une surjection pour $m<n$.
  Donc on peut supposer sans perte de généralité que $A_m=X_m$ pour $m<n$. Ainsi, $\pi_m(X/A,A/A)=0$ pour $m<n$.
  Donc, par le \sref{théorème de Hurewitz}{theoremeHurewitzpin}, on a également un isomorphisme~:
  $$\mathcal{H}_n(X/A,A/A,A/A):\pi_n(X/A,A/A,A/A)\ra H_n(X/A,A/A)$$
  Or le morphisme naturel $(X,A) \ra (X/A,A/A)$ induit un isomorphisme en homologie. Donc, par fonctorialité du morphisme de Hurewicz
  $\mathcal{H}_n$, $\pi_n(X,A,x)\ra \pi_n(X/A,A/A)$ est un isomorphisme.
  À l'aide du \sref{lemmedextension}{lemmedextension}, on construit un morphisme~:
  $$u:X/A\ra K(\Pi,n)$$
  qui soit un isomorphisme sur les $\pi_m$ pour $m\leq n$. On note $F\ra X$ la fibre homotopique du morphisme induit $\theta:X\ra K(\Pi,n)$.
  Comme $A\ra K(\Pi,n)$ est contractile, on a un morphisme de paires $(A,X)\ra (F,X)$. C'est un isomorphisme sur les groupes d'homotopie des paires
  pour tout $m\geq 1$. Donc, par la suite exacte longue en homotopie, $A\ra F$ est une équivalence d'homotopie. Donc la suite~:
  $$A\lra X\lra K(\Pi,n)$$
  est une suite fibre.
\end{proof}

\begin{theo}\label{Postnikovfibrationsprincipales}
  Soit $X$ un ensemble simplicial simple. Alors il existe une tour de Postnikov de fibrations principales pour $X$.
\end{theo}

\begin{proof}
  On se donne des $X(n)$, $i_n:X\ra X(n)$ et $q_n:X(n)\ra X(n-1)$ qui forment une tour de Postnikov pour $X$.
  Alors, la paire $(X(n),X(n-1))$ vérifie les hypothèses du \sref{lemme}{lemmePostnikovfibrationsprincipales}.
  En effet, $\pi_1(X(n-1),x)=\pi_1(X,x)$ agit trivialement sur les $\pi_m(X(n),x)$ et les $\pi_m(X(n-1),x)$, donc également
  sur les groupes $\pi_m(X(n),X(n-1),x)$ par la suite exacte longue en homotopie.
  Ceci conclut.
\end{proof}

\section*{Remerciements} Je tiens à remercier mon directeur de stage Frédéric Déglise pour ses explications et le temps qu'il m'a accordé.
\nocite{*}


\bibliographystyle{alpha}
\bibliography{references}

\end{document}